\chapter{Mengen und Relationen}
Die Mengenlehre ist gegen Ende des 19-ten Jahrhunderts aus dem Bestreben heraus
entstanden, die Mathematik auf eine solide Grundlage zu stellen.  Die Schaffung einer
solchen Grundlage wurde als notwendig erachtet, da der Begriff der Unendlichkeit den
Mathematikern zunehmends Kopfzerbrechen bereitete.

Begr\"{u}ndet wurde die Mengenlehre in wesentlichen Teilen von Georg Cantor (1845 -- 1918).
Die erste Definition des Begriffs der Menge lautete etwa wie folgt \cite{cantor:1895}:
\begin{center}
Eine \emph{Menge} ist eine \emph{wohldefinierte} Ansammlung von \emph{Elementen}.
\end{center}
Das Attribut ``\emph{wohldefiniert}'' dr\"{u}ckt dabei aus, dass wir f\"{u}r 
eine vorgegebene Menge $M$ und ein Objekt $x$ stets klar sein muss, ob das Objekt $x$
zu der Menge $M$ geh\"{o}rt oder nicht.  In diesem Fall schreiben wir \\[0.2cm]
\hspace*{1.3cm} $x \in M$ \\[0.2cm]
und lesen diese Formel als ``\emph{$x$ ist ein Element der Menge $M$}''.
Das Zeichen ``$\in$'' wird in der Mengenlehre also als zweisteliges Pr\"{a}dikats-Zeichen
gebraucht, f\"{u}r das sich eine Infix-Notation eingeb\"{u}rgert hat.
Um den Begriff der \emph{wohldefinierten Ansammlung von Elementen} mathematisch zu
pr\"{a}zisieren, f\"{u}hrte Cantor das sogenannte \emph{Komprehensions-Axiom} ein.
Wir k\"{o}nnen dieses zun\"{a}chst wie folgt formalisieren: Ist $p(x)$ eine Eigenschaft, die
ein Objekt $x$ haben kann, so k\"{o}nnen wir die Menge $M$ aller Objekte, welche die
Eigenschaft $p(x)$ haben, bilden.  Wie schreiben dann \\[0.2cm]
\hspace*{1.3cm} $M = \{ x \;|\; p(x) \}$ \\[0.2cm]
und lesen dies als ``$M$ ist die Menge aller $x$, f\"{u}r die $p(x)$ gilt''.
Eine Eigenschaft $p(x)$ ist dabei nichts anderes als eine Formel, in der die Variable $x$
vorkommt.
Wir veranschaulichen das Komprehensions-Axiom durch ein Beispiel:  Es sei $\mathbb{N}$
die Menge der nat\"{u}rlichen Zahlen. Ausgehend von der Menge $\mathbb{N}$  wollen wir die
Menge der \emph{geraden Zahlen} definieren. Zun\"{a}chst m\"{u}ssen wir dazu die Eigenschaft einer
Zahl $x$,
\emph{gerade} zu sein, durch eine Formel $p(x)$ mathematisch erfassen.  Eine nat\"{u}rliche Zahl $x$ ist
genau dann gerade, wenn es eine nat\"{u}rliche Zahl $y$ gibt, so dass $x$ das Doppelte von $y$
ist.  Damit k\"{o}nnen wir die Eigenschaft $p(x)$ folgenderma\3en
definieren: \\[0.2cm]
\hspace*{1.3cm} $p(x) \;:=\; (\exists y\in \mathbb{N}: x = 2 \cdot y)$. \\[0.2cm]
Also kann die Menge der geraden Zahlen als \\[0.2cm]
\hspace*{1.3cm} $\{ x \;|\; \exists y\in \mathbb{N}: x = 2 \cdot y \}$ \\[0.2cm]
geschrieben werden.

Leider f\"{u}hrt die uneingeschr\"{a}nkte Anwendung des Komprehensions-Axiom schnell zu
Problemen.  Betrachten wir dazu die Eigenschaft einer Menge, sich \underline{nicht} selbst zu
enthalten, wir setzen also
\\[0.2cm]
\hspace*{1.3cm}
 $p(x) := \neg(x \in x)$ 
\\[0.2cm]
und definieren die Menge $R$ als \\[0.2cm]
\hspace*{1.3cm} $R := \{ x \;|\; \neg (x \in x) \}$.  \\[0.2cm]
Intuitiv w\"{u}rden wir vielleicht erwarten, dass keine Menge sich selbst enth\"{a}lt.  Wir wollen
jetzt zun\"{a}chst f\"{u}r die eben definierte Menge $R$ \"{u}berpr\"{u}fen, wie die Dinge liegen.
Es k\"{o}nnen zwei F\"{a}lle auftreten:
\begin{enumerate}
\item Fall: $\neg (R \in R)$. Also enth\"{a}lt die Menge $R$ sich nicht selbst.
      Da die Menge $R$ aber als die Menge der Mengen definiert ist, die sich nicht selber
      enthalten, m\"{u}sste $R$ eine Element von $R$ sein, es m\"{u}sste also
      $R \in R$ gelten im Widerspruch zur Voraussetzung $\neg (R \in R)$.
\item Fall: $R \in R$. Setzen wir hier die Definition von $R$ ein, so haben wir \\[0.2cm]
      \hspace*{1.3cm}  $R \in \{ x \;|\; \neg(x \in x) \}$. \\[0.2cm]
      Das hei\3t dann aber gerade $\neg (R \in R)$ und steht im Widerspruch zur
      Voraussetzung $R \in R$.
\end{enumerate}
Wie wir es auch drehen und wenden, es kann weder $R \in R$ noch $\neg (R \in R)$ gelten. 
Als Ausweg k\"{o}nnen wir nur feststellen, dass das vermittels \\[0.2cm]
\hspace*{1.3cm} $\{ x \mid \neg (x \in x) \}$ \\[0.2cm]
definierte Objekt keine Menge ist.
Das hei\3t dann aber, dass das Komprehensions-Axiom
zu allgemein ist.  Wir folgern, dass nicht jede  in der Form \\[0.2cm]
\hspace*{1.3cm} $M = \{ x \mid p(x) \}$ \\[0.2cm]
angegebene Menge wohldefiniert ist.  Die Konstruktion der ``Menge''
 ``$\{x \mid \neg(x \in x)\}$''stammt von dem britischen Logiker und Philosophen 
\href{http://de.wikipedia.org/wiki/Bertrand_Russell}{Bertrand Russell} (1872 -- 1970).  Sie wird
deswegen auch als 
\href{http://de.wikipedia.org/wiki/Russellsche_Antinomie}{\emph{Russell'sche Antinomie}}
 bezeichnet. 


Um  Paradoxien wie die Russell'sche Antinomie  zu vermeiden, ist es erforderlich, bei der
Konstruktion von Mengen vorsichtiger vorzugehen.
Wir werden im Folgenden Konstruktions-Prinzipien f\"{u}r Mengen vorstellen,
die schw\"{a}cher sind als das Komprehensions-Axiom, die aber f\"{u}r die Praxis der Informatik
ausreichend sind.  Wir wollen dabei die dem Komprehensions-Axiom zugrunde liegende Notation 
beibehalten und Mengendefinitionen in der Form \\[0.2cm]
\hspace*{1.3cm} $M = \{ x \mid p(x) \}$  \\[0.2cm]
angeben.  Um Paradoxien zu vermeiden, werden wir nur bestimmte
Sonderf\"{a}lle dieser Mengendefinition zulassen.  Diese Sonderf\"{a}lle, sowie weitere M\"{o}glichkeiten
Mengen zu konstruieren, stellen wir jetzt vor.

\section{Erzeugung von Mengen durch explizites Auflisten}
Die einfachste M\"{o}glichkeit, eine Menge festzulegen, besteht in der expliziten
\emph{Auflistung} aller ihrer Elemente. Diese Elemente werden in den geschweiften
Klammern ``\texttt{\{}'' und ``\texttt{\}}'' eingefasst und durch Kommas getrennt.
Definieren wir beispielsweise \\[0.2cm]
\hspace*{1.3cm} $M := \{ 1, 2, 3 \}$, \\[0.2cm]
so haben wir damit festgelegt, dass die Menge $M$ aus den Elementen $1$, $2$ und $3$
besteht. In der Schreibweise des Komprehensions-Axioms k\"{o}nnen wir diese Menge als \\[0.2cm]
\hspace*{1.3cm} $M = \{ x \mid x = 1 \vee x = 2 \vee x = 3 \}$ \\[0.2cm]
angeben.
Als ein weiteres Beispiel f\"{u}r eine Menge, die durch explizite Aufz\"{a}hlung ihrer Elemente
angegeben werden kann, betrachten wir die Menge der kleinen Buchstaben, die wir wie folgt
definieren: \\[0.2cm]
\hspace*{1.3cm} 
$\{\mathtt{a}, \mathtt{b}, \mathtt{c}, \mathtt{d}, \mathtt{e},
 \mathtt{f}, \mathtt{g}, \mathtt{h}, \mathtt{i}, \mathtt{j}, \mathtt{k}, \mathtt{l},
 \mathtt{m}, \mathtt{n}, \mathtt{o}, \mathtt{p}, \mathtt{q}, \mathtt{r}, \mathtt{s},
 \mathtt{t}, \mathtt{u}, \mathtt{v}, \mathtt{w}, \mathtt{x}, \mathtt{y}, \mathtt{z\}}$.
 \\[0.2cm]
Als letztes Beispiel betrachten wir die leere Menge $\emptyset$, die durch
Aufz\"{a}hlung aller ihrer Elemente definiert werden kann:
\\[0.2cm]
\hspace*{1.3cm}
$\emptyset := \{\}$.
\\[0.2cm]
Die leere Menge enth\"{a}lt also \"{u}berhaupt keine Elemente.  Diese Menge spielt in der Mengenlehre eine
\"{a}hnliche Rolle wie die Zahl $0$ in der Zahlentheorie.

Wird eine Menge durch Auflistung ihrer Elemente definiert, so spielt die Reihenfolge, in der die
Elemente aufgelistet werden, keine Rolle.  Beispielsweise gilt
\\[0.2cm]
\hspace*{1.3cm}
$\{1,2,3\} = \{3,1,2\}$,
\\[0.2cm]
denn beide Mengen enthalten offenbar dieselben Elemente.

\section{Die Menge der nat\"{u}rlichen Zahlen}
Alle durch explizite Auflistung definierten Mengen haben offensichtlich nur endlich viele
Elemente.  Aus der mathematischen Praxis kennen wir aber auch Mengen mit unendlich vielen
Elementen.  Ein Beispiel ist die 
\href{http://en.wikipedia.org/wiki/Natural_number}{Menge der nat\"{u}rlichen Zahlen}, die wir mit $\mathbb{N}$
bezeichnen.  Im Gegensatz zu einigen anderen Autoren werde ich dabei die Zahl $0$ \underline{nicht}
als nat\"{u}rliche Zahl auffassen.  Mit den bisher behandelten Verfahren l\"{a}sst sich die Menge $\mathbb{N}$  nicht definieren.

Wir m\"{u}ssen daher die Existenz dieser Menge als Axiom fordern.  Genauer postulieren wir, dass es eine
Menge $\mathbb{N}$ gibt, welche die folgenden drei Eigenschaften hat:
\begin{enumerate}
\item $1 \in \mathbb{N}$.
\item Falls $n \in \mathbb{N}$ gilt, so gilt auch $n+1 \in \mathbb{N}$.
\item Au\3er den Zahlen, die auf Grund der ersten beiden Bedingungen Elemente der Menge $\mathbb{N}$ sind, enth\"{a}lt
      $\mathbb{N}$ keine weiteren Elemente.
\end{enumerate}
Anschaulich schreiben wir \\[0.2cm]
\hspace*{1.3cm} $\mathbb{N} := \{ 1, 2, 3, \cdots \}$. \\[0.2cm]
Neben der Menge $\mathbb{N}$ der nat\"{u}rlichen Zahlen verwenden wir noch die folgenden
Mengen von Zahlen: 
\begin{enumerate}
\item $\mathbb{N}_0$ ist die Menge der nicht-negativen ganzen Zahlen, es gilt also
      \\[0.2cm]
      \hspace*{1.3cm}
      $\mathbb{N}_0 := \{ 0 \} \cup \mathbb{N}$.
\item $\mathbb{Z}$ ist die Menge der ganzen Zahlen, es gilt
      \\[0.2cm]
      \hspace*{1.3cm}
      $\mathbb{Z} := \{ 0, 1, -1, 2, -2, 3, -3, \cdots \}$ 

\item $\mathbb{Q}$ ist die Menge der rationalen Zahlen, es gilt
      \\[0.2cm]
      \hspace*{1.3cm}
      $\Bigl\{ \bruch{p}{q} \mid\, p \in \mathbb{Z} \wedge q \in \mathbb{N} \Bigr\}$
\item $\mathbb{R}$ ist die Menge der reellen Zahlen.

      Eine mathematisch saubere Definition der reellen Zahlen erfordert einiges an Aufwand.  Wir
      werden die Konstruktion der reellen Zahlen erst in der Analysis-Vorlesung im zweiten Semester
      besprechen. 
\end{enumerate}


\section{Das Auswahl-Prinzip}
Das \emph{Auswahl-Prinzip} ist eine Abschw\"{a}chung des Komprehensions-Axiom.  Die Idee
ist, mit Hilfe einer Eigenschaft $p$ aus einer schon vorhandenen Menge $M$ die Menge $N$ der 
Elemente $x$ \emph{auszuw\"{a}hlen}, die eine bestimmte Eigenschaft $p(x)$ besitzen: \\[0.2cm]
\hspace*{1.3cm} $N = \{ x\in M \;|\; p(x) \}$ \\[0.2cm]
In der Notation des Komprehensions-Axioms schreibt sich diese Menge als \\[0.2cm]
\hspace*{1.3cm} $N = \{ x \mid x \in M \wedge p(x) \}$. \\[0.2cm]
Im Unterschied zu dem Komprehensions-Axiom k\"{o}nnen wir uns hier nur auf die Elemente einer
bereits vorgegebenen Menge $M$ beziehen und nicht auf v\"{o}llig beliebige Objekte.
\vspace{0.2cm}

\noindent
\textbf{Beispiel}:
Die Menge der geraden Zahlen kann mit dem Auswahl-Prinzip als die Menge 
\\[0.2cm]
\hspace*{1.3cm}
 $\{ x \in \mathbb{N} \;|\; \exists y\in \mathbb{N}: x = 2 \cdot y \}$. 
\\[0.2cm]
geschrieben werden.

\section{Potenz-Mengen}
Um den Begriff der \emph{Potenz-Menge} einf\"{u}hren zu k\"{o}nnen, m\"{u}ssen wir zun\"{a}chst 
\emph{Teilmengen} definieren.  Sind $M$ und $N$ zwei Mengen, so hei\3t $M$ eine
\emph{Teilmenge} von $N$ genau dann, wenn jedes Element der Menge $M$ auch ein Element der
Menge $N$ ist.  In diesem Fall schreiben wir $M \subseteq N$.  Formal k\"{o}nnen wir den
Begriff der Teilmenge durch die Formel
 \\[0.2cm]
\hspace*{1.3cm}
$M \subseteq N \;\stackrel{\mathrm{def}}{\Longleftrightarrow}\; \forall x: (x \in M \rightarrow x \in N)$ 
 \\[0.2cm]
definieren.

\example
Es gilt 
\\[0.2cm]
\hspace*{1.3cm}
$\{ 1, 3, 5\} \subseteq \{ 1, 2, 3, 4, 5 \}$
\\[0.2cm]
Weiter gilt f\"{u}r jede beliebige Menge $M$
\\[0.2cm]
\hspace*{1.3cm}
$\emptyset \subseteq M$. \eox


Unter der \emph{Potenz-Menge} einer Menge $M$ wollen wir nun die Menge aller Teilmengen
von $M$ verstehen.  Wir schreiben $2^M$ f\"{u}r die Potenz-Menge von $M$.  Dann gilt \\[0.2cm]
\hspace*{1.3cm} $2^M = \{ x \;|\; x \subseteq M \}$.
\vspace{0.2cm}

\example
Wir bilden  die Potenz-Menge der Menge $\{1,2,3\}$.  Es gilt: \\[0.2cm]
\hspace*{1.3cm} $2^{\{1,2,3\}} = \big\{ \{\},\, \{1\}, \, \{2\},\, \{3\},\, \{1,2\}, \, \{1,3\}, \, \{2,3\},\, \{1,2,3\}\big\}$. \\[0.2cm]
Diese Menge hat $8 = 2^3$ Elemente.  Allgemein kann durch \emph{Induktion} \"{u}ber die Anzahl der
Elemente der Menge $M$ gezeigt werden, dass die 
Potenz-Menge einer Menge $M$, die aus $m$ verschiedenen Elementen besteht, insgesamt $2^m$ 
Elemente enth\"{a}lt.  Bezeichnen wir die Anzahl der Elemente einer endlichen Menge mit
$\textsl{card}(M)$, so gilt also
\\[0.2cm]
\hspace*{1.3cm}
$\textsl{card}\left(2^M\right) = 2^{\textsl{card}(M)}$.
\\[0.2cm]
Dies erkl\"{a}rt die Schreibweise $2^M$ f\"{u}r die Potenz-Menge von $M$.  \eox

\section{Vereinigungs-Mengen}
Sind zwei Mengen $M$ und $N$ gegeben, so enth\"{a}lt die Vereinigung von $M$ und $N$ alle Elemente, die 
 in der Menge $M$ oder in der Menge $N$ liegen.  F\"{u}r diese Vereinigung schreiben wir $M \cup N$.
Formal kann die Vereinigung als 
\\[0.2cm]
\hspace*{1.3cm} $M \cup N := \{ x \;|\; x \in M \vee x \in N \}$. 
\\[0.2cm] 
definiert werden.


\example 
Ist  $M = \{1,2,3\}$ und $N = \{2,5\}$, so gilt: \\[0.2cm]
\hspace*{1.3cm} $\{1,2,3\} \cup \{2,5\} = \{1,2,3,5\}$.  \eox


Der Begriff der Vereinigung von Mengen l\"{a}sst sich verallgemeinern.  Betrachten
wir dazu eine Menge $X$, deren Elemente selbst wieder Mengen sind. Beispielsweise ist die Potenz-Menge 
einer Menge von dieser Art.  Wir k\"{o}nnen dann die Vereinigung aller Mengen, die Elemente
von der Menge $X$ sind, bilden.  Diese Vereinigung schreiben wir als $\bigcup X$.  Formal
kann diese Vereinigung als
\\[0.2cm]
\hspace*{1.3cm} $\bigcup X = \{ y \;|\; \exists x \in X: y \in x \}$.
\\[0.2cm]
 definiert werden:

\example
Die Menge $X$ sei wie folgt gegeben: \\[0.2cm]
\hspace*{1.3cm}
 $X = \big\{ \{\},\, \{1,2\}, \, \{1,3,5\}, \, \{7,4\}\,\big\}$. \\[0.2cm]
Dann gilt \\[0.2cm]
\hspace*{1.3cm}
 $\bigcup X = \{ 1, 2, 3, 4, 5, 7 \}$. \eox

\section{Schnitt-Menge}
Sind zwei Mengen $M$ und $N$ gegeben, so definieren wir den \emph{Schnitt} von
$M$ und $N$ als die Menge aller Elemente, die sowohl in $M$ als auch in $N$
auftreten.  Wir bezeichnen den Schnitt von $M$ und $N$ mit $M \cap N$.
Formal k\"{o}nnen wir $M \cap N$ als 
\\[0.2cm]
\hspace*{1.3cm} $M \cap N := \{ x \mid x \in M \wedge x \in N \}$.
\\[0.2cm]
definieren.


\example
Wir berechnen den Schnitt der  Mengen $M = \{ 1, 3, 5 \}$ und $N = \{ 2, 3, 5, 6 \}$.  Es gilt
\\[0.2cm]
\hspace*{1.3cm} $M \cap N = \{ 3, 5 \}$
\eox

\section{Differenz-Mengen}
 Sind zwei Mengen $M$ und $N$ gegeben, so bezeichnen wir die \emph{Differenz} von
 $M$ und $N$ als die Menge aller Elemente, die in $M$ aber nicht $N$
 auftreten.  Wir schreiben hierf\"{u}r $M \backslash N$.  Dieser Ausdruck wird als
\\[0.2cm]
\hspace*{1.3cm}
$M$ \emph{ohne} $N$
\\[0.2cm]
gelesen und kann formal als
 \\[0.2cm]
\hspace*{1.3cm} $M \backslash N := \{ x \mid x \in M \wedge x \not\in N \}$. 
 \\[0.2cm] 
definiert werden.


\example
Wir berechnen die Differenz der Mengen $M = \{ 1, 3, 5, 7 \}$ und $N = \{ 2, 3, 5, 6 \}$.  Es gilt
\\[0.2cm]
\hspace*{1.3cm} $M \backslash N = \{ 1, 7 \}$. \eox

\section{Bild-Mengen}
Es sei $M$ eine Menge und $f$ sei eine Funktion, die f\"{u}r alle $x$ aus $M$ definiert ist.
Dann hei\3t die Menge aller Abbilder $f(x)$ von Elementen $x$ aus der Menge $M$ das
\emph{Bild} von $M$ unter $f$.  Wir schreiben $f(M)$ f\"{u}r dieses Bild.
Formal kann $f(M)$ als
 \[ f(M) := \{ y \;|\; \exists x \in M: y = f(x) \} \]
definiert werden. In der Literatur findet sich f\"{u}r die obige Menge auch die Schreibweise
\[ f(M) = \bigl\{ f(x) \;|\; x \in M \}. \]

\example
Die Menge $Q$ aller positiven Quadrat-Zahlen kann als 
\[ Q := \{ y \mid \exists x \in \mathbb{N}: y = x^2\} \]
definiert werden.  Alternativ k\"{o}nnen wir auch 
\[ Q = \bigl\{ x^2 \mid x \in \mathbb{N} \bigr\} \]
schreiben.
\eox


\section{Kartesische Produkte}
Um den Begriff des kartesischen Produktes einf\"{u}hren zu k\"{o}nnen, ben\"{o}tigen wir zun\"{a}chst den Begriff
des geordneten Paares zweier Objekte $x$ und $y$.  Dieses wird  als \\[0.2cm]
\hspace*{1.3cm} $\langle x, y \rangle$ \\[0.2cm]
geschrieben.  Wir sagen, dass $x$ die \emph{erste Komponente} des Paares $\langle x, y \rangle$ ist, 
und $y$ ist die \emph{zweite Komponente}.  Zwei geordnete Paare $\langle x_1, y_1 \rangle$ und $\langle x_2, y_2 \rangle$
sind genau dann gleich, wenn sie komponentenweise gleich sind, d.h.~es gilt \\[0.2cm]
\hspace*{1.3cm} $\langle x_1, y_1 \rangle \,=\,\langle x_2, y_2 \rangle  \;\Leftrightarrow\; x_1 = x_2 \wedge y_1 = y_2$. \\[0.2cm]
Das kartesische Produkt zweier Mengen $M$ und $N$ ist nun die Menge aller geordneten
Paare, deren erste Komponente in $M$ liegt und deren zweite Komponente in $N$ liegt.
Das kartesische Produkt von $M$ und $N$ wird als $M \times N$ geschrieben, formal gilt: 
\[ M \times N := \big\{ z \mid \exists x\colon \exists y\colon z = \langle x,y\rangle \wedge x\in M \wedge y \in N \}. \]
Alternativ k\"{o}nnen wir auch schreiben
\[ M \times N := \big\{ \langle x,y\rangle \mid  x\in M \wedge y \in N \}. \]
\noindent
\textbf{Beispiel}:  Wir setzen $M = \{ 1, 2, 3 \}$ und $N = \{ 5, 7 \}$. Dann gilt\\[0.2cm]
\hspace*{1.3cm} 
$M \times N = \bigl\{ \pair(1,5),\pair(2,5),\pair(3,5),\pair(1,7),\pair(2,7),\pair(3,7)\bigr\}$.
\vspace{0.2cm}

Der Begriff des geordneten Paares l\"{a}sst sich leicht zum Begriff des $n$-Tupels verallgemeinern:
Ein $n$-Tupel hat die Form \\[0.2cm]
\hspace*{1.3cm} $\langle x_1, x_2, \cdots, x_n \rangle$. \\[0.2cm]
Analog kann auch der Begriff des kartesischen Produktes auf $n$ Mengen $M_1$, $\cdots$, $M_n$
verallgemeinert werden. Das sieht dann so aus: \\[0.2cm]
\hspace*{1.3cm} $M_1 \times \cdots \times M_n =\big\{ z \mid \exists x_1\colon
\cdots \exists x_n \colon \; z = \langle x_1,x_2,\cdots,x_n \rangle \wedge x_1
\in M_1 \wedge \cdots \wedge x_n \in M_n \big\}$. \\[0.2cm]
Ist $f$ eine Funktion, die auf $M_1 \times \cdots \times M_n$ definiert ist,
so vereinbaren wir folgende Vereinfachung der Schreibweise: \\[0.2cm]
\hspace*{1.3cm} $f(x_1, \cdots, x_n)$ steht f\"{u}r $f(\langle x_1, \cdots,
x_n\rangle)$. \\[0.2cm]
Gelegentlich werden $n$-Tupel auch als \emph{endliche Folgen} oder als
\emph{Listen} bezeichnet.  

\section{Gleichheit von Mengen}
Wir haben nun alle Verfahren, die wir zur Konstruktion von Mengen ben\"{o}tigen, vorgestellt.
 Wir kl\"{a}ren jetzt die Frage, wann zwei Mengen gleich sind.  Dazu
postulieren wir das folgende \emph{Extensionalit\"{a}ts-Axiom} f\"{u}r Mengen: 
\begin{center}
 {\sl Zwei Mengen sind genau dann gleich, wenn sie dieselben Elemente besitzen.}
\end{center}
Mathematisch k\"{o}nnen wir diesen Sachverhalt durch die Formel
\\[0.2cm]
\hspace*{1.3cm} $M = N \;\leftrightarrow\; \forall x: (x \in M \leftrightarrow x \in N)$ 
\\[0.2cm]
ausdr\"{u}cken.  Eine wichtige Konsequenz aus diesem Axiom ist die Tatsache, dass die Reihenfolge, mit der
Elemente in einer Menge aufgelistet werden, keine Rolle spielt.  Beispielsweise gilt \\[0.2cm]
\hspace*{1.3cm} $\{1,2,3\} = \{3,2,1\}$, \\[0.2cm]
denn beide Mengen enthalten dieselben Elemente.

Falls Mengen durch explizite Aufz\"{a}hlung ihrer Elemente definiert sind, ist die Frage nach
der Gleichheit zweier Mengen trivial.  Ist eine der Mengen mit Hilfe des Auswahl-Prinzips definiert, so
kann es beliebig schwierig sein zu entscheiden, ob zwei Mengen gleich sind.  Hierzu ein
Beispiel:  Es l\"{a}sst sich zeigen, dass \\[0.2cm]
\hspace*{1.3cm} 
$\{ n \in \mathbb{N} \mid \exists x, y, z\in\mathbb{N}: x^n + y^n = z^n \} = \{1,2\}$ \\[0.2cm]
gilt.  Allerdings ist der Nachweis dieser Gleichheit sehr schwer, denn er ist \"{a}quivalent
zum Beweis der \emph{Fermat'schen Vermutung}. Diese Vermutung wurde 1637
von {\sl Pierre de Fermat} aufgestellt und konnte erst 1995 von Andrew Wiles bewiesen werden.
Es gibt andere, \"{a}hnlich aufgebaute Mengen, wo bis heute unklar ist, welche Elemente in der
Menge liegen und welche nicht.


\section{Rechenregeln f\"{u}r das Arbeiten mit Mengen}
 Vereinigungs-Menge,  Schnitt-Menge und die Differenz zweier Mengen gen\"{u}gen Gesetzm\"{a}\3igkeiten, 
die in den folgenden Gleichungen zusammengefasst sind:
\\[0.2cm]
$\begin{array}{rlcl}
\quad 1. & M \cup \emptyset = M         & \hspace*{0.1cm} & M \cap \emptyset = \emptyset \\
2. & M \cup M = M         & \hspace*{0.1cm} & M \cap M = M          \\
3. & M \cup N = N \cup M  &  & M \cap N = N \cap M  \\
4. & (K \cup M) \cup N = K \cup (M \cup N) &  & (K \cap M) \cap N = K \cap (M \cap N) \\
5. & (K \cup M) \cap N = (K \cap N) \cup (M \cap N) &  & (K \cap M) \cup N = (K \cup N) \cap (M \cup N)  \\
6. & M \backslash \emptyset = M & & M \backslash M = \emptyset \\
7. & K \backslash (M \cup N) = (K \backslash M) \cap (K \backslash N) &&
     K \backslash (M \cap N) = (K \backslash M) \cup (K \backslash N) \\
8. & (K \cup M) \backslash N = (K \backslash N) \cup (M \backslash N) &&
     (K \cap M) \backslash N = (K \backslash N) \cap (M \backslash N) \\
9. & K \backslash (M \backslash N) = (K \backslash M) \cup (K \cap N) &&
     (K \backslash M) \backslash N = K \backslash (M \cup N) \\
10. & M \cup (N \backslash M) = M \cup N &&
      M \cap (N \backslash M) = \emptyset  \\
11. & M \cup (M \cap N) = M  &&
      M \cap (M \cup N) = M 

\end{array}$
\\[0.3cm]
Wir beweisen exemplarisch die Gleichung $K \backslash (M \cup N) = (K \backslash M) \cap (K \backslash N)$.
Um die Gleichheit zweier Mengen zu zeigen ist nachzweisen, dass beide Mengen dieselben Elemente enthalten.
Wir haben die folgende Kette von \"{a}quivalenzen: \\[0.3cm]
\hspace*{1.3cm} $
\begin{array}{ll}
                & x \in K \backslash (M \cup N)        \\
\leftrightarrow & x \in K \;\wedge\; \neg\; x \in M \cup N \\
\leftrightarrow & x \in K \;\wedge\; \neg\; (x \in M \vee x \in N) \\
\leftrightarrow & x \in K \;\wedge\;  (\neg\; x \in M) \wedge (\neg\; x \in N) \\
\leftrightarrow & (x \in K \wedge \neg\;x \in M) \;\wedge\; (x \in K \wedge \neg\;x \in N) \\
\leftrightarrow & (x \in K \backslash M) \;\wedge\; (x \in K \backslash N) \\
\leftrightarrow & x \in (K \backslash M) \cap (K \backslash N). \\
\end{array}$ \\[0.3cm]
Wir haben beim dritten Schritt dieser \"{a}quivalenz-Kette ausgenutzt, dass eine
Disjunktion der Form $F \vee G$ genau dann falsch ist, wenn sowohl $F$ als auch $G$ falsch ist,
formal gilt
\\[0.2cm]
\hspace*{1.3cm}
$\neg (F \vee G) \leftrightarrow \neg F \wedge \neg G$.
\\[0.2cm]
Wir werden diese \"{a}quivalenz im Rahmen der Logik-Vorlesung noch formal beweisen.

Die \"{u}brigen der oben aufgef\"{u}hrten Gleichungen k\"{o}nnen nach demselben Schema hergeleitet werden.

\exercise
Beweisen Sie die folgenden Gleichungen:
\begin{enumerate}
\item $K \backslash (M \cap N) = (K \backslash M) \cup (K \backslash N)$,
\item $M \cup (M \cap N) = M$,
\item $K \backslash (M \backslash N) = (K \backslash M) \cup (K \cap N)$,
\item $(K \backslash M) \backslash N = K \backslash (M \cup N)$. \exend
\end{enumerate} 
\vspace{0.2cm}

\noindent
Zur Vereinfachung der Darstellung von Beweisen vereinbaren wir die folgende Schreibweise:
Ist $M$ eine Menge und $x$ ein Objekt, so schreiben wir $x \notin M$  f\"{u}r
die Formel $\neg\; x \in M$, formal: \\[0.2cm]
\hspace*{1.3cm} $x \notin M \;\stackrel{de\!f}{\Longleftrightarrow}\; \neg\; x \in M$.
\\[0.2cm]
Eine analoge Notation verwenden wir auch f\"{u}r das Gleichheitszeichen:
$x \not= y \;\stackrel{de\!f}{\Longleftrightarrow}\; \neg\; (x = y)$.

\section{Bin\"{a}re Relationen}
Relationen treten in der Informatik an vielen Stellen auf.  Die wichtigste
Anwendung findet sich in der Theorie der relationalen Datenbanken.
Wir betrachten im Folgenden den Spezialfall der \emph{bin\"{a}ren Relationen} und beleuchten
das Verh\"{a}ltnis von bin\"{a}ren Relationen und Funktionen. Wir werden sehen, dass wir
Funktionen als spezielle bin\"{a}re Relationen auffassen k\"{o}nnen.  Damit stellt der Begriff der
bin\"{a}ren Relationen eine Verallgemeinerung des Funktions-Begriffs dar.

Zum Abschluss des Kapitels f\"{u}hren wir \emph{transitive Relationen} und
\emph{\"{a}quivalenz-Relationen} ein.  Dabei handelt es sich um grundlegende Konzepte, die jeder Informatiker
kennen sollte.

Ist eine Menge $R$ als Teilmenge des kartesischen Produkts zweier Mengen $M$ und $N$ gegeben, gilt
also \\[0.2cm]
\hspace*{1.3cm} $R \subseteq M \times N$, \\[0.2cm]
so bezeichnen wir $R$ auch als \emph{bin\"{a}re Relation}.  In diesem Fall definieren wir den
\emph{Definitions-Bereich} von $R$ als \\[0.2cm]
\hspace*{1.3cm} $\dom(R) := \{ x \mid \exists y \in N \colon \langle x, y \rangle \in R \}$.  
\\[0.2cm]
Entsprechend wird der \emph{Werte-Bereich} von $R$ als \\[0.2cm]
\hspace*{1.3cm} $\rng(R) := \{ y \mid \exists x \in M \colon \langle x, y \rangle \in R\}$ \\[0.2cm]
definiert.  

\example
Es sei $M := \{1,2,3\}$ und $N := \{1,2,3,4,5\}$.  Wir definieren die Relation $R \subseteq M \times N$ 
als
\\[0.2cm]
\hspace*{1.3cm}
 $R = \{ \pair(1,1), \pair(1,2), \pair(3,5) \}$.  
\\[0.2cm]
Dann gilt 
\\[0.2cm]
\hspace*{1.3cm} $\dom(R) = \{1,3\}$ \quad und \quad $\rng(R) = \{1,2,5\}$. 
\\[0.2cm]
Dieses Beispiel zeigt, dass der Begriff der \emph{Relation} eine Verallgemeinerung des Begriffs
der Funktion ist.  Das liegt daran, dass eine Funktion jedem Argument genau ein Wert zugeordnet
wird.  Bei einer Relation k\"{o}nnen hingegen einem Argument auch mehrere Werte zugeordnet werden.
Ebenso ist es bei einer Relation m\"{o}glich, dass einem Argument kein Wert zugeordnet wird.
Beispielsweise ordnet die oben angegebene Relation $R$ der Zahl $1$ sowohl die Zahl $1$ als auch die
Zahl $2$ zu, w\"{a}hrend der Zahl $2$ kein Wert zugeordnet wird.
\eox

\noindent
Das n\"{a}chste, stark vereinfachte Beispiel gibt einen Vorgeschmack von der Bedeutung bin\"{a}rer
Relationen in der Theorie der \emph{relationalen Datenbanken}.

\example
Ein Autoverk\"{a}ufer speichert in seiner Datenbank, welcher Kunde welches Auto gekauft hat.
Nehmen wir an, dass die Mengen \textsl{Auto} und \textsl{Kunde} wie folgt gegeben sind:
\\[0.2cm]
\hspace*{1.3cm}
$\textsl{Kunde} = \{ \mathrm{Bauer}, \mathrm{Maier}, \mathrm{Schmidt} \}$
\quad und \quad
$\textsl{Auto} = \{ \mathrm{Polo}, \mathrm{Fox}, \mathrm{Golf} \}$.
\\[0.2cm]
Dann k\"{o}nnte die bin\"{a}re Relation 
\\[0.2cm]
\hspace*{1.3cm}
$\textsl{Verkauf} \subseteq \textsl{Kunde} \times \textsl{Auto}$
\\[0.2cm]
beispielsweise durch die folgende Menge gegeben sein:
\\[0.2cm]
\hspace*{1.3cm}
$\{ \pair(\mathrm{Bauer}, \mathrm{Golf}), \pair(\mathrm{Bauer}, \mathrm{Fox}), \pair(\mathrm{Schmidt}, \mathrm{Polo})\}$.
\\[0.2cm]
Diese Relation w\"{u}rde ausdr\"{u}cken, dass der Kunde Bauer einen Golf und einen Fox erworben
hat, der Kunde Schmidt hat einen Polo gekauft und Herr Maier hat bisher noch kein Auto erworben.
In der Theorie der Datenbanken werden Relationen \"{u}blicherweise in Form von Tabellen
dargestellt. Die oben angegebene Relation h\"{a}tte dann die folgende Form:
\begin{center}
  \begin{tabular}[c]{|l|l|}
\hline
\textsl{Kunde} & \textsl{Auto} \\
\hline
\hline
  Bauer   & Golf \\
\hline
  Bauer   & Fox  \\
\hline
  Schmidt & Polo \\
\hline
  \end{tabular}
\end{center}
Die oberste Zeile, in der wir die Spalten-\"{u}berschriften ``\textsl{Kunde}'' und ``\textsl{Auto}''
angeben,  geh\"{o}rt selbst nicht zu der Relation, sondern wird als \emph{Relationen-Schema}
bezeichnet und die Relation zusammen mit ihrem Relationen-Schema nennen wir \emph{Tabelle}.  


\subsection{Funktionale Relationen}
Wir hatten schon gesehen, dass Relationen als Verallgemeinerungen von Funktionen aufgefasst werden
k\"{o}nnen.  Wir wollen als n\"{a}chstes untersuchen, unter welchen Umst\"{a}nden eine Relation als Funktion aufgefasst
werden kann.  Zu diesem Zweck folgt nun eine Definition.

\begin{Definition}[links-eindeutig, rechts-eindeutig]
  Wir nennen eine Relation $R \subseteq M \times N$ 
  \emph{rechts-eindeutig}, wenn folgendes gilt: \\[0.2cm]
  \hspace*{1.3cm} 
  $\forall x \el M \colon \forall y_1, y_2 \el N \colon \bigl(\langle x, y_1 \rangle \in R \wedge \langle x, y_2 \rangle \in R \rightarrow y_1 = y_2\bigr)$.
  \\[0.2cm]
  Bei einer rechts-eindeutigen Relation $R \subseteq M \times N$ gibt es also zu jedem $x\in M$ h\"{o}chstens ein $y \in N$, 
  so dass $\langle x, y \rangle \in R$ gilt.  Entsprechend nennen wir eine Relation $R \subseteq M \times N$ 
  \emph{links-eindeutig}, wenn gilt: \\[0.2cm]
  \hspace*{1.3cm} 
  $\forall y \el N \colon \forall x_1, x_2 \el M \colon \bigl(\langle x_1, y \rangle \in R \wedge \langle x_2, y \rangle \in R \rightarrow x_1 = x_2\bigr)$.
  \\[0.2cm]
  Bei einer links-eindeutigen Relation $R \subseteq M \times N$ 
  gibt es also zu jedem $y\in N$ h\"{o}chstens ein $x \in M$, so
  dass $\langle x, y \rangle \in R$ gilt. \eox
\end{Definition}


\noindent
\textbf{Beispiele}: Es sei $M = \{1,2,3\}$ und $N = \{4,5,6\}$.
\begin{enumerate}
\item Die Relation $R_1$ sei definiert durch \\[0.2cm]
      \hspace*{1.3cm} $R_1 = \{ \pair(1,4), \pair(1,6) \}$. \\[0.2cm]
      Diese Relation ist \underline{nicht} rechts-eindeutig, denn  $4 \not= 6$.
      Die Relation ist links-eindeutig, denn die rechten Seiten aller in $R_1$
      auftretenden Tupel sind verschieden.
\item Die Relation $R_2$ sei definiert durch \\[0.2cm]
      \hspace*{1.3cm} $R_2 = \{ \pair(1,4), \pair(2,6) \}$. \\[0.2cm]
      Diese Relation ist rechts-eindeutig, denn die linken Seiten aller in $R_2$ auftretenden
      Tupel sind verschieden.  Sie ist auch links-eindeutig, denn die rechten Seiten aller
      in $R_2$ auftretenden Tupel sind verschieden. 
\item Die Relation $R_3$ sei definiert durch \\[0.2cm]
      \hspace*{1.3cm} $R_3 = \{ \pair(1,4), \pair(2,6), \pair(3,6) \}$. \\[0.2cm]
      Diese Relation ist rechts-eindeutig, denn die linken Seiten aller in $R_2$       auftretenden
      Tupel sind verschieden.  Sie ist nicht links-eindeutig, denn es gilt
      $\pair(2,6) \el R$ und $\pair(3,6) \el R$, aber $2 \not= 3$.
\end{enumerate}

\begin{Definition}[links-total, rechts-total]
  Eine bin\"{a}re Relation $R \subseteq M \times N$ hei\3t \emph{links-total auf $M$}, wenn \\[0.2cm]
  \hspace*{1.3cm} $\forall x \in M \colon \exists y \in N \colon \pair(x,y) \in R$ \\[0.2cm]
  gilt. Dann gibt es f\"{u}r alle $x$ aus der Menge $M$ ein $y$ aus der Menge $N$, so dass
  $\pair(x,y)$ in der Menge  $R$ liegt.  Die Relation $R_3$ aus dem letzten Beispiel ist
  links-total, denn jedem Element aus $M$ wird durch $R_3$ ein Element aus $N$ zugeordnet.

  Analog nennen wir eine bin\"{a}re Relation $R \subseteq M \times N$  \emph{rechts-total auf $N$}, wenn \\[0.2cm]
  \hspace*{1.3cm} $\forall y \in N \colon \exists x \in M \colon \pair(x,y) \in R$ \\[0.2cm]
  gilt. Dann gibt es f\"{u}r alle $y$ aus der Menge $N$ ein $x$ aus der Menge $M$, so dass
  $\pair(x,y)$ in der Menge  $R$ liegt.  Die Relation $R_3$ aus dem letzten Beispiel ist
  nicht rechts-total, denn dem Element $5$ aus $N$ wird durch $R_3$ kein Element aus $M$
  zugeordnet, denn f\"{u}r alle $\pair(x,y) \in R_3$ gilt $y \not= 5$. \eox
\end{Definition}

\begin{Definition}
Eine Relation $R \subseteq M \times N$, die sowohl links-total auf $M$ als auch rechts-eindeutig
ist, nennen wir eine \emph{funktionale} Relation auf $M$.    \eox
\end{Definition}


\begin{Satz}
Ist $R \subseteq M \times N$ eine
funktionale Relation, so k\"{o}nnen wir eine Funktion $f_R\colon M \rightarrow N$ wie folgt
definieren: \\[0.2cm]
\hspace*{1.3cm} $f_R(x) := y \;\stackrel{de\!f}{\Longleftrightarrow}\; \pair(x,y) \in R$. 
\end{Satz}

\proof
Diese Definition funktioniert, denn aus der Links-Totalit\"{a}t von $R$ folgt, dass es f\"{u}r
jedes $x\in M$ auch ein $y \in N$ gibt, so dass $\pair(x,y) \in R$ ist.  Aus der
Rechts-Eindeutigkeit von $R$ folgt dann, dass dieses $y$ eindeutig bestimmt ist. \qed

\remark
Ist umgekehrt eine Funktion \mbox{$f:M \rightarrow N$} gegeben, so k\"{o}nnen wir dieser Funktion
eine Relation $\textsl{graph}(f) \subseteq M \times N$ zuordnen, indem wir definieren: 
\[ \textsl{graph}(f) := \bigl\{ \pair(x,f(x)) \mid  x\in M \bigr\}. \]
Die so definierte Relation $\textsl{graph}(f)$ ist links-total auf $M$, denn die Funktion $f$
berechnet ja f\"{u}r jedes $x \el M$ ein Ergebnis und die Relation ist rechts-eindeutig,
denn die Funktion berechnet f\"{u}r jedes Argument immer nur ein Ergebnis. \eox

Aufgrund der gerade diskutierten Korrespondenz zwischen Funktionen und Relationen
werden wir daher im Folgenden alle Funktionen als spezielle bin\"{a}re Relationen auffassen.
F\"{u}r die Menge aller Funktionen von $M$ nach $N$ schreiben wir auch $N^M$, genauer
definieren wir
\[ N^M := \{ R \subseteq M \times N \mid \mbox{$R$ funktional}\, \}. \]
Diese Schreibweise erkl\"{a}rt sich wie folgt: Sind $M$ und $N$ endliche Mengen mit $m$
bzw.~$n$ Elementen, so gibt es genau $n^m$ verschiedene Funktionen von $M$ nach $N$, es
gilt also
\[ \textsl{card}\left(N^M\right) = \textsl{card}(N)^{\textsl{card}(M)}. \]
Wir werden daher funktionale Relationen und die entsprechenden Funktionen identifizieren.
Damit ist dann f\"{u}r eine funktionale Relation $R \subseteq M \times N$ und ein $x \in M$
auch die Schreibweise $R(x)$ zul\"{a}ssig: Mit $R(x)$ bezeichnen wir das eindeutig bestimmte $y \in N$,
f\"{u}r das $\pair(x,y) \in R$ gilt.

\examples
\begin{enumerate}
\item Wir setzen $M = \{1,2,3\}$, $N = \{1,2,3,4,5,6,7,8,9\}$ und definieren \\[0.2cm]
      \hspace*{1.3cm} $R := \{ \pair(1,1),\pair(2,4),\pair(3,9) \}$. \\[0.2cm]
      Dann ist $R$ eine funktionale Relation auf $M$.  Diese Relation berechnet
      gerade die Quadrat-Zahlen auf der Menge $M$.
\item Diesmal setzen wir $M = \{1,2,3,4,5,6,7,8,9\}$ und $N = \{1,2,3\}$ und definieren \\[0.2cm]
      \hspace*{1.3cm} $R := \{ \pair(1,1),\pair(4,2),\pair(9,3) \}$. \\[0.2cm]
      Dann ist $R$ keine funktionale Relation auf $M$, denn $R$ ist nicht links-total auf
      $M$.  Beispielsweise wird das Element $2$ von der Relation $R$ auf kein Element aus 
      $N$ abgebildet.
\item Wir setzen nun  $M = \{1,2,3\}$, $N = \{1,2,3,4,5,6,7,8,9\}$ und definieren \\[0.2cm]
      \hspace*{1.3cm} $R := \{ \pair(1,1),\pair(2,3),\pair(2,4),\pair(3,9) \}$ \\[0.2cm]
      Dann ist $R$ keine funktionale Relation auf $M$, denn $R$ ist nicht rechts-eindeutig auf
      $M$.  Das liegt daran, dass das Element $2$ von der Relation $R$ sowohl auf $3$ als
      auch auf $4$ abgebildet wird. \eox
\end{enumerate}

\begin{Definition}[Bild]
  Ist $R \subseteq M \times N$ eine bin\"{a}re Relation und ist weiter $X \subseteq M$, so
  definieren wir das \emph{Bild von $X$ unter $R$} als \\[0.2cm]
  \hspace*{1.3cm} $R(X) := \{ y \mid \exists x \in X \colon \pair(x,y) \in R \}$. \eox
\end{Definition}

\subsection{Inverse Relation}
Zu einer  Relation $R \subseteq M \times N$ definieren wir die \emph{inverse} Relation 
$R^{-1} \subseteq N \times M$ wie folgt: \\[0.2cm]
\hspace*{1.3cm} $R^{-1} := \bigl\{ \pair(y,x) \mid \pair(x,y) \in R  \bigr\}$. \\[0.2cm]
Aus dieser Definition folgt sofort, dass $R^{-1}$ rechts-eindeutig ist genau dann, wenn
$R$ links-eindeutig ist.  Au\3erdem ist $R^{-1}$ links-total auf $N$ genau dann, wenn $R$
rechts-total auf $N$ ist.  Ist eine Relation sowohl links-eindeutig als auch rechts-eindeutig und
au\3erdem sowohl links-total auf $M$ als auch rechts-total auf $N$, so nennen wir sie auch \emph{bijektiv}.
In diesem Fall l\"{a}sst sich neben der Funktion
$f_R$ auch eine Funktion $f_{R^{-1}}$ definieren.  Die Definition der letzten Funktion
lautet ausgeschrieben: \\[0.2cm]
\hspace*{1.3cm} $f_{R^{-1}}(y) := x \;\stackrel{de\!f}{\Longleftrightarrow}\; \pair(y,x) \in R^{-1} \Longleftrightarrow \pair(x,y) \in R$. \\[0.2cm]
Diese Funktion ist dann aber genau die Umkehr-Funktion von $f_R$, es gilt \\[0.2cm]
\hspace*{1.3cm}
 $\forall y \in N \colon f_R\bigl(f_{R^{-1}}(y)\bigr) = y$ \quad und \quad
 $\forall x \in M \colon f_{R^{-1}}\bigl(f_R(x)\bigr) = x$. \\[0.2cm]
Dieser Umstand rechtfertigt im Nachhinein die Schreibweise $R^{-1}$.

\subsection{Komposition von Relationen}
\"{a}hnlich wie wir Funktionen verkn\"{u}pfen k\"{o}nnen, k\"{o}nnen auch Relationen verkn\"{u}pft werden.
Wir betrachten zun\"{a}chst Mengen $L$, $M$ und $N$.
Sind dort zwei Relationen $R \subseteq L \times M$ und $Q \subseteq M \times N$ definiert,
so ist das \emph{relationale Produkt} $R \circ Q$ wie folgt definiert: \\[0.2cm]
\hspace*{1.3cm}
$R \circ Q := \bigl\{ \pair(x,z) \mid \exists y \in M \colon(\pair(x,y) \in R \wedge \pair(y,z) \in Q) \bigr\}$ 
\\[0.2cm]
Offenbar gilt $R \circ Q \subseteq L \times N$.
Das relationale Produkt von $Q$ und $R$ wird gelegentlich auch als die \emph{Komposition} von
$Q$ und $R$ bezeichnet.  In der Theorie der Datenbanken werden Sie dem relationalen Produkt
wiederbegegnen, denn der 
\href{http://en.wikipedia.org/wiki/Join_(SQL)}{\emph{Join}-Operator}, der in
Datenbankabfrage-Sprachen wie 
\href{http://en.wikipedia.org/wiki/SQL}{\textsc{Sql}} benutzt wird, ist eine Verallgemeinerung des
relationalen Produkts.
\vspace{0.2cm}

\example
Es sei $L = \{1,2,3\}$, $M = \{4,5,6\}$ und $N = \{7,8,9\}$.  Weiter seien die Relationen
$Q$ und $R$ wie folgt gegeben: \\[0.2cm]
\hspace*{1.3cm} $R = \{ \pair(1,4), \pair(1,6), \pair(3,5) \}$ \quad und \quad
                $Q = \{ \pair(4,7), \pair(6,8), \pair(6,9) \}$. \\[0.2cm]
Dann gilt \\[0.2cm]
\hspace*{1.3cm} $R \circ Q = \{ \pair(1,7), \pair(1,8), \pair(1,9) \}$. \eox
\pagebreak

\exercise
Es sei $R \subseteq L \times M$ eine funktionale Relation auf $L$ und 
$Q \subseteq M \times N$ sei eine funktionale Relation auf $M$.  Zeigen Sie, dass dann auch 
$R \circ Q$ eine funktionale Relation auf $L$ ist und zeigen Sie weiter, dass die Funktion $f_{R \circ Q}$
wie folgt aus den Funktionen $f_R$ und $f_Q$ berechnet werden kann:
\\[0.2cm]
\hspace*{1.3cm}
$f_{R \circ Q}(x) = f_Q\bigl(f_R(x)\bigr)$. \eox


\noindent
\textbf{Bemerkung}: In einigen Lehrb\"{u}chern wird das relationale Produkt, 
das wir als $R \circ Q$ definiert haben, mit $Q \circ R$
bezeichnet. Damit lautet die Definition von $R \circ Q$  dann wie folgt:
Ist $R \subseteq M \times N$ und $Q \subseteq L \times M$, dann ist 
\\[0.2cm]
\hspace*{1.3cm}
$R \circ Q := \bigl\{ \pair(x,z) \mid \exists y \in M \colon(\pair(x,y) \in Q \wedge \pair(y,z) \in R) \bigr\}$.
\\[0.2cm]
Diese Definition hat den folgenden Vorteil: Falls $R$ und $Q$ funktionale Relationen sind und wenn
dann weiter $f$ und $g$ die diesen Relationen zugeordneten Funktionen sind, wenn also 
\\[0.2cm]
\hspace*{1.3cm}
$Q = \textsl{graph}(f)$ \quad und \quad $R = \textsl{graph}(g)$
\\[0.2cm]
gilt, dann haben wir f\"{u}r die Komposition der Funktionen $f$ und $g$, die durch $(g \circ
f)(x) = g\bigl(f(x)\bigr)$ 
definiert ist, die Gleichung
\\[0.2cm]
\hspace*{1.3cm}
$\textsl{graph}(g \circ f) = R \circ Q = \textsl{graph}(g) \circ \textsl{graph}(f)$.
\\[0.2cm]
Die von uns verwendete Definition hat den Vorteil, dass die Berechnung des 
\emph{transitiven Abschlusses}  einer Relation, die wir sp\"{a}ter noch geben werden, intuitiver wird.
\eox 


\example
Das n\"{a}chste Beispiel zeigt die Verwendung des relationale Produkts im Kontext einer
Datenbank.  Wir nehmen an, dass die Datenbank eines Autoh\"{a}ndler unter anderem die
folgenden beiden Tabellen enth\"{a}lt.
\begin{center}
\textsl{Kauf}:  \begin{tabular}[t]{|l|l|}
\hline
\textsl{Kunde} & \textsl{Auto} \\
\hline
\hline
  Bauer   & Golf \\
\hline
  Bauer   & Fox  \\
\hline
  Schmidt & Polo \\
\hline
  \end{tabular}
\qquad \textsl{Preis}:
  \begin{tabular}[t]{|l|l|}
\hline
\textsl{Auto} & \textsl{Betrag} \\
\hline
\hline
  Golf    & $20\,000$ \\
\hline
  Fox     & $10\,000$ \\
\hline
  Polo    & $13\,000$ \\
\hline
  \end{tabular}
\end{center}
Dann ist das relationale Produkt der in den Tabellen \textsl{Kauf} und \textsl{Preis}
dargestellten Relationen durch die in der folgenden Tabelle dargestellten Relation
gegeben:

\begin{center}
  \begin{tabular}[t]{|l|l|}
\hline
\textsl{Kunde} & \textsl{Betrag} \\
\hline
\hline
  Bauer   & $20\,000$ \\
\hline
  Bauer   & $10\,000$ \\
\hline
  Schmidt & $13\,000$ \\
\hline
  \end{tabular} 
\end{center}
Diese Relation k\"{o}nnte dann zur Rechnungsstellung weiter verwendet werden. \eox


\subsection{Eigenschaften des relationalen Produkts}
\begin{Satz}
  Die Komposition von Relationen ist \emph{assoziativ}:  Sind \\[0.2cm]
  \hspace*{1.3cm} 
  $R \subseteq K \times L$, \quad $Q \subseteq L \times M$ \quad und \quad 
  $P \subseteq M \times N$ \\[0.2cm]
  bin\"{a}re Relationen, so gilt \\[0.2cm]
  \hspace*{1.3cm} $(R \circ Q) \circ P = R \circ (Q \circ P)$. 
\end{Satz}
\pagebreak

\proof
Wir zeigen, dass
\begin{equation}
  \label{eq:ass0}
   \pair(x,u) \in (R \circ Q) \circ P \leftrightarrow \pair(x,u) \in R \circ (Q \circ P)   
\end{equation}
gilt.  Dazu formen wir zun\"{a}chst die linke Seite $\pair(x,u) \in (R \circ Q) \circ P$ der \"{a}quivalenz
\ref{eq:ass0} um.  Es gilt
\[
\begin{array}{cll}
                  & \pair(x,u) \in (R \circ Q) \circ P \\[0.2cm]
  \leftrightarrow & \exists z: \bigl(\pair(x,z) \in R \circ Q \wedge \pair(z,u) \in P\bigr) &
                    \mbox{nach Def. von}\; (R \circ Q) \circ P \\[0.2cm]
  \leftrightarrow & \exists z: \bigl(\bigl(\exists y: \pair(x,y) \in R \wedge \pair(y,z) \in Q\bigr) \wedge \pair(z,u) \in P\bigr) &
                    \mbox{nach Def. von}\; R \circ Q \\
\end{array}
\]
Da die Variable $y$ in der Formel $\pair(z,u) \in P$ nicht auftritt, k\"{o}nnen wir den
Existenz-Quantor \"{u}ber $y$ auch herausziehen, so dass wir die obige Kette von \"{a}quivalenzen zu
\begin{equation}
  \label{eq:ass1}
  \leftrightarrow \quad 
  \exists z: \exists y: \bigl(\pair(x,y) \in R \wedge \pair(y,z) \in Q \wedge \pair(z,u) \in P\bigr)
\end{equation}
fortsetzen k\"{o}nnen.  Wir formen nun die rechte Seite der \"{a}quivalenz
\ref{eq:ass0} um:
\[
\begin{array}{cll}
                & \pair(x,u) \in R \circ (Q \circ P) \\[0.2cm] 
\leftrightarrow & \exists y: \bigl(\pair(x,y) \in R \wedge \pair (y,u) \in Q \circ P\bigr) &
                  \mbox{nach Def. von}\; R \circ (Q \circ P) \\[0.2cm]
\leftrightarrow & \exists y: \bigl(\pair(x,y) \in R \wedge 
                  \exists z: \bigl(\pair(y,z) \in Q \wedge \pair(z,u) \in P\bigr)\bigr) &
                  \mbox{nach Def. von}\; Q \circ P \\
\end{array}
\]
Da die Variable $z$ in der Formel $\pair(x,y) \in R$ nicht vorkommt, k\"{o}nnen wir den
Existenz-Quantor \"{u}ber $z$ auch vorziehen und k\"{o}nnen daher diese Kette von \"{a}quivalenzen als
\begin{equation}
  \label{eq:ass2}
  \leftrightarrow \quad
  \exists z: \exists y: \bigl(\pair(x,y) \in R \wedge \pair(y,z) \in Q \wedge \pair(z,u) \in P\bigr)
\end{equation}
fortsetzen. Die Formeln (\ref{eq:ass1}) und (\ref{eq:ass2}) sind identisch.  Damit ist die \"{a}quivalenz
(\ref{eq:ass0}) nachgewiesen und der Beweis der Assoziativit\"{a}t des Kompositions-Operators
ist erbracht.  \hspace*{\fill} $\Box$
\vspace{0.2cm}

\begin{Satz}
  Sind zwei Relationen $R \subseteq L \times M$ und $Q \subseteq M \times N$ gegeben, so gilt \\[0.2cm]
  \hspace*{1.3cm} $(R \circ Q)^{-1} = Q^{-1} \circ R^{-1}$. \\[0.2cm]
  Beachten Sie, dass sich die Reihenfolge von $Q$ und $R$ hier vertauscht!  
\end{Satz}

\proof
Es ist zu zeigen, dass f\"{u}r alle Paare $\pair(z,x) \in N \times L$ die  \"{a}quivalenz \\[0.2cm]
\hspace*{1.3cm} 
$\pair(z,x) \in (Q \circ R)^{-1} \quad \leftrightarrow \quad\pair(z,x) \in R^{-1} \circ Q^{-1}$ 
\\[0.2cm]
gilt.  Den Nachweis erbringen wir durch die folgende Kette von \"{a}quivalenz-Umformungen: 
\[ 
\begin{array}{cl}
                & \pair(z,x) \in (R \circ Q)^{-1}                                             \\[0.2cm]
\leftrightarrow & \pair(x,z) \in R \circ Q                                                    \\[0.2cm]
\leftrightarrow & \exists y \in M \colon \bigl(\pair(x,y) \in R \wedge \pair(y,z) \in Q\bigr) \\[0.2cm]
\leftrightarrow & \exists y \in M \colon \bigl(\pair(y,z) \in Q \wedge \pair(x,y) \in R\bigr) \\[0.2cm]
\leftrightarrow & \exists y \in M \colon \bigl(\pair(z,y) \in Q^{-1} \wedge \pair(y,x) \in R^{-1}\bigr) 
                  \\[0.2cm]
\leftrightarrow & \pair(z,x) \in Q^{-1} \circ R^{-1}   \hspace*{\fill} \Box 
\end{array}
\]
\pagebreak

\begin{Satz}[Distributiv-Gesetze f\"{u}r das relationale Produkt]
  Sind $R_1$ und $R_2$   Relationen auf $L \times M$ und ist $Q$ eine Relation auf $M \times N$, so
  gilt \\[0.2cm] 
  \hspace*{1.3cm}
  $(R_1 \cup R_2) \circ Q = (R_1 \circ Q) \cup (R_2 \circ Q)$. 
  \\[0.2cm]
  Analog gilt ebenfalls \\[0.2cm]
  \hspace*{1.3cm} 
  $R \circ (Q_1 \cup Q_2) = (R \circ Q_1) \cup (R \circ Q_2)$, 
  \\[0.2cm]
  falls $R$ eine Relation auf $L \times M$ und $Q_1$ und $Q_2$ Relationen auf $M \times N$
  sind.  Um Gleichungen der obigen Art ohne Klammern schreiben zu k\"{o}nnen vereinbaren wir, dass der 
  Kompositions-Operator $\circ$ st\"{a}rker bindet als $\cup$ und $\cap$.  
\end{Satz}

\proof
Wir beweisen  das erste Distributivgesetz, indem wir 
\begin{equation}
  \label{eq:dis0}
\pair(x,z) \in (R_1 \cup R_2) \circ Q \;\leftrightarrow\; \pair(x,z) \in R_1 \circ Q \cup R_2 \circ Q   
\end{equation}
zeigen.  Wir formen zun\"{a}chst den Ausdruck $\pair(x,z) \in (R_1 \cup R_2) \circ Q$ um:
\[
\begin{array}{cll}
                  & \pair(x,z) \in (R_1 \cup R_2) \circ Q  \\[0.2cm]
  \leftrightarrow & \exists y: \bigl(\pair(x,y) \in R_1 \cup R_2 \wedge \pair(y,z) \in Q\bigr) 
                  & \mbox{nach Def. von}\; (R_1 \cup R_2) \circ Q \\[0.2cm]
  \leftrightarrow & \exists y: \bigl(\bigl(\pair(x,y) \in R_1 \vee \pair(x,y) \in R_2\bigr) \wedge \pair(y,z) \in Q\bigr) 
                  & \mbox{nach Def. von}\; R_1 \cup R_2 \\[0.2cm]
\end{array}
\]
Diese Formel stellen wir mit Hilfe des Distributiv-Gesetzes der Aussagen-Logik um.
In der Aussagenlogik werden wir im Rahmen der Informatik-Vorlesung sehen, dass f\"{u}r beliebige Formeln
$F_1$, $F_2$ und $G$ die \"{a}quivalenz 
\[ (F_1 \vee F_2) \wedge G \;\leftrightarrow\; (F_1 \wedge G) \vee (F_2 \wedge G) \]
gilt.  Die Anwendung dieses Gesetzes liefert:
\begin{eqnarray}
  \nonumber
  & & \exists y: \bigl(\bigl(\underbrace{\pair(x,y) \in R_1}_{F_1} \vee \underbrace{\pair(x,y) \in R_2}_{F_2}\bigr) \wedge \underbrace{\pair(y,z) \in Q}_G\bigr) 
\\[0.2cm] 
  \label{eq:dis1}
  & \leftrightarrow &
    \exists y: \bigl(\bigl(\underbrace{\pair(x,y) \in R_1}_{F_1} \wedge \underbrace{\pair(y,z) \in Q}_G \bigr) \vee 
               \bigl(\underbrace{\pair(x,y) \in R_2}_{F_2} \wedge \underbrace{\pair(y,z) \in Q}_G \bigr) \bigr)   
\end{eqnarray}
Wir formen nun den Ausdruck $\pair(x,z) \in R_1 \circ Q \cup R_2 \circ Q$ um:
\[
\begin{array}{cll}
                & \pair(x,z) \in R_1 \circ Q \cup R_2 \circ Q \\[0.2cm]
\leftrightarrow & \pair(x,z) \in R_1 \circ Q \;\vee\; \pair(x,z) \in R_2 \circ Q &
                  \mbox{nach Def. von}\;\cup \\[0.2cm]
\leftrightarrow & \bigl(\exists y: (\pair(x,y) \in R_1 \wedge \pair(y,z) \in Q)\bigr) \;\vee\; 
                  \bigl(\exists y: (\pair(x,y) \in R_2 \wedge \pair(y,z) \in Q)\bigr) \\[0.2cm]
                & \mbox{nach Def. von}\; R_1 \circ Q \; \mbox{und}\; R_2 \circ Q \\[0.2cm]
\end{array}
\]
Diese letzte Formel stellen wir mit Hilfe eines Distributiv-Gesetzes f\"{u}r die
Pr\"{a}dikaten-Logik um.  In der Pr\"{a}dikaten-Logik werden wir sp\"{a}ter sehen, dass f\"{u}r beliebige
Formeln $F_1$ und $F_2$ die \"{a}quivalenz
\[ \exists y: \bigl(F_1 \vee F_2\bigr) \;\leftrightarrow\; \bigl(\exists y: F_1\bigr) \vee \bigl(\exists y: F_2\bigr) \]
g\"{u}ltig ist.  Damit folgt dann
\begin{eqnarray}
\nonumber
 & & \exists y: \bigl(\underbrace{\pair(x,y) \in R_1 \wedge \pair(y,z) \in Q}_{F_1}\bigr) \;\vee\; 
     \exists y: \bigl(\underbrace{\pair(x,y) \in R_2 \wedge \pair(y,z) \in Q}_{F_2}\bigr) \\[0.2cm]
  \label{eq:dis2}
 & \leftrightarrow &
     \exists y: \Bigl(\bigl(\underbrace{\pair(x,y) \in R_1 \wedge \pair(y,z) \in Q}_{F_1}\bigr) \;\vee\; 
                \bigl(\underbrace{\pair(x,y) \in R_2 \wedge \pair(y,z) \in Q}_{F_2}\bigr)\Bigr) 
\end{eqnarray}
Da die Formeln \ref{eq:dis1} und \ref{eq:dis2} identisch sind, ist der Beweis des
Distributiv-Gesetzes 
\\[0.2cm]
\hspace*{1.3cm}
 $(R_1 \cup R_2) \circ Q = R_1 \circ Q \cup R_2 \circ Q$
\\[0.2cm]
erbracht. \hspace*{\fill} $\Box$
\vspace{0.2cm}

\exercise
Es seien $M$, $N$ und $L$ Mengen und es gelte $R \subseteq M \times N$ und $Q_1,Q_2 \subseteq N \times L$. 
\begin{enumerate}[(a)]
\item Beweisen oder widerlegen Sie die Gleichung
      \\[0.2cm]
      \hspace*{1.3cm}  $R \circ (Q_1 \cup Q_2) = R \circ Q_1 \cup R \circ Q_2$.
\item Beweisen oder widerlegen Sie die Behauptung 
      \\[0.2cm]
      \hspace*{1.3cm}
      $R \circ (Q_1 \cap Q_2) = R \circ Q_1 \cap R \circ Q_2$. \eox
\end{enumerate}


\begin{Definition}[Identische Relation]
  Ist $M$ eine Menge, so definieren wir die \emph{identische Relation} $\mathrm{id}_M \subseteq M \times M$
  wie folgt: \\[0.2cm]
  \hspace*{1.3cm} $\mathrm{id}_M := \bigl\{ \pair(x,x) \mid x \in M \bigr\}$.  \eox
\end{Definition}

\example
Es sei $M = \{1,2,3\}$.  Dann gilt \\[0.2cm]
\hspace*{1.3cm}  $\mathrm{id}_M := \bigl\{ \pair(1,1),  \pair(2,2),  \pair(3,3) \bigr\}$.  \eox
\vspace*{0.2cm}

\remark
Aus der Definition folgt sofort \\[0.2cm]
\hspace*{1.3cm} $\mathrm{id}_M^{-1} = \mathrm{id}_M$. \eox

\begin{Satz}
  Ist $R \subseteq M \times N$  eine bin\"{a}re Relation, so gilt 
  \[ R \circ \mathtt{id}_N = R \quad \mbox{und} \quad \mathrm{id}_M \circ R = R. \] 
\end{Satz}

\proof
Wir weisen nur die zweite Gleichung nach, denn der Nachweis der ersten Gleichung verl\"{a}uft analog zu
dem Beweis der zweiten Gleichung.  Nach Definition des relationalen Produkts
gilt
\[ \mathrm{id}_M \circ R = \bigl\{ \pair(x,z) \mid \exists y:\bigl(\pair(x,y) \in \mathrm{id}_M \wedge \pair(y,z) \in R \bigr)\bigr\}. \]
Nun ist $\pair(x,y) \in \mathrm{id}_M$ genau dann, wenn $x = y$ ist, also gilt
\[ \mathrm{id}_M \circ R = \bigl\{ \pair(x,z) \mid \exists y:\bigl( x = y \wedge \pair(y,z) \in R\bigr) \bigr\}. \]
Es gilt die folgende \"{a}quivalenz
\[ \exists y:\bigl(x = y \wedge \pair(y,z) \in R \bigr) \leftrightarrow \pair(x,z) \in R. \]
Diese \"{a}quivalenz ist leicht einzusehen:  Falls $\exists y:\bigl( x = y \wedge \pair(y,z) \in R\bigr)$
gilt, so muss das $y$ dessen Existenz gefordert wird, den Wert $x$ haben und dann gilt
nat\"{u}rlich auch $\pair(x,z) \in R$.  Gilt andererseits $\pair(x,z) \in R$, so definieren
wir $y := x$.  F\"{u}r das so definierte $y$ gilt offensichtlich 
$x = y \wedge \pair(y,z) \in R$.  Unter Verwendung der oberen \"{a}quivalenz haben wir 
\[ \mathrm{id}_M \circ R = \bigl\{ \pair(x,z) \mid \pair(x,z) \in R \bigr\}. \]
Wegen  $R \subseteq M \times N$ besteht $R$ nur aus geordneten
Paaren und daher gilt
\[ R = \bigl\{ \pair(x,z) \mid \pair(x,z) \in R \bigr\}. \]
Damit ist $\mathrm{id}_M \circ R = R$ gezeigt. \qed
\pagebreak

\exercise
Es sei $R \subseteq M \times N$.  Welche Eigenschaften muss die Relation $R$ besitzen, damit die Gleichung 
\\[0.2cm]
\hspace*{1.3cm}
$R \circ R^{-1} = \mathrm{id}_M$ 
\\[0.2cm]
richtig ist? Unter welchen Bedingungen gilt
\\[0.2cm]
\hspace*{1.3cm}
$R^{-1} \circ R = \mathrm{id}_N$?
\eox
% Die erste Gleichung gilt genau dann, wenn $R$ links-total und links-eindeutig ist.


\section{Bin\"{a}re Relationen auf einer Menge}
Wir betrachten im Folgenden den Spezialfall von Relationen $R \subseteq M \times N$, f\"{u}r
den $M = N$ gilt.  Wir definieren:
Eine Relation $R \subseteq M \times M$ hei\3t eine Relation \emph{auf} der Menge $M$.
Im Rest dieses Abschnittes betrachten wir nur noch solche Relationen. 
Statt $M \times M$ schreiben wir dann k\"{u}rzer $M^2$.

Ist $R$ eine Relation auf $M$ und sind $x, y \in M$, so verwenden wir gelegentlich die
Infix-Schreibweise und schreiben statt
$\pair(x,y) \in R$ auch $x\, R\, y$.  Beispielsweise l\"{a}sst sich die Relation $\leq$ 
auf $\mathbb{N}$ wie folgt definieren: 
\\[0.2cm]
\hspace*{1.3cm}
 $\leq\; := \bigl\{ \pair(x,y) \in \mathbb{N} \times \mathbb{N} \mid \exists z \in \mathbb{N}_0 \colon x + z = y \bigr\}$.
\\[0.2cm]
Statt $\pair(x,y) \in\; \leq$ hat sich in der Mathematik die Schreibweise $x \leq y$ eingeb\"{u}rgert.
  
\begin{Definition}[reflexiv] 
Eine Relation $R \subseteq M \times M$  ist \emph{reflexiv} auf der Menge $M$ falls gilt: \\[0.2cm]
\hspace*{1.3cm} $\forall x\in M \colon \pair(x,x) \in R$. \eox
\end{Definition}

\begin{Satz}
Eine Relation $R \subseteq M \times M$ ist genau dann reflexiv auf $M$, wenn $\mbox{\rm id}_M \subseteq R$ gilt.
\end{Satz}

\proof
Es gilt
\\[0.2cm]
\hspace*{1.3cm}
$
\begin{array}[b]{cl}
                & \mathrm{id}_M \subseteq R \\[0.2cm]
\leftrightarrow & \bigl\{ \pair(x,x) \mid x \in M \bigr\} \subseteq R \\[0.2cm] 
\leftrightarrow & \forall x \in M: \pair(x,x) \in R \\[0.2cm] 
\leftrightarrow & \mbox{$R$ ist reflexiv auf $M$.}  
\end{array}
$
\qed

\begin{Definition}[symmetrisch]
Eine Relation $R \subseteq M \times M$  ist \emph{symmetrisch} falls gilt: \\[0.2cm]
\hspace*{1.3cm} 
$\forall x,y\el M \colon \bigl(\pair(x,y) \in R \rightarrow\pair(y,x)\in R\bigr)$. \eox
\end{Definition}

\begin{Satz}
Eine Relation $R \subseteq M \times M$ ist genau dann symmetrisch, wenn $R^{-1} \subseteq R$ gilt.
\end{Satz}

\proof
Die \"{a}quivalenz der beiden Bedingungen wird offensichtlich, wenn wir die Inklusions-Bedingung
$R^{-1} \subseteq R$ expandieren, indem wir die Gleichungen
\[ R^{-1} = \bigl\{ \pair(y,x) \mid \pair(x,y) \in R \bigr\} \quad \mbox{und} \quad
   R      = \bigl\{ \pair(x,y) \mid \pair(x,y) \in R \bigr\} \]
ber\"{u}cksichtigen, denn dann hat die Inklusions-Bedingung die Form
\[ \bigl\{ \pair(y,x) \mid \pair(x,y) \in R \bigr\} \subseteq
   \bigl\{ \pair(x,y) \mid \pair(x,y) \in R \bigr\}. \]
Nach der Definition der Teilmengen-Beziehung ist diese Bedingung gleichwertig zu der Formel
\[ \forall x, y \in M \colon \bigl(\pair(y,x) \in R \rightarrow\pair(x,y) \in R\bigr). 
\]
Vertauschen wir hier die Rollen der Variablen $x$ und $y$, so ist dies gerade die Bedingung, dass
die Relation $R$ symmetrisch ist. \qed

\begin{Definition}[anti-symmetrisch]
Eine Relation $R \subseteq M \times M$  ist \emph{anti-symmetrisch} falls gilt: \\[0.2cm]
\hspace*{1.3cm} 
$\forall x, y \in M \colon \bigl(\pair(x,y) \in R \wedge \pair(y,x) \in R \rightarrow x = y\bigr)$. \eox
\end{Definition}

\begin{Satz}
Eine Relation $R \subseteq M \times M$  ist genau dann anti-symmetrisch, wenn
$R \cap R^{-1} \subseteq \mbox{\rm id}_{M}$ gilt.
\end{Satz}

\proof
\begin{enumerate}
\item[``$\Rightarrow$'':]
      Wir nehmen zun\"{a}chst an, dass $R$ anti-symmetrisch ist und folglich
      \[ \forall x, y \in M \colon \bigl(\pair(x,y) \in R \wedge \pair(y,x) \in R \rightarrow x = y\bigr) \]
      gilt und zeigen, dass aus dieser Voraussetzung die Inklusions-Beziehung
      \\[0.2cm]
      \hspace*{1.3cm}
      $R \cap R^{-1} \subseteq \mathrm{id}_{M}$ 
      \\[0.2cm]
      folgt.  Es gelte also
      \\[0.2cm]
      \hspace*{1.3cm}
      $\pair(x,y) \in R \cap R^{-1}$.
      \\[0.2cm]
      Daraus folgt
      \\[0.2cm]
      \hspace*{1.3cm}
      $\pair(x,y) \in R \wedge \pair(x,y) \in R^{-1}$.
      \\[0.2cm]
      Nach Definition von $R^{-1}$ folgt daraus
      \\[0.2cm]
      \hspace*{1.3cm}
      $\pair(x,y) \in R \wedge \pair(y,x) \in R$.
      \\[0.2cm]
      Nun folgt aber aus der Anti-Symmetrie von $R$ 
      \\[0.2cm]
      \hspace*{1.3cm}
      $x=y$
      \\[0.2cm]
      und das impliziert
      \\[0.2cm]
      \hspace*{1.3cm}
      $\pair(x,y) \in \mathrm{id}_M$.
      \\[0.2cm]
      Damit ist die Inklusion
      \\[0.2cm]
      \hspace*{1.3cm}
      $R \cap R^{-1} \subseteq \mathrm{id}_{M}$ 
      \\[0.2cm]
      gezeigt.
\item[``$\Leftarrow$'':]
      Wir nehmen nun an, dass $R \cap R^{-1} \subseteq \mathrm{id}_{M}$ gilt und zeigen, dass daraus
      die G\"{u}ltigkeit von 
      \[ \forall x, y \in M \colon \bigl(\pair(x,y) \in R \wedge \pair(y,x) \in R \rightarrow x = y\bigr) \]
      folgt.  Seien also $x,y \in M$ und es gelte
      \\[0.2cm]
      \hspace*{1.3cm}
      $\pair(x,y) \in R$ und $\pair(y,x) \in R$.
      \\[0.2cm]
      Wir m\"{u}ssen zeigen, dass daraus
      \\[0.2cm]
      \hspace*{1.3cm}
      $x=y$ 
      \\[0.2cm]
      folgt.  Aus $\pair(y,x) \in R$ folgt 
      $\pair(x,y) \in R^{-1}$.  Also gilt $\pair(x,y) \in R \cap R^{-1}$.
      Aus der Inklusions-Beziehung $R \cap R^{-1} \subseteq \mathrm{id}_{M}$ folgt dann
      $\pair(x,y) \in \mathrm{id}_M$ und daraus folgt sofort $x = y$. 
      \qed
\end{enumerate}

\exercise
\begin{enumerate}[(a)]
\item Geben Sie eine Menge $M$ und eine Relation $R \subseteq M \times M$ an, so dass $R$
      sowohl anti-symmetrisch als auch symmetrisch ist.
\item Geben Sie eine Menge $M$ und eine Relation $R \subseteq M \times M$ an, so dass $R$
      weder anti-symmetrisch noch symmetrisch ist. \eox
\end{enumerate}

\remark
In der Literatur finden Sie noch den Begriff der \emph{Asymmetrie}.  Dort wird eine Relation 
$R \subseteq M^2$ als \emph{asymmetrisch} definiert, wenn 
\\[0.2cm]
\hspace*{1.3cm}
$\forall x, y \in M: \bigl(\pair(x,y) \in R \rightarrow \pair(y,x) \not\in R \bigr)$
\\[0.2cm]
gilt.  Beachten Sie, dass der Begriff der \emph{Asymmetrie} \underline{nicht} die Negation des Begriffs der
\emph{Symmetrie} ist:  Zwar kann eine symmetrische Relation nicht asymmetrisch sein, aber es gibt
Relationen, die weder asymmetrisch noch symmetrisch sind.  In dieser Vorlesung spielt der Begriff
der \emph{Asymmetrie} keine Rolle.  F\"{u}r Sie reicht es zu wissen, dass \emph{Anti-Symmetrie} und
\emph{Asymmetrie} zwei verschiedene Begriffe sind.
\eox

\begin{Definition}[transitiv]
Eine Relation $R \subseteq M \times M$  ist \emph{transitiv} falls gilt: \\[0.2cm]
\hspace*{1.3cm} 
$\forall x, y, z \in M \colon \bigl(\pair(x,y) \in R \wedge \pair(y,z) \in R \rightarrow \pair(x,z) \in R\bigr)$. 
\eox
\end{Definition}

\begin{Satz}
Eine Relation $R \subseteq M \times M$  ist genau dann transitiv, wenn
$R \circ R \subseteq R$ ist.
\end{Satz}

\proof
\begin{enumerate}
\item[``$\Rightarrow$'':] 
      Wir nehmen zun\"{a}chst an, dass $R$ transitiv ist und damit
      \[ \forall x, y, z \in M \colon \bigl(\pair(x,y) \in R \wedge \pair(y,z) \in R \rightarrow \pair(x,z) \in R\bigr) \]
      gilt und zeigen, dass daraus $R \circ R \subseteq R$ folgt.
      Sei also
      \\[0.2cm]
      \hspace*{1.3cm}
      $\pair(x,z) \in R \circ R$.  
      \\[0.2cm]
      Nach Definition des relationalen Produkts gibt es
      dann ein $y$, so dass
      \\[0.2cm]
      \hspace*{1.3cm}
      $\pair(x,y) \in R$ und $\pair(y,z) \in R$ 
      \\[0.2cm]
      gilt. Da $R$ transitiv ist, folgt daraus
      \\[0.2cm]
      \hspace*{1.3cm}
      $\pair(x,z) \in R$ 
      \\[0.2cm] 
      und das war zu zeigen.
\item[``$\Leftarrow$'':]
      Wir nehmen nun an, dass die Inklusion $R \circ R \subseteq R$ gilt und zeigen, dass daraus
      \[ \forall x, y, z \in M \colon \bigl(\pair(x,y) \in R \wedge \pair(y,z) \in R \rightarrow \pair(x,z) \in R\bigr) \]
      folgt.  Seien also $x,y,z \in M$ mit
      \\[0.2cm]
      \hspace*{1.3cm}
      $\pair(x,y) \in R$ und $\pair(y,z) \in R$ 
      \\[0.2cm]
      gegeben.  Nach Definition des relationalen Produkts gilt dann
      \\[0.2cm]
      \hspace*{1.3cm}
      $\pair(x,z) \in R \circ R$ 
      \\[0.2cm]
      und aus
      der Voraussetzung $R \circ R \subseteq R$ folgt nun
      \\[0.2cm]
      \hspace*{1.3cm}
      $\pair(x,z) \in R$. 
      \qed
\end{enumerate} 

\examples 
In den ersten beiden Beispielen sei $M = \{1,2,3\}$.
\begin{enumerate}
\item $R_1 = \{ \pair(1,1), \pair(2,2), \pair(3,3) \}$.

      $R_1$ ist reflexiv auf $M$, symmetrisch, anti-symmetrisch und transitiv.
\item $R_2 = \{ \pair(1,2), \pair(2,1), \pair(3,3) \}$.

      $R_2$ ist nicht reflexiv auf $M$, da $\pair(1,1) \not\in R_2$.
      $R_2$ ist symmetrisch. 
      $R_2$ ist nicht anti-symmetrisch, denn aus $\pair(1,2) \in R_2$ und 
      $\pair(2,1) \in R_2$ m\"{u}sste $2=1$ folgen.
      Schlie\3lich ist $R_2$ auch nicht transitiv, denn aus $\pair(1,2) \in R_2$ und 
      $\pair(2,1) \in R_2$ m\"{u}sste $\pair(1,1) \in R_2$ folgen.

      In den beiden folgenden Beispielen sei $M = \mathbb{N}$.
\item $R_3 := \{ \pair(n,m) \in \mathbb{N}^2 \mid n \leq m \}$.

      $R_3$ ist reflexiv auf $\mathbb{N}$, denn f\"{u}r alle nat\"{u}rlichen Zahlen $n \in \mathbb{N}$ gilt 
      $n \leq n$.  $R_3$ ist nicht symmetrisch, denn beispielsweise gilt
      $1 \leq 2$, aber es gilt nicht $2 \leq 1$.  Allerdings ist $R_3$ anti-symmetrisch,
      denn wenn sowohl $n \leq m$ als auch $m \leq n$ gilt, dann muss  $m = n$ gelten.
      Schlie\3lich ist $R_3$ auch transitiv, denn aus $k \leq m$ und $m \leq n$ folgt
      nat\"{u}rlich $k \leq n$.
\item $R_4 := \{ \pair(m,n) \in \mathbb{N}^2 \mid \exists k\in \mathbb{N}: m\cdot k = n \}$

      F\"{u}r zwei positive Zahlen $m$ und $n$ gilt $\pair(m,n) \in R_4$ genau dann, wenn $m$ ein
      Teiler von $n$ ist.  Damit ist klar, dass $R_4$ reflexiv auf $\mathbb{N}$ ist, denn jede Zahl 
      teilt sich selbst.  Nat\"{u}rlich ist $R_4$ nicht symmetrisch, denn $1$ ist ein Teiler
      von $2$ aber nicht umgekehrt.  Daf\"{u}r ist $R_4$ aber anti-symmetrisch, denn wenn 
      sowohl $m$ ein Teiler von $n$ ist und auch $n$ ein Teiler vom $m$, so muss $m = n$
      gelten.  Schlie\3lich ist $R_4$ auch transitiv: Ist $m$ ein Teiler von $n$ und
      $n$ ein Teiler von $o$, so ist nat\"{u}rlich $m$ ebenfalls ein Teiler von $o$.
\end{enumerate}

\section{Der transitive Abschluss einer Relation}
Ist $R$ eine Relation auf einer Menge $M$, die nicht transitiv ist, so k\"{o}nnen wir $R$ zu einer
transitiven Relation erweitern.  Zu diesem Zweck  definieren wir zun\"{a}chst den Begriff der \emph{Potenz}
einer Relation auf $M$.

\begin{Definition}[Potenz einer Relation, $R^n$]
  F\"{u}r eine Relation $R \subseteq M^2$ definieren wir f\"{u}r alle $n \in \mathbb{N}_0$ die \emph{Potenz} $R^n$
  durch Induktion \"{u}ber $n$ wie folgt:  
  \begin{enumerate}
  \item Induktions-Anfang: $n= 0$.  Wir setzen \\[0.2cm]
        \hspace*{1.3cm} $R^0 := \mathrm{id}_M$.
  \item Induktions-Schritt: $n \rightarrow n + 1$. Nach Induktions-Voraussetzung ist $R^n$
        bereits definiert. Daher k\"{o}nnen wir $R^{n+1}$ als \\[0.2cm]
        \hspace*{1.3cm} $R^{n+1} = R \circ R^n$
        \\[0.2cm]
        definieren. \eox
\end{enumerate}
\end{Definition}

Es zeigt sich, dass f\"{u}r die Potenzen einer Relation und das relationalen Produkt ein \"{a}hnlicher
Zusammenhang besteht, wie f\"{u}r die Potenzen einer Zahl und das gew\"{o}hnliche Produkt, denn es gilt das
folgende Gesetz.

\begin{Satz}[Potenz-Gesetz des relationalen Produkts] \hspace*{\fill} \\
  Es sei $R \subseteq M \times M$.  F\"{u}r beliebige ganze Zahlen $k,l \in \mathbb{N}_0$ gilt:
  \[ R^k \circ R^l = R^{k+l}. \]
\end{Satz}

\proof  Wir f\"{u}hren den Beweis durch Induktion nach $k$.
\begin{enumerate}
\item[I.A.:] $k = 0$.  Es gilt
             \\[0.2cm]
             \hspace*{1.3cm}
             $R^0 \circ R^l = \textsl{id}_M \circ R^l = R^l = R^{0+l}$. $\surd$
\item[I.S.:] $k \mapsto k+1$.  Es gilt
             \[
             \begin{array}{lcll}
               R^{k+1} \circ R^l & = & (R \circ R^k) \circ R^l &
                                       \mbox{nach Def. von}\; R^{k+1} \\
                                 & = & R \circ (R^k \circ R^l) &
                                       \mbox{aufgrund des Assoziativ-Gesetzes f\"{u}r $\circ$}                                       \\
                                 & = & R \circ R^{k+l} &
                                       \mbox{nach Induktions-Voraussetzung} \\
                                 & = & R^{(k+l)+1} &
                                       \mbox{nach Def. von}\; R^{n+1} \\
                                 & = & R^{(k+1)+l}.\; \surd & \hspace*{\fill} \Box
             \end{array}
             \]
\end{enumerate}
             
Die n\"{a}chste Definition zeigt, dass wir eine beliebige Relation $R \subseteq M^2$ zu einer
transitiven Relation erweitern k\"{o}nnen.

\begin{Definition}[transitiver Abschluss einer Relation, $R^+$]\hspace*{\fill} \\
  Es sei $R \subseteq M^2$.
  Wir definieren den \emph{transitiven Abschluss} der Relation $R$ als die Menge \\[0.2cm]
  \hspace*{1.3cm} $R^+ := \bigcup\limits_{n\in\mathbb{N}} R^n$. \\
  Dabei ist f\"{u}r eine Folge $(A_n)_n$ von Mengen der Ausdruck $\bigcup\limits_{n\in\mathbb{N}} A_n$ 
  als \\[0.2cm]
  \hspace*{1.3cm}
  $\bigcup\limits_{n\in\mathbb{N}} A_n := \bigl\{ x \,\big|\, \exists n \in \mathbb{N}: x \in A_n \bigr\}$.
  \\[0.2cm]
  definiert.  Etwas weniger formal (aber daf\"{u}r anschaulicher) k\"{o}nnen wie auch schreiben
  \\[0.2cm]
  \hspace*{1.3cm}
 $\bigcup\limits_{n\in\mathbb{N}} A_n := \bigcup\limits_{i=1}^{\infty} A_n := A_1 \cup A_2 \cup A_3 \cup \cdots $. \eox
\end{Definition}

\begin{Satz}
Es sei $M$ eine Menge und $R \subseteq M \times M$ eine bin\"{a}re Relation auf $M$.
Dann hat die oben definierte Relation $R^+$ die folgenden Eigenschaften:
\begin{enumerate}
\item $R^+$ ist transitiv.
\item $R^+$   ist die bez\"{u}glich der Inklusions-Ordnung $\subseteq$ kleinste Relation
      $T$ auf $M$, die einerseits transitiv ist und andererseits die Relation $R$ enth\"{a}lt.
      Anders ausgedr\"{u}ckt: Ist $T$ ein transitive Relation auf $M$ mit $R \subseteq T$, so
      muss $R^+ \subseteq T$ gelten.
\end{enumerate}
\end{Satz}

\noindent
\textbf{Beweis}:
\begin{enumerate}
\item Wir zeigen zun\"{a}chst, dass $R^+$ transitiv ist. Dazu m\"{u}ssen wir die G\"{u}ltigkeit der Formel
      \[ \forall x, y, z: \bigl(\pair(x,y) \in R^+ \wedge \pair(y,z) \in R^+ \rightarrow \pair(x,z) \in R^+\bigr) \]
nachweisen.  Wir nehmen also an, dass $\pair(x,y) \in R^+$ und $\pair(y,z) \in R^+$
gilt und zeigen, dass aus dieser Voraussetzung auf $\pair(x,z) \in R^+$ geschlossen
werden kann.  Nach Definition von $R^+$ haben wir 
\[ \pair(x,y) \in \bigcup\limits_{n\in\mathbb{N}} R^n \quad \mbox{und} \quad
   \pair(y,z) \in \bigcup\limits_{n\in\mathbb{N}} R^n.
\]
Nach der Definition der Menge $\bigcup\limits_{n\in\mathbb{N}} R^n$ gibt es dann nat\"{u}rliche Zahlen $k,l\in\mathbb{N}$, so dass
\[ \pair(x,y) \in R^k \quad \mbox{und} \quad \pair(y,z) \in R^l \]
gilt.  Aus der Definition des relationalen Produktes folgt nun
\[  \pair(x,z) \in R^k \circ R^l. \]
Aufgrund des Potenz-Gesetzes f\"{u}r das relationale Produkt gilt 
\[ R^k \circ R^l = R^{k+l}. \]
Also haben wir $\pair(x,z) \in R^{k+l}$ und daraus folgt sofort
\[  \pair(x,z) \in \bigcup\limits_{n\in\mathbb{N}} R^n. \]
Damit gilt $\pair(x,z) \in R^+$ und das war zu zeigen. $\surd$

\item 
Um zu zeigen, dass $R^+$ die kleinste Relation ist, die einerseits transitiv ist
und andererseits $R$ enth\"{a}lt, nehmen wir an, dass $T$ eine transitive Relation ist,
f\"{u}r die $R \subseteq T$ gilt.  Wir m\"{u}ssen dann zeigen, dass $R^+ \subseteq T$ gilt.  Um diesen
Nachweis zu f\"{u}hren, zeigen wir zun\"{a}chst, dass f\"{u}r alle 
nat\"{u}rlichen Zahlen $n\in\mathbb{N}$ die folgende Inklusion gilt:
\[ R^n \subseteq T. \]
Wir f\"{u}hren den Nachweis dieser Behauptung durch vollst\"{a}ndige Induktion \"{u}ber $n\in\mathbb{N}$.
\begin{enumerate}
\item[I.A.:] $n=1$.  

               Dann ist $R^1 \subseteq T$ zu zeigen.  Wegen
               $R^1 = R \circ R^0 = R \circ \mathrm{id}_M = R$
               folgt dies aber unmittelbar aus der Voraussetzung $R \subseteq T$. $\surd$
\item[I.S.:] $n \mapsto n+1$.  

             Nach Induktions-Voraussetzung wissen wir, dass
              \[ R^n \subseteq T \]
             gilt.
             Wir multiplizieren diese Inklusion auf beiden Seiten von links relational mit $R$
             und haben dann
             \[ R^{n+1} = R \circ R^n \subseteq R \circ T. \]
             Multiplizieren wir die Voraussetzung $R \subseteq T$ von rechts relational mit
             $T$, so finden wir
             \[ R \circ T \subseteq T \circ T. \]
             Weil $T$ transitiv ist, gilt 
             \[ T \circ T \subseteq T. \]
             Insgesamt haben wir also die folgende Kette von Inklusionen
             \[ R^{n+1} \subseteq R \circ T \subseteq T \circ T \subseteq T. \]
             Damit folgt $R^{n+1} \subseteq T$ und der Induktions-Beweis ist
             abgeschlossen. $\surd$
\end{enumerate}
Wir schlie\3en den Beweis ab, indem wir
\\[0.2cm]
\hspace*{1.3cm}
 $R^+ \subseteq T$ 
\\[0.2cm]
zeigen.  Sei $\pair(x,y) \in R^+$.
Nach Definition von $R^+$ muss es dann eine  nat\"{u}rliche Zahl $n$ geben, so
dass $\pair(x,y) \in R^n$ ist.  Wegen $R^n \subseteq T$ folgt daraus aber
$\pair(x,y) \in T$ und damit ist auch der zweite Teil des Beweises abgeschlossen. \qed
\end{enumerate}

\example
Es sei $\textsl{Mensch}$ die Menge alle Menschen, die jemals gelebt haben.  Wir definieren
die Relation \textsl{Eltern} auf $M$ indem wir 
\\[0.2cm]
\hspace*{1.3cm}
$\textsl{Eltern} := \{ \pair(x,y) \in \textsl{Mensch}^2 \mid \mbox{$x$ ist Vater von $y$ oder 
                                                                   $x$ ist Mutter von $y$} \}$
\\[0.2cm]
setzen.  Dann besteht der transitive Abschluss der Relation \textsl{Eltern} aus allen Paaren 
$\pair(x,y)$, f\"{u}r die $x$ ein Vorfahre von $y$ ist:
\\[0.2cm]
\hspace*{1.3cm}
$\textsl{Eltern}^+ = \{ \pair(x,y) \in \textsl{Mensch}^2 \mid \mbox{$x$ ist Vorfahre von $y$} \}$.
\eox
\pagebreak

\example
Es sei \textsl{F} die Menge aller Flugh\"{a}fen.  Wir definieren auf der Menge \textsl{F} eine
Relation \textsl{D} durch
\\[0.2cm]
\hspace*{1.3cm}
$\textsl{D} := \{ \pair(x,y) \in \textsl{F} \times \textsl{F} \mid
                  \mbox{Es gibt einen Direktflug von $x$ nach $y$} \}$.
\\[0.2cm]
$D$ bezeichnet also die direkten Verbindungen.  Die Relation $D^2$ ist dann definiert als
\\[0.2cm]
\hspace*{1.3cm}
$D^2 := D \circ D = \{ \pair(x,z) \in \textsl{F} \times \textsl{F} \mid 
          \exists y \in \textsl{F}: \pair(x,y) \in D \wedge \pair(y,z) \in D \}$.
\\[0.2cm]
Das sind aber gerade die Paare $\pair(x,z)$, f\"{u}r die es einen Luftweg von $x$ nach $z$
gibt, der genau einen Zwischenstopp enth\"{a}lt.  Entsprechend enth\"{a}lt $D^3$ die Paare
$\pair(x,z)$, f\"{u}r die man mit zwei Zwischenstopps von $x$ nach $y$ kommt und allgemein
enth\"{a}lt $D^k$ die Paare $\pair(x,z)$, f\"{u}r die man mit $k-1$ Zwischenstopps von dem
Flughafen $x$ zu dem Flughafen $z$ kommt.
Der transitive Abschluss von $D$ enth\"{a}lt
dann alle Paare $\pair(x,y)$, f\"{u}r die es \"{u}berhaupt eine M\"{o}glichkeit gibt, auf dem Luftweg
von $x$ nach $y$ zu kommen. \eox

\exercise
Auf der Menge $\mathbb{N}$ der nat\"{u}rlichen Zahlen wird die Relation $R$ wie folgt definiert:
\\[0.2cm]
\hspace*{1.3cm}
$R = \{ \pair(k, k + 1) \mid k \in \mathbb{N} \}$.
\\[0.2cm]
Berechnen Sie die folgenden Relationen:
\begin{enumerate}[(a)]
\item $R^2$,
\item $R^3$,
\item $R^n$ f\"{u}r beliebige $n \in \mathbb{N}$,
\item $R^+$. \exend
\end{enumerate}

\exercise
Wir definieren die Relation $R$ auf der Menge $\mathbb{N}$ der nat\"{u}rlichen Zahlen als
\\[0.2cm]
\hspace*{1.3cm}
$R := \{ \pair(n, 2 \cdot n) \mid n \in \mathbb{N} \}$.
\\[0.2cm]
Berechnen Sie den transitiven Abschluss $R^+$.
\exend


\section{\"{a}quivalenz-Relationen \label{section:aequivalenz_relation}}
\begin{Definition}[\"{a}quivalenz-Relation]
Eine Relation $R \subseteq M \times M$  ist eine 
\emph{\"{a}quivalenz-Relation} auf $M$ genau dann, wenn folgende Bedingungen erf\"{u}llt sind:
\begin{enumerate}
\item $R$ ist reflexiv auf $M$,
\item $R$ ist symmetrisch und
\item $R$ ist transitiv. \eox
\end{enumerate}
\end{Definition}

Der Begriff der \"{a}quivalenz-Relationen verallgemeinert den Begriff der Gleichheit, denn
ein triviales Beispiel f\"{u}r eine \"{a}quivalenz-Relation auf $M$ ist die Relation $\mathrm{id}_M$.
Das folgende Beispiel zeigt eine nicht-triviale \"{a}quivalenz-Relation, die sp\"{a}ter noch eine wichtige
Rolle spielen wird.

\example
Wir betrachten die Menge $\mathbb{Z}$ der ganzen Zahlen
zusammen mit der Relation $\approx_n$, die wir f\"{u}r nat\"{u}rliche Zahlen $n\in\mathbb{N}$ 
als \\[0.2cm]
\hspace*{1.3cm}
 $\approx_n \;:=\; \{ \pair(x,y) \in \mathbb{Z}^2 \mid \exists k \in \mathbb{Z} \colon k \cdot n = x - y \}$
\\[0.2cm]
definieren.  Wir zeigen, dass die Relation $\approx_n$ f\"{u}r $n \in \mathbb{N}$ eine \"{a}quivalenz-Relation auf
$\mathbb{Z}$ definiert. 
\pagebreak
\begin{enumerate}
\item Um zu zeigen, dass $\approx_n$ reflexiv ist, m\"{u}ssen wir nachweisen, dass f\"{u}r alle
      $x \in \mathbb{Z}$ die Beziehung
      \\[0.2cm]
      \hspace*{1.3cm}
      $\pair(x,x) \in\; \approx_n$.        
      \\[0.2cm]
      gilt.  Nach Definition von $\approx_n$
      ist dies \"{a}quivalent zu \\[0.2cm]
      \hspace*{1.3cm}
      $\pair(x,x) \in \bigl\{ \pair(x,y) \in \mathbb{Z}^2 \mid \exists k \in \mathbb{Z}: k \cdot n = x - y \bigr\}$.
      \\[0.2cm]
      F\"{u}r eine vorgegebene Zahl $x$ ist dies gleichbedeutend mit \\[0.2cm]
      \hspace*{1.3cm}
      $\exists k \in \mathbb{Z}: k \cdot n = x - x$.
      \\[0.2cm]
      Offenbar erf\"{u}llt $k=0$ diese Gleichung, denn es gilt: \\[0.2cm]
      \hspace*{1.3cm}
      $0\cdot n = 0 = x - x$. 
      \\[0.2cm]
      Damit ist die Reflexivit\"{a}t nachgewiesen. $\surd$
\item Um die Symmetrie von $\approx_n$ nachzuweisen nehmen wir an, dass 
      $\pair(x,y) \in\; \approx_n$ ist.  Dann gibt es also ein $k \in \mathbb{Z}$, so dass
      \\[0.2cm]
      \hspace*{1.3cm}      
      $k\cdot n = x - y$
      \\[0.2cm] 
      gilt.  Multiplizieren wir diese Gleichung mit $-1$, so erhalten wir
      \\[0.2cm]
      \hspace*{1.3cm}      
      $(-k)\cdot n = y - x$.
      \\[0.2cm]
      Dies zeigt aber, dass $\pair(y,x) \in\; \approx_n$ ist und damit ist die Symmetrie
      nachgewiesen. $\surd$
\item Zum Nachweis der Transitivit\"{a}t von $\approx$ nehmen wir an, dass
      sowohl $\pair(x,y) \in\; \approx_n$ als auch $\pair(y,z) \in\; \approx_n$
      gelten.  Dann gibt es also $k_1,k_2 \in \mathbb{Z}$ so dass 
      \\[0.2cm]
      \hspace*{1.3cm}      
      $k_1 \cdot n = x - y$ \quad und \quad $k_2 \cdot n = y - z$ 
      \\[0.2cm]
      gelten.  Wir m\"{u}ssen zeigen, dass $\pair(x,z) \in \approx_n$ gilt.  Dazu m\"{u}ssen wir zeigen,
      dass es ein $k_3 \in \mathbb{Z}$ gibt, so dass
      \\[0.2cm]
      \hspace*{1.3cm}
      $k_3 \cdot n = x - z$
      \\[0.2cm]
      gilt.  Addieren wir die beiden Gleichungen f\"{u}r $x-y$ und $y-z$, so sehen wir, dass
      \\[0.2cm]
      \hspace*{1.3cm}      
      $(k_1 + k_2) \cdot n = x - z$
      \\[0.2cm]
      gilt.  Definieren wir  $k_3 := k_1 + k_2$, so gilt also $k_3\cdot n = x - z$ und damit haben wir
      \\[0.2cm]
      \hspace*{1.3cm}
      $\pair(x,z) \in\; \approx_n$ 
      \\[0.2cm]
      nachgewiesen und folglich die Transitivit\"{a}t von $\approx_n$ gezeigt. $\surd$ 
      \qed
\end{enumerate}

\exercise
Beweisen Sie, dass f\"{u}r alle ganzen Zahlen $x$ und $y$ die Beziehung
\\[0.2cm]
\hspace*{1.3cm}
 $x \approx_n y \;\leftrightarrow\; x \modulo n = y \modulo n$
\\[0.2cm]
gilt.
\exend


\exercise
Auf der Menge $\mathbb{N} \times \mathbb{N}$ definieren wir eine Relation $R$ wie folgt:
\\[0.2cm]
\hspace*{1.3cm}
$R := \bigl\{ \bigl\langle \pair(x_1, y_1) ,\pair(x_2, y_2) \bigr\rangle \in  (\mathbb{N} \times \mathbb{N})^2 \mid x_1 + y_2 = x_2 + y_1 \bigr\}$
\\[0.2cm]
Zeigen Sie, dass $R$ dann eine \"{a}quivalenz-Relation ist.
\exend
\pagebreak

\begin{Satz} Es seien $M$ und $N$ Mengen und 
\\[0.2cm]
\hspace*{1.3cm}
$f : M \rightarrow N$
\\[0.2cm]
sei eine Funktion.  Definieren wir die Relation $R_f \subseteq M \times M$ als
\\[0.2cm]
\hspace*{1.3cm}
$R_f := \bigl\{ \pair(x,y) \in M \times M \mid f(x) = f(y) \bigr\}$,
\\[0.2cm]
so ist $R_f$ eine \"{a}quivalenz-Relation.  
\end{Satz}


\noindent
\textbf{Beweis}: Wir weisen der Reihe nach die Reflexivit\"{a}t, Symmetrie und Transitivit\"{a}t der Relation $R_f$ nach:
\begin{enumerate}
\item $R_f$ ist reflexiv auf $M$, denn es gilt 
      \\[0.2cm]
      \hspace*{1.3cm}
      $\forall x \in M: f(x) = f(x)$.
      \\[0.2cm]
      Daraus folgt sofort 
      \\[0.2cm]
      \hspace*{1.3cm}
      $\forall x \in M: \pair(x,x) \in R_f$. $\surd$
\item Um die Symmetrie von $R_f$ nachzuweisen, m\"{u}ssen wir
      \\[0.2cm]
      \hspace*{1.3cm}
      $\forall x,y \in M: (\pair(x,y) \in R_f \rightarrow \pair(y,x) \in R_f)$
      \\[0.2cm]
      zeigen.  Sei also $\pair(x,y) \in R_f$. Dann gilt nach Definition von $R_f$
      \\[0.2cm]
      \hspace*{1.3cm}
      $f(x) = f(y)$.
      \\[0.2cm]
      Daraus folgt sofort 
      \\[0.2cm]
      \hspace*{1.3cm}
      $f(y) = f(x)$
      \\[0.2cm]
      und nach Definition von $R_f$ ist das \"{a}quivalent zu 
      \\[0.2cm]
      \hspace*{1.3cm}
      $\pair(y,x) \in R_f$. $\surd$
\item Um die Transitivit\"{a}t von $R_f$ nachzuweisen, m\"{u}ssen wir 
      \\[0.2cm]
      \hspace*{1.3cm}
      $\forall x,y,z \in M: \bigl(\pair(x,y) \in R_f \wedge \pair(y,z) \in R_f \rightarrow 
                            \pair(x,z) \in R_f\bigr)
      $
      \\[0.2cm]
      zeigen.  Gelte also 
      \\[0.2cm]
      \hspace*{1.3cm}
      $\pair(x,y) \in R_f \wedge \pair(y,z) \in R_f$.
      \\[0.2cm]
      Nach Definition von $R_f$ hei\3t das 
      \\[0.2cm]
      \hspace*{1.3cm}
      $f(x) = f(y) \wedge f(y) = f(z)$.
      \\[0.2cm]
      Daraus folgt sofort 
      \\[0.2cm]
      \hspace*{1.3cm}
      $f(x) = f(z)$.
      \\[0.2cm]
      Nach Definition der Relation $R_f$ haben wir also 
      \\[0.2cm]
      \hspace*{1.3cm}
      $\pair(x,z) \in R_f$. $\surd$
      \qed
\end{enumerate}


\remark
Ist $f: M \rightarrow N$ eine Funktion und gilt
\\[0.2cm]
\hspace*{1.3cm}
$R_f = \bigl\{ \pair(x,y) \in M \times M \mid f(x) = f(y) \bigr\}$ 
\\[0.2cm]
so sagen wir, dass $R_f$ die \emph{von $f$ auf M erzeugte} \"{a}quivalenz-Relation ist.  Wir werden
sp\"{a}ter sehen, dass es zu jeder  \"{a}quivalenz-Relation eine Funktion gibt, die diese
\"{a}quivalenz-Relation erzeugt. \eox 
\pagebreak

\example 
Die \"{a}quivalenz-Relation $\approx_n$ wird von der Funktion
\\[0.2cm]
\hspace*{1.3cm}
$x \mapsto x \modulo n$
\\[0.2cm]
erzeugt, denn wir haben in einer Aufgabe gezeigt, dass f\"{u}r alle $x,y \in \mathbb{Z}$ 
\\[0.2cm]
\hspace*{1.3cm}
$x \approx_n y \;\leftrightarrow\; x \modulo n = y \modulo n$
\\[0.2cm]
gilt.
\eox


\example
Es sei $M$ die Menge aller Menschen und $S$ sei die Menge aller Staaten.  Nehmen wir zur
Vereinfachung an, dass jeder Mensch genau eine Staatsb\"{u}rgerschaft hat, so k\"{o}nnen wir eine Funktion
\\[0.2cm]
\hspace*{1.3cm}
$\textsl{sb}: M \rightarrow S$
\\[0.2cm]
definieren, die jedem Menschen $x$ seine Staatsb\"{u}rgerschaft $\textsl{sb}(x)$ zuordnet.  Bei der durch
diese Funktion definierten \"{a}quivalenz-Relation sind dann alle die Menschen \"{a}quivalent, welche dieselbe
Staatsb\"{u}rgerschaft haben.  \eox

\begin{Definition}[\"{a}quivalenz-Klasse]
Ist $R$ eine \"{a}quivalenz-Relation auf $M$, so definieren wir f\"{u}r alle $x \in M$ 
die Menge $[x]_R$ durch \\[0.2cm]
\hspace*{1.3cm} $[x]_R \;:=\; \bigl\{ y \in M \mid x \mathop{R} y \bigr\}$. \qquad
(Wir schreiben hier $x R y$ als Abk\"{u}rzung f\"{u}r $\pair(x, y) \in R$.) 
\\[0.2cm]
Die Menge $[x]_R$ bezeichnen wir als die von $x$ erzeugte \emph{\"{a}quivalenz-Klasse}.  \eox
\end{Definition}

\begin{Satz}[Charakterisierung der \"{a}quivalenz-Klassen] \label{satz:aequivalenz-klassen}
\hspace*{\fill} \\
Ist $R \subseteq M \times M$ eine \"{a}quivalenz-Relation, so gilt: 
\begin{enumerate}
\item $\forall x \el M \colon x \el [x]_R$,
\item $\forall x, y \el M \colon \bigl(x \mathop{R} y \rightarrow [x]_R = [y]_R\bigr)$,
\item $\forall x, y \el M \colon \bigl(\neg x \mathop{R} y \rightarrow [x]_R \cap [y]_R = \emptyset\bigr)$.
\end{enumerate}
\end{Satz}


\remark
Da f\"{u}r $x,y\el M$ entweder $x \mathop{R} y$ oder $\neg (x \mathop{R} y)$ gilt, zeigen die
letzten beiden Eigenschaften, dass zwei \"{a}quivalenz-Klassen entweder gleich
oder disjunkt sind:
\\[0.2cm]
\hspace*{1.3cm}
$\forall x, y \el M: \bigl([x]_R = [y]_R \vee [x]_R \cap [y]_R = \emptyset\bigr)$. \eox

\proof
Wir beweisen die Behauptungen in derselben Reihenfolge wie oben angegeben.
\begin{enumerate}
\item Wir haben $x \el [x]_R$ genau dann, wenn 
      $x \in \bigl\{ y \el M \mid x \mathop{R} y \bigr\}$ gilt und letzteres ist
      \"{a}quivalent zu $x \mathop{R} x$.
      Die Eigenschaft $x \mathop{R} x$ folgt aber unmittelbar aus der Reflexivit\"{a}t der \"{a}quivalenz-Relation. $\surd$
\item Sei  $x \mathop{R} y$.  Um $[x]_R = [y]_R$ nachzuweisen zeigen wir
      $[x]_R \subseteq [y]_R$ und $[y]_R \subseteq [x]_R$.

      \begin{enumerate}
      \item Zeige $[x]_R \subseteq [y]_R$:
        
            Sei $u \el [x]_R$.  Dann gilt
            \\[0.2cm]
            \hspace*{1.3cm}
            $x \mathop{R} u$.  
            \\[0.2cm]
            Aus der Voraussetzung
            $x \mathop{R} y$ folgt wegen der Symmetrie der Relation $R$, dass auch
            \\[0.2cm]
            \hspace*{1.3cm}
            $y \mathop{R} x$ 
            \\[0.2cm]
            gilt.  Aus 
            $y \mathop{R} x$ und $x \mathop{R} u$ folgt wegen der Transitivit\"{a}t der Relation
            $R$, dass
            \\[0.2cm]
            \hspace*{1.3cm}
            $y \mathop{R} u$ 
            \\[0.2cm]
            gilt.  Nach der Definition der
            Menge $[y]_R$ folgt damit $u \el [y]_R$. 
            Damit ist $[x]_R \subseteq [y]_R$ nachgewiesen.
      \item Zeige $[y]_R \subseteq [x]_R$:

            Um $[y]_R \subseteq [x]_R$ zu zeigen nehmen wir $u \el [y]_R$ an.
            Dann gilt $y \mathop{R} u$.  Aus der Voraussetzung $x \mathop{R} y$ und
            $y \mathop{R} u$ folgt wegen der Transitivit\"{a}t der Relation $R$ sofort
            $x \mathop{R} u$.  Dann gilt aber $u \el [x]_R$ und damit ist auch die Inklusion
            $[y]_R \subseteq [x]_R$ nachgewiesen. 
      \end{enumerate}
      Damit haben wir insgesamt die Gleichung $[x]_R = [y]_R$ gezeigt. $\surd$
\item Sei nun $\neg (x \mathop{R} y)$ vorausgesetzt.  Um nachzuweisen,
      dass  $[x]_R \cap [y]_R = \emptyset$ ist nehmen wir an, dass
      es ein $z \el [x]_R \cap [y]_R$ gibt.  Aus dieser Annahme werden wir einen
      Widerspruch zu der Voraussetzung $\neg (x \mathop{R} y)$ 
        herleiten.  Sei also $z \el[x]_R$ und $z \el [y]_R$.  Nach Definition
        der \"{a}quivalenz-Klassen $[x]_R$ und $[y]_R$ gilt dann 
      \\[0.2cm]
      \hspace*{1.3cm}      
      $x \mathop{R} z$ \quad und \quad $y \mathop{R} z$.
      \\[0.2cm]
      Aufgrund der Symmetrie von $R$ k\"{o}nnen wir $y \mathop{R} z$ umdrehen und haben dann
      \\[0.2cm]
      \hspace*{1.3cm}      
      $x \mathop{R} z$ \quad und \quad $z \mathop{R} y$.
      \\[0.2cm]
      Aus der Transitivit\"{a}t der \"{a}quivalenz-Relation $R$ folgt jetzt
      $x \mathop{R} y$.
      Dies steht aber im Widerspruch zu der Voraussetzung $\neg (x \mathop{R} y)$.
      Damit ist die Annahme, dass es ein $z \el [x]_R \cap [y]_R$ gibt, widerlegt.
      Folglich ist die Menge $[x]_R \cap [y]_R$ leer. $\surd$
      \qed
\end{enumerate}

\begin{Korollar}
Aus dem letzten Satzes folgt sofort, dass 
\\[0.2cm]
\hspace*{1.3cm}
$\pair(x,y) \in R \;\leftrightarrow\; [x]_R = [y]_R$
\\[0.2cm]
gilt. \eox
\end{Korollar}

\remark
Ist $R \subseteq M \times M$ eine \"{a}quivalenz-Relation auf M, so k\"{o}nnen wir eine Funktion
\\[0.2cm]
\hspace*{1.3cm}
$f_R : M \rightarrow 2^M$
\\[0.2cm]
durch die Festlegung
\\[0.2cm]
\hspace*{1.3cm}
$f_R(x) := [x]_R = \bigl\{ y \in M \,|\, x R y \bigr\}$
\\[0.2cm]
definieren.   Das Korollar zum letzten Satz zeigt dann, dass die Funktion $f_R$ die \"{a}quivalenz-Relation
$R$ erzeugt, denn es gilt
\\[0.2cm]
\hspace*{1.3cm}
$
\begin{array}[t]{lcl}
R_{f_R} & = & \bigl\{ \pair(x, y) \in M \times M \mid f_R(x) = f_R(y)  \bigr\}  \\[0.2cm]
        & = & \bigl\{ \pair(x, y) \in M \times M \mid [x]_R = [y]_R    \bigr\}  \\[0.2cm]
        & = & \bigl\{ \pair(x, y) \in M \times M \mid \pair(x,y) \in R \bigr\}  \\[0.2cm]
        & = & R,
\end{array}
$
\\[0.2cm]
denn  $R \subseteq M \times M$. \eox


\begin{Definition}[Quotienten-Raum]
Ist $M$ eine Menge und $R$ eine \"{a}quivalenz-Relation auf $M$ so definieren wir die Menge $M/R$ (lese:
$M$ \emph{modulo} $R$) als die Menge der von $R$ auf $M$ erzeugten \"{a}quivalenz-Klassen:
\\[0.2cm]
\hspace*{1.3cm}
$M/R := \bigl\{ [x]_R \mid x \in M \bigr\}$.
\\[0.2cm]
Die Menge $M/R$ der von $R$ auf $M$ erzeugten \"{a}quivalenz-Klassen nennen wir den
\emph{Quotienten-Raum} vom $M$ \"{u}ber $R$.  \eox
\end{Definition}

\example
Setzen wir das letzte Beispiel fort, in dem alle die Menschen \"{a}quivalent waren, die dieselbe
Staatsb\"{u}rgerschaft haben, so finden wir, dass die \"{a}quivalenz-Klassen, die von dieser
\"{a}quivalenz-Relation erzeugt werden, gerade aus den Menschen bestehen, die dieselbe
Staatsb\"{u}rgerschaft besitzen.  \eox


\begin{Definition}[Partition] 
Ist ${\cal P} \subseteq 2^M$ eine Menge von Teilmengen von $M$, so sagen wir, dass ${\cal P}$ eine 
\emph{Partition} von $M$ ist, falls $\mathcal{P}$ die folgenden Eigenschaften hat:
\begin{enumerate}
\item \emph{Vollst\"{a}ndigkeits-Eigenschaft} 
      \\[0.2cm]
      \hspace*{1.3cm}
      $\forall x \el M : \exists K \el {\cal P} : x \el K$,

      jedes Element aus $M$ findet sich  in einer Menge aus ${\cal P}$ wieder.
\item \emph{Separations-Eigenschaft} 
      \\[0.2cm]
      \hspace*{1.3cm}
      $\forall K, L \el {\cal P} : \bigl(K \cap L =\emptyset \vee K = L\bigr)$,
      \\[0.2cm]
      zwei Mengen aus ${\cal P}$ sind also entweder disjunkt oder schon identisch.
\end{enumerate}
Gelegentlich wird eine Partition einer Menge $M$ auch als \emph{Zerlegung} von $M$
bezeichnet. \eox
\end{Definition}

\remark
Der letzte Satz (Satz \ref{satz:aequivalenz-klassen}) hat gezeigt, dass f\"{u}r jede \"{a}quivalenz-Relation
$R$ auf einer Menge $M$ der Quotienten-Raum 
\\[0.2cm]
\hspace*{1.3cm}
$M/R = \bigl\{ [x]_R \mid x \in M \bigr\}$
\\[0.2cm]
eine Partition der Menge $M$ darstellt.  Der n\"{a}chste Satz zeigt, dass auch die Umkehrung gilt, denn
aus jeder Partition einer Menge l\"{a}sst sich eine \"{a}quivalenz-Relation erzeugen.
\eox

\begin{Satz} 
Es sei $M$ eine Menge und $\mathcal{P} \subseteq 2^M$ eine
Partition von $M$.  Definieren wir die Relation $R$ durch
\\[0.2cm]
\hspace*{1.3cm}
$R := \bigl\{ \pair(x,y) \in M \times M \mid \exists K \in \mathcal{P}: \bigl(x \in K \wedge y \in K\bigr) \bigr\}$,
\\[0.2cm]
so ist $R$ eine \"{a}quivalenz-Relation auf $M$.
\end{Satz}

\proof
Wir haben zu zeigen dass die Relation $R$ reflexiv, symmetrisch und
transitiv ist.
\begin{enumerate}
\item Reflexivit\"{a}t: Zu zeigen ist 
      \\[0.2cm]
      \hspace*{1.3cm}
      $\forall x \in M: x \mathop{R} x$.
      \\[0.2cm]
      Das ist nach Definition der Relation $R$ \"{a}quivalent zu der Formel 
      \\[0.2cm]
      \hspace*{1.3cm}
      $\forall x \in M: \exists K \in \mathcal{P}: \bigl(x \in K \wedge x \in K\bigr)$
      \\[0.2cm]
      Das k\"{o}nnen wir sofort zu der Formel
      \\[0.2cm]
      \hspace*{1.3cm}
      $\forall x \in M: \exists K \in \mathcal{P}: x \in K$
      \\[0.2cm]
      vereinfachen.  Diese Formel ist nichts anderes als die Vollst\"{a}ndigkeit
      der Partition $\mathcal{P}$. $\surd$
\item Symmetrie: Zu zeigen ist 
      \\[0.2cm]
      \hspace*{1.3cm}
      $\forall x, y \in M:\bigl( x \mathop{R} y \rightarrow y \mathop{R} x\bigr)$.
      \\[0.2cm]
      Wir nehmen also an, dass 
      \\[0.2cm]
      \hspace*{1.3cm}
      $x \mathop{R} y$ 
      \\[0.2cm]
      gilt.  Nach Definition
      der Relation $R$ ist das \"{a}quivalent zu 
      \\[0.2cm]
      \hspace*{1.3cm}
      $\exists K \in \mathcal{P}: \bigl(x \in K \wedge y \in K\bigr)$.
      \\[0.2cm]
      Diese Formel ist offenbar \"{a}quivalent zu 
      \\[0.2cm]
      \hspace*{1.3cm}
      $\exists K \in \mathcal{P}: \bigl(y \in K \wedge x \in K\bigr)$
      \\[0.2cm]
      und nach Definition der Relation $R$ folgt nun 
      \\[0.2cm]
      \hspace*{1.3cm}
      $y \mathop{R} x$. $\surd$
\item Transitivit\"{a}t: Zu zeigen ist 
      \\[0.2cm]
      \hspace*{1.3cm}
      $\forall x,y,z \in M:\bigl( x \mathop{R} y \wedge y \mathop{R} z \rightarrow x \mathop{R} z\bigr)$.
      \\[0.2cm]
      Wir nehmen also an, dass 
      \\[0.2cm]
      \hspace*{1.3cm}
      $x \mathop{R} y \wedge y \mathop{R} z$
      \\[0.2cm]
      gilt.  Das ist nach Definition der Relation $R$ \"{a}quivalent zu
      \\[0.2cm]
      \hspace*{1.3cm}
      $\exists K \in \mathcal{P}: \bigl(x \in K \wedge y \in K\bigr) \wedge 
       \exists L \in \mathcal{P}: \bigl(y \in L \wedge z \in L\bigr)$.
      \\[0.2cm]
      Dann gibt es aber offenbar zwei Mengen $K,L\in\mathcal{P}$, so dass
      \\[0.2cm]
      \hspace*{1.3cm}
      $x \in K \wedge y \in K \cap L \wedge z \in L$
      \\[0.2cm]
      gilt.  Damit ist $K \cap L \not= \emptyset$ und aus der 
      Separations-Eigenschaft der Partition $\mathcal{P}$ folgt 
      \\[0.2cm]
      \hspace*{1.3cm}
      $K = L$.
      \\[0.2cm]
      Damit haben wir 
      \\[0.2cm]
      \hspace*{1.3cm}
      $\exists K \in \mathcal{P}: \bigl(x \in K \wedge z \in K\bigr)$
      \\[0.2cm]
      gezeigt und nach Definition der Relation $R$ hei\3t das 
      $x \mathop{R} z$. $\surd$
      \qed
\end{enumerate}



\section{Partielle und totale Ordnungen}
In dem letzten Abschnitt dieses Kapitels definieren wir Ordnungs-Relationen.  Darunter verstehen wir
Relationen, die sich \"{a}hnlich verhalten wie die Relation $\leq$ auf den Zahlen.

\begin{Definition}[partielle Ordnung]
  Eine Relation $R \subseteq M \times M$  ist eine 
  \emph{partielle Ordnung (im Sinne von $\leq$) auf $M$} genau dann, wenn die Relation $R$
  \begin{enumerate}
  \item reflexiv auf $M$,
  \item anti-symmetrisch und
  \item transitiv ist.
  \end{enumerate}
  Die Relation ist dar\"{u}ber hinaus eine \emph{totale Ordnung auf $M$}, wenn zus\"{a}tzlich
  \\[0.2cm]
  \hspace*{1.3cm} $\forall x \el M : \forall y \el M :\bigl( x\mathop{R}y \vee y \mathop{R} x\bigr)$
  \\[0.2cm]
  gilt. \eox
\end{Definition}

\remark
Eine totale Ordnung wird gelegentlich auch als \emph{lineare Ordnung} bezeichnet. \eox

\example 
Die Teilbarkeitsrelation $\mathop{\mathtt{div}}$ kann auf den nat\"{u}rlichen Zahlen
wie folgt definiert werden 
\\[0.2cm]
\hspace*{1.3cm}
$ \mathop{\mathtt{div}} := 
   \bigl\{ \pair(x,y) \in \mathbb{N} \times \mathbb{N} \mid \exists k \in \mathbb{N}: k \cdot x = y\bigr\}$.
\\[0.2cm] 
Wir zeigen, dass diese Relation eine partielle Ordnung auf $\mathbb{N}$ ist und weisen dazu
Reflexivit\"{a}t, Anti-Symmetrie und Transitivit\"{a}t nach.
\begin{enumerate}
\item Reflexivit\"{a}t: Zu zeigen ist 
      \\[0.2cm]
      \hspace*{1.3cm}
      $\forall x \in \mathbb{N}: x \mathop{\mathtt{div}} x$.
      \\[0.2cm]
      Nach Definition der Relation $\mathop{\mathtt{div}}$ ist  das \"{a}quivalent zu
      \\[0.2cm]
      \hspace*{1.3cm}
      $\exists k \in \mathbb{N}: k \cdot x = x$. 
      \\[0.2cm]
      Wir m\"{u}ssen also ein $k$ finden, so dass die Gleichung $k \cdot x = x$ gilt.
      Setzen wir $k=1$, so ist diese Gleichung sicher erf\"{u}llt und damit ist die Reflexivit\"{a}t
      gezeigt. $\surd$
\item Anti-Symmetrie: Zu zeigen ist 
      \\[0.2cm]
      \hspace*{1.3cm}
      $\forall x, y \in \mathbb{N}:\bigl( x \mathop{\mathtt{div}} y \wedge y \mathop{\mathtt{div}} x \rightarrow x = y\bigr)$
      \\[0.2cm] 
      Wir nehmen also an, dass 
      \\[0.2cm]
      \hspace*{1.3cm}
      $x \mathop{\mathtt{div}} y \wedge y \mathop{\mathtt{div}} x$
      \\[0.2cm]
      gilt (und werden $x=y$ zeigen).  Nach Definition der Relation $\mathop{\mathtt{div}}$
      ist die Annahme \"{a}quivalent zu 
      \\[0.2cm]
      \hspace*{1.3cm}
      $\bigl(\exists k_1 \in \mathbb{N}: k_1 \cdot x = y \bigr) \wedge
       \bigl(\exists k_2 \in \mathbb{N}: k_2 \cdot y = x \bigr)$ 
      \\[0.2cm]
      Also gibt es nat\"{u}rliche Zahlen $k_1$ und $k_2$, so dass 
      \\[0.2cm]
      \hspace*{1.3cm}
      $k_1 \cdot x = y \wedge k_2 \cdot y = x$
      \\[0.2cm]
      gilt.  Setzen wir diese Gleichungen ineinander ein, so erhalten wir 
      \\[0.2cm]
      \hspace*{1.3cm}
      $k_1 \cdot k_2 \cdot y = y$ \quad und \quad
      $k_2 \cdot k_1 \cdot x = x$.
      \\[0.2cm] 
      Da $x$ und $y$ als nat\"{u}rliche Zahlen von $0$ verschieden sind, muss dann
      \\[0.2cm]
      \hspace*{1.3cm}
      $k_1 \cdot k_2 = 1$
      \\[0.2cm]
      gelten.  Da aus $k_1 \cdot k_2 = 1$ sofort $k_1 = 1$ und $k_2 = 1$ folgt, denn auch $k_1$ und
      $k_2$ sind ja nat\"{u}rliche Zahlen, 
      k\"{o}nnen wir wegen der urspr\"{u}nglichen Gleichungen $k_1 \cdot x = y$ und $k_2 \cdot y = x$
      sofort auf $x = y$ schlie\3en.  Damit ist die Anti-Symmetrie gezeigt. $\surd$
\item Transitivit\"{a}t: Zu zeigen ist 
      \\[0.2cm]
      \hspace*{1.3cm}
      $\forall x, y, z \in \mathbb{N}:\bigl( x \mathop{\mathtt{div}} y \wedge y \mathop{\mathtt{div}} z \rightarrow x \mathop{\mathtt{div}} z\bigr)$
      \\[0.2cm] 
      Wir nehmen also an, dass 
      \\[0.2cm]
      \hspace*{1.3cm}
      $x \mathop{\mathtt{div}} y \wedge y \mathop{\mathtt{div}} z$
      \\[0.2cm]
      gilt (und werden $x \mathop{\mathtt{div}} z$ zeigen).  Nach Definition der Relation $\mathop{\mathtt{div}}$
      ist die Annahme \"{a}quivalent zu 
      \\[0.2cm]
      \hspace*{1.3cm}
      $\bigl(\exists k_1 \in \mathbb{N}: k_1 \cdot x = y \bigr) \wedge
       \bigl(\exists k_2 \in \mathbb{N}: k_2 \cdot y = z \bigr)$ 
      \\[0.2cm]
      Also gibt es nat\"{u}rliche Zahlen $k_1$ und $k_2$, so dass 
      \\[0.2cm]
      \hspace*{1.3cm}
      $k_1 \cdot x = y \wedge k_2 \cdot y = z$
      \\[0.2cm]
      gilt.  Setzen wir die erste Gleichung in die zweite  ein, so erhalten wir 
      \\[0.2cm]
      \hspace*{1.3cm}
      $k_2 \cdot k_1 \cdot x = z$.
      \\[0.2cm] 
      Setzen wir $k_3 := k_2 \cdot k_1$, so haben wir also $k_3 \cdot x = z$
      und das zeigt 
      \\[0.2cm]
      \hspace*{1.3cm}
      $x \mathop{\mathtt{div}} z$.
      \\[0.2cm]
      Damit haben wir die Transitivit\"{a}t nachgewiesen.  $\surd$
\end{enumerate}
Die Relation $\mathtt{div}$ ist keine totale Ordnung, denn beispielsweise gilt weder
$2 \mathop{\mathtt{div}} 3$ noch $3 \mathop{\mathtt{div}} 2$.  \exend
\pagebreak

\exercise
Auf der Menge der nat\"{u}rlichen Zahlen $\mathbb{N}$ definieren wir die Relation $\leq$ 
wie folgt: 
\\[0.2cm]
\hspace*{1.3cm}
$\leq := \bigl\{ \pair(x,y) \in \mathbb{N} \times \mathbb{N} \mid \exists k \in \mathbb{N}_0: x + k = y \bigr\}$.
\\[0.2cm]
Zeigen Sie, dass die Relation $\leq$ eine totale Ordnung auf $\mathbb{N}$ ist.
\exend

\exercise
Auf der Potenz-Menge der nat\"{u}rlichen Zahlen definieren wir die Relation
$\subseteq$ als 
\\[0.2cm]
\hspace*{1.3cm}
$\subseteq := 
\bigl\{ \pair(A,B) \in 2^\mathbb{N} \times 2^\mathbb{N}\mid \exists C \in 2^\mathbb{N}: A \cup C = B \bigr\}$
\\[0.2cm]
Zeigen Sie, dass die Relation $\subseteq$ auf $2^\mathbb{N}$ zwar eine partielle, aber keine
totale Ordnung ist.
\exend


\exercise
Auf der Menge $\mathbb{N} \times \mathbb{N}$ definieren wir die Relation $\sqsubseteq$ durch die Festlegung
\\[0.2cm]
\hspace*{1.3cm}
$\pair(x_1, y_1) \sqsubseteq \pair(x_2, y_2)$ \quad g.d.w. \quad 
$x_1 < x_2 \vee (x_1 = x_2 \wedge y_1 \leq y_2)$.  
\\[0.2cm]
Zeigen Sie, dass $\sqsubseteq$ eine totale Ordnung auf $\mathbb{N} \times \mathbb{N}$ ist.
\exend
\vspace*{0.3cm}

\noindent
Wir schlie\3en damit den theoretischen Teil unseres Ausflugs in die Mengenlehre und verweisen f\"{u}r weitere
Details auf die Literatur, wobei ich Ihnen hier besonders das Buch von Seymour Lipschutz
\cite{lipschutz:1998} empfehlen m\"{o}chte.  

%%% Local Variables: 
%%% mode: latex
%%% TeX-master: "lineare-algebra"
%%% End: 
