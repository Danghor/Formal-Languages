\documentclass{article}
\usepackage[latin1]{inputenc}
\usepackage{german}
\usepackage{fancyvrb}
\usepackage{amssymb}
\usepackage{a4wide}

\newcommand{\qote}[1]{``\texttt{#1}''}
\newcommand{\quoted}[1]{\;\mbox{\symbol{34}\texttt{#1}\symbol{34}}\;}

\newcounter{aufgabe}
\newcommand{\exercise}{\vspace*{0.3cm}
\stepcounter{aufgabe}

\noindent
\textbf{Aufgabe \arabic{aufgabe}}: }

\renewcommand{\labelenumi}{(\alph{enumi})}
\renewcommand{\labelenumii}{\arabic{enumii}.}

\begin{document}


\noindent
{\large Aufgaben zu den Vorlesungen  ``Theoretische Informatik \texttt{I$\!$I$\!$I}'' und ``{\sl Compilerbau}''}
\vspace{0.5cm}

\exercise
Ein Kassenzettel bestehe aus Zeilen, die in etwa wie folgt aussehen:
\begin{tabbing}
\hspace*{1.3cm} \= \textsl{Tullamore Dew}: \= 91.50 Euro \\
                \> \textsl{Jim Beam}:      \> 17.34 Franken \\
                \> \textsl{Stroh Rum}:     \> 22.33 Euro
\end{tabbing}
Die Preise sind also teilweise in Euro, teilweise
in Schweizer Franken angegeben.  
Entwickeln Sie ein \textsl{JFlex}-Programm, dass einen Kassenzettel einliest und 
anschlie{\ss}end den Gesamtpreis in Euro ausgibt.  
Nehmen Sie dabei an, dass ein Euro einen Wert von 1,20 Franken hat.

\exercise
\begin{enumerate}
\item Geben Sie einen deterministischen endlichen Automaten $F$ an, so dass $L(F)$ aus genau den
      W\"ortern der Sprache $\{a,b\}^*$ besteht,  bei denen auf jedes \qote{a} mindestens ein \qote{b} folgt.
\item Geben Sie einen regul\"aren Ausdruck an, der diese Sprache beschreibt.
\item Berechnen Sie, ausgehend von dem unter (b) angegebenen regul\"aren Ausdruck,
      einen nicht-deterministischen endlichen Automaten, der dieselbe Sprache erkennt.

      Benutzen Sie dabei das in der Vorlesung diskutierte Verfahren.
\item \"Uberf\"uhren Sie den unter (c) angegebenen Automaten in einen
      deterministischen endlichen Automaten.

      Benutzen Sie dabei das in der Vorlesung diskutierte Verfahren.
\item Minimieren Sie die Anzahl der Zust\"ande des in (d) berechneten Automaten.

      Benutzen Sie dabei das in der Vorlesung diskutierte Verfahren.
\end{enumerate}

\exercise
Die Sprache $L_{P}$ beinhaltet alle W\"orter aus der Sprache $\{ \texttt{a}, \texttt{b} \}^*$,
die Palindrome sind.  Definieren wir f\"ur einen String $w = c_1 c_2 \cdots c_{n-1} c_n$ den
String $w_r$ (gelesen: \emph{w r\"uckw\"arts}) durch
\\[0.2cm]
\hspace*{1.3cm}
$w^r := c_n c_{n-1} \cdots c_2 c_1$,
\\[0.2cm]
so kann die Sprache $L_P$ durch 
\\[0.2cm]
\hspace*{1.3cm}
$L_{P} := \bigl\{ w \in \{ \texttt{a}, \texttt{b} \}^* \mid w = w^r \bigr\}$
\\[0.2cm]
definiert werden.
\begin{enumerate}
\item Zeigen Sie mit Hilfe des Pumping-Lemmas f\"ur regul\"are Sprachen, dass die Sprache $L_{P}$
      der Palindrome keine regul\"are Sprache ist.
\item \"Uberpr\"ufen Sie, ob die Sprache $L_P$ eine kontextfreie Sprache ist und beweisen Sie
      Ihre Behauptung.
\end{enumerate}
\pagebreak 

\exercise
Es sei $\Sigma = \{ \texttt{a}, \texttt{b}, \mathtt{c} \}$.  \begin{enumerate}
\item Die Sprache $L_1$ werde durch die Gleichung
      \\[0.2cm]
      \hspace*{1.3cm}
      $L_1 :=  \{ \texttt{a}^k \texttt{b}^l \texttt{c}^{l + k} \in \Sigma^* \mid  k,l \in \mathbb{N} \}$
      \\[0.2cm] 
      definiert. \"Uberpr\"ufen Sie, ob die Sprache $L_1$ regul\"ar ist und beweisen Sie Ihre Behauptung.
\item \"Uberpr\"ufen Sie, ob die oben definierte Sprache $L_1$ kontextfrei ist und beweisen Sie Ihre Behauptung.
\item Die Sprache $L_2$ werde durch die Gleichung
      \\[0.2cm]
      \hspace*{1.3cm}
      $L_2 :=  \{ \texttt{a}^k \texttt{b}^l \texttt{c}^{l \cdot k} \in \Sigma^* \mid  k,l \in \mathbb{N} \}$
      \\[0.2cm] 
      definiert. \"Uberpr\"ufen Sie, ob die Sprache $L_2$ regul\"ar ist und beweisen Sie Ihre Behauptung.
\item \"Uberpr\"ufen Sie, ob die oben definierte Sprache $L_2$ kontextfrei ist und beweisen Sie Ihre Behauptung.
\end{enumerate}


\exercise
Pr\"afix-Ausdr\"ucke sind Ausdr\"ucke, in denen die arithmetischen Operatoren \qote{+}, \qote{-},
\qote{*} und \qote{/} als zweistellige Pr\"afix-Operatoren benutzt werden.  Au{\ss}er den
genannten Operatoren sollen die Ausdr\"ucke nur nat\"urliche Zahlen enthalten.  Die Operatoren stehen
immer vor ihren Argumenten, die nat\"urlich selber komplexe Pr\"afix-Ausdr\"ucke sein k\"onnen.
Beispielsweise ist
\\[0.2cm]
\hspace*{1.3cm}
$+ + 2 * 3\; 4 - 1\; 2$
\\[0.2cm]
ein Pr\"afix-Ausdruck, der den selben Wert hat wie der Term
\\[0.2cm]
\hspace*{1.3cm}
$\bigl(2 + (3 * 4)) + (1 - 2)$.
\begin{enumerate}
\item Geben Sie eine \textbf{Ebnf}-Grammatik f\"ur Pr\"afix-Ausdr\"ucke an.
\item Geben Sie eine \textsc{Antlr}-Grammatik an, mit deren Hilfe Sie einen
      Pr\"afix-Ausdruck auswerten k\"onnen.  
\item Geben Sie eine \textsc{JavaCup}-Grammatik an, mit deren Hilfe Sie einen
      Pr\"afix-Ausdruck auswerten k\"onnen.  Zus\"atzlich sollen Sie mit Hilfe von \textsl{JFlex} einen
      Scanner entwickeln, der die in der \textsc{JavaCup}-Grammatik verwendeten Nicht-Terminale
      erzeugt.
\end{enumerate}

\exercise
Betrachten Sie die folgende Sprache:
\[ L = \{ a^k b^l \mid k,l \in \mathbb{N} \wedge k \not= l \} \]
\begin{enumerate}
\item Zeigen Sie, dass diese Sprache nicht regul\"ar ist.

      \textbf{Hinweis}: Benutzen Sie das Theorem von Nerode.
\item Zeigen Sie, dass diese Sprache kontextfrei ist.
\end{enumerate}
\pagebreak

\exercise
Die Grammatik $G = \langle \{ A, B \}, \{ \quoted{u}, \quoted{x}, \quoted{y}, \quoted{z} \}, R, A \rangle$
habe  die folgenden Regeln:
\\[0.2cm]
\hspace*{1.3cm}
$\begin{array}[t]{lcl}
  A & \rightarrow & B \quoted{x} \\
    & \mid        & \quoted{y} B \quoted{z} \\
    & \mid        & \quoted{u} \quoted{z} \\
    & \mid        & \quoted{y} \quoted{u} \quoted{x} \\[0.1cm]
  B & \rightarrow &\quoted{u} 
 \end{array}
$
\\[0.2cm]
Bearbeiten Sie die folgenden Teilaufgaben:
\begin{enumerate}
\item \"Uberpr\"ufen Sie, ob diese Grammatik eine LL(1)-Grammatik ist und begr\"unden Sie
      Ihre Antwort.
\item \"Uberpr\"ufen Sie, ob diese Grammatik eine LR-Grammatik ist und begr\"unden Sie
      Ihre Antwort.
\end{enumerate}


%\exercise
%In einer hypothetischen Programmiersprache bestehen die Ausdr\"ucke aus nat\"urlichen Zahlen,
%Variablen und bin\"aren Infix-Operatoren der Form ``\texttt{\#}$_k$'' mit 
%$k \in \{1, \cdots, n \}$.  Die Zahl $n$ ist hier eine beliebige vorgegebene nat\"urliche Zahl.
%Die Zahl $k$ gibt die Pr\"azedenz des Operators ``\texttt{\#}$_k$''
%an:  Je gr\"o{\ss}er $k$ ist, um so st\"arker bindet der Operator.  Au{\ss}erdem wollen wir
%voraussetzen, dass die Operatoren  ``\texttt{\#}$_k$'' alle links-assoziativ sind.
%\vspace{0.2cm}

%\noindent
%Geben Sie eine Grammatik an, mit der sich die Struktur von Ausdr\"ucken der oben
%beschriebenen Art eindeutig beschreiben l\"asst.


\exercise
Die Grammatik $G = \langle \{ S \}, \{ \mathtt{+}, \mathtt{-}, \mathtt{a} \}, R, S \rangle$
habe  die folgenden Regeln:
\[ S \rightarrow S\, S\, \mathtt{+} \mid S\, S\, \mathtt{-} \mid \mathtt{a}. \]
\begin{enumerate}
\item Berechnen Sie die Mengen $\textsl{First}(S)$ und $\textsl{Follow}(S)$.
\item Berechnen Sie die Menge der SLR-Zust\"ande f\"ur diese Grammatik.
\item Berechnen Sie die Funktionen $\textsl{action}()$ und $\textsl{goto}()$ f\"ur diese Grammatik und
      stellen Sie das Ergebnis in einer Tabelle dar.
\item Untersuchen Sie, ob diese Grammatik mehrdeutig ist.
\end{enumerate}

\exercise
Nehmen Sie an, dass die im Skript eingef\"uhrte Sprache \textsl{Integer}-\texttt{C} um eine 
\texttt{do-while}-Schleife erweitert werden soll, deren Syntax durch die folgende Grammatik-Regel gegeben ist:
\\[0.2cm]
\hspace*{1.3cm}
$\textsl{statement} \rightarrow \quoted{do}\; \textsl{statement}\; \quoted{while}\;
 \quoted{(} \;\textsl{boolExpr} \;\quoted{)}$.
\\[0.2cm]
Die Semantik dieses Konstruktes soll mit der Semantik des entsprechenden Konstruktes in
der Sprache \texttt{C} \"ubereinstimmen.
Geben Sie eine Gleichung an, die beschreibt, wie eine \texttt{do-while}-Schleife in
\textsl{Java-Assembler} \"ubersetzt werden kann.

\end{document}

%%% Local Variables: 
%%% mode: latex
%%% TeX-master: t
%%% End: 
