\documentclass{article}
\usepackage[latin1]{inputenc}
\usepackage{amssymb}
\usepackage{fancyvrb}
\usepackage{german}
\usepackage{a4wide}
\usepackage{epsfig}
\usepackage{fleqn}

\pagestyle{plain}

\newcounter{aufgabe}

\renewcommand{\labelenumi}{(\alph{enumi})}
\newcommand{\quoted}[1]{\;\mbox{``\texttt{#1}''}\;}

\newcommand{\exercise}{\vspace*{0.1cm}

\stepcounter{aufgabe}
\vspace*{0.3cm}

\noindent
\textbf{Aufgabe \arabic{aufgabe}}: }

\newcommand{\solution}{
\vspace*{0.3cm}

\noindent
\textbf{L\"osung}: }


\begin{document}


\noindent
{\large  Aufgaben zur Klausurvorbereitung f\"ur die Vorlesung ``{\sl Compilerbau}''}
\vspace{0.5cm}

\noindent
\textbf{Hinweis}: Bei der Klausur m\"ussen alle Aufgabenbl\"ater mit abgegeben werden, sonst
ist die Klausur ung\"ultig!
\vspace{0.5cm}

\exercise
Die Grammatik $G = \langle \{ E \}, \{ \quoted{+}, \quoted{*}, \quoted{x} \}, R, E \rangle$
habe  die folgenden Regeln:
\[ E \rightarrow E\, E\, \quoted{+}\, \mid E\, E\, \quoted{*}\, \mid \quoted{x}. \]
Wenden Sie den Algorithmus von Earley auf den String ``\texttt{xx+x*}'' an.

\exercise
Die Grammatik $G = \langle \{ S \}, \{ \mathtt{+}, \mathtt{-}, \mathtt{a} \}, R, S \rangle$
habe  die folgenden Regeln:
\[ S \rightarrow S\, S\, \mathtt{+} \mid S\, S\, \mathtt{-} \mid \mathtt{a}. \]
\begin{enumerate}
\item Berechnen Sie die Mengen $\textsl{First}(S)$ und $\textsl{Follow}(S)$.
\item Berechnen Sie die Menge der SLR-Zust\"ande f\"ur diese Grammatik.
\item Berechnen Sie die Funktionen $\textsl{action}()$ und $\textsl{goto}()$ f\"ur diese Grammatik.
\item Berechnen Sie die Menge der LR-Zust\"ande f\"ur diese Grammatik.
\item Untersuchen Sie, ob diese Grammatik mehrdeutig ist.
\end{enumerate}

\exercise
Die Grammatik $G = \langle \{ A, B \}, \{ \quoted{u}, \quoted{x}, \quoted{y}, \quoted{z} \}, R, A \rangle$
habe  die folgenden Regeln:
\\[0.2cm]
\hspace*{1.3cm}
$\begin{array}[t]{lcl}
  A & \rightarrow & B \quoted{x} \\
    & \mid        & \quoted{y} B \quoted{z} \\
    & \mid        & \quoted{u} \quoted{z} \\
    & \mid        & \quoted{y} \quoted{u} \quoted{x} \\[0.1cm]
  B & \rightarrow &\quoted{u} 
 \end{array}
$
\\[0.2cm]
Bearbeiten Sie die folgenden Teilaufgaben:
\begin{enumerate}
\item \"Uberpr\"ufen Sie, ob die  diese Grammatik eine LL(1)-Grammatik ist und begr\"unden Sie
      Ihre Antwort.
\item \"Uberpr\"ufen Sie, ob die  diese Grammatik eine LL($*$)-Grammatik ist und begr\"unden Sie
      Ihre Antwort.
\item \"Uberpr\"ufen Sie, ob die  diese Grammatik eine SLR-Grammatik ist und begr\"unden Sie
      Ihre Antwort.
\end{enumerate}
\pagebreak

\exercise
Wir definieren \emph{geschachtelte Listen} rekursiv als solche Listen, 
deren Elemente nat\"urliche Zahlen oder geschachtelte Listen sind.  Die Elemente in
geschachtelten Listen sollen durch Kommata getrennt werden und die Listen selber sollen
durch die eckigen Klammern ``\texttt{[}'' und ``\texttt{]}'' begrenzt sein.
Beispiele f\"ur geschachtelte Listen sind also:
\begin{enumerate}
\item \texttt{[ 1, [ 1, [], [ 2, 3]], 7]}
\item \texttt{[ [], [ [], [], [ 4], 5], []]}
\end{enumerate}
L\"osen Sie die folgenden Teilaufgaben:
\begin{enumerate}
\item Geben Sie eine Grammatik f\"ur geschachtelte Listen an.
\item Definieren Sie geeignete Klassen, mit deren Hilfe Sie geschachtelte Listen 
      repr\"asentieren k\"onnen.
\item Geben Sie einen \textsl{JavaCup}-Parser an, der eine geschachtelte Liste einliest
      und einen abstrakten Syntax-Baum der Liste berechnet.
\end{enumerate}


\exercise
Der Typ $\texttt{list}(T)$ sei wie folgt definiert:
\\[0.2cm]
\hspace*{1.3cm}
\texttt{type list(X) := nil + cons(X, list(X));}
\\[0.2cm]
Die Funktion \texttt{addLast} habe die folgende Signatur:
\\[0.2cm]
\hspace*{1.3cm}
\texttt{signature addLast: list(T) * T -> list(T);}
\\[0.2cm]
und die Variablen \texttt{x} und \texttt{z} haben den Typ \texttt{int}.
\begin{enumerate}
\item Berechnen Sie 
      \\[0.2cm]
      \hspace*{1.3cm}
      $\textsl{typeEqs}\bigl(\texttt{addLast(cons(x, nil), z): list(int)}\bigr)$.
\item L\"osen Sie die in Teil (a) berechneten Typ-Gleichungen.
\end{enumerate}

\exercise
Nehmen Sie an, dass die im Skript eingef\"uhrte Sprache \textsl{Integer}-\texttt{C} um eine 
\texttt{do-while}-Schleife erweitert werden soll, deren Syntax durch die folgende Grammatik-Regel gegeben ist:
\\[0.2cm]
\hspace*{1.3cm}
$\textsl{statement} \rightarrow \quoted{do}\; \textsl{statement}\; \quoted{while}\;
 \quoted{(} \;\textsl{boolExpr} \;\quoted{)}$.
\\[0.2cm]
Die Semantik dieses Konstruktes soll mit der Semantik des entsprechenden Konstruktes in
der Sprache \texttt{C} \"ubereinstimmen.
\begin{enumerate}
\item Geben Sie eine Gleichung an, die beschreibt, wie eine \texttt{do-while}-Schleife in
      \textsl{Java-Byte-Code} \"ubersetzt werden kann.
\item Geben Sie die Methode $\textsl{compile}()$ an, die das entsprechende
      Konstrukt \"ubersetzt.  Gehen Sie dabei davon aus, dass Sie diese Methode innerhalb
      einer Klasse \texttt{DoWhile} implementieren, wobei diese Klasse f\"ur \textsc{Ep}
      wie folgt spezifiziert ist:
      \\[0.2cm]
      \hspace*{1.3cm}
      $\texttt{Statement} \;=\; \cdots \;+\; \texttt{DoWhile}(\texttt{Statement}\; stmnt,\;
      \mathtt{BoolExpr}\; cond)\; +\; \cdots; $      
\end{enumerate}



\end{document}

%%% Local Variables: 
%%% mode: latex
%%% TeX-master: t
%%% End: 
