\documentclass{article}
\usepackage[latin1]{inputenc}
\usepackage{amssymb}
\usepackage{fancyvrb}
\usepackage{german}
\usepackage{a4wide}
\usepackage{epsfig}
\usepackage{fleqn}

\pagestyle{plain}

\def\pair(#1,#2){\langle #1, #2 \rangle}

\newcounter{aufgabe}

\renewcommand{\labelenumi}{(\alph{enumi})}
\renewcommand{\labelenumii}{\arabic{enumii}.}

\newcommand{\exercise}{\vspace*{0.1cm}

\stepcounter{aufgabe}
\vspace*{0.3cm}

\noindent
\textbf{Aufgabe \arabic{aufgabe}}: }

\newcommand{\solution}{
\vspace*{0.3cm}

\noindent
\textbf{L�sung}: }
\newcommand{\quoted}[1]{\mbox{``\texttt{#1}''}}
\def\pair(#1,#2){\langle #1, #2 \rangle}


\begin{document}


\noindent
{\large  L�sungen zu den Aufgaben zur Klausurvorbereitung}
\vspace{0.5cm}

\exercise
Die Grammatik $G = \langle \{ E \}, \{ \quoted{+}, \quoted{*}, \quoted{x} \}, R, E \rangle$
habe  die folgenden Regeln:
\[ E \rightarrow E\, E\, \quoted{+}\, \mid E\, E\, \quoted{*}\, \mid \quoted{x}. \]
Wenden Sie den Algorithmus von Earley auf den String ``\texttt{xx+x*}'' an.

\solution
Wir definieren  $s := \mbox{``\texttt{xx+x*}''}$.
Der Earley-Algorithmus berechnet die folgenden Mengen:
\begin{enumerate}
\item $Q_0 = \{
       \begin{array}[t]{l}
         \pair(\widehat{S} \rightarrow \bullet\; E, 0), \\
         \pair(E \rightarrow \bullet \;E\; E\; \quoted{+}, 0), \\
         \pair(E \rightarrow \bullet \;E\; E\; \quoted{*}, 0), \\
         \pair(E \rightarrow \bullet \quoted{x}, 0)\;  \}
       \end{array}
       $
\item Wegen $s[1] = \quoted{x}$ gilt:

       $Q_1 = \{
       \begin{array}[t]{l}
         \pair(E \rightarrow  \quoted{x}\bullet, 0), \\
         \pair(\widehat{S} \rightarrow  E\;\bullet, 0), \\
         \pair(E \rightarrow  E\bullet E\; \quoted{+}, 0), \\
         \pair(E \rightarrow  E\bullet E\; \quoted{*}, 0), \\
         \pair(E \rightarrow \bullet\; E\; E\; \quoted{+}, 1), \\
         \pair(E \rightarrow \bullet\; E\; E\; \quoted{*}, 1), \\
         \pair(E \rightarrow \bullet \quoted{x}, 1)\; \}
       \end{array}
       $
\item Wegen $s[2] = \quoted{x}$ gilt:

      $Q_2 = \{
       \begin{array}[t]{l}    
         \pair(E \rightarrow  \quoted{x}\bullet, 1), \\
         \pair(E \rightarrow  E\; E\bullet \quoted{*}, 0), \\
         \pair(E \rightarrow  E\; E\bullet \quoted{+}, 0), \\
         \pair(E \rightarrow  E\bullet E\; \quoted{*}, 1), \\
         \pair(E \rightarrow  E\bullet E\; \quoted{+}, 1), \\
         \pair(E \rightarrow \bullet\; E\; E\; \quoted{*}, 2), \\
         \pair(E \rightarrow \bullet\; E\; E\; \quoted{+}, 2), \\
         \pair(E \rightarrow \bullet \quoted{x}, 2)\; \}
       \end{array}
       $
\item Wegen $s[3] = \quoted{+}$ gilt:

      $Q_3 = \{
       \begin{array}[t]{l} 
         \pair(E \rightarrow  E\; E\; \quoted{+}\bullet, 0), \\
         \pair(\widehat{S} \rightarrow  E\;\bullet, 0), \\
         \pair(E \rightarrow  E\bullet E\; \quoted{*}, 0), \\
         \pair(E \rightarrow  E\bullet E\; \quoted{+}, 0), \\
         \pair(E \rightarrow \bullet\; E\; E\; \quoted{*}, 3), \\
         \pair(E \rightarrow \bullet\; E\; E\; \quoted{+}, 3), \\
         \pair(E \rightarrow \bullet \quoted{x}, 3)\; \}
       \end{array}
      $
\item Wegen $s[4] = \quoted{x}$ gilt:

      $Q_4 = \{
       \begin{array}[t]{l}
         \pair(E \rightarrow  \quoted{x}\bullet, 3), \\
         \pair(E \rightarrow  E\; E\bullet \quoted{*}, 0), \\
         \pair(E \rightarrow  E\; E\bullet \quoted{+}, 0)\, \\
         \pair(E \rightarrow  E\bullet E\; \quoted{*}, 3), \\
         \pair(E \rightarrow  E\bullet E\; \quoted{+}, 3), \\
         \pair(E \rightarrow \bullet\; E\; E\; \quoted{*}, 4), \\
         \pair(E \rightarrow \bullet\; E\; E\; \quoted{+}, 4), \\
         \pair(E \rightarrow \bullet \quoted{x}, 4)\; \}
       \end{array}
      $

\item Wegen $s[5] = \quoted{*}$ gilt:

      $Q_5 = \{
       \begin{array}[t]{l}    
         \pair(E \rightarrow  E\; E\; \quoted{*}\bullet, 0), \\
         \pair(\widehat{S} \rightarrow  E\;\bullet, 0), \\
         \pair(E \rightarrow  E\bullet E\; \quoted{*}, 0), \\
         \pair(E \rightarrow  E\bullet E\; \quoted{+}, 0), \\
         \pair(E \rightarrow \bullet\; E\; E\; \quoted{*}, 5), \\
         \pair(E \rightarrow \bullet\; E\; E\; \quoted{+}, 5), \\
         \pair(E \rightarrow \bullet \quoted{x}, 5)\; \}
       \end{array}
      $

      Da dieser Zustand das Earley-Item $\pair(\widehat{S} \rightarrow  E\;\bullet, 0)$
      enth�lt, liegt der String $s$ in der von der Grammatik $G$ erzeugten Sprache.
      
\end{enumerate}

\exercise
Die Grammatik $G = \langle \{ E \}, \{ \quoted{+}, \quoted{-}, \quoted{a} \}, R, E \rangle$
habe  die folgenden Regeln:
\[ E \rightarrow E\, E\, \quoted{+} \mid E\, E\, \quoted{-} \mid \quoted{a}. \]
\begin{enumerate}
\item Berechnen Sie die Mengen $\textsl{First}(E)$ und $\textsl{Follow}(E)$.
\item Berechnen Sie die Menge der SLR-Zust�nde f�r diese Grammatik.
\item Berechnen Sie die Funktionen $\textsl{action}()$ und $\textsl{goto}()$ f�r diese Grammatik.
\item Berechnen Sie die Menge der LR-Zust�nde f�r diese Grammatik.
\item Untersuchen Sie, ob diese Grammatik mehrdeutig ist.
\end{enumerate}

\solution
\begin{enumerate}
\item Es gilt offenbar
      \\[0.2cm]
      \hspace*{1.3cm}
      $\textsl{First}(E) = \{ a \}$ \quad und \quad
      $\textsl{Follow}(E) = \{ \quoted{+}, \quoted{-}, \quoted{a}, \symbol{36} \}$.

      Bemerkung: Das Zeichen ``$\symbol{36}$'' liegt in der Menge $\textsl{Follow}(E)$, weil 
      $E$ das Start-Symbol der Grammatik ist.
\item Wir erhalten die folgenden Zust�nde:
      \begin{enumerate}
      \item Wir definieren $s_0 = \textsl{closure}\bigl(\{\widehat{S} \rightarrow \star E\}\bigr)$ und 
            finden
            \\[0.2cm]
            \hspace*{1.3cm}
            $s_0 = \{\; \widehat{S} \rightarrow \bullet\; E,\;
                      E \rightarrow \bullet \; E\; E \quoted{+},\; 
                      E \rightarrow \bullet \; E\; E \quoted{-},\;
                      E \rightarrow \bullet \quoted{a}\; 
            \}$.
      \item Wir definieren $s_1 = \textsl{goto}(s_0, E)$ und finden
            \\[0.2cm]
            \hspace*{1.3cm}
            $
            \begin{array}[t]{lcl}
            s_1 \;= & \{ & \widehat{S} \rightarrow E\; \bullet,             \\
                    &    & E \rightarrow E \bullet E \quoted{+},            \\
                    &    & E \rightarrow E \bullet E \quoted{-},            \\
                    &    & E \rightarrow \bullet\; E\, E \quoted{+},        \\
                    &    & E \rightarrow \bullet\; E\, E \quoted{-},        \\
                    &    & E \rightarrow \bullet \quoted{a}                \\
                    & \}.
            \end{array}
            $

      \item Wir definieren $s_2 = \textsl{goto}(s_1, E)$ und finden
            \\[0.2cm]
            \hspace*{1.3cm}
            $
            \begin{array}[t]{lcl}
            s_2 \;= & \{ & E \rightarrow E\; E \bullet \quoted{+},  \\
                    &    & E \rightarrow E\; E \bullet \quoted{-},   \\
                    &    & E \rightarrow E \bullet E \quoted{+},    \\
                    &    & E \rightarrow E \bullet E \quoted{-},    \\
                    &    & E \rightarrow \bullet\; E\; E \quoted{+},  \\
                    &    & E \rightarrow \bullet\; E\; E \quoted{-},  \\
                    &    & E \rightarrow \bullet \quoted{a}           \\
                    & \}.
            \end{array}
            $
      \item Wir definieren $s_3 = \textsl{goto}(s_2, \quoted{a})$ und finden
            \\[0.2cm]
            \hspace*{1.3cm}
            $s_3 = \{ E \rightarrow \quoted{a} \bullet \}$.
      \item Wir definieren $s_4 = \textsl{goto}(s_2, \quoted{+})$ und finden
            \\[0.2cm]
            \hspace*{1.3cm}
            $s_4 = \{ E \rightarrow E\, E \quoted{+} \bullet \}$.
      \item Wir definieren $s_5 = \textsl{goto}(s_2, \quoted{-})$ und finden
            \\[0.2cm]
            \hspace*{1.3cm}
            $s_5 = \{ E \rightarrow E \, E \quoted{-} \bullet \}$.
      \item Als n�chstes berechnen wir: 
        \begin{enumerate}
        \item $\textsl{goto}(s_0, \quoted{a}) = s_3$.
        \item $\textsl{goto}(s_1, \quoted{a}) = s_3$.
        \item $\textsl{goto}(s_2, E) = s_2$.
        \end{enumerate}
            

            Damit haben wir nun alle interessanten Werte der Funktion $\textsl{goto}()$ berechnet,
            denn f�r alle bisher nicht explizit angegebenen Werte liefert diese Funktion die leere Menge.
      \end{enumerate}

\item Damit erhalten wir f�r die Funktion $\textsl{action}()$ die folgende Tabelle:
      \begin{enumerate}
      \item $\textsl{action}(s_0, \quoted{a}) = \pair(\texttt{shift}, s_3)$
      \item $\textsl{action}(s_1, \quoted{\symbol{36}}) = \mathtt{accept}$
      \item $\textsl{action}(s_1, \quoted{a}) = \pair(\texttt{shift}, s_3)$
      \item $\textsl{action}(s_2, \quoted{+}) = \pair(\texttt{shift}, s_4)$
      \item $\textsl{action}(s_2, \quoted{-}) = \pair(\texttt{shift}, s_5)$
      \item $\textsl{action}(s_2, \quoted{a}) = \pair(\texttt{shift}, s_3)$
      \item $\textsl{action}(s_3, \quoted{\symbol{36}}) = \pair(\texttt{reduce}, E \rightarrow \quoted{a})$
      \item $\textsl{action}(s_3, \quoted{+}) = \pair(\texttt{reduce}, E \rightarrow \quoted{a})$
      \item $\textsl{action}(s_3, \quoted{-}) = \pair(\texttt{reduce}, E \rightarrow \quoted{a})$
      \item $\textsl{action}(s_3, \quoted{a}) = \pair(\texttt{reduce}, E \rightarrow \quoted{a})$
      \item $\textsl{action}(s_4, \quoted{\symbol{36}}) = \pair(\texttt{reduce}, E \rightarrow E\, E \quoted{+})$
      \item $\textsl{action}(s_4, \quoted{+}) = \pair(\texttt{reduce}, E \rightarrow E\, E \quoted{+})$
      \item $\textsl{action}(s_4, \quoted{-}) = \pair(\texttt{reduce}, E \rightarrow E\, E \quoted{+})$
      \item $\textsl{action}(s_4, \quoted{a}) = \pair(\texttt{reduce}, E \rightarrow E\, E \quoted{+})$
      \item $\textsl{action}(s_5, \quoted{\symbol{36}}) = \pair(\texttt{reduce}, E \rightarrow E\, E \quoted{-})$
      \item $\textsl{action}(s_5, \quoted{+}) = \pair(\texttt{reduce}, E \rightarrow E\, E \quoted{-})$
      \item $\textsl{action}(s_5, \quoted{-}) = \pair(\texttt{reduce}, E \rightarrow E\, E \quoted{-})$
      \item $\textsl{action}(s_5, \quoted{a}) = \pair(\texttt{reduce}, E \rightarrow E\, E \quoted{-})$
      \end{enumerate}
      F�r die Funktion $\textsl{goto}()$ finden wir:
      \begin{enumerate}
      \item $\textsl{goto}(s_0, E) = s_1$
      \item $\textsl{goto}(s_1, E) = s_2$
      \item $\textsl{goto}(s_2, E) = s_2$
      \end{enumerate}

\item Wir erhalten die folgenden Zust�nde:
      \begin{enumerate}
      \item Wir setzen wieder 
            $s_0 = \textsl{closure}\Bigl(\bigl\{\widehat{S} \rightarrow \bullet E: \symbol{36}\bigr\}\Bigr)$ 
            und erhalten diesmal
            \\[0.2cm]
            \hspace*{1.3cm}
            $
            \begin{array}[t]{lcl}  
            s_0 \;= & \bigl\{ & \widehat{S} \rightarrow \bullet\; E: \quoted{\symbol{36}},                  \\
                    &         & E \rightarrow \bullet\; E\; E \quoted{+}: \{\quoted{\symbol{36}}, \quoted{a}\}, \\
                    &         & E \rightarrow \bullet\; E\; E \quoted{-}: \{\quoted{\symbol{36}}, \quoted{a}\},  \\
                    &         & E \rightarrow \bullet \quoted{a}: \{\quoted{\symbol{36}}, \quoted{a}\}       \\
                    & \bigr\}.
            \end{array}
            $
      \item Wir definieren $s_1 = \textsl{goto}(s_0, E)$ und erhalten
            \\[0.2cm]
            \hspace*{1.3cm}
            $
            \begin{array}[t]{lcl} 
         s_1 \; = & \bigl\{ & \widehat{S} \rightarrow E \bullet: \quoted{\symbol{36}},    \\
                  &         & E \rightarrow E \bullet E \quoted{+}: \{\quoted{\symbol{36}}, \quoted{a}\},       \\
                  &         & E \rightarrow E \bullet E \quoted{-}: \{\quoted{\symbol{36}}, \quoted{a}\},       \\
                  &         & E \rightarrow \bullet\; E\; E \quoted{+}: \{\quoted{+}, \quoted{-}, \quoted{a}\}, \\
                  &         & E \rightarrow \bullet\; E\; E \quoted{-}: \{\quoted{+}, \quoted{-}, \quoted{a}\}, \\
                  &         & E \rightarrow \bullet \quoted{a}: \{\quoted{+}, \quoted{-}, \quoted{a}\}          \\
                  & \bigr\}.
            \end{array}
            $
       \item Wir definieren $s_2 = \textsl{goto}(s_1, E)$ und erhalten
             \\[0.2cm]
             \hspace*{1.3cm}
             $
             \begin{array}[t]{lcl} 
          s_2 \;= & \bigl\{ & E \rightarrow E\; E \bullet \quoted{+}: \{\quoted{\symbol{36}}, \quoted{a}\},   \\
                  &         & E \rightarrow E\; E \bullet \quoted{-}: \{\quoted{\symbol{36}}, \quoted{a}\},    \\
                  &         & E \rightarrow E \bullet E \quoted{+}: \{\quoted{+}, \quoted{-}, \quoted{a}\},   \\
                  &         & E \rightarrow E \bullet E \quoted{-}: \{\quoted{+}, \quoted{-}, \quoted{a}\},   \\
                  &         & E \rightarrow \bullet\; E\; E \quoted{+}: \{\quoted{+}, \quoted{-}, \quoted{a}\}, \\
                  &         & E \rightarrow \bullet\; E\; E \quoted{-}: \{\quoted{+}, \quoted{-}, \quoted{a}\}, \\
                  &         & E \rightarrow \bullet \quoted{a}: \{\quoted{+}, \quoted{-}, \quoted{a}\},       \\
                  & \bigr\}.
             \end{array}
             $
       \item Wir definieren $s_3 = \textsl{goto}(s_2, E)$ und erhalten
             \\[0.2cm]
             \hspace*{1.3cm}
             $
             \begin{array}[t]{lcl} 
          s_3 \;= & \bigl\{ & E \rightarrow E\; E \bullet \quoted{+}: \{\quoted{+}, \quoted{-}, \quoted{a}\},   \\
                  &         & E \rightarrow E\; E \bullet \quoted{-}: \{\quoted{+}, \quoted{-}, \quoted{a}\},   \\
                  &         & E \rightarrow E \bullet E \quoted{+}: \{\quoted{+}, \quoted{-}, \quoted{a}\},     \\
                  &         & E \rightarrow E \bullet E \quoted{-}: \{\quoted{+}, \quoted{-}, \quoted{a}\},     \\
                  &         & E \rightarrow \bullet\; E\; E \quoted{+}: \{\quoted{+}, \quoted{-}, \quoted{a}\}, \\
                  &         & E \rightarrow \bullet\; E\; E \quoted{-}: \{\quoted{+}, \quoted{-}, \quoted{a}\}, \\
                  &         & E \rightarrow \bullet \quoted{a}: \{\quoted{+}, \quoted{-}, \quoted{a}\}          \\
                  & \bigr\}.
             \end{array}
             $

             Beachten Sie, dass $s_2 \not= s_3$ ist, denn die Menge der Folge-Token sind f�r die markierten
             Regeln $ E \rightarrow E\; E \bullet \quoted{+}$ und $ E \rightarrow E\; E \bullet \quoted{-}$ 
             in den Mengen $s_2$ und $s_3$ unterschiedlich.
       \item Wir definieren $s_4 = \textsl{goto}(s_0, \quoted{a})$ und erhalten
             \\[0.2cm]
             \hspace*{1.3cm}
             $s_4 = \{ E \rightarrow \quoted{a} \bullet: \{\quoted{\symbol{36}}, \quoted{a}\} \}$.
       \item Wir definieren $s_5 = \textsl{goto}(s_2, \quoted{a})$ und erhalten 
             \\[0.2cm]
             \hspace*{1.3cm}
             $s_5 = \bigl\{ E \rightarrow \quoted{a} \bullet: \{\quoted{+}, \quoted{-}, \quoted{a}\} \bigr\}$
       \item Wir definieren $s_6 = \textsl{goto}(s_2, \quoted{+})$ und erhalten 
             \\[0.2cm]
             \hspace*{1.3cm}
             $s_6 = \bigl\{ E \rightarrow E\, E \quoted{+} \bullet: \{\quoted{\symbol{36}}, \quoted{a}\} \bigr\}$
       \item Wir definieren $s_7 = \textsl{goto}(s_3, \quoted{+})$ und erhalten 
             \\[0.2cm]
             \hspace*{1.3cm}
             $s_7 = \bigl\{ E \rightarrow E\, E \quoted{+} \bullet: \{\quoted{+}, \quoted{-}, \quoted{a}\} \bigr\}$
       \item Wir definieren $s_8 = \textsl{goto}(s_2, \quoted{-})$ und erhalten 
             \\[0.2cm]
             \hspace*{1.3cm}
             $s_8 = \bigl\{ E \rightarrow E\, E \quoted{-} \bullet: \{\quoted{\symbol{36}}, \quoted{a}\} \bigr\}$
       \item Wir definieren $s_9 = \textsl{goto}(s_3, \quoted{-})$ und erhalten 
             \\[0.2cm]
             \hspace*{1.3cm}
             $s_9 = \bigl\{ E \rightarrow E\, E \quoted{-} \bullet: \{\quoted{+}, \quoted{-}, \quoted{a}\} \bigr\}$
       \end{enumerate}
\item Bei der Berechnung der Funktion $\textsl{action}()$ in Teil (c) der Aufgabe haben wir gesehen, dass
      es keine Konflikte gibt.  Daher ist die Grammatik eine SLR-Grammatik und damit sicher nicht mehrdeutig.
\end{enumerate}
\pagebreak

\exercise
Die Grammatik $G = \langle \{ A, B \}, \{ \quoted{u}, \quoted{x}, \quoted{y}, \quoted{z} \}, R, A \rangle$
habe  die folgenden Regeln:
\\[0.2cm]
\hspace*{1.3cm}
$\begin{array}[t]{lcl}
  A & \rightarrow & B \quoted{x} \\
    & \mid        & \quoted{y} B \quoted{z} \\
    & \mid        & \quoted{u} \quoted{z} \\
    & \mid        & \quoted{y} \quoted{u} \quoted{x} \\[0.1cm]
  B & \rightarrow &\quoted{u} 
 \end{array}
$
\\[0.2cm]
Bearbeiten Sie die folgenden Teilaufgaben:
\begin{enumerate}
\item �berpr�fen Sie, ob die  diese Grammatik eine LL(1)-Grammatik ist und begr�nden Sie
      Ihre Antwort.
\item �berpr�fen Sie, ob die  diese Grammatik eine LL($*$)-Grammatik ist und begr�nden Sie
      Ihre Antwort.
\item �berpr�fen Sie, ob die  diese Grammatik eine SLR-Grammatik ist und begr�nden Sie
      Ihre Antwort.
\end{enumerate}

\solution
\begin{enumerate}
\item Die Grammatik ist keine $LL(1)$-Grammatik, denn zwischen den beiden Regeln
      \\[0.2cm]
      \hspace*{1.3cm}
      $A \rightarrow \quoted{y} B \quoted{z}$ \quad und \quad
      $A \rightarrow \quoted{y} \quoted{u} \quoted{x}$ 
      \\[0.2cm]
      gibt es einen Konflikt, wir haben
      \\[0.2cm]
      \hspace*{1.3cm}
      $\textsl{First}(\quoted{y} B \quoted{z}) = \{ \quoted{y} \}$ \quad und \quad
      $\textsl{First}(\quoted{y} \quoted{u} \quoted{z}) = \{ \quoted{y} \}$ 
      \\[0.2cm]
      und damit folgt
      \\[0.2cm]
      \hspace*{1.3cm}
      $\textsl{First}(\quoted{y} B \quoted{z}) \cap \textsl{First}(\quoted{y} \quoted{u} \quoted{z}) = 
      \{ \quoted{y} \} \not= \{\}$.
\item Um zu �berpr�fen, ob die Grammatik eine LL($*$)-Grammatik ist, ersetzen wir zun�chst die Variable $B$ 
      durch ihre Definition.  F�r $A$ erhalten wir dann die folgenden Grammatik-Regeln:
      \\[0.2cm]
      \hspace*{1.3cm}
      $\begin{array}[t]{lcl}
        A & \rightarrow & \quoted{u} \quoted{x} \\
          & \mid        & \quoted{y} \quoted{u} \quoted{z} \\
          & \mid        & \quoted{u} \quoted{z} \\
          & \mid        & \quoted{y} \quoted{u} \quoted{x} 
        \end{array}
       $
      \\[0.2cm]
      Basteln wir daraus nun, wie im Skript beschrieben, einen endlichen Automaten, so ist leicht zu
      sehen, dass die akzeptierenden Zust�nde homogen sind.  Damit handelt es sich bei der
      angegebenen Grammatik um eine $LL(*)$-Grammatik.
\item Die angegebene Grammatik ist keine SLR-Grammatik.  Um das zu sehen, erweitern wir die Grammatik
      um die Regel $\widehat{S} \rightarrow A$ und berechnen den Zustand
      \\[0.2cm]
      \hspace*{1.3cm}
      $s_0 = \textsl{closure}( \{ \widehat{S} \rightarrow A \})$.
      \\[0.2cm]
      Wir finden
      \\[0.2cm]
      \hspace*{1.3cm}
      $
      \begin{array}[t]{lcl}  
       s_0 \;= & \{ & S \rightarrow \bullet\; A,                              \\
               &    & A \rightarrow \bullet\; B \quoted{x},                   \\
               &    & A \rightarrow \bullet \quoted{y} B \quoted{z},          \\
               &    & A \rightarrow \bullet \quoted{u} \quoted{z},            \\
               &    & A \rightarrow \bullet \quoted{y} \quoted{u} \quoted{x}, \\
               &    & B \rightarrow \bullet \quoted{u}                        \\
               & \}. 
      \end{array}
      $
      \\[0.2cm]
      Wir berechnen nun $\textsl{goto}(s_0, \quoted{u})$ und erhalten
      \\[0.2cm]
      \hspace*{1.3cm}
      $s_1 = \{ A \rightarrow \quoted{u} \bullet \quoted{z},\;
                B \rightarrow \quoted{u} \bullet
             \}
      $.
      \\[0.2cm]
      Bei der Berechnung von $\textsl{action}(s_1, \quoted{z})$ tritt nun eine Shift-Reduce-Konflikt auf,
      denn es gilt
      \\[0.2cm]
      \hspace*{1.3cm}
      $\textsl{follow}(B) = \{ \quoted{x}, \quoted{z} \}$.

      \noindent
      \textbf{Bemerkung}:  Die in der Aufgabe angegebene Grammatik ist sowohl eine LR-Grammatik als
      auch eine LALR-Grammatik.  Letzteres l�sst sich mit \textsl{Bison} oder \textsl{JavaCup} nachweisen 
      und Ersteres folgt aus der Tatsache, dass jede LALR-Grammatik auch eine LR-Grammatik ist.
\end{enumerate}


\exercise
Nehmen Sie an, dass die im Skript eingef�hrte Sprache \textsl{Integer}-\texttt{C} um eine 
\texttt{do-while}-Schleife erweitert werden soll, deren Syntax durch die folgende Grammatik-Regel gegeben ist:
\\[0.2cm]
\hspace*{1.3cm}
$\textsl{statement} \rightarrow \quoted{do}\; \textsl{statement}\; \quoted{while}\;
 \quoted{(} \;\textsl{boolExpr} \;\quoted{)}$.
\\[0.2cm]
Die Semantik dieses Konstruktes soll mit der Semantik des entsprechenden Konstruktes in
der Sprache \texttt{C} �bereinstimmen.
\begin{enumerate}
\item Geben Sie eine Gleichung an, die beschreibt, wie eine \texttt{do-while}-Schleife in
      \textsl{Java-Byte-Code} �bersetzt werden kann.
\item Geben Sie die Methode $\textsl{compile}()$ an, die das entsprechende
      Konstrukt �bersetzt.  Gehen Sie dabei davon aus, dass Sie diese Methode innerhalb
      einer Klasse \texttt{DoWhile} implementieren, wobei diese Klasse f�r \textsc{Ep}
      wie folgt spezifiziert ist:
      \\[0.2cm]
      \hspace*{1.3cm}
      $\texttt{Statement} \;=\; \cdots \;+\; \texttt{DoWhile}(\texttt{Statement}\; stmnt,\;
      \mathtt{BoolExpr}\; cond)\; +\; \cdots; $      
\end{enumerate}

\solution
\begin{enumerate}
\item Die �bersetzung einer \texttt{do-while}-Schleife der Form
      \[ \textsl{do}\; \textsl{statement}\; \texttt{while}\;(\textsl{cond})\; \quoted{;} \]
      orientiert sich an der folgenden Spezifikation:
      \[
      \begin{array}[t]{lcl}
        \textsl{compile}\bigl(\mathtt{do}\; \textsl{stmnt}\; \texttt{while}\; (\textsl{cond})\;\bigr) 
        & = & 
        [\;\textsl{loop}\texttt{:}\;]            \\
        & + & \textsl{stmnt}.\textsl{compile}()    \\    
        & + & \textsl{cond}.\textsl{compile}()         \\
        & + & [\;\texttt{ifeq}\;\textsl{next}\;] \\
        & + & [\;\texttt{goto}\;\textsl{loop}\;]    \\    
        & + & [\;\textsl{next}\texttt{:}\;]         
      \end{array}
      \]
\item Die Methode $\textsl{compile}()$ kann in der Klasse \texttt{DoWhile} wie folgt implementiert werden:
      \begin{Verbatim}[ frame         = lines, 
                        framesep      = 0.3cm, 
                        labelposition = bottomline,
                        numbers       = left,
                        numbersep     = -0.2cm,
                        xleftmargin   = 0.0cm,
                        xrightmargin  = 1.1cm,
                      ]
      public List<AssemblerCmd> compile() {
          List<AssemblerCmd> result = new LinkedList<AssemblerCmd>();
          LABEL        loopLabel = new LABEL();
          LABEL        nextLabel = new LABEL();
          AssemblerCmd ifeq      = new IFEQ(nextLabel.getLabel());
          AssemblerCmd gotoLoop  = new GOTO(loopLabel.getLabel());
          result.add(loopLabel);
          result.addAll(mStmnt.compile());
          result.addAll(mCond.compile());
          result.add(ifeq);
          result.add(gotoLoop);
          result.add(nextLabel);
          return result;
      }
      \end{Verbatim}
\end{enumerate}

\end{document}

%%% Local Variables: 
%%% mode: latex
%%% TeX-master: t
%%% End: 
