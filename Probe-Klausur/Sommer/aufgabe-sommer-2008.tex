\documentclass{article}
\usepackage[latin1]{inputenc}
\usepackage{amssymb}
\usepackage{fancyvrb}
\usepackage{german}
\usepackage{a4wide}
\usepackage{epsfig}
\usepackage{fleqn}

\pagestyle{empty}

\newcounter{aufgabe}

\renewcommand{\labelenumi}{(\alph{enumi})}

\newcommand{\exercise}{\vspace*{0.1cm}
\stepcounter{aufgabe}
\vspace*{0.3cm}

\noindent
\textbf{Aufgabe \arabic{aufgabe}}: }

\newcommand{\solution}{
\vspace*{0.3cm}

\noindent
\textbf{L\"osung}: }

\begin{document}


\noindent
{\large  Aufgaben zur Klausurvorbereitung f\"ur die Vorlesung ``{\sl Compilerbau}''}
\vspace{0.5cm}

\noindent
\textbf{Hinweis}: Bei der Klausur m\"ussen alle Aufgabenbl\"ater mit abgegeben werden, sonst
ist die Klausur ung\"ultig!
\vspace{0.5cm}

\exercise
Die Grammatik $G = \langle \{ S \}, \{ \mathtt{+}, \mathtt{*}, \mathtt{a} \}, R, S \rangle$
habe  die folgenden Regeln:
\[ S \rightarrow S\, S\, \mathtt{+} \mid S\, S\, \mathtt{*} \mid \mathtt{a}. \]
\begin{enumerate}
\item Berechnen Sie die Menge der SLR-Zust\"ande f\"ur diese Grammatik.
\item Berechnen Sie die Funktionen $\textsl{action}()$ und $\textsl{goto}()$ f\"ur diese Grammatik.
\end{enumerate}

\exercise
Geben Sie ein \textsl{JFlex}-Datei an, mit deren Hilfe Sie alle Hexadezimal-Zahlen, die in
einer Datei vorkommen, aufaddieren k\"onnen.  Die Hexadezimal-Zahlen sollen dabei in der Form
\[ \mathtt{0xAB123} \]
dargestellt werden.

\exercise
Geben Sie die \textsc{Cup}-Spezifikation eines Parsers an, der Boole'sche Ausdr\"ucke
auswertet.  Die Boole'schen Ausdr\"ucke sollen mit Hilfe der Operatoren ``\texttt{\&}''
(logisches und), ``\texttt{|}'' (logisches oder) und ``\texttt{!}'' (logisches nicht) aus
den Terminalen ``\texttt{true}'' und ``\texttt{false}'' aufgebaut sein. 

\exercise
Der Typ $\texttt{list}(T)$ sei wie folgt definiert:
\\[0.2cm]
\hspace*{1.3cm}
\texttt{type list(X) := nil + cons(X, list(X));}
\\[0.2cm]
Die Funktion \texttt{addLast} habe die folgende Signatur:
\\[0.2cm]
\hspace*{1.3cm}
\texttt{signature addLast: list(T) * T -> list(T);}
\\[0.2cm]
und die Variablen \texttt{x} und \texttt{z} haben den Typ \texttt{int}.
\begin{enumerate}
\item Berechnen Sie 
      \\[0.2cm]
      \hspace*{1.3cm}
      $\textsl{typeEqs}\bigl(\texttt{addLast(cons(x, nil), z): list(int)}\bigr)$.
\item L\"osen Sie die in Teil (a) berechneten Typ-Gleichungen.
\end{enumerate}


\exercise
\framebox{\epsfig{file=beichel.eps,scale=0.1}}
Das Alphabet $\Sigma$ werde durch
\[ \Sigma = \{ \texttt{a}, \texttt{b}, \texttt{c} \} \]
definiert und auf $\Sigma^*$ werden nachfolgend verschiedene Sprachen definiert, f\"ur die Sie jeweils 
angeben sollen, ob es sich um eine kontextfreie Sprache handelt oder nicht.  Falls es sich
um eine kontextfreie Sprache handelt, m\"ussen Sie eine Grammatik angeben, die diese Sprache
erzeugt, andernfalls sollen Sie mit Hilfe des Pumping-Lemmas nachweisen, dass
die Sprache nicht kontextfrei ist.
\begin{enumerate}
\item $L_1 = \{ \mathtt{a}^m \mathtt{b}^n \mathtt{c}^{m+n} \mid m,n \in \mathbb{N} \}$.
\item $L_2 = \{ \mathtt{a}^m \mathtt{b}^n \mathtt{c}^{m * n} \mid m,n \in \mathbb{N} \}$.  
\end{enumerate}


\end{document}

%%% Local Variables: 
%%% mode: latex
%%% TeX-master: t
%%% End: 
