\documentclass{article}
\usepackage[latin1]{inputenc}
\usepackage{amssymb}
\usepackage{fancyvrb}
\usepackage{german}
\usepackage{a4wide}
\usepackage{epsfig}
\usepackage{fleqn}


\newcounter{aufgabe}

\renewcommand{\labelenumi}{(\alph{enumi})}
\newcommand{\qed}{\hspace*{\fill} $\Box$}

\newcommand{\exercise}{\vspace*{0.1cm}
\stepcounter{aufgabe}
\vspace*{0.3cm}

\noindent
\textbf{Aufgabe \arabic{aufgabe}}: }

\newcommand{\solution}{
\vspace*{0.3cm}

\noindent
\textbf{L\"osung}: }

\begin{document}


\noindent
{\large  L\"osungen zu den Aufgaben zur Klausurvorbereitung ``{\sl Compilerbau}''}
\vspace{0.5cm}

\noindent
\textbf{Hinweis}: Bei der Klausur m\"ussen alle Aufgabenbl\"ater mit abgegeben werden, sonst
ist die Klausur ung\"ultig!
\vspace{0.5cm}

\exercise
Die Grammatik $G = \langle \{ S \}, \{ \mathtt{+}, \mathtt{*}, \mathtt{a} \}, R, S \rangle$
habe  die folgenden Regeln:
\[ S \rightarrow S\, S\, \mathtt{+} \mid S\, S\, \mathtt{*} \mid \mathtt{a}. \]
\begin{enumerate}
\item Berechnen Sie die Menge der SLR-Zust\"ande f\"ur diese Grammatik.
\item Berechnen Sie die Funktionen $\textsl{action}()$ und $\textsl{goto}()$ f\"ur diese Grammatik.
\end{enumerate}

\exercise
Geben Sie ein \textsl{JFlex}-Datei an, mit deren Hilfe Sie alle Hexadezimal-Zahlen, die in
einer Datei vorkommen, aufaddieren k\"onnen.  Die Hexadezimal-Zahlen sollen dabei in der Form
\[ \mathtt{0xAB123} \]
dargestellt werden.

\exercise
Geben Sie die \textsc{Cup}-Spezifikation eines Parsers an, der Boole'sche Ausdr\"ucke
auswertet.  Die Boole'schen Ausdr\"ucke sollen mit Hilfe der Operatoren ``\texttt{\&}''
(logisches und), ``\texttt{|}'' (logisches oder) und ``\texttt{!}'' (logisches nicht) aus
den Terminalen ``\texttt{true}'' und ``\texttt{false}'' aufgebaut sein. 
\pagebreak

\exercise
Der Typ $\texttt{list}(T)$ sei wie folgt definiert:
\\[0.2cm]
\hspace*{1.3cm}
\texttt{type list(X) := nil + cons(X, list(X));}
\\[0.2cm]
Die Funktion \texttt{addLast} habe die folgende Signatur:
\\[0.2cm]
\hspace*{1.3cm}
\texttt{signature addLast: list(T) * T -> list(T);}
\\[0.2cm]
und die Variablen \texttt{x} und \texttt{z} haben den Typ \texttt{int}.
\begin{enumerate}
\item Berechnen Sie 
      \\[0.2cm]
      \hspace*{1.3cm}
      $\textsl{typeEqs}\bigl(\texttt{addLast(cons(x, nil), z): list(int)}\bigr)$.
\item L\"osen Sie die in Teil (a) berechneten Typ-Gleichungen.
\end{enumerate}

\solution
\begin{enumerate}
\item Wir berechnen zun\"achst die Typ-Gleichungen nach der im Skript angegebenen Definition.
      \[
      \begin{array}[t]{cl}
           & \textsl{typeEqs}(\texttt{addLast(cons(x, nil), z): list(int)}) \\[0.2cm]
         = &       \{ \mathtt{list}(T) = \mathtt{list(int)} \} 
              \cup \textsl{typeEqs}(cons(x, nil), \mathtt{list}(T))
              \cup \textsl{typeEqs}(z: \mathtt{int})                     \\[0.2cm]
         = &       \{ \mathtt{list}(T) = \mathtt{list(int)} \} 
              \cup \{ \texttt{list}(S) = \mathtt{list}(T) \}\; \cup \\
           &       \textsl{typeEqs}(x, S)
              \cup \textsl{typeEqs}(\mathtt{nil}, \mathtt{list}(S))
              \cup \textsl{typeEqs}(z, \mathtt{int})                  \\[0.2cm]
         = &       \{ \mathtt{list}(T) = \mathtt{list(int)},\; 
                      \texttt{list}(S) = \mathtt{list}(T),\;           \mathtt{int} = S,\;
                      \mathtt{list}(R) = \mathtt{list}(S),\;
                      \mathtt{int} = \mathtt{int} \}

      \end{array}
      \]
\item Wir l\"osen die oben berechneten Typ-Gleichungen nach dem im Skript angegebenen Verfahren.
      \\[0.2cm]
      \hspace*{-0.3cm}  
      $\begin{array}[t]{cl}
           &  \bigl\langle \{  \mathtt{list}(T) = \mathtt{list(int)},\; 
                      \texttt{list}(S) = \mathtt{list}(T),\;           \mathtt{int} = S,\;
                      \mathtt{list}(R) = \mathtt{list}(S),\;
                      \mathtt{int} = \mathtt{int} \},\; [] \bigr\rangle \\[0.2cm]
         \leadsto &  \bigl\langle \{ T = \mathtt{int},\; 
                      \texttt{list}(S) = \mathtt{list}(T),\;           \mathtt{int} = S,\;
                      \mathtt{list}(R) = \mathtt{list}(S),\;
                      \mathtt{int} = \mathtt{int} \},\; [] \bigr\rangle \\[0.2cm]
         \leadsto &  \bigl\langle \{ 
                      \texttt{list}(S) = \mathtt{list}(\mathtt{int}),\;           
                      \mathtt{int} = S,\;
                      \mathtt{list}(R) = \mathtt{list}(S),\;
                      \mathtt{int} = \mathtt{int} \},\; [ T \mapsto \mathtt{int}] \bigr\rangle \\[0.2cm]
         \leadsto &  \bigl\langle \{ 
                      S = \mathtt{int},\;           
                      \mathtt{int} = S,\;
                      \mathtt{list}(R) = \mathtt{list}(S),\;
                      \mathtt{int} = \mathtt{int} \},\; [ T \mapsto \mathtt{int}] \bigr\rangle \\[0.2cm]
         \leadsto &  \bigl\langle \{            
                      \mathtt{int} = \mathtt{int},\;
                      \mathtt{list}(R) = \mathtt{list}(\mathtt{int}),\;
                      \mathtt{int} = \mathtt{int} \},\; 
                      [ T \mapsto \mathtt{int},\; S \mapsto \mathtt{int}] \bigr\rangle \\[0.2cm]
         \leadsto &  \bigl\langle \{            
                      \mathtt{list}(R) = \mathtt{list}(\mathtt{int}),\;
                      \mathtt{int} = \mathtt{int} \},\; 
                      [ T \mapsto \mathtt{int},\; S \mapsto \mathtt{int}] \bigr\rangle \\[0.2cm]
         \leadsto &  \bigl\langle \{            
                      R = \mathtt{int},\;
                      \mathtt{int} = \mathtt{int} \},\; 
                      [ T \mapsto \mathtt{int},\; S \mapsto \mathtt{int}] \bigr\rangle \\[0.2cm]
         \leadsto &  \bigl\langle \{            
                      \mathtt{int} = \mathtt{int} \},\; 
                      [ T \mapsto \mathtt{int},\; S \mapsto \mathtt{int},\; R \mapsto \mathtt{int}] \bigr\rangle \\[0.2cm]
         \leadsto &  \bigl\langle \{ \},\; 
                      [ T \mapsto \mathtt{int},\; S \mapsto \mathtt{int},\; R \mapsto \mathtt{int}] \bigr\rangle
         \end{array}
      $
      \\[0.2cm]
      Damit ist die Substitution 
      $[ T \mapsto \mathtt{int},\; S \mapsto \mathtt{int},\; R \mapsto \mathtt{int}]$ eine 
      L\"osung der Typ-Gleichungen und wir k\"onnen folgern, dass der Term tats\"achlich den angegebenen Typ
      hat.
      \qed
\end{enumerate}

\pagebreak
\exercise
Das Alphabet $\Sigma$ werde durch
\[ \Sigma = \{ \texttt{a}, \texttt{b}, \texttt{c} \} \]
definiert und auf $\Sigma^*$ werden nachfolgend verschiedene Sprachen definiert, f\"ur die Sie jeweils 
angeben sollen, ob es sich um eine kontextfreie Sprache handelt oder nicht.  Falls es sich
um eine kontextfreie Sprache handelt, m\"ussen Sie eine Grammatik angeben, die diese Sprache
erzeugt und diese Behauptung beweisen.
Andernfalls sollen Sie mit Hilfe des Pumping-Lemmas nachweisen, dass
die Sprache nicht kontextfrei ist.
\begin{enumerate}
\item $L_1 = \{ \mathtt{a}^m \mathtt{b}^n \mathtt{c}^{m+n} \mid m,n \in \mathbb{N} \}$.
\item $L_2 = \{ \mathtt{a}^m \mathtt{b}^n \mathtt{c}^{m * n} \mid m,n \in \mathbb{N} \}$.  
\end{enumerate}

\solution
\begin{enumerate}
\item Die Sprache $L_1$ ist eine kontextfreie Sprache, denn Sie wird von der folgenden
      Grammatik erzeugt:
      \begin{eqnarray*}
        S & \rightarrow & \texttt{a} S \texttt{c} \\
          & \mid        & B                       \\[0.2cm]
        B & \rightarrow & \texttt{b} B \texttt{c} \\
          & \mid        & \varepsilon 
      \end{eqnarray*}
      Wir zeigen zun\"achst durch Induktion \"uber $n$, dass aus
      \[ B \Rightarrow^n w \]
      folgt, dass $w$ die Form $w = \texttt{b}^{n-1} \texttt{c}^{n-1}$ haben muss.
      \begin{enumerate}
      \item[I.A.:] $n=1$.  Wir haben 
                   \[ B \Rightarrow^1 \varepsilon \]
                   und daraus folgt $w = \varepsilon$.  Andererseits haben wir 
                   \[ \texttt{b}^{n-1} \texttt{c}^{n-1} = \texttt{b}^{1-1} \texttt{c}^{1-1} = 
                      \texttt{b}^{0} \texttt{c}^{0} = \varepsilon \varepsilon = \varepsilon.
                   \]   
      \item[I.S.:] $n \mapsto n + 1$.  Nun hat die Ableitung von $w$ die Form
                   \[ B \Rightarrow \texttt{b} B \texttt{c} \Rightarrow^n \texttt{b} v \texttt{c}. \]
                   Einerseits zeigt dies, dass
                   \[ w = \texttt{b} v \texttt{c} \]
                   ist, andererseits, sehen wir, dass 
                   \[ B \Rightarrow^n v \]
                   gelten muss und daraus folgt nach Induktions-Voraussetzung 
                   \[ v = \texttt{b}^{n-1} \texttt{c}^{n-1} \]
                   Damit haben wir insgesamt
                   \[ w = \texttt{b} \texttt{b}^{n-1} \texttt{c}^{n-1} \texttt{c} = 
                          \texttt{b}^{n} \texttt{c}^{n}
                   \]
                   und damit ist gezeigt, dass
                   \[ L(B) \subseteq \{ \texttt{b}^{n} \texttt{c}^{n} \mid n \in \mathbb{N} \} \]
                   gilt.
      \end{enumerate}
      Wir zeigen nun durch Induktion nach der L\"ange der Herleitung, dass aus
      \[ S \Rightarrow^* w \]
      folgt, dass $w \in L_1$ ist.
      \begin{enumerate}
      \item[1. Fall:]  Es gilt
            \[ S \Rightarrow B \Rightarrow^* w. \]
            Nach dem eben Bewiesenen gibt es dann ein $n \in \mathbb{N}$, so dass 
            \[ w = \texttt{b}^{n} \texttt{c}^{n} \]
            gilt und damit haben wir trivialerweise
            \[ w = \texttt{a}^{0}\texttt{b}^{n} \texttt{c}^{0+n}, \]
            was $w \in L_1$ impliziert.
      \item[2. Fall:]  Es gilt
            \[ S \Rightarrow \texttt{a} S \texttt{c} \Rightarrow^* \texttt{a} v \texttt{c}. \] 
            dann haben wir einerseits
            \[ w =  \texttt{a} v \texttt{c} \]
            und andererseits gibt es nach Induktions-Voraussetzung $n,m \in \mathbb{N}$, so dass
            \[ v =  \texttt{a}^m \texttt{b}^{n} \texttt{c}^{m+n} \]
            gilt.  Daraus folgt aber sofort
            \[ w =  \texttt{a}^{m+1} \texttt{b}^{n} \texttt{c}^{m+n+1} \] 
            und folglich gilt auch in diesem Fall $w \in L_1$.
      \end{enumerate}
      Zum Schluss muss noch gezeigt werden, dass alle W\"orter aus 
      $L_1$ tats\"achlich von der oben angegebenen Grammatik erzeugt werden.  Das ist aber offensichtlich,
      denn das Wort $\texttt{a}^{m} \texttt{b}^{n} \texttt{c}^{m+n}$ l\"asst sich dadurch erzeugen,
      dass zun\"achst $m$ mal die Regel $S \rightarrow \texttt{a} S \texttt{c}$ verwendet wird.
      Darauf folgt eine Anwendung der Regel $S \rightarrow B$ und
      anschlie{\ss}end folgen $n$ Anwendungen der Regel $B \rightarrow \texttt{b} B \texttt{c}$ 
      und ganz zum Schluss wird noch die Regel $B \rightarrow \varepsilon$ angewendet. 
      \[
      \begin{array}[t]{lcl}
        S & \Rightarrow & \texttt{a} S \texttt{c} \Rightarrow \cdots \Rightarrow \texttt{a}^m S \texttt{c}^m \\
          & \Rightarrow & \texttt{a}^m B \texttt{c}^m \\
          & \Rightarrow & \texttt{a}^m \texttt{b} B \texttt{c} \texttt{c}^m
            \Rightarrow \cdots
            \Rightarrow \texttt{a}^m \texttt{b}^n B \texttt{c}^n \texttt{c}^m \\
          & \Rightarrow & \texttt{a}^m \texttt{b}^n \varepsilon \texttt{c}^n \texttt{c}^m
                          = \texttt{a}^m \texttt{b}^n \texttt{c}^{m+n} 
      \end{array}
      \]
      \qed
\item Die Sprache $L_2$ ist nicht kontextfrei.  Wir f\"uhren den Beweis indirekt und nehmen an, dass
      $L_2$ kontextfrei w\"are.  Nach dem Pumping-Lemma existiert dann eine nat\"urliche Zahl $n$, so dass
      f\"ur alle $s \in L$, f\"ur die $|s| \geq n$ ist, eine Zerlegung $s = uvwxyz$ existiert, f\"ur die
      au{\ss}erdem folgendes gilt:
      \begin{enumerate}
      \item $|vwx| \leq n$,
      \item $vx \not= \varepsilon$,
      \item $uv^kwx^ky \in L$ f\"ur alle $k \in \mathbb{N}$.
      \end{enumerate}
      Wir k\"onnen o.B.d.A. fordern, dass $n \geq 1$ gilt.
      Wir betrachten nun den String
      \[ s = \texttt{a}^{2\cdot n} \texttt{b}^{2\cdot n} \texttt{c}^{4\cdot n^2}. \]
      Offenbar gilt $s \in L_2$ und wegen $|s| = 2 \cdot n + 2 \cdot n + 4 \cdot n^2 > n$ erf\"ullt $s$ die
      Voraussetzung des Pumping-Lemmas.  Wir finden also eine Zerlegung
      \[ s = u v w x y \]
      mit den oben angegebenen Eigenschaften.  Wegen $|vwx| \leq n$ und 
      $| \texttt{b}^{2\cdot n} | = 2\cdot n > n$ kann der Teilstring $vwx$ den String $\texttt{b}^{2\cdot n}$ 
      nicht vollst\"andig \"uberdecken.  Daher muss einer der beiden folgenden F\"alle vorliegen:
      \[ vwx \leq \texttt{a}^{2\cdot n}\texttt{b}^{2\cdot n} \quad \mbox{oder} \quad
         vwx \leq \texttt{b}^{2\cdot n}\texttt{c}^{4\cdot n^2}.
      \]
      Wir betrachten die beiden F\"alle getrennt.
      \begin{enumerate}
      \item $vwx \leq \texttt{a}^{2\cdot n}\texttt{b}^{2\cdot n}$

            In diesem Fall enth\"alt der String $vwx$ den Buchstaben $\texttt{c}$ nicht.
            Nach dem Pumping-Lemma gilt $uwy \in L$. Daher haben wir einerseits
            \[ \textsl{count}(uwy, \texttt{c}) =  \textsl{count}(uvwxy, \texttt{c}) = 4 \cdot n^2, \]
            Andererseits folgt aus $vx \not= \varepsilon$, dass 
            \[ 
            \begin{array}[t]{ll}
               \textsl{count}(uwy, \texttt{a}) < \textsl{count}(uvwxy, \texttt{a}) = 2 \cdot n &
               \quad \mbox{oder} \\
               \textsl{count}(uwy, \texttt{b}) < \textsl{count}(uvwxy, \texttt{b}) = 2 \cdot n
            \end{array}
            \]
            gilt.  Dann gilt auf jeden Fall
            \[ \textsl{count}(uwy, \texttt{a}) \cdot \textsl{count}(uwy, \texttt{b}) < 
               (2 \cdot n) \cdot (2 \cdot n) = 4\cdot n^2 = \textsl{count}(uwy, \texttt{c})
            \]
            und daraus w\"urde im Widerspruch zum Pumping-Lemma $uwy \not\in L_2$ folgen.
      \item $vwx \leq \texttt{b}^{2\cdot n}\texttt{c}^{4\cdot n^2}$.
          
            In diesem Fall enth\"alt der String $vwx$ den Buchstaben $\mathtt{a}$ nicht.
            Wir k\"onnen annehmen, dass der String $vwx$ den Buchstaben $\mathtt{c}$ enth\"alt, denn
            andernfalls w\"urde $vwx \leq \texttt{a}^{2\cdot n}\texttt{b}^{2\cdot n}$ gelten und den Fall
            haben wir oben schon behandelt.  Da einerseits $|vy| \leq n$ und andererseits $|vx| > 0$ gilt,
            haben wir insgesamt die Ungleichungs-Kette
            \[ 4\cdot n^2 - n \leq \textsl{count}(uwy,\texttt{c}) < 4\cdot n^2. \]
            Au{\ss}erdem gilt $\textsl{count}(uwy,\texttt{a}) = \textsl{count}(s,\texttt{a}) = 2\cdot n$,
            denn $vx$ besteht ja nur aus den Buchstaben \texttt{b} und \texttt{c}.
            Wegen $\textsl{count}(uwy,\texttt{c}) < 4\cdot n^2$ muss dann
            \[ \textsl{count}(uwy,\texttt{b}) < 2\cdot n \]
            gelten, denn sonst h\"atte das Produkt 
            \[ \textsl{count}(uwy, \texttt{a}) \cdot \textsl{count}(uwy, \texttt{b}) \]
            ja den Wert $4\cdot n^2$
            und  $uwy$ w\"urde nicht in der Sprache $L_2$ liegen.  Also haben wir
            folgende Ungleichungs-Kette:
            \[ 
            \begin{array}[t]{lcl}
              \textsl{count}(uwy, \texttt{a}) \cdot \textsl{count}(uwy, \texttt{b}) 
              & \leq & (2\cdot n) \cdot (2\cdot n - 1) \\
              & =    & 4 \cdot n^2 - 2 \cdot n \\
              & <    & 4 \cdot n^2 - n \\
              & \leq & \textsl{count}(uwy, \texttt{c}).
            \end{array}
            \]
            Daraus folgt aber im Widerspruch zum Pumping-Lemma, dass der String $uwy$ nicht in der
            Sprache $L_2$ enthalten ist. \qed
      \end{enumerate}
\end{enumerate}

\end{document}

%%% Local Variables: 
%%% mode: latex
%%% TeX-master: t
%%% End: 
