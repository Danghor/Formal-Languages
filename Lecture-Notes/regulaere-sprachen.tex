\chapter{The Theory of Regular Languages \label{chapter:regulaere-sprachen}}
A formal language $L \subseteq \Sigma^*$ is called a \emph{regular language} if there is a regular
expression $r$ such that the language $L$ is specified by $r$, i.e.~if
\\[0.2cm]
\hspace*{1.3cm}
$L = L(r)$ 
\\[0.2cm]
holds.  In Chapter \ref{chapter:finit-state-machines.tex} we have shown that the regular languages
are those languages that are recognized by a finite state machine.  In this chapter, we show
that regular languages have certain closure properties:
\begin{enumerate}
\item The union $L_1 \cup L_2$ of two regular languages $L_1$ and $L_2$ is a regular language.
\item The intersection $L_1 \cap L_2$ of two regular languages $L_1$ und $L_2$ is a regular language.
\item The complement $\Sigma^* \backslash L$ of a regular language $L$ is a regular language.
\end{enumerate}
As an application of these closure properties we then show how it is possible to decide whether two
regular expressions are equivalent, i.e.~we present an algorithm that takes two regular expressions
$r_1$ and $r_2$ as input and checks, whether 
\\[0.2cm]
\hspace*{1.3cm}
$r_1 \doteq r_2$
\\[0.2cm]
holds.  After that, we discuss the limits of regular languages.  To this end, we prove the
\href{http://en.wikipedia.org/wiki/Pumping_lemma_for_regular_languages}{\emph{pumping lemma}}.
Using the pumping lemma we will be able to show that the language
\\[0.2cm]
\hspace*{1.3cm} $\{ \mathtt{a}^n \mathtt{b}^n \mid n \in \mathbb{N} \}$
\\[0.2cm]
is not regular.

\section{Abschluss-Eigenschaften regul\"arer Sprachen}
In diesem Abschnitt zeigen wir, dass regul\"are Sprachen unter den Boole'schen Operationen
\emph{Vereinigung}, \emph{Durchschnitt} und \emph{Komplement} abgeschlossen sind.  Wir beginnen mit
der Vereinigung.  

\begin{Satz}
  Sind $L_1$ und $L_2$ regul\"are Sprachen, so ist auch die Vereinigung $L_1 \cup L_2$ 
  eine regul\"are Sprache.
\end{Satz}

\proof
Da $L_1$ und $L_2$ regul\"are Sprachen sind, gibt es regul\"are Ausdr\"ucke $r_1$ und $r_2$, so dass
\\[0.2cm]
\hspace*{1.3cm}
$L_1 = L(r_1)$ \quad und \quad $L_2 = L(r_2)$
\\[0.2cm]
gilt.  Wir definieren $r := r_1 + r_2$.  Offenbar gilt
\\[0.2cm]
\hspace*{1.3cm}
$L(r) = L(r_1 + r_2) = L(r_1) \cup L(r_2) = L_1 \cup L_2$.
\\[0.2cm]
Damit ist klar, dass auch $L_1 \cup L_2$ eine regul\"are Sprache ist. \qed

\begin{Satz} \label{satz:schnitt}
  Sind $L_1$ und $L_2$ regul\"are Sprachen, so ist auch der Durchschnitt $L_1 \cap L_2$ 
  eine regul\"are Sprache.
\end{Satz}

\proof W\"ahrend der letzte Satz unmittelbar aus der Definition der regul\"aren Ausdr\"ucke
gefolgert werden kann, m\"ussen wir nun etwas weiter ausholen.  Im vorletzten Kapitel haben wir
gesehen, dass es zu jedem regul\"aren Ausdruck $r$ einen \"aquivalenten deterministischen
endlichen Automaten $A$ gibt, der die durch $r$ spezifizierte Sprache akzeptiert und wir
k\"onnen au{\ss}erdem annehmen, dass dieser Automat vollst\"andig ist.  Es seien $r_1$ und $r_2$
regul\"are Ausdr\"ucke, die die Sprachen $L_1$ und $L_2$ spezifizieren:
\\[0.2cm]
\hspace*{1.3cm}
$L_1 = L(r_1)$ \quad und \quad $L_2 = L(r_2)$.
\\[0.2cm]
Dann konstruieren wir zun\"achst zwei vollst\"andige deterministische endliche Automaten $A_1$
und $A_2$, die diese Sprachen akzeptieren, es gilt also
\\[0.2cm]
\hspace*{1.3cm}
$L(A_1) = L_1$ \quad und \quad $L(A_2) = L_2$.
\\[0.2cm]
Wir werden f\"ur die Sprache $L_1 \cap L_2$ einen Automaten $A$ bauen, der diese Sprache
akzeptiert.  Da es zu jedem Automaten auch einen regul\"aren Ausdruck gibt, 
der die Sprache beschreibt, die von dem Automaten akzeptiert wird, 
haben wir damit dann gezeigt, dass die Sprache $L_1 \cap L_2$ regul\"ar ist.
Als Baumaterial f\"ur den Automaten $A$, der die Sprache $L_1 \cap L_2$ akzeptiert, 
verwenden wir nat\"urlich die Automaten $A_1$ und $A_2$.  Wir nehmen an, dass
\\[0.2cm]
\hspace*{1.3cm}
$A_1 = \langle Q_1, \Sigma, \delta_1, q_1, F_1 \rangle$ \quad und \quad
$A_2 = \langle Q_2, \Sigma, \delta_2, q_2, F_2 \rangle$
\\[0.2cm]
gilt und definieren $A$ als ein verallgemeinertes kartesisches Produkt der Automaten $A_1$ und $A_2$:
\\[0.2cm]
\hspace*{1.3cm}
$A := \langle Q_1 \times Q_2, \Sigma, \delta, \pair(q_1,q_2), F_1 \times F_2 \rangle$,
\\[0.2cm]
wobei die Zustands-\"Ubergangs-Funktion 
\\[0.2cm]
\hspace*{1.3cm}
 $\delta : (Q_1 \times Q_2) \times \Sigma \rightarrow Q_1 \times Q_2$ 
\\[0.2cm]
durch die Gleichung
\\[0.2cm]
\hspace*{1.3cm}
$\delta\bigl( \pair(p_1, p_2), c \bigr) := \bigl\langle\delta_1(p_1,c), \delta_2(p_2,c)\bigr\rangle$
\\[0.2cm]
definiert wird.  Der so definierte endliche Automat $A$ simuliert gleichzeitig die 
beiden Automaten $A_1$ und $A_2$ indem er parallel berechnet, in
welchem Zustand die Automaten $A_1$ und $A_2$ jeweils sind.  Damit das m\"oglich ist, bestehen die Zust\"ande
von $A$ aus Paaren $\pair(p_1,p_2)$, so dass $p_1$ ein Zustand von $A_1$ und $p_2$ ein
Zustand von $A_2$ ist und die Funktion $\delta$ berechnet den Nachfolgezustand zu
$\pair(p_1,p_2)$, indem gleichzeitig die Nachfolgezust\"ande von $p_1$ und $p_2$ berechnet werden.
Ein String wird genau dann akzeptiert, wenn sowohl der Automat $A_1$ als auch der Automat $A_2$ einen
akzeptierenden Zustand erreicht haben.  Daher wird die Menge der akzeptierenden Zust\"ande 
wie folgt definiert:
\\[0.2cm]
\hspace*{1.3cm}
$F := \bigl\{ \pair(p_1,p_2) \in Q_1 \times Q_2 \mid p_1 \in F_1 \wedge p_2 \in F_2 \bigr\} = F_1 \times F_2$.
\\[0.2cm]
Damit gilt f\"ur alle $s \in \Sigma^*$:
\\[0.2cm]
\hspace*{1.3cm}
$
\begin{array}[t]{cl}
                & s \in L(A)                                                           \\[0.1cm]
\Leftrightarrow & \delta(\pair(q_1,q_2), s) \in F                                      \\[0.1cm]
\Leftrightarrow & \langle \delta_1(q_1,s), \delta_2(q_2, s) \rangle \in F_1 \times F_2 \\[0.1cm]
\Leftrightarrow & \delta_1(q_1,s) \in F_1 \wedge  \delta_2(q_2, s) \in F_2             \\[0.1cm]
\Leftrightarrow & s \in L(A_1) \wedge  s \in L(A_2)                                    \\[0.1cm]
\Leftrightarrow & s \in L(A_1) \cap L(A_2)                                             \\[0.1cm]
\Leftrightarrow & s \in L_1 \cap L_2                                                 
\end{array}
$
\\[0.2cm]
Also haben wir nachgewiesen, dass
\\[0.2cm]
\hspace*{1.3cm}
 $L(A) = L_1 \cap L_2$ 
\\[0.2cm]
gilt und das war zu zeigen. \qed

\remark
Prinzipiell w\"are es m\"oglich, f\"ur regul\"are Ausdr\"ucke eine Funktion 
\\[0.2cm]
\hspace*{1.3cm}
$\wedge: \textsl{RegExp} \times \textsl{RegExp} \rightarrow \textsl{RegExp}$
\\[0.2cm]
zu definieren, so dass f\"ur den Ausdruck $r_1 \wedge r_2$ die Beziehung
\\[0.2cm]
\hspace*{1.3cm}
$L(r_1 \wedge r_2) = L(r_1) \cap L(r_2)$
\\[0.2cm]
gilt:  Zun\"achst berechnen wir zu $r_1$ und $r_2$ \"aquivalente nicht-deterministische
endliche Automaten, \"uberf\"uhren diese Automaten dann in einen vollst\"andigen
deterministischen Automaten, bilden wie oben gezeigt das kartesische Produkt dieser
Automaten und gewinnen schlie{\ss}lich aus diesem Automaten einen regul\"aren 
Ausdruck zur\"uck.  Der so gewonnene regul\"are Ausdruck w\"are allerdings so gro{\ss},
dass diese Funktion in der Praxis nicht implementiert wird, denn bei der \"Uberf\"uhrung
eines nicht-deterministischen in einen deterministischen Automaten kann der Automat
stark anwachsen und der regul\"are Ausdruck, der sich aus einem Automaten ergibt, kann schon
bei verh\"altnism\"a{\ss}ig kleinen Automaten sehr un\"ubersichtlich werden.

\begin{Satz}
  Ist $L$ eine regul\"are Sprache \"uber dem Alphabet $\Sigma$, so ist auch das Komplement
  von $L$, die Sprache $\Sigma^* \backslash L$ eine regul\"are Sprache.
\end{Satz}

\proof Wir gehen \"ahnlich vor wie beim Beweis des letzten Satzes und nehmen an, dass ein
vollst\"andiger deterministischer endlicher Automat $A$ gegeben ist, der die Sprache $L$
akzeptiert:
\\[0.2cm]
\hspace*{1.3cm} $L = L(A)$.
\\[0.2cm]
Wir konstruieren einen Automaten $\hat{A}$, der ein Wort $w$ genau dann akzeptiert, wenn
$A$ dieses Wort nicht akzeptiert.  Dazu ist es lediglich erforderlich das Komplement der
Menge der akzeptierenden Zust\"ande von $A$ zu bilden.  Sei also
\\[0.2cm]
\hspace*{1.3cm} $A = \langle Q, \Sigma, \delta, q_0, F \rangle$.
\\[0.2cm]
Dann definieren wir
\\[0.2cm]
\hspace*{1.3cm} $\hat{A} = \langle Q, \Sigma, \delta, q_0, Q \backslash F \rangle$.
\\[0.2cm]
Offenbar gilt
\\[0.2cm]
\hspace*{1.3cm} $\begin{array}[t]{cl}
  & w \in L(\hat{A})                      \\[0.1cm]
  \Leftrightarrow & \delta(q_0, w) \in Q \backslash F     \\[0.1cm]
  \Leftrightarrow & \delta(q_0, w) \not\in F \\[0.1cm]
  \Leftrightarrow & w \not\in L(A)
 \end{array}
$
\\[0.2cm]
und daraus folgt die Behauptung. \qed

\begin{Korollar} \label{kor:mengendif}
  Sind $L_1$ und $L_2$ regul\"are Sprachen, so ist auch die Mengen-Differenz $L_1 \backslash L_2$ 
  eine regul\"are Sprache.
\end{Korollar}

\proof
Es sei $\Sigma$ das Alphabet, das den Sprachen $L_1$ und $L_2$ zu Grunde liegt.  Dann gilt
\\[0.2cm]
\hspace*{1.3cm}
$L_1 \backslash L_2 = L_1 \cap (\Sigma^* \backslash L_2)$,
\\[0.2cm]
denn ein Wort $w$ ist genau dann in $L_1 \backslash L_2$, wenn $w$ einerseits in  $L_1$
und andererseits im Komplement von $L_2$ liegt.  Nach dem letzten Satz wissen wir, dass mit
$L_2$ auch das Komplement $\Sigma^* \backslash L_2$ regul\"ar ist.  Da der Durchschnitt
zweier regul\"arer Sprachen wieder regul\"ar ist, ist damit auch $L_1 \backslash L_2$ regul\"ar.
\qed
\vspace*{0.3cm}

Insgesamt haben wir jetzt gezeigt, dass regul\"are Sprachen unter den Boole'schen
Mengen-Operationen abgeschlossen sind.

\exerciseEng
Assume $\Sigma$ to be some alphabet.  For a string $s=c_1 c_2 \cdots c_{n-1} c_n \in \Sigma^*$ the \emph{reversal}
of $s$ is written $s^R$ and it is defined as
\\[0.2cm]
\hspace*{1.3cm}
$s^R := c_n c_{n-1} \cdots c_2 c_1$.
\\[0.2cm]
For example, if $s = \mathtt{abc}$, then $s^R = \mathtt{cba}$. The reversal $L^R$ of a language $L \subseteq \Sigma^*$ is defined as
\\[0.2cm]
\hspace*{1.3cm}
$L^R := \{ s^R \mid s \in L \}$.
\\[0.2cm]
Next, assume that the language $L \subseteq \Sigma^*$ is regular.  Prove that then $L^R$ is a regular
language, too. \eox

\section{Erkennung leerer Sprachen \label{section:leer}}
In diesem Abschnitt untersuchen wir f\"ur einen gegeben deterministischen endlichen Automaten
\\[0.2cm]
\hspace*{1.3cm}
$A = \langle Q, \Sigma, \delta, q_0, F \rangle$
\\[0.2cm]
die Frage, ob die von $A$ erkannte Sprache leer ist, ob also $L(A) = \{\}$ gilt.  Dazu fassen wir den
endlichen Automaten als einen Graphen auf: Die
Knoten dieses Graphen sind die Zust\"ande von $A$ und zwischen zwei Zust\"anden $q_1$ und
$q_2$ gibt es genau dann eine Kante, die $q_1$ mit $q_2$ verbindet, wenn es einen
Buchstaben $c \in \Sigma$ gibt, so dass $\delta(q_1, c) = q_2$ gilt.  Die Sprache $L(A)$
ist genau dann leer, wenn es in diesem Graphen keinen Pfad gibt, der von dem Start-Zustand
$q_0$ ausgeht und in einem akzeptierenden Zustand endet, wenn also die akzeptierenden
Zust\"ande von dem Start-Zustand aus nicht erreichbar sind.

Daher berechnen wir zur Beantwortung der Frage, ob $L(A)$ leer ist,
die Menge $R$ der von dem Start-Zustand $q_0$ erreichbaren Zust\"ande.  Diese Berechnung
kann am einfachsten iterativ erfolgen:
\begin{enumerate}
\item $q_0 \in R$.
\item $p_1 \in R \wedge \delta(p_1,c) = p_2 \;\rightarrow\; p_2 \in R$.

      Dieser Schritt wird solange wiederholt, bis wir der Menge $R$ keine neuen Zust\"ande
      mehr hinzuf\"ugen k\"onnen.
\end{enumerate}
Die Sprache $L(A)$ ist genau dann leer, wenn keiner der akzeptierenden Zust\"ande erreichbar
ist, mit anderen Worten haben wir
\\[0.2cm]
\hspace*{1.3cm}
$L(A) = \{\} \;\Leftrightarrow\; R \cap F = \{\}$.
\\[0.2cm]
Damit haben wir einen Algorithmus zur Beantwortung der Frage, ob $L(A) = \{\}$ ist:
Wir bilden die Menge alle vom Start-Zustand $q_0$ erreichbaren Zust\"ande und \"uberpr\"ufen
dann, ob diese Menge einen akzeptierenden Zustand enth\"alt.
\vspace*{0.3cm}

\noindent
\textbf{Bemerkung}:  Ist die regul\"are Sprache $L$ nicht durch einen endlichen Automaten $A$,
sondern durch einen regul\"aren Ausdruck $r$ gegeben, so l\"asst sich durch einen einfachen
rekursiven Algorithmus, der dem Aufbau des regul\"aren Ausdrucks folgt, entscheiden, ob
$L(r)$ leer ist.
\begin{enumerate}
\item $L(\emptyset) = \{\}$.
\item $L(\varepsilon) \not= \{\}$.
\item $L(c) \not= \{\}$ \quad f\"ur alle $c \in \Sigma$.
\item $L(r_1 \cdot r_2) = \{\} \;\Leftrightarrow\; L(r_1) = \{\} \vee L(r_2) = \{\}$.
\item $L(r_1 + r_2) = \{\} \;\Leftrightarrow\; L(r_1) = \{\} \wedge L(r_2) = \{\}$.
\item $L(r^*) \not= \{\}$.
\end{enumerate}


\section{Equivalence of Regular Expressions}
In Chapter \ref{chapter:regular-expressions} we had defined two regular expressions $r_1$ and $r_2$ to be equivalent 
(written $r_1 \doteq r_2$), if the languages specified by $r_1$ and $r_2$ are identical:
\\[0.2cm]
\hspace*{1.3cm}
$r_1 \doteq r_2 \stackrel{\mbox{\scriptsize def}}{\Longleftrightarrow} L(r_1) = L(r_2)$. 
\\[0.2cm]
In this section, we present an algorithm that receives two regular expressions $r_1$ and $r_2$ as input and then
checks, whether $r_1 \doteq r_2$ holds. 


\begin{Theorem}
  If $r_1$ and $r_2$ are regular expressions, then the question whether $r_1 \doteq r_2$ holds, is decidable.
\end{Theorem}

\proofEng
We present an algorithm that decides, whether $L(r_1) = L(r_2)$.  First, we observe that 
$L(r_1)$ and $L(r_2)$ are the same sets iff the set differences $L(r_2) \backslash L(r_1)$ and $L(r_1) \backslash L(r_2)$
are both empty:
\begin{eqnarray*}
                  L(r_1) = L(r_2) 
&\Leftrightarrow& L(r_1) \subseteq L(r_2)         \;\wedge\; L(r_2) \subseteq L(r_1)          \\
&\Leftrightarrow& L(r_1) \backslash L(r_2) = \{\} \;\wedge\; L(r_2) \backslash L(r_1) = \{\}  
\end{eqnarray*}
Next, assume that $F_1$ and $F_2$ are deterministic \textsc{Fsms} such that
\\[0.2cm]
\hspace*{1.3cm}
$L(F_1) = L(r_1)$ \quad and \quad $L(F_2) = L(r_2)$
\\[0.2cm]
holds.  We have seen in Chapter \ref{chapter:finit-state-machines.tex} how $F_1$ and $F_2$ can be
constructed from $r_1$ and $r_2$. According to the corollary \ref{kor:mengendif} the languages
$L(r_1) \backslash L(r_2)$ and $L(r_2) \backslash L(r_1)$ are regular and we have seen how to
construct \textsc{Fsm}s $F_{12}$ and $F_{21}$ such that
\\[0.2cm]
\hspace*{1.3cm}
$L(r_1) \backslash L(r_2) = L(F_{12})$ \quad and \quad $L(r_2) \backslash L(r_1) = L(F_{21})$ 
\\[0.2cm]
holds.  Hence we have
\\[0.2cm]
\hspace*{1.3cm}
$r_1 \doteq r_2 \;\Leftrightarrow\; L(F_{12}) = \{\} \wedge  L(F_{21}) = \{\}$
\\[0.2cm]
and according to Section \ref{section:leer} this question is decidable by checking whether any of
the accepting states of $F_{12}$ or $F_{21}$ are reachable from the start state.
\qed

\section{Implementing an Equivalence Checker}
Figure \ref{fig:equivalence.stlx} on page \pageref{fig:equivalence.stlx} shows how to
implement the algorithm sketched in the previous section.  The details are discussed
below.

\begin{figure}[!ht]
\centering
\begin{Verbatim}[ frame         = lines, 
                  framesep      = 0.3cm, 
                  firstnumber   = 1,
                  labelposition = bottomline,
                  numbers       = left,
                  numbersep     = -0.2cm,
                  xleftmargin   = 0.8cm,
                  xrightmargin  = 0.8cm,
                ]
    regExpEquiv := procedure(r1, r2, sigma) {
        fsm1      := regexp2DFA(r1, sigma);
        fsm2      := regexp2DFA(r2, sigma);    
        r1MinusR2 := fsmComplement(fsm1, fsm2);
        r2MinusR1 := fsmComplement(fsm2, fsm1);
        return isEmpty(r1MinusR2) && isEmpty(r2MinusR1);
    };
    regexp2DFA := procedure(r, sigma) {
        converter := regexp2NFA(sigma);
        [states, sigma, delta, q0, qf] := converter.toNFA(r);    
        return nfa2dfa([states, sigma, delta, q0, { qf }]);    
    };
    fsmComplement := procedure(f1, f2) {
        [states1, sigma, delta1, q1, a1] := f1;
        [states2,     _, delta2, q2, a2] := f2;
        states := states1 >< states2;
        delta  := {};
        for ([q1, q2] in states, c in sigma) {
            delta[[q1, q2], c] := [delta1[q1, c], delta2[q2, c]];
        }
        return [states, sigma, delta, [q1, q2], a1 >< (states2 - a2)];
    };
    isEmpty := procedure(fsm) {
        [states, sigma, delta, q0, accepting] := fsm;
        reachable := fixpoint({q0}, q |=> { delta[q, c] : c in sigma });
        return reachable * accepting == {};
    };
\end{Verbatim}
\vspace*{-0.3cm}
\caption{An algorithm to compare regular expressions.}
\label{fig:equivalence.stlx}
\end{figure}


\begin{enumerate}
\item The function \texttt{regExpEquiv} is called with three arguments: 
      \begin{enumerate}
      \item \texttt{r1} and \texttt{r2} are regular expressions that have to be
            compared.  These regular expressions are represented as terms in the same way
            as in Figure \ref{fig:regexp-2-nfa.stlx:toNFA} in the definition of the
            function \texttt{regexp2NFA}.
      \item \texttt{sigma} is the alphabet.
      \end{enumerate}
      The implementation of \texttt{regExpEquiv} is straightforward:
      \begin{enumerate}
      \item \texttt{r1} and \texttt{r2} are converted into deterministic finite state
            machines \texttt{fsm1} and \texttt{fsm2}, respectively.
      \item Next, we construct the finite state machines \texttt{r1MinusR2} and
            \texttt{r2MinusR1}.
            
            \texttt{r1MinusR2} accepts all strings that are accepted by \texttt{fsm1} but
            are rejected by \texttt{fsm2},  while
            \texttt{r2MinusR1} accepts all strings that are accepted by \texttt{fsm2} but
            are rejected by \texttt{fsm1}.  Therefore
            \\[0.2cm]
            \hspace*{1.3cm}
            $L(\mathtt{r1MinusR2}) = L(r_1) \backslash L(r_2)$ \quad and \quad
            $L(\mathtt{r1MinusR2}) = L(r_1) \backslash L(r_2)$.
      \item The regular expressions \texttt{r1} and \texttt{r2} are equivalent iff
            \\[0.2cm]
            \hspace*{1.3cm}
            $L(r_1) \subseteq L(r_2)$ \quad and \quad $L(r_2) \subseteq L(r_1)$
            \\[0.2cm]
            holds.  This is the case if and only if
            \\[0.2cm]
            \hspace*{1.3cm}
            $L(r_1) \backslash L(r_2) = \{\}$ \quad and \quad
            $L(r_2) \backslash L(r_1) = \{\}$
            \\[0.2cm]
            and these conditions are checked using the function \texttt{isEmpty}.
      \end{enumerate}
\item The function \texttt{regexp2DFA} takes a regular expression $r$ together
      with an alphabet \texttt{sigma} and constructs a deterministic finite state machine
      that accepts the language described by $r$.  This is done by first converting
      $r$ into a non-deterministic finite state machine \texttt{nfa} via the function
      \texttt{regexp2NFA}.  The non-deterministic finite state machine \texttt{nfa} is then
      converted into a deterministic finite state machine with the help of the function
      \texttt{nfa2dfa}.  However, there is one minor problem to solve:  The function
      \texttt{toNFA} returns a finite state machine that has exactly one accepting state.
      For this reason this function does return a 5-tuple of the form
      \\[0.2cm]
      \hspace*{1.3cm}
      $\langle Q, \Sigma, \delta, q_0, q_f \rangle$
      \\[0.2cm]
      where $q_f$ is the single accepting state.  However, the function \texttt{nfa2dfa} expects its argument
      to be a 5-tuple of the form
      \\[0.2cm]
      \hspace*{1.3cm}
      $\langle Q, \Sigma, \delta, q_0, A \rangle$
      \\[0.2cm]
      where $A$ is the set of accepting states.  Therefore, we had to extract the components of 
      NFA returned by \texttt{toNFA} and turn the single accepting state $q_f$ into the set 
      $\{ q_f \}$ in order to call the function \texttt{nfa2dfa}.

      The function \texttt{toNFA} has already been shown in
      Figure \ref{fig:regexp-2-nfa.stlx:toNFA} on page
      \pageref{fig:regexp-2-nfa.stlx:toNFA}, while the function \texttt{nfa2dfa}
      is shown in Figure \ref{fig:nfa2dfa.stlx} on page \pageref{fig:nfa2dfa.stlx}.
\item The function \texttt{fsmComplement} has two arguments $f_1$ and $f_2$.  These
      arguments are deterministic finite state machines.  The function returns a new
      finite state machine $F$ that accepts the language
      \\[0.2cm]
      \hspace*{1.3cm}
      $L(f_1) \backslash L(f_2)$.
      \\[0.2cm]
      The finite state machine $F$ simulates the two finite state machines $f_1$ and $f_2$
      in parallel.  Therefore, the states of $F$ are pairs of the form $\pair(p_1, p_2)$
      where $p_1$ is a state of $f_1$ while $p_2$ is a state of $f_2$.  The transition
      function $\delta$ of $F$ is a composition of the transition function $\delta_1$ of
      $f_1$  and $\delta_2$ of $f_2$ that is defined as follows:
      \\[0.2cm]
      \hspace*{1.3cm}
      $\delta\bigr(\pair(q_1, q_2), c\bigr) := \bigl\langle \delta_1(q_1,c),\, \delta_2(q_2,c)\bigr\rangle$.
      \\[0.2cm]
      A state $\pair(p_1, p_2)$ is an accepting state of $F$ iff $p_1$ is an accepting state of $f_1$ but
      $p_2$ is not an accepting state of $f_2$.
\item The input of the function \texttt{isEmpty} is a deterministic finite state machine
      \texttt{fsm}.  The function checks whether the language accepted by \texttt{fsm} is
      the empty set.  To this end, it computes the set of all states that are reachable
      from the start state.  If any of these states is accepting, then the language
      of \texttt{fsm} is not empty.  The computation of the reachable states is done via a fixpoint
      iteration.  The function \texttt{fixpoint} that is used here has already been discussed in
      Figure \ref{fig:nfa2dfa.stlx} on page \pageref{fig:nfa2dfa.stlx}.
\end{enumerate}

\section{Limits of Regular Languages}
In this section we present a theorem that can be used to show that certain languages are
\underline{not} regular.  This theorem is know as the 
\href{https://en.wikipedia.org/wiki/Pumping_lemma_for_regular_languages}{\emph{pumping lemma for regular languages}}.

\begin{Theorem}[Pumping Lemma for Regular Languages] \lb
  Assume $L$ is a regular language.  Then there exists a natural number $n \in \mathbb{N}$ such that
  every string $s \in L$ that has a length of at least $n$ can be split into three substrings $u$,
  $v$, and $w$ such that the following holds:
  \begin{enumerate}
  \item $s= uvw$,
  \item $v \not= \varepsilon$,
  \item $|uv| \leq n$,
  \item $\forall h \in \mathbb{N}_0: uv^hw \in L$.
  \end{enumerate}
  This theorem can be written as a single formula:  If $L$ is a regular language, then 
  \\[0.2cm]
  \hspace*{1.3cm}
  $\exists n \in \mathbb{N}: \forall s \in L : \Bigl(|s| \geq n \rightarrow \exists u,v,w\in \Sigma^* :
   s = uvw \wedge v \not= \varepsilon \wedge |uv| \leq n \wedge 
    \forall h \in \mathbb{N}_0: uv^h w \in L
   \Bigr)
  $.
\end{Theorem}

\proofEng
As $L$ is a regular language, there exists a deterministic \textsc{Fsm}
\\[0.2cm]
\hspace*{1.3cm}
$F = \langle Q, \Sigma, \delta, q_0, A \rangle$,
\\[0.2cm]
such that $L = L(F)$.  The number $n$ whose existence is claimed in the Pumping Lemma is defined as
the number of states of $F$: 
\\[0.2cm]
\hspace*{1.3cm}
$n := \textsl{card}(Q)$.
\\[0.2cm]
Next, assume a string $s \in L$ is given such that $|s| \geq n$.  Then there are $m := |w|$
characters $c_i$ such that
\\[0.2cm]
\hspace*{1.3cm}
$s = c_1 c_2 \cdots c_m$.
\\[0.2cm]
Since $|s| \geq n$ we have $m \geq n$.  On reading the characters $c_i$ the \textsc{Fsm} changes its
states as follows:
\\[0.2cm]
\hspace*{1.3cm}
$q_0 \stackrel{c_1}{\longmapsto} q_1 \stackrel{c_2}{\longmapsto} q_2 \stackrel{c_3}{\longmapsto} \cdots \stackrel{c_m}{\longmapsto} q_m$
\\[0.2cm]
and since we have  $s \in L$ we conclude that  $q_m$ must be an accepting state, i.e.~$q_m \in A$.
As $m \geq n$ and $n$ is the total number of states of $F$, not all of the states 
\\[0.2cm]
\hspace*{1.3cm}
$q_0$, $q_1$, $q_2$, $\cdots$, $q_m$
\\[0.2cm]
can be different.
Because of
\\[0.2cm]
\hspace*{1.3cm}
$\textsl{card}\bigl(\{0,1,\cdots,n\}\bigr) = n+1$
\\[0.2cm]
we know, that even in the list
\\[0.2cm]
\hspace*{1.3cm}
$[q_0,q_1,q_2,\cdots, q_{n}]$
\\[0.2cm]
at least one state has to occur at least twice.  Hence there are $k, l \in \{0,\cdots,n\}$ such that
\\[0.2cm]
\hspace*{1.3cm}
$q_k = q_l \wedge k < l$.
\\[0.2cm]
Next, we define the strings $u$, $v$, and $w$ as follows:
\\[0.2cm]
\hspace*{1.3cm}
$u := c_1 \cdots c_k$, \quad $v := c_{k+1} \cdots c_l$, \quad and \quad $w := c_{l+1} \cdots c_{m}$.
\\[0.2cm]
As $k < l$ we have that $v \not= \varepsilon$ and $l \leq n$ implies $|uv| \leq n$.
Furthermore, we have the following:
\begin{enumerate}
\item Reading the string $u$ changes the state of the \textsc{Fsm} $F$ from the start state $q_0$ to
      the state $q_k$, we have
      \begin{equation}
        \label{eq:pumping1}
        q_0 \stackrel{u}{\longmapsto} q_k.    
      \end{equation}
\item Reading the string $v$ changes the state of the \textsc{Fsm} $F$ from the state $q_k$ to the
      state $q_l$.  As we have $q_l = q_k$, this implies
      \begin{equation}
        \label{eq:pumping2}
      q_k \stackrel{v}{\longmapsto} q_k.        
      \end{equation}
\item Reading the string $w$ changes the state of the \textsc{Fsm} $F$ from the state  $q_l = q_k$
      to the accepting state $q_m$:
      \begin{equation}
        \label{eq:pumping3}
        q_k \stackrel{w}{\longmapsto} q_m.        
      \end{equation}
\end{enumerate}
From $q_k \stackrel{v}{\longmapsto} q_k$ we conclude
\\[0.2cm]
\hspace*{1.3cm}
$q_k \stackrel{v}{\longmapsto} q_k \stackrel{v}{\longmapsto} q_k$, \quad hence \quad $q_k \stackrel{v^2}{\longmapsto} q_k$.
\\[0.2cm]
As we can repeat reading $v$ in state $q_k$ any number of times, we have
\begin{equation}
  \label{eq:pumping4}
  q_k \stackrel{v^h}{\longmapsto} q_k  \quad \mbox{for all $h \in \mathbb{N}_0$.}
\end{equation}
Combining the equations (\ref{eq:pumping1}), (\ref{eq:pumping3}), and (\ref{eq:pumping4})  we have
\\[0.2cm]
\hspace*{1.3cm}
$q_0 \stackrel{u}{\longmapsto} q_k \stackrel{v^h}{\longmapsto} q_k \stackrel{w}{\longmapsto} q_m$.
\\[0.2cm]
This can be condensed to
\\[0.2cm]
\hspace*{1.3cm}
$q_0 \stackrel{uv^hw}{\longmapsto} q_m$
\\[0.2cm]
and since the state $q_m$ is an accepting state we conclude that $uv^hw \in L$ holds for any $h \in \mathbb{N}_0$. \qed



\begin{Proposition}
  The alphabet  $\Sigma$ is defined as $\Sigma = \{ \quoted{a}, \quoted{b} \}$.
  Define the language $L$ as the set of all strings of the form $\mathtt{a}^k\mathtt{b}^k$ where $k$
  is some natural number:
  \\[0.2cm]
  \hspace*{1.3cm}
  $L = \bigl\{ \mathtt{a}^k\mathtt{b}^k \mid k \in \mathbb{N} \bigr\}$.
  \\[0.2cm]
  Then the language  $L$ is not regular.
\end{Proposition}

\proofEng
The proof is a proof by contradiction. We assume that $L$ is a regular language.  According to the
Pumping Lemma there exists a fixed natural number $n$ such that every $s \in L$ that satisfies  $|s|
\geq n$ can be written as
\\[0.2cm]
\hspace*{1.3cm}
$s = uvw$
\\[0.2cm]
such that
\\[0.2cm]
\hspace*{1.3cm}
$|uv| \leq n$, \quad $v \not= \varepsilon$, \quad and \quad $\forall h \in \mathbb{N}_0: uv^h w \in L$
\\[0.2cm]
holds.  Let us define the string $s$ as
\\[0.2cm]
\hspace*{1.3cm}
$s := \mathtt{a}^{n} \mathtt{b}^{n}$.
\\[0.2cm]
Obviously we have $|s| = 2 \cdot n \geq n$.  Hence there are strings $u$, $v$, and $w$
such that 
\\[0.2cm]
\hspace*{1.3cm}
$\mathtt{a}^{n}\mathtt{b}^{n} = uvw$, \quad $|uv| \leq n$, \quad $v \not= \varepsilon$, 
\quad and \quad $\forall h \in \mathbb{N}_0: uv^h w \in L$.
\\[0.2cm]
As $|uv| \leq n$, the string $uv$ is a prefix not only of $s$ but even of $\mathtt{a}^n$. Therefore,
and since $v \not= \varepsilon$ we know that the string $v$ must have the form
\\[0.2cm]
\hspace*{1.3cm}
$v = \mathtt{a}^k$ \quad for some $k \in \mathbb{N}$.
\\[0.2cm]
If we take the formula $\forall h \in \mathbb{N}_0: uv^h w \in L$ and set  $h:=0$, we conclude that
\begin{equation}
  \label{eq:pumping5}
 uw \in L. 
\end{equation}
In order to facilitate our argument, we define the function
\\[0.2cm]
\hspace*{1.3cm}
$\textsl{count}: \Sigma^* \times \Sigma \rightarrow \mathbb{N}_0$.
\\[0.2cm]
Given a  string $t$ and a character $c$ the function $\textsl{count}(t,c)$ counts how often the
character $c$ occurs in the string $t$.  For the language  $L$ we have
\\[0.2cm]
\hspace*{1.3cm}
$t \in L \Rightarrow \textsl{count}(t,\squoted{a}) = \textsl{count}(t, \squoted{b})$. 
\\[0.2cm]
On one hand we have:
\[  
\begin{array}{lcl}
\textsl{count}(uw,\squoted{a}) & = & \textsl{count}(uvw,\squoted{a}) - \textsl{count}(v,\squoted{a}) \\
 & = & \textsl{count}(s,\squoted{a}) - \textsl{count}(v,\squoted{a}) \\
 & = & \textsl{count}(\mathtt{a}^n\mathtt{b}^n,\squoted{a}) - \textsl{count}(\mathtt{a}^k,\squoted{a}) \\
 & = & n - k  \\
 & < & n   \\
\end{array}
\]
But on the other hand we have
\[  
\begin{array}{lcl}
\textsl{count}(uw,\squoted{b}) & = & \textsl{count}(uvw,\squoted{b}) - \textsl{count}(v,\squoted{b}) \\
                               & = & \textsl{count}(s,\squoted{b}) - \textsl{count}(v,\squoted{b}) \\
 & = & \textsl{count}(\mathtt{a}^n\mathtt{b}^n,\squoted{b}) - \textsl{count}(\mathtt{a}^k,\squoted{b}) \\
                               & = & n  - 0\\
                               & = & n  
\end{array}
\]
Therefore, we have
\\[0.2cm]
\hspace*{1.3cm}
$\textsl{count}(uw,\squoted{a}) < \textsl{count}(uw,\squoted{b})$
\\[0.2cm]
and this shows that the string $uw$ is not a member of the language $L$ because for all strings in $L$ 
the same number of occurrences of the character ``\texttt{a}'' is the same as the number of
occurrences of the character ``\texttt{b}''.  This contradiction shows that the language $L$ cannot
be regular.
\qed

\remark
The previous proposition shows that the expressive power of regular languages is quite weak.
We could easily adapt the previous proposition to show that the language
\\[0.2cm]
\hspace*{1.3cm}
$\bigl\{ \mathtt{(}^n \mathtt{)}^n \mid n \in \mathbb{N} \bigr\}$
\\[0.2cm]
is not regular.  Hence, regular expressions are unable to check even such simple questions as to
whether the parentheses in an expressions are balanced.  Therefore, the concept of regular
expressions is not strong enough to describe the syntax of a programming language.
The next chapter introduces the notion of \emph{context-free languages}.  These languages
are powerful enough to describe modern programming languages. 

\exerciseEng
The language  $L_{\mathrm{square}}$ is the set of all strings of the form $\mathtt{a}^n$ where $n$
is a square, we have
\\[0.2cm]
\hspace*{1.3cm}
$L_{\mathrm{square}} = \bigl\{ a^{m} \mid \exists k \in \mathbb{N}_0: m = k^2 \bigr\}$
\\[0.2cm]
Prove that the language  $L_{\mathrm{square}}$ is not a regular language.
\eox
\vspace*{0.1cm}

\noindent
\textbf{Hint}:  When looking for a counter example, you should try to set $h:=2$.


\solution
Wir f\"uhren den Beweis indirekt und nehmen an, dass $L_{\mathrm{square}}$ regul\"ar
w\"are.  Nach dem Pumping-Lemma gibt es dann eine positive nat\"urliche Zahl $n$ (dies war die Anzahl
der Zust\"ande des deterministischen Automaten, der die Sprache erkennt), so dass sich jeder String
$s \in L_{\mathrm{square}}$ mit $|s| \geq n$ in drei Teilstrings $u$, $v$ und $w$ aufspalten l\"asst, so dass gilt:
\begin{enumerate}
\item $s = uvw$,
\item $|uv| \leq n$,
\item $v \not= \varepsilon$,
\item $\forall h \in \mathbb{N}_0: uv^hw \in L_{\mathrm{square}}$. 
\end{enumerate} 
Wir betrachten nun den String $s = a^{n^2}$.  F\"ur die L\"ange dieses Strings gilt offenbar
\[ |s| = \big| a^{n^2} \big| = n^2 \geq n. \]
Also gibt es eine Aufspaltung von $s$ der Form $s = uvw$ mit den oben angegebenen Eigenschaften.
Da $a$ der einzige Buchstabe ist, der in $s$ vorkommt, k\"onnen in $u$, $v$ und $w$ auch keine anderen
Buchstaben vorkommen und es muss nat\"urliche Zahlen $x$, $y$ und $z$ geben, so dass 
\[ u = a^x,\; v = a^y\; \mbox{und}\; w = a^z \]
gilt.  Wir untersuchen, welche Konzequenzen sich daraus f\"ur die Zahlen $x$, $y$ und $z$ ergeben.
\begin{enumerate}
\item Die Zerlegung $s = uvw$ schreibt sich als $a^{n^2} = a^xa^ya^z$ und daraus folgt
      \begin{equation}
        \label{eq:e1}
         n^2 = x + y + z.     
      \end{equation}
\item Die Ungleichung $|uv| \leq n$ ist jetzt \"aquivalent zu $x +y \leq n$, woraus 
      \begin{equation}
        \label{eq:e2}
        y \leq n
      \end{equation}
      folgt.
\item Die Bedingung $v \not= \varepsilon$ liefert
      \begin{equation}
        \label{eq:e3}
        y > 0.
      \end{equation}
\item Aus der Formel $\forall h \in \mathbb{N}_0: uv^hw \in L_{\mathrm{square}}$ folgt
      \begin{equation}
        \label{eq:e4}
        \forall h \in \mathbb{N}_0: a^xa^{y\cdot h}a^z \in L_{\mathrm{square}}. 
      \end{equation}
\end{enumerate}
Die letzte Gleichung muss insbesondere auch f\"ur den Wert $h=2$ gelten:
\[ a^xa^{y\cdot 2}a^z \in L_{\mathrm{square}}.  \]
Nach Definition der Sprache $L_{\mathrm{square}}$ gibt es dann eine nat\"urliche Zahl
$k$, so dass gilt
\begin{equation}
  \label{eq:e5}
  x + 2\cdot y + z = k^2
\end{equation}
Addieren wir in Gleichung (\ref{eq:e1}) auf beiden Seiten $y$, so erhalten wir insgesamt
\[ n^2 + y = x + 2\cdot y + z = k^2. \]
Wegen $y > 0$ folgt daraus 
\begin{equation}
  \label{eq:e6}
  n < k.    
\end{equation}
Andererseits haben wir
\[ 
\begin{array}[t]{lcll}
 k^2  & =    & x + 2 \cdot y + z       & \mbox{nach Gleichung (\ref{eq:e5}})   \\
      & =    & x + y + z + y           & \mbox{elementare Umformung}           \\
      & \leq & x + y + z + n           & \mbox{nach Ungleichung (\ref{eq:e2}}) \\
      & =    & n^2 + n                 & \mbox{nach Gleichung (\ref{eq:e1}})   \\
      & <    & n^2 + 2 \cdot n + 1     & \mbox{da $n+1 > 0$}                   \\ 
      & =    & (n + 1)^2               & \mbox{elementare Umformung}           \\ 
\end{array}
\]
Damit haben wir insgesamt  $k^2 < (n+1)^2$ gezeigt und das impliziert
\begin{equation}
  \label{eq:e7}
  k < n+1.
\end{equation}
Zusammen mit Ungleichung (\ref{eq:e6}) liefert Ungleichung (\ref{eq:e7}) nun die Ungleichungs-Kette
\[ n < k < n + 1. \]
Da andererseits $k$ eine nat\"urliche Zahl sein muss und $n$ ebenfalls eine nat\"urliche Zahl ist,
haben wir jetzt einen Widerspruch, denn zwischen $n$ und $n+1$ gibt es keine nat\"urliche Zahl.
\qed
\pagebreak

\renewcommand{\labelenumi}{(\alph{enumi})}

\exerciseEngStar
The language $L$ is defined as
\\[0.2cm]
\hspace*{1.3cm}
$L := \{ \mathtt{a}^m \mathtt{b}^n \mathtt{c}^n \mid m,n \in \mathbb{N} \} \cup 
      \{ \mathtt{b}^m \mathtt{c}^n \mid m,n \in \mathbb{N} \} 
$.
\begin{enumerate}
\item Prove that $L$ is not regular.
\item Prove that $L$ satisfies the pumping lemma.  \eox
\end{enumerate}

\renewcommand{\labelenumi}{\arabic{enumi}.}

\exerciseEng
Define $\Sigma := \{\mathtt{a},\mathtt{b}\}$.  
Prove that the language
\\[0.2cm]
\hspace*{1.3cm}
$L := \bigl\{ \mathtt{a}^p \mid \mbox{$p$ is a prime number} \bigr\}$
\\[0.2cm]
is not regular.  \eox
\vspace{0.3cm}

\noindent
\textbf{Beweis}:
Wir f\"uhren den Beweis indirekt und nehmen an, $L$ w\"are regul\"ar.  Nach
dem Pumping-Lemma gibt es dann eine Zahl $n$, so dass es f\"ur alle Strings $s \in L$, 
deren L\"ange gr\"o{\ss}er-gleich $n$ ist, eine Zerlegung
\\[0.2cm]
\hspace*{1.3cm}
$s = uvw$
\\[0.2cm]
mit den folgenden drei Eigenschaften gibt:
\begin{enumerate}
\item $v \not= \varepsilon$, 
\item $|uv| \leq n$ \quad und
\item $\forall h \in \mathbb{N}_0: u v^h w \in L$.
\end{enumerate}
Wir w\"ahlen nun eine Primzahl $p$, die gr\"o{\ss}er-gleich  $n + 2$ ist und setzen $s = \mathtt{a}^p$.
Dann gilt $|s| = p \geq n$ und die Voraussetzung des Pumping-Lemmas ist erf\"ullt.
Wir finden also eine Zerlegung von $\mathtt{a}^p$ der Form
\\[0.2cm]
\hspace*{1.3cm}
$\mathtt{a}^p = uvw$ 
\\[0.2cm]
mit den oben angegebenen Eigenschaften.
Aufgrund der Gleichung $s = uvw$ k\"onnen die Teilstrings $u$, $v$ und $w$ nur aus dem
Buchstaben ``\texttt{a}'' bestehen.  Also gibt es nat\"urliche Zahlen $x$, $y$, und
$z$ so dass gilt:
\\[0.2cm]
\hspace*{1.3cm}
$u = \mathtt{a}^x$, \quad $v = \mathtt{a}^y$ \quad und \quad $w = \mathtt{a}^z$.
\\[0.2cm]
F\"ur  $x$, $y$ und $z$ gilt dann Folgendes:
\begin{enumerate}
\item $x + y + z = p$,
\item $y \not= 0$,
\item $x + y \leq n$,
\item $\forall h \in \mathbb{N}_0: x + h \cdot y + z \in \mathbb{P}$.
\end{enumerate}
Setzen wir in der letzten Gleichung f\"ur $h$ den Wert $(x + z)$ ein, so erhalten wir
\\[0.2cm]
\hspace*{1.3cm}
$x + (x + z)\cdot y + z \in P$.
\\[0.2cm]
Wegen $x + (x + z)\cdot y + z = (x + z) \cdot (1 + y)$ h\"atten wir dann
\\[0.2cm]
\hspace*{1.3cm}
$(x + z) \cdot (1 + y) \in \mathbb{P}$.
\\[0.2cm]
Das kann aber nicht sein, denn wegen $y > 0$ ist der Faktor $1 + y$ von 1
verschieden und wegen $x + y \leq n$ und $x + y + z = p$ und $p \geq n + 2$ wissen wir, dass
$z \geq 2$ ist, so dass auch der Faktor $(x + z)$ von 1 verschieden ist.  Damit kann das Produkt
$(x + z) \cdot (1 + y)$ aber keine Primzahl mehr sein und wir haben einen Widerspruch zu der
Annahme, dass $L$ regul\"ar ist. \qed


\exercise
Die Sprache $L_{\mathrm{power}}$ beinhaltet alle W\"orter der Form $\mathtt{a}^n$ f\"ur die $n$ eine
Zweier-Potenz ist, es gilt also
\\[0.2cm]
\hspace*{1.3cm}
$L_{\mathrm{power}} = \bigl\{ \mathtt{a}^{2^k} \mid k \in \mathbb{N}_0 \bigr\}$
\\[0.2cm]
Zeigen Sie mit Hilfe des Pumping-Lemmas, dass die Sprache $L_{\mathrm{power}}$ keine regul\"are Sprache ist.
\eox

\exercise
F\"ur zwei nat\"urliche Zahlen $k$ und $l$ bezeichne der Ausdruck $\mathtt{ggt}(k, l)$ den gr\"o{\ss}ten
gemeinsamen Teiler von $k$ und $l$.  Wir definieren die Sprache $L_{\mathrm{ggt}}$ als
\\[0.2cm]
\hspace*{1.3cm}
$L_{\mathrm{ggt}} := \bigl\{ \mathtt{a}^k\mathtt{b}^l \mid k \in \mathbb{N} \wedge l \in \mathbb{N} \wedge \mathtt{ggt}(k, l) = 1 \bigr\}$.
\\[0.2cm]
Zeigen Sie, dass die Sprache $L_{\mathrm{ggt}}$ keine regul\"are Sprache ist.
\vspace{0.2cm}

\noindent
\textbf{Hinweis}:  Nutzen Sie aus, dass regul\"are Sprachen unter Komplement-Bildung abgeschlossen sind.
\eox


\paragraph{Historical Remark}
The pumping lemma is a special case of a general theorem that has been proved by Bar-Hillel, Perles and
Shamir in 1961 \cite{barhillel:1961}.


%%% Local Variables: 
%%% mode: latex
%%% TeX-master: "formal-languages.tex"
%%% End: 
