\chapter{Assembler}
A compiler translates programs written in a high level language like \texttt{C} or
\textsl{Java} into some low level representation.  This low level representation can be
either machine code or some form of assembler code.  For the programming language \textsl{Java},
the command \texttt{javac} compiles a program written in \textsl{Java} into
\textsl{Java} byte code.  This byte code is then executed using the 
\href{http://en.wikipedia.org/wiki/Java_virtual_machine}{\textsl{Java} virtual machine}
(\textsc{Jvm}).

The compiler that we are going to
develop in the next chapter generates a particular form of assembler code know as
\href{http://jasmin.sourceforge.net}{\textsc{Jvm} assembler code}.
This assembler code can be translated directly into
\href{http://en.wikipedia.org/wiki/Java_bytecode}{\textsl{Java} byte code}, which 
is also the byte code generated by \texttt{javac}.  The program for translating \textsc{Jvm}
assembler into bytecode is called \href{http://jasmin.sourceforge.net}{\textsl{Jasmin}}.
The  byte code produced by \textsl{Jasmin} can be executed using the
\texttt{java} command just like any other ``\texttt{.class}''-file.  
This chapter will discuss the syntax and
semantics of \textsl{Jasmin} assembler code.



\section{Introduction into \textsc{Jasmin} Assembler}
To get used to the syntax of \textsl{Jasmin} assembler, we start with a small program that prints the
string ``\texttt{Hello World!}'' on the standard output stream.  Figure \ref{fig:Hello.jas} on page
\pageref{fig:Hello.jas} shows this program. We discuss this program line by line.

\begin{figure}[!ht]
\centering
\begin{Verbatim}[ frame         = lines, 
                  framesep      = 0.3cm, 
                  firstnumber   = 1,
                  labelposition = bottomline,
                  numbers       = left,
                  numbersep     = -0.2cm,
                  xleftmargin   = 0.8cm,
                  xrightmargin  = 0.8cm,
                ]
    .class public Hello
    .super java/lang/Object
    
    .method public <init>()V
        aload 0
        invokenonvirtual java/lang/Object/<init>()V
        return
    .end method
    
    .method public static main([Ljava/lang/String;)V
        .limit locals 1
        .limit stack  2
        getstatic     java/lang/System/out Ljava/io/PrintStream;
        ldc           "Hello World!"
        invokevirtual java/io/PrintStream/println(Ljava/lang/String;)V 
        return
    .end method
\end{Verbatim}
\vspace*{-0.3cm}
\caption{An assembler program to print ``\texttt{Hello World!}''.}
\label{fig:Hello.jas}
\end{figure}

\begin{enumerate}
\item Line 1 defines the name of the class file that is to be produced by the assembler.
      In this case, the class name is \texttt{Hello}.  Therefore, \textsl{Jasmin} will translate
      this file into the class file \texttt{Hello.class}.
\item Line 2 names the super class.  In our examples, this will always be the class
      \texttt{Object}.  Since this class resides in the package \texttt{java.lang}, the super class
      has to be specified as
      \\[0.2cm]
      \hspace*{1.3cm}
      \texttt{java/lang/Object}.
      \\[0.2cm]
      Observe that the character ``\texttt{.}'' in the class name ``\texttt{java.lang.Object}''
      has to be replaced by the character ``\texttt{/}''.
\item Line 4 to 8 initialize the program.  This code is always the same and corresponds to a
      constructor for the class \texttt{Hello}.  As this code is copied verbatim to the beginning 
      of every class file, we will not discuss it further. 
\item Line 10 -- 17 defines the method \texttt{main} that does the actual work.
  \begin{enumerate}
  \item Line 10 declares the name of the method and its signature.  The string
        \\[0.2cm]
        \hspace*{1.3cm}
        \texttt{main([Ljava/lang/String;)V}
        \\[0.2cm]
        specifies the signature:
        \begin{enumerate}
        \item The string ``\texttt{main}''  is the name of the method that is defined.
        \item The string ``\texttt{[}'' specifies that the first argument is an array.
        \item The string ``\texttt{L}'' specifies that this array consists of objects.
        \item The string ``\texttt{java/lang/String;}'' specifies that these objects are of class
              \texttt{java.lang.String}.
        \item The string ``\texttt{V}'' specifies that the return type of the method \texttt{main}
              is void.
       \end{enumerate}
  \item Line 11 declares that the method \texttt{main} uses one local variable.  This variable
        corresponds to the argument of the method main.  The assembler file shown in Figure
        \ref{fig:Hello.jas} on page \pageref{fig:Hello.jas} corresponds to the \textsl{Java} code
        shown in Figure \ref{fig:Hello.java} below.  The method \texttt{main} has one local
        variable, which is the argument \texttt{args}.
        The information on the number of local variables is needed by the \textsl{Java} virtual
        machine in order to allocate room for these variables on the stack.

\begin{figure}[!ht]
\centering
\begin{Verbatim}[ frame         = lines, 
                  framesep      = 0.3cm, 
                  firstnumber   = 1,
                  labelposition = bottomline,
                  numbers       = left,
                  numbersep     = -0.2cm,
                  xleftmargin   = 0.8cm,
                  xrightmargin  = 0.8cm,
                ]
    public class Hello {
        public static void main(String[] args) {
    	System.out.println("Hello World!");
        }
    }
\end{Verbatim}
\vspace*{-0.3cm}
\caption{Printing \texttt{Hello world} in \textsl{Java}.} 
\label{fig:Hello.java}
\end{figure}

  \item Line 12 specifies that the stack contains a maximum of two objects.  It is easy to see that
        this is true, since line 13 pushes the object
        \\[0.2cm]
        \hspace*{1.3cm}
        \texttt{java.lang.System.out}
        \\[0.2cm]
        onto the stack.  This is an Object of class \texttt{java.io.PrintStream}.
        Then, line 14 pushes a reference to the string ``\texttt{Hello World}'' onto the stack.
  \item Line 15 calls the method \texttt{println}, which is a method of the class
        \texttt{java.io.PrintStream}.  
  \item Line 16 returns from the method \texttt{main}.
  \item Line 17 is a declaration signifying the end of the code corresponding to the method 
        \texttt{main}.
  \end{enumerate}
\end{enumerate}
In order to execute this assembler program, we first have to translate the assembler program into a
\texttt{class}-file.  This can be done using the command
\\[0.2cm]
\hspace*{1.3cm}
\texttt{java -jar /usr/local/lib/jasmin.jar Hello.jas}
\\[0.2cm]
Of course, this command only works if the directory \texttt{/usr/local/lib/} does contain the file
\texttt{jasmin.jar}.  Executing this command creates the file \texttt{Hello.class}.  This class file
can then be executed by typing
\\[0.2cm]
\hspace*{1.3cm}
\texttt{java Hello}
\\[0.2cm]
in the command line, provided the environment variable \texttt{CLASSPATH} contains the current
directory, i.e.~the \texttt{CLASSPATH} has to contain the directory ``\texttt{.}''.

We will proceed to discuss the different assembler commands in more detail later.  To this end, we
first have to discuss some background: One of the design goal of the programming language
\textsl{Java} was compatibility.  The idea was 
that it should be possible to execute \textsl{Java} class files on any computer.  Therefore, the
\textsl{Java} designers decided to create a so called 
\href{http://en.wikipedia.org/wiki/Virtual_machine}{\emph{virtual machine}}.  A virtual machine is
a computer architecture that, instead of being implemented in silicon, simulated. 
Programs written in \textsl{Java} are first compiled into so called \emph{class files}.  These class
files correspond to the machine code of the \textsl{Java} virtual machine (\textsc{Jvm}).  
The architecture of the virtual machine is a 
\href{http://en.wikipedia.org/wiki/Stack_machine}{\emph{stack machine}}. 
A stack machine does not have any registers to store variables.  Instead, there is a stack and all
variables reside on the stack.  Any command takes its arguments from the top of the stack and
replaces these arguments with the result of the operation performed by the command.
For example, if we want to add two variables, then we first have to
load these variables onto the stack.  Next, performing the add operation will replace these two
arguments with their sum.


\section{Assembler Instructions}
We proceed to discuss some of the \textsc{JVM} instructions.  Since there are more than 160 in
total, we can only discuss a subset of these instructions.  We restrict ourselves to those
instructions that deal with integers:  For example, there is an instruction \texttt{iadd} that adds
two 32 bit integers.  There are also instructions like \texttt{fadd} that adds floating point
numbers and \texttt{dadd} that adds double precision floating point values but we won't discuss
them.  Before we are able to discuss the different instructions we have to discuss how the main
memory is organized in the \textsc{Jvm}: In the \textsc{Jvm}, the memory is split into three parts:
\begin{enumerate}
\item The \emph{program memory} contains the program code as a sequence of bytes.
\item The operands of the different machine instructions are put onto the \emph{stack}.
      Furthermore, the stack contains the arguments and the local variables of a procedure.
      However, in the context of the \textsc{Jvm} the procedures are called \emph{methods}.

      The register \texttt{SP} points to the top of the stack.
      If a method is called, the arguments of the method are placed on the stack.  
      The register \texttt{LV} (\emph{local variables}) points to the first argument of the current  
      method.    On top of the arguments, the local variables of the method are put on the stack.
      Both the arguments and the local variables can be accessed via the register \texttt{LV}
      by specifying their offset from the first argument.
      We will discuss the register \texttt{LV} in more detail when we discuss the invocation of methods. 
\item The \emph{heap} is used for dynamically allocated memory.  Newly created objects are
      located in the heap.
\item The \emph{constant pool} contains the definitions of constants and also the adresses of
      methods in program memory.
\end{enumerate}
In the following, we will be mostly concerned with the stack and the heap.
We proceed to discuss some of the assembler instructions.

\subsubsection{Arithmetical and Logical Instructions}

\paragraph{The instruction \squoted{\texttt{iadd}}}
adds those values that are on top of the stack and replaces these values by their sum.
Figure \ref{fig:ijvm-add} on page \pageref{fig:ijvm-add} show how this command works.
The left part of the figure shows the stack as it is before the command \texttt{iadd} is executed,
while the right part of the figure depicts the situation after the execution of this command.  Note
that after the command is executed, the stack pointer points to the position where formerly the
first argument had been stored.


\setlength{\unitlength}{0.5cm}
\begin{figure}[!ht]
  \centering
\framebox{
\begin{picture}(20,14)
\put(4.5,1){before \texttt{iadd}}
\put(4.91,3){$\vdots$}
\put(7.91,3){$\vdots$}
\put(5,4){\vector(0,1){8}}
\put(8,4){\vector(0,1){8}}
\put(5,8){\line(1,0){3}}
\put(5,9){\line(1,0){3}}
\put(5,10){\line(1,0){3}}
\put(2,9.3){\texttt{SP}}
\put(3.2,9.5){\vector(1,0){1.5}}
\put(6.3,9.3){$b$}
\put(6.3,8.3){$a$}

\put(13.8,1){after \texttt{iadd}}
\put(13.91,3){$\vdots$}
\put(16.91,3){$\vdots$}
\put(14,4){\vector(0,1){8}}
\put(17,4){\vector(0,1){8}}
\put(14,8){\line(1,0){3}}
\put(14,9){\line(1,0){3}}
\put(14,10){\line(1,0){3}}
\put(11,8.3){\texttt{SP}}
\put(12.2,8.5){\vector(1,0){1.5}}
%\put(15.3,9.3){$b$}
\put(14.7,8.3){$a + b$}

\end{picture}}
  \caption{The effect of \texttt{iadd}.}
  \label{fig:ijvm-add}
\end{figure}

\paragraph{The instruction \squoted{\texttt{isub}}}
subtracts the integer value on top of the stack from the value that is found on the position next to
the top of the stack.
Figure \ref{fig:ijvm-sub} on page \pageref{fig:ijvm-sub} pictures this command.
The left part of the figure shows the stack as it is before the command \texttt{isub} is executed,
while the right part of the figure depicts the situation after the execution of this command.  Note
that after the command is executed, the stack pointer points to the position where formerly the
first argument had been stored.



\setlength{\unitlength}{0.5cm}
\begin{figure}[!ht]
  \centering
\framebox{
\begin{picture}(20,14)
\put(4.5,1){before \texttt{isub}}
\put(4.91,3){$\vdots$}
\put(7.91,3){$\vdots$}
\put(5,4){\vector(0,1){8}}
\put(8,4){\vector(0,1){8}}
\put(5,8){\line(1,0){3}}
\put(5,9){\line(1,0){3}}
\put(5,10){\line(1,0){3}}
\put(2,9.3){\texttt{SP}}
\put(3.2,9.5){\vector(1,0){1.5}}
\put(6.3,9.3){$b$}
\put(6.3,8.3){$a$}

\put(13.8,1){after \texttt{isub}}
\put(13.91,3){$\vdots$}
\put(16.91,3){$\vdots$}
\put(14,4){\vector(0,1){8}}
\put(17,4){\vector(0,1){8}}
\put(14,8){\line(1,0){3}}
\put(14,9){\line(1,0){3}}
\put(14,10){\line(1,0){3}}
\put(11,8.3){\texttt{SP}}
\put(12.2,8.5){\vector(1,0){1.5}}
%\put(15.3,9.3){$b$}
\put(14.7,8.3){$a - b$}

\end{picture}}
  \caption{The effect of \texttt{isub}.}
  \label{fig:ijvm-sub}
\end{figure}


\paragraph{The instruction \squoted{\texttt{imul}}}
multiplies the two integer values which are on top of the stack.  This instruction works similar to
the instruction \texttt{iadd}.  If the product does not fit in 32 bits, only the lowest 32 bits of
the result are written onto the stack. 




\paragraph{The instruction \squoted{\texttt{idiv}}}
divides the integer value that is found on the position next to the  top of the stack by the value
on top of the stack. 


\paragraph{The instruction \squoted{\texttt{irem}}}
computes the remainder $a \,\texttt{\%}\, b$ of the division of $a$ by $b$ where $a$ and $b$ are
integer values found on top of the stack.


\paragraph{The instruction \squoted{\texttt{iand}}}
computes the bitwise conjunction of the values on top of the stack.


\paragraph{The instruction \squoted{\texttt{ior}}}
computes the bitwise disjunction of the integer values that are on top of the stack.


\paragraph{The instruction \squoted{\texttt{ixor}}}
computes the bitwise \href{http://en.wikipedia.org/wiki/Exclusive_or}{\emph{exclusive or}} of the integer values that are on top of the stack.

\subsubsection{Shift Instructions}

\paragraph{The instruction \squoted{\texttt{ishl}}}
shifts the value $a$ to the left by $b[4:0]$ bits.  Here, $a$ and $b$ are assumed to be the two
values on top of the stack:  $b$ is the value on top of the stack and $a$ is the value below $b$.
$b[4:0]$ denotes the natural number that results from the 5 lowest bits of $b$.  Figure
\ref{fig:ijvm-shl} on page \pageref{fig:ijvm-shl} pictures this command.

\setlength{\unitlength}{0.5cm}
\begin{figure}[!ht]
  \centering
\framebox{
\begin{picture}(20,14)
\put(4.5,1){before \texttt{ishl}}
\put(4.91,3){$\vdots$}
\put(7.91,3){$\vdots$}
\put(5,4){\vector(0,1){8}}
\put(8,4){\vector(0,1){8}}
\put(5,8){\line(1,0){3}}
\put(5,9){\line(1,0){3}}
\put(5,10){\line(1,0){3}}
\put(2,9.3){\texttt{SP}}
\put(3.2,9.5){\vector(1,0){1.5}}
\put(6.3,9.3){$b$}
\put(6.3,8.3){$a$}

\put(13.8,1){after \texttt{ishl}}
\put(13.91,3){$\vdots$}
\put(18.91,3){$\vdots$}
\put(14,4){\vector(0,1){8}}
\put(19,4){\vector(0,1){8}}
\put(14,8){\line(1,0){5}}
\put(14,9){\line(1,0){5}}
\put(14,10){\line(1,0){5}}
\put(11,8.3){\texttt{SP}}
\put(12.2,8.5){\vector(1,0){1.5}}
%\put(15.3,9.3){$b$}
\put(14.7,8.3){$a \,\texttt{<<}\, b[4:0]$}

\end{picture}}
  \caption{The effect of \texttt{ishl}.}
  \label{fig:ijvm-shl}
\end{figure}

\paragraph{The instruction \squoted{\texttt{ishr}}}
shifts the value $a$ to the right by $b[4:0]$ bits.  Here, $a$ and $b$ are assumed to be the two
values on top of the stack:  $b$ is the value on top of the stack and $a$ is the value below $b$.
$b[4:0]$ denotes the natural number that results from the 5 lowest bits of $b$.  
Note that this instruction performs an 
\href{http://en.wikipedia.org/wiki/Arithmetic_shift}{\emph{arithmetic shift}}, 
i.e.~the sign bit is preserved.

\subsection{Instructions to Manipulate the Stack}

\paragraph{The instruction \squoted{\texttt{dup}}}
duplicates the value that is on top of the stack.  Figure \ref{fig:ijvm-dup} on page
\pageref{fig:ijvm-dup} pictures this command.

\setlength{\unitlength}{0.5cm}
\begin{figure}[!ht]
  \centering
\framebox{
\begin{picture}(20,14)
\put(4.5,1){before \texttt{dup}}
\put(4.91,3){$\vdots$}
\put(7.91,3){$\vdots$}
\put(5,4){\vector(0,1){8}}
\put(8,4){\vector(0,1){8}}
\put(5,8){\line(1,0){3}}
\put(5,9){\line(1,0){3}}
\put(2,8.3){\texttt{SP}}
\put(3.2,8.5){\vector(1,0){1.5}}
\put(6.3,8.3){$a$}

\put(14.3,1){after \texttt{dup}}
\put(13.91,3){$\vdots$}
\put(16.91,3){$\vdots$}
\put(14,4){\vector(0,1){8}}
\put(17,4){\vector(0,1){8}}
\put(14,8){\line(1,0){3}}
\put(14,9){\line(1,0){3}}
\put(14,10){\line(1,0){3}}
\put(11,9.3){\texttt{SP}}
\put(12.2,9.5){\vector(1,0){1.5}}
\put(15.3,9.3){$a$}
\put(15.3,8.3){$a$}

\end{picture}}
  \caption{The effect of \texttt{dup}.}
  \label{fig:ijvm-dup}
\end{figure}


\paragraph{The instruction \squoted{\texttt{pop}}}
removes the value that is on top of the stack.  Figure \ref{fig:ijvm-pop} on page
\pageref{fig:ijvm-pop} pictures this command.


\setlength{\unitlength}{0.5cm}
\begin{figure}[!ht]
  \centering
\framebox{
\begin{picture}(20,14)
\put(4.5,1){before \texttt{pop}}
\put(4.91,3){$\vdots$}
\put(7.91,3){$\vdots$}
\put(5,4){\vector(0,1){8}}
\put(8,4){\vector(0,1){8}}
\put(5,8){\line(1,0){3}}
\put(5,9){\line(1,0){3}}
\put(5,10){\line(1,0){3}}
\put(2,9.3){\texttt{SP}}
\put(3.2,9.5){\vector(1,0){1.5}}
\put(6.3,9.3){$b$}
\put(6.3,8.3){$a$}

\put(14.3,1){after \texttt{pop}}
\put(13.91,3){$\vdots$}
\put(16.91,3){$\vdots$}
\put(14,4){\vector(0,1){8}}
\put(17,4){\vector(0,1){8}}
\put(14,8){\line(1,0){3}}
\put(14,9){\line(1,0){3}}
\put(14,10){\line(1,0){3}}
\put(11,8.3){\texttt{SP}}
\put(12.2,8.5){\vector(1,0){1.5}}
\put(15.3,9.3){$b$}
\put(15.3,8.3){$a$}

\end{picture}}
  \caption{The effect of \texttt{pop}.}
  \label{fig:ijvm-pop}
\end{figure}


\paragraph{The instruction \squoted{\texttt{nop}}}
does nothing.  The name is short for \squoted{no operation}.


\paragraph{The instruction \squoted{\texttt{bipush} $b$}}
pushed the byte $b$ that is given as argument onto the stack.
Figure \ref{fig:ijvm-bipush} on page \pageref{fig:ijvm-bipush} pictures this command.


\setlength{\unitlength}{0.5cm}
\begin{figure}[!ht]
  \centering
\framebox{
\begin{picture}(20,14)
\put(4.5,1){before \texttt{bipush} $b$}
\put(4.91,3){$\vdots$}
\put(7.91,3){$\vdots$}
\put(5,4){\vector(0,1){8}}
\put(8,4){\vector(0,1){8}}
\put(5,8){\line(1,0){3}}
\put(5,9){\line(1,0){3}}
\put(2,8.3){\texttt{SP}}
\put(3.2,8.5){\vector(1,0){1.5}}
\put(6.3,8.3){$a$}

\put(13.8,1){after \texttt{bipush} $b$}
\put(13.91,3){$\vdots$}
\put(16.91,3){$\vdots$}
\put(14,4){\vector(0,1){8}}
\put(17,4){\vector(0,1){8}}
\put(14,8){\line(1,0){3}}
\put(14,9){\line(1,0){3}}
\put(14,10){\line(1,0){3}}
\put(11,9.3){\texttt{SP}}
\put(12.2,9.5){\vector(1,0){1.5}}
\put(15.3,9.3){$b$}
\put(15.3,8.3){$a$}

\end{picture}}
  \caption{The effect of \texttt{bipush} $b$.}
  \label{fig:ijvm-bipush}
\end{figure}
 



\paragraph{The instruction \squoted{\texttt{getstatic} $v$ $c$}}
takes two parameters:  $v$ is the name of a static variable and $c$ is the type of this variable.
For example,  
\\[0.2cm]
\hspace*{1.3cm}
\texttt{getstatic java/lang/System/out Ljava/io/PrintStream;}
\\[0.2cm] 
pushes a reference to the \texttt{PrintStream} that is known as 
\\[0.2cm]
\hspace*{1.3cm}
\texttt{java.lang.System.out}
\\[0.2cm]
onto the stack.


\paragraph{The instruction \squoted{\texttt{iload} $v$}}
reads the local variable with index  $v$ and pushes it on top of the stack.
Figure \ref{fig:ijvm-load} on page \pageref{fig:ijvm-load} pictures this command.
Note that \texttt{LV} denotes the register pointing to the beginning of the local variables of a
method. 
  

\setlength{\unitlength}{0.5cm}
\begin{figure}[!ht]
  \centering
\framebox{
\begin{picture}(20,18)
\put(4.5,1){before \texttt{iload} $v$}
\put(4.91,3){$\vdots$}
\put(7.91,3){$\vdots$}
\put(5,4){\vector(0,1){12}}
\put(8,4){\vector(0,1){12}}
\put(5,12){\line(1,0){3}}
\put(5,13){\line(1,0){3}}
\put(2,12.3){\texttt{SP}}
\put(3.2,12.5){\vector(1,0){1.5}}
\put(6.3,12.3){$a$}

\put(0.7,8.3){$\texttt{LV}+v$}
\put(3.2,8.5){\vector(1,0){1.5}}
\put(5,8){\line(1,0){3}}
\put(5,9){\line(1,0){3}}
\put(6.3,8.3){$y$}

\put(2,5.3){\texttt{LV}}
\put(3.2,5.5){\vector(1,0){1.5}}
\put(5,5){\line(1,0){3}}
\put(5,6){\line(1,0){3}}
\put(6.3,5.3){$x$}

\put(13.8,1){after \texttt{iload} $v$}
\put(13.91,3){$\vdots$}
\put(16.91,3){$\vdots$}
\put(14,4){\vector(0,1){12}}
\put(17,4){\vector(0,1){12}}
\put(14,12){\line(1,0){3}}
\put(14,13){\line(1,0){3}}
\put(14,14){\line(1,0){3}}
\put(11,13.3){\texttt{SP}}
\put(12.2,13.5){\vector(1,0){1.5}}
\put(15.3,12.3){$a$}
\put(15.3,13.3){$y$}

\put(9.7,8.3){$\texttt{LV}+v$}
\put(12.2,8.5){\vector(1,0){1.5}}
\put(14,8){\line(1,0){3}}
\put(14,9){\line(1,0){3}}
\put(15.3,8.3){$y$}

\put(11,5.3){\texttt{LV}}
\put(12.2,5.5){\vector(1,0){1.5}}
\put(14,5){\line(1,0){3}}
\put(14,6){\line(1,0){3}}
\put(15.3,5.3){$x$}

\end{picture}}
  \caption{The effect of \texttt{iload} $v$.}
  \label{fig:ijvm-load}
\end{figure}

\paragraph{The instruction \squoted{\texttt{istore} $v$}}
removes the value which is on top of the stack and stores this value at the location for the local
variable with number $v$.  Hence, $v$ is interpreted as an index into the local variable table of
the method.
Figure \ref{fig:ijvm-store} on page \pageref{fig:ijvm-store} pictures this command.


\setlength{\unitlength}{0.5cm}
\begin{figure}[!ht]
  \centering
\framebox{
\begin{picture}(20,18)
\put(4.5,1){before \texttt{istore} $v$}
\put(4.91,3){$\vdots$}
\put(7.91,3){$\vdots$}
\put(5,4){\vector(0,1){12}}
\put(8,4){\vector(0,1){12}}
\put(5,12){\line(1,0){3}}
\put(5,13){\line(1,0){3}}
\put(5,14){\line(1,0){3}}
\put(2,13.3){\texttt{SP}}
\put(3.2,13.5){\vector(1,0){1.5}}
\put(6.3,12.3){$a$}
\put(6.3,13.3){$b$}

\put(0.7,8.3){$\texttt{LV}+v$}
\put(3.2,8.5){\vector(1,0){1.5}}
\put(5,8){\line(1,0){3}}
\put(5,9){\line(1,0){3}}
\put(6.3,8.3){$y$}

\put(2,5.3){\texttt{LV}}
\put(3.2,5.5){\vector(1,0){1.5}}
\put(5,5){\line(1,0){3}}
\put(5,6){\line(1,0){3}}
\put(6.3,5.3){$x$}

\put(14.3,1){after \texttt{istore} $v$}
\put(13.91,3){$\vdots$}
\put(16.91,3){$\vdots$}
\put(14,4){\vector(0,1){12}}
\put(17,4){\vector(0,1){12}}
\put(14,12){\line(1,0){3}}
\put(14,13){\line(1,0){3}}
\put(14,14){\line(1,0){3}}
\put(11,12.3){\texttt{SP}}
\put(12.2,12.5){\vector(1,0){1.5}}
\put(15.3,12.3){$a$}
\put(15.3,13.3){$b$}

\put(9.7,8.3){$\texttt{LV}+v$}
\put(12.2,8.5){\vector(1,0){1.5}}
\put(14,8){\line(1,0){3}}
\put(14,9){\line(1,0){3}}
\put(15.3,8.3){$b$}

\put(11,5.3){\texttt{LV}}
\put(12.2,5.5){\vector(1,0){1.5}}
\put(14,5){\line(1,0){3}}
\put(14,6){\line(1,0){3}}
\put(15.3,5.3){$x$}

\end{picture}}
  \caption{The effect of \texttt{istore} $v$.}
  \label{fig:ijvm-store}
\end{figure}


\paragraph{The instruction \squoted{\texttt{ldc} $c$}}
pushes the constant $c$ onto the stack.  This constant can be an integer, a single precision floating
point number, or a (pointer to) a string.  If $c$ is a string, this string is actually stored in the
so called \emph{constant pool} and in this case the command ``\texttt{ldc} $c$'' will only push a
pointer to the string onto the stack.
 
\subsubsection{Branching Commands}
In this subsection we discuss those commands that change the control flow.

\paragraph{The instruction \squoted{\texttt{goto} $l$}}
jumps to the label $l$.  Here the label $l$ is a label name that has to be declared inside the
method containing this \texttt{goto} command.  A label with name \textsl{target} is declared using the syntax
\\[0.2cm]
\hspace*{1.3cm}
\textsl{target}\texttt{:}
\\[0.2cm]
The next section presents an example assembler program that demonstrates this command.
 
\paragraph{The instruction \squoted{\texttt{if\_icmpeq} $l$}}
checks whether the value on top of the stack is the same as the value preceding it.  If this is the case, the program will branch
to the label $l$.  Otherwise, the control flow is not changed.  Observe that both values that are
compared are removed from the stack.

\paragraph{The instruction \squoted{\texttt{if\_icmpne} $l$}}
checks whether the value on top of the stack is different from the value preceding it.  If this is the case, the program will branch
to the label $l$.  Otherwise, the control flow is not changed.  Observe that both values that are
compared are removed from the stack.

\paragraph{The instruction \squoted{\texttt{if\_icmplt} $l$}}
checks whether the value that is below the top of the stack is less than the value on top of the stack.  If this is the case, the program will branch
to the label $l$.  Otherwise, the control flow is not changed.  Observe that both values that are
compared are removed from the stack.

\paragraph{The instruction \squoted{\texttt{if\_icmple} $l$}}
checks whether the value that is below the top of the stack is less or equal than the value on top
of the stack.  If this is the case, the program will branch 
to the label $l$.  Otherwise, the control flow is not changed.  Observe that both values that are
compared are removed from the stack.

There are similar commands called \texttt{if\_icmpgt} and \texttt{if\_icmpge}.

\paragraph{The instruction \squoted{\texttt{ifeq} $l$}}
checks whether the value on top of the stack is zero.  If this is the case, the program will branch
to the label $l$.  Otherwise, the control flow is not changed.  Observe that the value that is
tested is removed from the stack.


\paragraph{The instruction \squoted{\texttt{iflt} $l$}}
checks whether the value on top of the stack is less than zero.  If this is the case, the program
will branch to the label $l$.  Otherwise, the control flow is not changed.  Observe that the value
that is tested is removed from the stack.


\paragraph{The instruction \squoted{\texttt{invokevirtual} $m$}}
is used to call the method $m$.  Here $m$ has to fully specify the name of the method.
For example, in order to invoke the method \texttt{println} of the class
\texttt{java.io.PrintStream} we have to write 
\\[0.2cm]
\hspace*{1.3cm}
\texttt{invokevirtual java/io/PrintStream/println(I)V}
\\[0.2cm]
Before the command \texttt{invokevirtual} is executed, we have to put the arguments of the method
onto the stack.  For example, if we want to invoke the method \texttt{println} that takes an integer
argument, we first have to push an object of type \texttt{PrintStream} onto the stack.  Furthermore,
we need to push an integer onto the stack.

\paragraph{The instruction \squoted{\texttt{invokestatic} $m$}}
is used to call the method $m$.  Here $m$ has to fully specify the name of the method.
Furthermore, $m$ needs to be s static method.
For example, in order to invoke a  method called \texttt{sum} that resides in the class \texttt{Sum}
and that takes one integer argument we have to write 
\\[0.2cm]
\hspace*{1.3cm}
\texttt{invokestatic  Sum/sum(I)I}
\\[0.2cm]
In the type specification ``\texttt{sum(I)I}'' the first ``\texttt{I}'' specifies that \texttt{sum}
takes one integer argument, while the second ``\texttt{I}'' specifies that the method \texttt{sum}
returns an integer.

Before the command \texttt{invokestatic} is executed, we have to put the arguments of the method
onto the stack.  For example, if we want to invoke the method \texttt{sum} described  above, then we
have to push an integer onto the stack.


\paragraph{The instruction \squoted{\texttt{ireturn}}}
returns from the method that is currently invoked.  This method also returns a value to the calling
procedure.  In order for \texttt{ireturn} to be able to return a value $v$, this value $v$ has to be
pushed onto the stack before the command \texttt{ireturn} is executed.

In general, if a method taking $n$ arguments $a_1$, $\cdots$, $a_n$ is to be called, then first the
arguments $a_1$, $\cdots$, $a_n$ have to be pushed onto the stack.  When the method $m$ is done and
has computed its result $r$, the arguments $a_1$, $\cdots$, $a_n$ will have been replaced with the
single value $r$.

\paragraph{The instruction \squoted{\texttt{return}}}
returns from the method that is currently invoked.  However, in contrast to the command
\texttt{ireturn}, this command is used if the method that has been invoked  does not return a result. 

\section{An Example Program}
Figure \ref{fig:sum-jasmin.c} shows a small \texttt{C} program that computes the sum
\\[0.2cm]
\hspace*{1.3cm}
$\sum\limits_{i=1}^{6^2} i$.
\\[0.2cm]
The function $\mathtt{sum}(n)$ computes the sum $\sum_{i=1}^{n} i$ and the function
\texttt{main} calls this function with the argument $6^2$.  Figure \ref{fig:Sum.jas} on page
\pageref{fig:Sum.jas} shows how this program can be translated 
into assembler.  We discuss the implementation line by line.


\begin{figure}[!ht]
\centering
\begin{Verbatim}[ frame         = lines, 
                  framesep      = 0.3cm, 
                  firstnumber   = 1,
                  labelposition = bottomline,
                  numbers       = left,
                  numbersep     = -0.2cm,
                  xleftmargin   = 0.8cm,
                  xrightmargin  = 0.8cm,
                ]
    #import "stdio.h"
    
    int sum(int n) {
        int s;
        s = 0;
        while (n != 0) {
            s = s + n;
            n = n - 1;    
        };
        return s;
    }
    int main() {
        printf("%d\n", sum(6*6));
        return 0;
    }
\end{Verbatim}
\vspace*{-0.3cm}
\caption{A \texttt{C} function to compute  $\sum\limits_{i=1}^{36} i$.}
\label{fig:sum-jasmin.c}
\end{figure}

\begin{figure}[!ht]
\centering
\begin{Verbatim}[ frame         = lines, 
                  framesep      = 0.3cm, 
                  firstnumber   = 1,
                  labelposition = bottomline,
                  numbers       = left,
                  numbersep     = -0.2cm,
                  xleftmargin   = 0.8cm,
                  xrightmargin  = 0.8cm,
                ]
    .class public Sum
    .super java/lang/Object
    
    .method public <init>()V
        aload 0
        invokenonvirtual java/lang/Object/<init>()V
        return
    .end method
    
    .method public static main([Ljava/lang/String;)V
        .limit locals 1
        .limit stack  3
        getstatic     java/lang/System/out Ljava/io/PrintStream;
        ldc           6
        dup
        imul
        invokestatic  Sum/sum(I)I
        invokevirtual java/io/PrintStream/println(I)V
        return
    .end method
    
    .method public static sum(I)I
        .limit locals 2
        .limit stack  2
        ldc    0
        istore 1                    ; s = 0
    loop:
        iload  0                    ; n
        ifeq   finish               ; if (n == 0) goto finish
        iload  1                    ; s
        iload  0                    ; n
        iadd
        istore  1                   ; s = s + n
        iload   0                   ; n
        ldc     1
        isub
        istore  0                   ; n = n - 1
        goto    loop
    finish:
        iload   1
        ireturn                     ; return s
    .end method
\end{Verbatim}
\vspace*{-0.3cm}
\caption{An assembler program to compute the sum $\sum\limits_{i=1}^{36} i$.}
\label{fig:Sum.jas}
\end{figure}

\begin{enumerate}
\item Line 1 specifies the name of the generated class which is to be \texttt{Sum}.
\item Line 2 specifies that the class \texttt{Sum} is a subclass of the class
      \texttt{java.lang.Object}. 
\item Lines 4 -- 8 initialize the class.  The code used here is the same as in the example printing 
      ``\texttt{Hello World!}''.
\item Line 10 declares the method \texttt{main}.
\item Line 11 specifies that there is just one local variable.
\item Line 12 specifies that the stack will contain at most 3 temporary values.
\item Line 13 pushes the object \texttt{java.lang.out} onto the stack.
      We need this object later in order to invoke \texttt{println}.
\item Line 14 pushes the number 6 onto the stack.
\item Line 15 duplicates the value 6.  Therefore, after line 15 is executed, the stack contains three
      elements: The object \texttt{java.lang.out},  the number 6, and again the number 6.
\item Line 16 multiplies the two values on top of the stack and replaces them with their product,
      which happens to be 36.
\item Line 17 calls the method \texttt{sum} defined below.  After this call has finished, the number 
      36 on top of the stack is replaced with the value $\mathtt{sum}(36)$.
\item Line 18 prints the value that is on top of the stack. 
\item Line 22 declares the method sum.  This method takes one integer argument and returns an
      integer as result. 
\item Line 23 specifies that the method \texttt{sum} has two local variables: The first local
      variable is the parameter $n$ and the second local variable corresponds to the variable $s$ in the
      \texttt{C} program.
\item The effect of lines 25 and 26 is to initialize this variable $s$ with the value $0$.
\item Line 28 pushes the local variable $n$ on the stack so that line 29 is able to test whether $n$
      is already $0$.  If $n = 0$, the program branches to the label \texttt{finish} in line 39,
      pushes the result $s$ onto the stack (line 40) and returns.  If $n$ is not yet $0$, the
      execution proceeds normally to line 30.
\item Line 30 and line 31 push the sum $s$ and the variable $n$ onto the stack.  These
      values are then added and the result is written to the local variable $s$ in line 33.
      The combined effect of these instructions is therefore to perform the assignment
      \\[0.2cm]
      \hspace*{1.3cm}
      \texttt{s = s + n;}
\item The instructions in line 34 up to line 37 implement the assignment 
      \\[0.2cm]
      \hspace*{1.3cm}
      \texttt{n = n - 1;}
\item In line 38 we jump back to the beginning of the while loop and test whether $n$ has become
      zero.  
\item The declaration in line 42 terminates the definition of the method \texttt{sum}.
\end{enumerate}

\exerciseEng
Implement an assembler program that computes the factorial function.  Test your program by printing
$n!$ for $n = 1, \cdots, 10$.

%%% Local Variables: 
%%% mode: latex
%%% TeX-master: "formal-languages.tex"
%%% End: 
