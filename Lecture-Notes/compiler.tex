\chapter{Entwicklung eines einfachen Compilers} 
In diesem Kapitel konstruieren wir einen Compiler, der ein Fragment der Sprache \texttt{C}
in \textsl{Java}-Assembler �bersetzt.  
Das von dem Compiler �bersetzte Fragment der Sprache \texttt{C} bezeichnen wir als
\textsl{Integer}-\texttt{C}, denn es steht dort nur der Datentyp \texttt{int} zur Verf�gung.

Ein Compiler besteht prinzipiell aus den folgenden Komponenten:
\begin{enumerate}
\item Der Scanner liest die zu �bersetzende Datei ein und zerlegt diese in eine Folge von Token.
      
      Wir werden unseren Scanner mit Hilfe des Werkzeugs \textsl{JFlex} entwickeln.
\item Der Parser liest die Folge von Token und produziert als Ergebnis einen abstrakten Syntax-Baum.

      Wir werden den Parser mit Hilfe von \textsl{JavaCup} generieren.
\item Der Typ-Checker �berpr�ft den abstrakten Syntax-Baum auf Typ-Fehler.

      Da die von uns �bersetzte Sprache nur einen einzelnen Datentyp enth�lt, er�brigt
      sich diese Phase f�r den von uns entwickelten Compiler.
\item In realen Compilern erfolgt nun eine \emph{Optimierungsphase}, die wir aus Zeitgr�nden aber nicht betrachten.
\item Der Code-Generator �bersetzt schlie�lich den Parse-Baum in eine Folge von
      \textsc{Java}-Assembler-Befehlen.  
\end{enumerate}
Bei unseren Compiler sind wir an dieser Stelle schon fertig.  Bei Compilern, deren Zielcode ein
\textsc{Risc}-Assembler-Programm ist, wird normalerweise zun�chst auch ein Code erzeugt, der dem
\textsc{Jvm}-Code �hnelt.  Ein solcher Code wird als \emph{Zwischen-Code} bezeichnet.  Es bleibt
dann die Aufgabe eines sogenannten \emph{Backends}, daraus ein Assembler-Programm f�r einen gegebene
Prozessor-Architektur zur erzeugen.  Die schwierigste Aufgabe besteht hier darin, f�r die
verwendeten Variablen eine Register-Zuordnung zu finden, bei der m�glichst alle Variablen in
Registern vorgehalten werden k�nnen.  
Aus Zeitgr�nden k�nnen wir das Thema der Register-Zuordnung in dieser Vorlesung nicht mehr behandeln.
% Dieses Thema ist allerdings so komplex, dass wir es in einem separaten Kapitel diskutieren.

\section{Die Programmiersprache \textsl{Integer}-\texttt{C}}

\begin{figure}[!ht]
  \begin{center}
  \begin{minipage}[t]{12.5cm}
  \begin{eqnarray*}    
\textsl{program}      & \rightarrow & \;\textsl{function}^* \\[0.2cm]
\textsl{function}     & \rightarrow & \quoted{int} \textsc{Id} \quoted{(} \textsl{paramList} \quoted{)} \quoted{\{} \textsl{decl}\,^*\; (\textsl{stmnt}\; \squoted{;})^* \;\squoted{\}} \\[0.2cm]
\textsl{paramList}    & \rightarrow & \;(\squoted{int}\, \textsc{Id}\; (\squoted{,}\, \quoted{int} \textsc{Id})^*)?  \\[0.2cm]
\textsl{decl}  & \rightarrow & \quoted{int} \textsc{Id} \quoted{;}  \\[0.2cm]
\textsl{stmnt}    & \rightarrow & \quoted{\{} (\textsl{stmnt}\; \squoted{;})^* \quoted{\}}  \\[0.0cm]
                      & \mid        & \;\textsc{Id} \quoted{=} \textsl{expr} \\[0.0cm]
                      & \mid        &  \quoted{if} \quoted{(} \textsl{boolExpr} \quoted{)} \textsl{stmnt} \\[0.0cm]
                     & \mid         &  \quoted{if} \quoted{(} \textsl{boolExpr} \quoted{)} \textsl{stmnt} \quoted{else} \textsl{stmnt} \\[0.0cm]
                     & \mid        &  \quoted{while} \quoted{(} \textsl{boolExpr} \quoted{)} \textsl{stmnt} \\[0.0cm]
                     & \mid        &  \quoted{return} \textsl{expr}     \\[0.0cm]
                     & \mid        &  \;\textsl{expr}                  \\[0.2cm]
   \textsl{boolExpr} & \rightarrow & \textsl{expr} \;(\squoted{==} \mid \squoted{!=} \mid \squoted{<=} \mid \squoted{>=} \mid \squoted{<} \mid \squoted{>})\;  \textsl{expr} \\
         & \mid        &  \quoted{!} \textsl{boolExpr}             \\[0.0cm]
         & \mid        &  \textsl{boolExpr} \;(\squoted{\&\&} \mid \squoted{||})\; \textsl{boolExpr}   \\[0.2cm]
 \textsl{expr} & \rightarrow & \textsl{expr} \;(\squoted{+} \mid \squoted{-} \mid \squoted{*} \mid \squoted{/})\; \textsl{expr} \\[0.0cm]
     & \mid        &  \quoted{(} \textsl{expr} \quoted{)}                 \\[0.0cm]
     & \mid        &  \textsc{Number}                             \\[0.0cm]
     & \mid        &  \textsc{Id}\; (\squoted{(} (\textsl{expr}\; (\squoted{,}\; \textsl{expr})^*)?\;\squoted{)})?  
  \end{eqnarray*}
  \vspace*{-0.5cm}

  \end{minipage}
  \end{center}
  \caption{Eine Grammatik f�r \textsl{Integer}-\textsc{C}}
\label{fig:compiler.cup}
\end{figure}

\noindent
Wir stellen nun die Sprache \textsl{Integer}-\texttt{C} vor, die unser Compiler �bersetzen
soll.  In diesem Zusammenhang sprechen wir auch von der \emph{Quellsprache} unseres Compilers.
Abbildung \ref{fig:compiler.cup} zeigt die Grammatik der Quellsprache in 
erweiterter Backus-Naur-Form (\textsc{Ebnf}). 
Die Grammatik f�r \textsl{Integer}-\texttt{C} verwendet die folgenden beiden Terminale:
\begin{enumerate}
\item \textsc{Id} steht f�r eine Folge von Ziffern, Buchstaben und dem Unterstrich, die 
      mit einem Buchstaben beginnt.  Eine \textsc{Id} bezeichnet entweder eine Variable oder
      den Namen einer Funktion.
\item \textsc{Number} steht f�r eine Folge von Ziffern, die als Dezimalzahl interpretiert wird.
\end{enumerate}
Nach der oben angegebenen Grammatik ist ein Programm eine Liste von Funktionen.
Eine Funktion besteht aus der Deklaration der Signatur, worauf in geschweiften Klammern
eine Liste von Deklarationen (\textsl{decl}) und Befehlen (\textsl{stmt}) folgt.  Jeder Befehl wird
durch ein Semikolon abgeschlossen.  Der Aufbau der einzelnen Befehle ist dann
�hnlich wie bei der Sprache \textsc{Sl}, f�r die wir im Kapitel \ref{chapter:interpreter}
einen Interpreter entwickelt haben.

Die in Abbildung \ref{fig:compiler.cup} gezeigte Grammatik ist mehrdeutig:
\begin{enumerate}
\item Die Grammatik hat das \emph{Dangling-Else-Problem}.

      Da wir im Kapitel \ref{chapter:cup} bereits gesehen haben, wie dieses Problem
      professionell gel�st werden kann, nehme ich mir hier die Freiheit, \textsc{JavaCup} mit der Option
      \\[0.2cm]
      \hspace*{1.3cm}
      ``\texttt{-expect 1}
      \\[0.2cm]
      aufzurufen und dadurch die Fehlermeldung zu unterdr�cken, denn wir hatten ja bereits gesehen, das
      \textsc{Cup} per default den durch die Mehrdeutigkeit entstehenden Shift-Reduce-Konflikt in unserem
      Sinne aufl�st. 
\item F�r die bei arithmetischen und Boole'schen Ausdr�cken verwendeten Operatoren
      m�ssen Pr�zedenzen festgelegt werden.
\end{enumerate}

\begin{figure}[!ht]
\centering
\begin{Verbatim}[ frame         = lines, 
                  framesep      = 0.3cm, 
                  labelposition = bottomline,
                  numbers       = left,
                  numbersep     = -0.2cm,
                  xleftmargin   = 0.8cm,
                  xrightmargin  = 0.8cm,
                ]
    int sum(int n) {
        int s;
        s = 0;
        while (n != 0) {
            s = s + n;
            n = n - 1;    
        };
        return s;
    }  
    int main() {
        int n;
        n = 6 * 6;
        println(sum(n));
    }
\end{Verbatim}
\vspace*{-0.3cm}
\caption{Ein einfaches \textsc{Integer}-\texttt{C}-Programm.}
\label{fig:sum.c}
\end{figure}

\noindent
Abbildung \ref{fig:sum.c} zeigt ein einfaches \textsc{Integer}-\texttt{C}-Programm.  Die Funktion
$\textsl{sum}(n)$ berechnet die Summe
\\[0.2cm]
\hspace*{1.3cm}
 $\sum\limits_{i=1}^n i$
\\[0.2cm]
und die Funktion $\textsl{main}()$ ruft die Funktion \texttt{sum} mit dem Argument $6 \cdot 6$ auf.
Die in dem Programm verwendete Funktion \texttt{println} gibt ihr Argument gefolgt von einem
Zeilenumbruch aus.


\section{Entwicklung von Scanner und Parsers}
Scanner und Parser werden mit Hilfe von \textsl{JFlex} und \textsc{JavaCup} in der
gleichen Art und Weise entwickelt, wie wir das bereits mehrfach in dieser Vorlesung
gesehen haben. 
Abbildung \ref{fig:compiler.jflex} zeigt die Implementierung des Scanners.  Der Scanner
erkennt die verwendeten Operatoren und Schl�sselw�rter, sowie Variablen und nat�rliche
Zahlen.  Gegen�ber den bisher gesehenen \textsl{JFlex}-Scannnern gibt es hier keine
erw�hnenswerten Unterschiede.  Daher werden wir den Scanner nicht weiter diskutieren.


\begin{figure}[!ht]
\centering
\begin{Verbatim}[ frame         = lines, 
                  framesep      = 0.3cm, 
                  labelposition = bottomline,
                  numbers       = left,
                  numbersep     = -0.2cm,
                  xleftmargin   = 0.0cm,
                  xrightmargin  = 0.0cm,
                ]
    import java_cup.runtime.*;    
    %%
    %char
    %line
    %column
    %cup
    %{   
        private Symbol symbol(int type) {
            return new Symbol(type, yychar, yychar + yylength());
        }   
        private Symbol symbol(int type, Object value) {
            return new Symbol(type, yychar, yychar + yylength(), value);
        }
    %}
    %%
       
    "+"                   { return symbol( sym.PLUS      ); } 
    "-"                   { return symbol( sym.MINUS     ); } 
    "*"                   { return symbol( sym.TIMES     ); } 
    "/"                   { return symbol( sym.SLASH     ); } 
    "("                   { return symbol( sym.LPAREN    ); } 
    ")"                   { return symbol( sym.RPAREN    ); }
    "{"                   { return symbol( sym.LBRACE    ); }
    "}"                   { return symbol( sym.RBRACE    ); }
    ","                   { return symbol( sym.COMMA     ); }
    ";"                   { return symbol( sym.SEMICOLON ); }
    "="                   { return symbol( sym.ASSIGN    ); }
    "=="                  { return symbol( sym.EQUALS    ); }
    "!="                  { return symbol( sym.NEQUALS   ); }
    "<"                   { return symbol( sym.LT        ); }
    ">"                   { return symbol( sym.GT        ); }
    "<="                  { return symbol( sym.LE        ); }
    ">="                  { return symbol( sym.GE        ); }
    "&&"                  { return symbol( sym.AND       ); }
    "||"                  { return symbol( sym.OR        ); }
    "!"                   { return symbol( sym.NOT       ); }
    "int"                 { return symbol( sym.INT       ); }
    "return"              { return symbol( sym.RETURN    ); }
    "if"                  { return symbol( sym.IF        ); }
    "else"                { return symbol( sym.ELSE      ); }
    "while"               { return symbol( sym.WHILE     ); }
    
    [a-zA-Z][a-zA-Z_0-9]* { return symbol(sym.IDENTIFIER, yytext());          }
    [0-9]|[1-9][0-9]*     { return symbol(sym.NUMBER, new Integer(yytext())); }
    [ \t\v\n\r]           { /* skip white space */ }   
    "//" [^\n]*           { /* skip comments    */ }   
    [^] { throw new Error("Illegal character '" + yytext() + 
                          "' at line " + yyline + ", column " + yycolumn); }
\end{Verbatim}
\vspace*{-0.3cm}
\caption{Der Scanner f�r \textsl{Integer}-\texttt{C}}
\label{fig:compiler.jflex}
\end{figure}


\begin{figure}[!ht]
\centering
\begin{Verbatim}[ frame         = lines, 
                  framesep      = 0.3cm, 
                  labelposition = bottomline,
                  numbers       = left,
                  numbersep     = -0.2cm,
                  xleftmargin   = 0.0cm,
                  xrightmargin  = 0.0cm,
                ]
    Program = Program(List<Function> functionList);
    
    Function = Function(String            name, 
                        List<String>      parameterList, 
                        List<Declaration> mDeclarations,
                        List<Statement>   body);
    
    Statement = Block(List<Statement> statementList)
              + Assign(String var, Expr expr)
              + IfThen(BoolExpr boolExpr, Statement statement)
              + IfThenElse(BoolExpr condition, Statement then, Statement else)
              + While(BoolExpr condition, Statement statement)
              + Return(Expr expr)
              + ExprStatement(Expr expr);
    
    Declaration = Declaration(String var);
    
    Expr = Sum(Expr lhs, Expr rhs)
         + Difference(Expr lhs, Expr rhs)
         + Product(Expr lhs, Expr rhs)
         + Quotient(Expr lhs, Expr rhs)
         + MyNumber(Integer number)
         + Variable(String name)
         + FunctionCall(String name, List<Expr> args);
    
    BoolExpr = Equation(Expr lhs, Expr rhs)
             + Inequation(Expr lhs, Expr rhs)
             + LessOrEqual(Expr lhs, Expr rhs)
             + GreaterOrEqual(Expr lhs, Expr rhs)
             + LessThan(Expr lhs, Expr rhs)
             + GreaterThan(Expr lhs, Expr rhs)
             + Negation(BoolExpr expr)
             + Conjunction(BoolExpr lhs, BoolExpr rhs)
             + Disjunction(BoolExpr lhs, BoolExpr rhs);
\end{Verbatim}
\vspace*{-0.3cm}
\caption{Spezifikation der Klassen zur Darstellung des Syntax-Baums.}
\label{fig:program.ep}
\end{figure}

Bevor wir die Implementierung des Parsers diskutieren k�nnen, m�ssen wir angeben, durch welche
Klassen wir den Syntax-Baum darstellen wollen.  Wir beschreiben diese Klassen mit Hilfe
der in Abbildung \ref{fig:program.ep} auf Seite \pageref{fig:program.ep} gezeigten
Spezifikation.  Diese Spezifikation ist wie folgt zu lesen:
\begin{enumerate}
\item In line 1, the equation
      \\[0.2cm]
      \hspace*{1.3cm}
      \texttt{Program = Program(List<Function> functionList)}
      \\[0.2cm]
      specifies that the class \texttt{Program} has one member variable called
      \texttt{mFunctionList}.  This variable has the type \texttt{List<Function>}.
      Therefore, an object of class \texttt{Program} is essentially a list of objects of
      class \texttt{Function}.
\item Similarly, in line 3 to 6 the equation
      \begin{verbatim}
    Function = Function(String            name, 
                        List<String>      parameterList, 
                        List<Declaration> mDeclarations,
                        List<Statement>   body);
      \end{verbatim}
      \vspace*{-0.5cm}

      specifies that the class \texttt{Function} has four attributes:
      \begin{enumerate}
      \item \texttt{mName} is a \texttt{String} storing the name of the function,
      \item \texttt{mParameterList} is the list of formal parameters of the function,
      \item \texttt{mDeclarations} is the list of local variable declarations occurring in
            the body of the function, while
      \item \texttt{mBody} is the list of statements that make up the definition of the function.
      \end{enumerate}
      \pagebreak

\item Next, the equation
      \begin{verbatim}
    Statement = Block(List<Statement> statementList)
              + Assign(String var, Expr expr)
              + IfThen(BoolExpr boolExpr, Statement statement)
              + IfThenElse(BoolExpr condition, Statement then, Statement else)
              + While(BoolExpr condition, Statement statement)
              + Return(Expr expr)
              + ExprStatement(Expr expr);
      \end{verbatim}
      tells us that there is an abstract class \texttt{Statement} and that the classes
      \texttt{Block}, \texttt{Assign}, $\cdots$, \texttt{ExprStatement} are derived from
      this class.
      \begin{enumerate}
      \item \texttt{Block} is a class representing a list of statements enclosed in
            the curly braces \squoted{\{} and \squoted{\}}.  This list of statemments is
            stored in the member variable \texttt{mStatementList}.
      \item \texttt{Assign} is a class representing an assignment statement.  Therefore,
            this class has two attributes:
            \begin{itemize}
            \item \texttt{mVar} is the name of the variable on the left hand side of the
                  assignment and
            \item \texttt{mExpr} is the expression on the right hand side of the  assignment.
            \end{itemize}
      \item \texttt{IfThen} is a class representing an \texttt{if-then} statement without
            a trailing \texttt{else} clause.  This class has two member variables:
            \begin{itemize}
            \item \texttt{mBoolExpr} is the Boolean expression controlling whether the
            \item \texttt{mStatement} is to be executed.
            \end{itemize}
      \item \texttt{IfThenElse} is a class representing an \texttt{if-then-else}
            statement. This class has three member variables:
            \begin{itemize}
            \item \texttt{mBoolExpr} is the Boolean expression controlling the execution.
            \item \texttt{mThen} is the statement that is executed if
                  \texttt{mBoolExpr} evaluates as \texttt{true}.
            \item \texttt{mElse} is the statement that is executed if
                  \texttt{mBoolExpr} evaluates as \texttt{false}.
            \end{itemize}
      \item \texttt{While} is a class representing a \texttt{while} loop. 
            This class has two member variables:
            \begin{itemize}
            \item \texttt{mBoolExpr} is the Boolean expression controlling the loop.
            \item \texttt{mStatement} is a statement that is executed as long as
                  \texttt{mBoolExpr} evaluates to \texttt{true}.
            \end{itemize}
      \item \texttt{Return} is a class representing a \texttt{return} statement. 
            This class has the member variable \texttt{mExpr}.  Evaluation of this expression
            yields the value to be returned.  Note that in the grammar given in Figure
            \ref{fig:compiler.cup} on page \pageref{fig:compiler.cup} the expression following the
            return statement is not optional.
      \item \texttt{ExprStatement} is a class representing an expression that is to be
             evaluated as a statement.        
      \end{enumerate}
\item The equation in line 16 specifies that the class \texttt{Declaration} has one member
      variable with name \texttt{mVar}.  This variable stores the name of the variable
      that is declared in the variable declaration associated with the corresponding
      object of class \texttt{Declaration}.
\item Similarly, the equations defining \texttt{Expr} and \texttt{BoolExpr} specify the
      representation of arithmetical and Boolean expressions.
\end{enumerate}

Damit k�nnen wir nun die \textsl{Java}-\textsc{Cup}-Datei angeben, mit der wir den Syntaxbaum
erzeugen.  Wir haben diese Datei aus Platzgr�nden in drei Teile aufgespalten:
\begin{enumerate}
\item Abbildung \ref{fig:compiler.cup-1} zeigt die Spezifikation der Terminale, Nicht-Terminale und
      Operator-Pr�zedenzen.  Bei den Pr�zedenzen ist es sinnvoll diese so zu spezifizieren, dass die
      arithmetischen Operatoren st�rker binden als die logischen Operatoren.
\item Abbildung \ref{fig:compiler.cup-2} und \ref{fig:compiler.cup-3} zeigen die eigentliche
      Grammatik und die Erzeugung des abstrakten Syntaxbaums.  In der Grammatik lassen 
      wir auch zu, dass die Liste der Funktionen leer ist.  Das vereinfacht die
      Konstruktion der Liste etwas.
\end{enumerate}
Observe that the actions only construct the abstract syntax tree.  Everything else is then
delegated to appropriate methods in the classes representing the syntax tree.


\begin{figure}[!ht]
\centering
\begin{Verbatim}[ frame         = lines, 
                  framesep      = 0.3cm, 
                  labelposition = bottomline,
                  numbers       = left,
                  numbersep     = -0.2cm,
                  xleftmargin   = 0.0cm,
                  xrightmargin  = 0.0cm,
                ]
    import java_cup.runtime.*;
    import java.util.*;
    
    /* Terminals (tokens returned by the scanner). */
    terminal           COMMA, PLUS, MINUS, TIMES, SLASH, LPAREN, RPAREN, LBRACE, RBRACE;
    terminal           ASSIGN, EQUALS, LT, GT, LE, GE, NEQUALS, AND, OR, NOT;
    terminal           IF, ELSE, WHILE, RETURN, SEMICOLON; 
    terminal           INT;
    terminal String    IDENTIFIER;
    terminal Integer   NUMBER;
    
    /* Non-terminals */
    nonterminal Program           program;
    nonterminal List<Function>    functionList;
    nonterminal Function          function;
    nonterminal List<String>      paramList, neParamList;
    nonterminal Declaration       declaration;
    nonterminal List<Declaration> declarations;
    nonterminal Statement         statement;
    nonterminal List<Statement>   statementList;
    nonterminal Expr              expr;
    nonterminal List<Expr>        exprList, neExprList;
    nonterminal BoolExpr          boolExpr;
    
    precedence left     OR;
    precedence left     AND;
    precedence right    NOT;
    precedence left     PLUS, MINUS;
    precedence left     TIMES, SLASH;
\end{Verbatim}
\vspace*{-0.3cm}
\caption{Deklaration der Terminale, Nicht-Terminale und Operator-Pr�zedenzen}
\label{fig:compiler.cup-1}
\end{figure}

\begin{figure}[!ht]
\centering
\begin{Verbatim}[ frame         = lines, 
                  framesep      = 0.3cm, 
                  firstnumber   = last,
                  labelposition = bottomline,
                  numbers       = left,
                  numbersep     = -0.2cm,
                  xleftmargin   = 0.0cm,
                  xrightmargin  = 0.0cm,
                ]
    program ::= functionList:l {: RESULT = new Program(l); :} ;
    
    functionList ::= /* epsilon */ {: RESULT = new LinkedList<Function>(); :}
                  |  functionList:l function:f {: l.add(f); RESULT = l;    :}
                  ;
    
    function ::= INT IDENTIFIER:f LPAREN paramList:p RPAREN 
                 LBRACE declarations:d statementList:l RBRACE
                 {: RESULT = new Function(f, p, d, l); :}
              ;
    
    paramList ::= /* epsilon */ {: RESULT = new LinkedList<String>(); :}
               |  neParamList:l {: RESULT = l;                        :}
               ;
    
    neParamList ::= INT IDENTIFIER:v                     
                    {: RESULT = new LinkedList<String>(); 
                       RESULT.add(v); 
                    :}
                 |  neParamList:l COMMA INT IDENTIFIER:v {: RESULT = l; RESULT.add(v);  :}
                 ;           
    
    declaration ::= INT IDENTIFIER:v SEMICOLON {: RESULT = new Declaration(v); :} ;
    
    declarations ::= /* epsilon */                {: RESULT = new LinkedList<Declaration>(); :}
                  |  declarations:l declaration:d {: RESULT = l; RESULT.add(d);              :}
                  ;
    
    statement ::= LBRACE statementList:l RBRACE {: RESULT = new Block(l);     :}
               |  IDENTIFIER:v ASSIGN expr:e    {: RESULT = new Assign(v, e); :}
               |  IF LPAREN boolExpr:b RPAREN statement:s                 
                  {: RESULT = new IfThen(b, s); :}        
               |  IF LPAREN boolExpr:b RPAREN statement:t ELSE statement:e
                  {: RESULT = new IfThenElse(b, t, e); :}
               |  WHILE LPAREN boolExpr:b RPAREN statement:s
                  {: RESULT = new While(b, s); :}
               |  RETURN expr:e    {: RESULT = new Return(e);        :}
               |  expr:e           {: RESULT = new ExprStatement(e); :}
               ;
    
    statementList ::= /* epsilon */ {: RESULT = new LinkedList<Statement>(); :}
                   |  statement:s SEMICOLON statementList:l 
                      {: RESULT = new LinkedList<Statement>();
                         RESULT.add(s); 
                         RESULT.addAll(l);
                      :}
                   ;
\end{Verbatim}
\vspace*{-0.3cm}
\caption{Der erste Teil der \textsl{Java}-\textsc{Cup}-Grammatik.}
\label{fig:compiler.cup-2}
\end{figure}


\begin{figure}[!ht]
\centering
\begin{Verbatim}[ frame         = lines, 
                  framesep      = 0.3cm, 
                  firstnumber   = last,
                  labelposition = bottomline,
                  numbers       = left,
                  numbersep     = -0.2cm,
                  xleftmargin   = 0.0cm,
                  xrightmargin  = 0.0cm,
                ]
    expr ::= expr:l PLUS  expr:r                 {: RESULT = new Sum(       l, r); :}
          |  expr:l MINUS expr:r                 {: RESULT = new Difference(l, r); :}
          |  expr:l TIMES expr:r                 {: RESULT = new Product(   l, r); :}
          |  expr:l SLASH expr:r                 {: RESULT = new Quotient(  l, r); :}
          |  LPAREN expr:e RPAREN                {: RESULT = e;                    :}
          |  NUMBER:n                            {: RESULT = new MyNumber(n);      :}
          |  IDENTIFIER:v                        {: RESULT = new Variable(v);      :}
          |  IDENTIFIER:n LPAREN exprList:l RPAREN 
             {: RESULT = new FunctionCall(n, l); :}
          ;
    
    exprList ::= /* epsilon */ {: RESULT = new LinkedList<Expr>(); :}
              |  neExprList:l  {: RESULT = l;                      :}
              ;
    
    neExprList ::= expr:e 
                   {: List<Expr> l = new LinkedList<Expr>();
                      l.add(e);
                      RESULT = l;
                   :}                  
                |  neExprList:l COMMA expr:e {: l.add(e); RESULT = l; :}
                ;
    
    boolExpr ::= expr:l EQUALS  expr:r     {: RESULT = new Equation(      l, r); :}
              |  expr:l NEQUALS expr:r     {: RESULT = new Inequation(    l, r); :}
              |  expr:l LE      expr:r     {: RESULT = new LessOrEqual(   l, r); :}
              |  expr:l GE      expr:r     {: RESULT = new GreaterOrEqual(l, r); :}
              |  expr:l LT      expr:r     {: RESULT = new LessThan(      l, r); :}
              |  expr:l GT      expr:r     {: RESULT = new GreaterThan(   l, r); :}
              |  NOT boolExpr:e            {: RESULT = new Negation(      e   ); :}
              |  boolExpr:l AND boolExpr:r {: RESULT = new Conjunction(   l, r); :}
              |  boolExpr:l OR  boolExpr:r {: RESULT = new Disjunction(   l, r); :}
              ;
\end{Verbatim}
\vspace*{-0.3cm}
\caption{Der zweite Teil der \textsl{Java}-\textsc{Cup}-Grammatik.}
\label{fig:compiler.cup-3}
\end{figure}



%%% Local Variables: 
%%% mode: latex
%%% TeX-master: "formal-languages"
%%% End: 
