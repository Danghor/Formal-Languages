\section{sed}
Das Werkzeug \texttt{sed} ist ein \emph{Stream-Editor}, der auf regul\"aren Ausdr\"ucken
basiert.  Ein \emph{Stream-Editor} ist ein \emph{Filter}, also ein Programm, das von der
Standard-Eingabe liest und auf die Standard-Ausgabe schreibt.  Das Programm kann
wie folgt aufgerufen werden:
\\[0.2cm]
\hspace*{1.3cm}
\texttt{sed} \textsl{cmd}
\\[0.2cm]
Hierbei bezeichnet \textsl{cmd} ein \texttt{sed}-Komando.  Wir werden nur ein einziges
\texttt{sed}-Komando diskutieren: Das Ersetzungs-Kommando.  Dieses Kommando hat die Form
\\[0.2cm]
\hspace*{1.3cm}
\texttt{s/}\textsl{Regexp}/\textsl{Replacement}\texttt{/g}
\\[0.2cm]
Hierbei ist \textsl{Regexp} ein regul\"arer Ausdruck und \textsl{Replacement} ist ein
Substitutions-Ausdruck.  Der Aufruf
\\[0.2cm]
\hspace*{1.3cm}
\texttt{sed s/}\textsl{Regexp}/\textsl{Replacement}\texttt{/g}
\\[0.2cm]
ersetzt sucht in der Eingabe nach einem Auftreten des regul\"aren Ausdrucks \textsl{Regexp}
und ersetzt diese durch \textsl{Replacement}.  Beispielsweise ersetzt das Kommando 
\\[0.2cm]
\hspace*{1.3cm}
\texttt{sed s/Stephan/Stefan/g}
\\[0.2cm]
alle Auftreten von \texttt{Stephan} durch \texttt{Stefan}.  Dieses Kommando
k\"onnte wie folgt in eine \emph{Unix-Pipe} eingebunden werden:
\\[0.2cm]
\hspace*{1.3cm}
\texttt{cat $f_1$ | sed s/Stephan/Stefan/g > $f_2$} 
\\[0.2cm]
Diese Pipe w\"urde die Datei $f_1$ lesen, in dem gelesenen Text Datei jedes Auftreten von
``\texttt{Stephan}'' durch ``\texttt{Stefan}'' ersetzen und den so ver\"anderten Text in die
Datei $f_2$ schreiben.

\subsection{Regul\"are Ausdr\"ucke in \texttt{sed}}
Die Syntax der in \texttt{sed} verwendeten regul\"aren Ausdr\"ucke weicht von der
im letzten Kapitel eingef\"uhrten Syntax stark ab.  Den regul\"aren Ausdr\"ucken in \texttt{sed}
liegt das \textsc{Ascii}-Alphabet zu Grunde.  Die Buchstaben des \textsc{Ascii}-Alphabets
werden in zwei Gruppen eingeteilt:
\begin{enumerate}
\item Operator-Symbole,
\item alle \"ubrigen Zeichen,
\end{enumerate}
Die folgenden Zeichen sind folgenden Operator-Symbole verwendet: 
\\[0.2cm]
\hspace*{1.3cm}
   ``\texttt{*}'', ``\texttt{.}'', ``\texttt{\symbol{92}}'', ``\texttt{\symbol{94}}'', 
   ``\texttt{\symbol{36}}'', \texttt{[}, \texttt{]}
\\[0.2cm]
Regul\"are Ausdr\"ucke in \texttt{sed} werden nun wie folgt definiert.
\begin{enumerate}
\item Jeder Buchstabe $c$, das von den oben aufgelisteten Operator-Symbolen verschieden ist,
      ist ein regul\"are Ausdruck.
\item Ist $o$ ein Operator-Symbol, so ist ``$\texttt{\symbol{92}}o$'' ein regul\"arer
      Ausdruck, der f\"ur das Operator-Symbol $o$ steht.
\item Das Zeichen ``\texttt{.}'' ist ein regul\"arer Ausdruck, der f\"ur ein beliebiges
      Zeichen steht, die zugeh\"orige Sprache besteht also aus allen W\"ortern der L\"ange Eins.
\item Sind $r_1$ und $r_2$ regul\"are Ausdr\"ucke, so ist auch die Konkatenation $r_1r_2$
      ein regul\"arer Ausdruck.  Im letzten Kapitel hatten wir  diesen Ausdruck $r_1 \cdot r_2$
      geschrieben.
\item Sind $r_1$ und $r_2$ regul\"are Ausdr\"ucke, so ist  $r_1\,\mathtt{\symbol{92}|}\,r_2$
      ein regul\"arer Ausdruck.  Im letzten Kapitel hatten wir stattdessen $r_1 + r_2$
      geschrieben.
\item Ist $r$ ein regul\"arer Ausdruck, so ist auch $r\mathtt{*}$ ein regul\"arer Ausdruck.
      Wir hatten diesen Ausdruck als $r^*$ geschrieben.
\item Ist $r$ ein regul\"arer Ausdruck, so ist auch
      $\mathtt{\symbol{92}(}r\mathtt{\symbol{92})}$
      ein regul\"arer Ausdruck.  Wir hatten stattdessen $(r)$ geschrieben, aber weil die
      beiden Klammer-Symbole ``\texttt{(}'' und ``\texttt{)}'' in \texttt{sed} keine
      Operator-Symbole sind, m\"ussen sie durch das Voranstellen eines Backslashs gesch\"utzt 
      werden.
\item Ist $r$ ein regul\"arer Ausdruck, so ist auch $r\mathtt{\symbol{92}+}$ ein regul\"arer
      Ausdruck.   Dieser Ausdruck ist nur eine Abk\"urzung f\"ur den Ausdruck
      \\[0.2cm]
      \hspace*{1.3cm}
      $rr\mathtt{*}$.
      \\[0.2cm]
      Anschaulich steht $r\mathtt{\symbol{92}+}$ dass der durch $r$ spezifizierte
      Ausdruck mindestens einmal auftreten muss.
\item Ist $r$ ein regul\"arer Ausdruck, so ist auch $r\mathtt{\symbol{92}?}$ ein regul\"arer
      Ausdruck.  Der Ausdruck spezifiziert,
       dass der durch $r$ spezifizierte
      Ausdruck einmal oder keinmal auftreten muss.
\item Sind $c_1, \cdots, c_n$ Buchstaben, so ist
      $\mathtt{[}c_1,\cdots, c_n\mathtt{]}$ ein regul\"arer Ausdruck, der als Abk\"urzung
      f\"ur den Ausdruck $c_1 \mathtt{\symbol{92}|} \cdots \mathtt{\symbol{92}|} c_n$
      verwendet werden kann.  Beispielsweise spezifiziert der Ausdruck
      \\[0.2cm]
      \hspace*{1.3cm}
      \texttt{[aeiou]}
      \\[0.2cm]
      genau die Vokale.
\item Sind $c_1$ und $c_n$ zwei Buchstaben, so ist
      $\mathtt{[}c_1\mathtt{-}c_n\mathtt{]}$ ein regul\"arer Ausdruck, der 
      die Menge aller Zeichen zwischen $c_1$ und $c_n$ beschreibt.  Beispielsweise
      spezifiziert der Ausdruck 
      \\[0.2cm]
      \hspace*{1.3cm}
      \texttt{[a-z]}
      \\[0.2cm]
      die Menge aller kleinen Buchstaben.
\item Sind $c_1, \cdots, c_n$ Buchstaben, so ist
      $\mathtt{[\symbol{94}}c_1,\cdots, c_n\mathtt{]}$ ein regul\"arer Ausdruck, 
      der f\"ur alle Buchstaben steht, die von $c_1, \cdots, c_n$ verschieden sind.
\end{enumerate}
Wir geben einige Beispiele f\"ur regul\"are Ausdr\"ucke:
\begin{enumerate}
\item Der regul\"are Ausdruck \texttt{Hugo} spezifiziert das Wort ``\texttt{Hugo}''.
\item Der regul\"are Ausdruck \texttt{Hugo*} spezifiziert die W\"orter 
      ``\texttt{Hug}'',
      ``\texttt{Hugo}'',
      ``\texttt{Hugoo}'',
      ``\texttt{Hugooo}'', $\cdots$.
      Beachten Sie bei diesem Beispiel, dass der Operator ``\texttt{*}'' st\"arker bindet
      als der (unsichtbare) Konkatenations-Operator.
\item Der regul\"are Ausdruck \texttt{a*b} spezifiziert alle W\"orter,
      die mit beliebig vielen \texttt{a}'s beginnen, denen dann noch ein einzelnes
      \texttt{b} folgt.
\item Der regul\"are Ausdruck 
      \texttt{\symbol{92}(\symbol{92}[a-z]\symbol{92}|[A-Z]\symbol{92})\symbol{92}[a-z]} beschreibt W\"orter,
      die mit einem gro{\ss}en oder kleinen Buchstaben des lateinischen Alphabets beginnen
      und anschlie{\ss}end nur aus kleinen Buchstaben dieses Alphabets bestehen.  Dieser
      regul\"are Ausdruck kann auch einfacher als \texttt{[a-zA-Z][a-z]*} geschrieben werden.
\item Der regul\"are Ausdruck \texttt{[a-zA-Z][a-zA-Z0-9\_]*} beschreibt die Syntax von
      Variablen-Namen in der Sprache \texttt{C}:  In \texttt{C} muss eine Variable
      mit einem lateinischen Buchstaben beginnen, danach k\"onnen lateinische Buchstaben,
      Ziffern, oder das Zeichen ``\texttt{\_}'' folgen.
\end{enumerate}

\subsection{Replacements}
Wir hatten anfangs gesagt, dass wir Aufrufe von \texttt{sed} der Form
\\[0.2cm]
\hspace*{1.3cm}
\texttt{sed s/}\textsl{Regexp}/\textsl{Replacement}\texttt{/g}
\\[0.2cm]
diskutieren wollen.  Im letzten Abschnitt hatten wir diskutiert, was wir f\"ur den Ausdruck
\textsl{Regexp} einsetzen k\"onnen, nun wenden wir uns dem Ausdruck \textsl{Replacement} zu.
Der Ausdruck \textsl{Replacement} spezifiziert den Text, durch den der Text, der dem
regul\"aren Ausdruck \textsl{Regexp} entspricht ersetzt werden kann.  Dabei kann innerhalb
von dem Ausdruck \textsl{Replacement} auf geklammerte Teile zur\"uckgegriffen werden.
\texttt{\symbol{92}1} bezieht sich auf den ersten in Klammern eingefassten  Teilausdruck,
\texttt{\symbol{92}2} auf den zweiten,
\texttt{\symbol{92}3} auf den dritten, etc.

F\"ur eine ausf\"uhrlichere Diskussion von \texttt{sed} verweisen wir auf das Buch
von Dougherty und Robbins \cite{dougherty:1997}, sowie auf die Manual-Page und die Seiten
im Unix-Info-System.


%%% Local Variables: 
%%% mode: latex
%%% TeX-master: "formale-sprachen"
%%% End: 
