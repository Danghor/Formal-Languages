\chapter{Types and Type Checking$^*$}
There are two fundamentally different approaches to typing.  Either the user is
required to declare the types of variables or instead the program has to check the types
of objects at runtime.  In the first case, the programming language is called 
\emph{statically typed}, in the second case it is called \emph{dynamically typed}.
\begin{enumerate}
\item Statically typed languages like \textsl{Java} or \texttt{C} require that the user
      declares the types of functions and variables.  The compiler is then able to check
      that these variables will indeed have the declared type at runtime.  This approach
      has the following advantages:
      \begin{enumerate}
      \item A number of runtime errors can be excluded.  For example, if a variable is 
            declared as a \texttt{float} in \texttt{C}, we are guaranteed that the program
            will not try to store a string in this variable.
      \item The program does not need to check the type of variables at runtime and can
            this be more efficient.
      \item Adding type information serves as a form of documentation that can be checked
            automatically.  This enhances the readability of typed programs.
      \item Typed programs are easier to maintain as many violations of interfaces between
            different parts of a typed program will actually manifest itself as type
            errors.  Therefore, the compiler is able to detect these violations.
      \end{enumerate}
\item Dynamically typed languages like \textsl{Perl}, \textsl{Python}, or
      \textsl{JavaScript} do not require the programmer to declare any types.  Rather, the types
      are checked at runtime.  For example, if a program in a dynamically typed language
      contains an expression of the form
      \\[0.2cm]
      \hspace*{1.3cm}
      $x + y$,
      \\[0.2cm]
      the compiler generates code that checks the type of $x$ and $y$ at runtime.  Then,
      if $x$ and $y$ are discovered to be integers, the program performs an integer
      addition.  However, if $x$ and $y$ happen to be strings,  these strings are 
      concatenated.  Dynamic typing  has the following advantages:
      \begin{enumerate}
      \item Programs written in dynamically typed languages are typically shorter
            than the corresponding programs in statically typed languages.  
      \item If the types used in an algorithm are very complex, then using an untyped
            language is sometimes the only way to code an algorithm in the intended way. 
            The reason is that in some cases the expressiveness of current type system is 
            not sufficient to be able to code certain algorithms conveniently.
      \end{enumerate}
\end{enumerate}
The disadvantages of statically typed languages are the advantages of dynamically typed
languages and vice versa.  Therefore, dynamically typed languages are often used for
prototyping, while statically typed languages are used for production systems.

In practice, most statically typed languages offer some escape mechanism to cover the
cases where the type system gets to unwieldy.  For example, in \texttt{C} the programmer
can declare a variable $x$ to have type \texttt{void*} and then cast this pointer to any other
pointer type.  In \textsl{Java}, programmer can declare a variable $x$ as having type \texttt{Object}.
This variable can then hold any object and when this variable is used the user 
needs to cast it to the appropriate type.  

In this chapter we are going to show how the compiler is able to type check a statically
typed program.  To this end,  we first introduce a very simple statically typed programming
language and then we develop a type checker for this language.


\section{Eine Beispielsprache}
Wir stellen jetzt die Sprache \textsc{Ttl} (\emph{typed term language} vor.  Dabei handelt es sich
um eine sehr einfache Beispielsprache, die es dem Benutzer erm\"oglicht
\begin{itemize}
\item Typen zu definieren,
\item Funktionen zu deklarieren und
\item Terme anzugeben,
\end{itemize}
f\"ur die dann die Typ-Korrektheit nachgewiesen wird.
Die Sprache \textsc{Ttl} ist sehr einfach gehalten,
damit wir uns auf die wesentlichen Ideen der Typ-\"Uberpr\"ufung konzentrieren k\"onnen.
Daher k\"onnen wir in \textsc{Ttl} auch keine wirklichen Programme schreiben,
sondern einzig und allein \"uberpr\"ufen, ob Ausdr\"ucke wohlgetypt sind.

\begin{figure}[!ht]
\centering
\begin{Verbatim}[ frame         = lines, 
                  framesep      = 0.3cm, 
                  labelposition = bottomline,
                  numbers       = left,
                  numbersep     = -0.2cm,
                  xleftmargin   = 0.8cm,
                  xrightmargin  = 0.8cm,
                  commandchars  = \\\{\}
                ]
    \underline{t}yp\underline{e} list(X) := nil + cons(X, list(X));
    
    \underline{si}g\underline{nature} concat: list(T) * list(T) -> list(T);
    \underline{si}g\underline{nature} x: int;
    \underline{si}g\underline{nature} y: int;
    \underline{si}g\underline{nature} z: int;
    
    concat(nil, nil): list(int);
    concat(cons(x, nil), cons(y, cons(z, nil))): list(int);
\end{Verbatim}
\vspace*{-0.3cm}
\caption{Ein \textsc{Ttl}-Beispiel-Programm}
\label{fig:types.ttl}
\end{figure}

Abbildung \ref{fig:types.ttl} zeigt ein einfaches Beispiel-Programm.  Die Schl\"usselw\"orter habe ich
unterstrichen.  Wir diskutieren dieses Programm jetzt im Detail.
\begin{enumerate}
\item In Zeile 1 definieren wir den \emph{generischen} Typ $\texttt{list}(X)$.  Das $X$ ist hier der
      Typ-Parameter, der sp\"ater durch einen konkreten Typ wie z.B. \texttt{int},
      \texttt{string} oder $\mathtt{list}(\mathtt{string})$ ersetzt werden kann.

      Semantisch ist die Zeile als induktive Definition zu lesen, durch die eine Menge von Termen
      definiert wird, wobei $X$ eine gegebene Menge bezeichnet. 
      $\textsl{nil}$ und $\textsl{cons}$ werden in diesem Zusammenhang als Funktions-Zeichen verwendet.
      Formal hat die induktive Definition die folgende Gestalt:
      \begin{enumerate}
      \item Induktions-Anfang:  Der Term $\textsl{nil}$ ist ein Element der Menge $\textsl{List}(X)$:
            \[ \textsl{nil} \in \textsl{list}(X) \]
      \item Induktions-Schritt:  Falls $a$ ein Element der Menge $X$ und $l$ ein Element der
            Menge $\textsl{list}(X)$ ist, dann ist der Term $\textsl{cons}(a,l)$ ebenfalls
            ein Element der Menge $\textsl{list}(X)$:
            \[ a \in X \wedge l \in \textsl{list}(X) \rightarrow \textsl{cons}(a,l) \in \textsl{list}(X). \]
      \end{enumerate}
\item In Zeile 3 deklarieren wir die Funktion \textsl{concat} als zweistellige Funktion.  Die
      beiden Argumente haben jeweils den Typ $\texttt{list}(T)$ und das Ergebnis hat ebenfalls
      diesen Typ.
\item In den Zeilen 4 -- 6 legen wir fest, dass die Variablen $x$, $y$ und $z$ jeweils den Typ
      \texttt{int} haben.
\item In Zeile 8 und 9 werden schlie{\ss}lich die beiden Terme
      \[ \mathtt{concat(nil, nil)} \quad \mbox{und} \quad
         \mathtt{concat(cons(x, nil), cons(y, cons(z, nil)))}
      \] 
      angegeben und es wird behauptet, dass diese den Typ $\texttt{list}(\texttt{int})$ haben.
      Die Aufgabe der Typ-\"Uberpr\"ufung besteht darin, diese Aussage zu verifizieren.
\end{enumerate}


\begin{figure}[!ht]
\centering
\begin{Verbatim}[ frame         = lines, 
                  framesep      = 0.3cm, 
                  labelposition = bottomline,
                  numbers       = left,
                  numbersep     = -0.2cm,
                  xleftmargin   = 0.0cm,
                  xrightmargin  = 0.0cm,
                ]
    grammar ttl;
    
    program   : typeDef* signature* typedTerm*
              ;
    
    typeDef   : 'type' FCT ':=' type ('+' type)* ';'
              | 'type' FCT '(' PARAM (',' PARAM)* ')' ':=' type ('+' type)* ';' 
              ;
    
    type      : FCT '(' type (',' type)* ')'
              | FCT                               
              | PARAM                              
              ;
    
    signature : 'signature' FCT ':' type ('*' type)* '->' type ';'                
              | 'signature' FCT ':' type ';'
              ;
    
    term      : FCT '(' term (',' term)* ')'
              | FCT    
              ;
    
    typedTerm : term ':' type ';' 
              ;
    
    PARAM   : ('A'..'Z')('a'..'z'|'A'..'Z'|'_'|'0'..'9')*;
    FCT     : ('a'..'z')('a'..'z'|'A'..'Z'|'_'|'0'..'9')*;
\end{Verbatim}
\vspace*{-0.3cm}
  \caption{Eine EBNF-Grammatik f\"ur die getypte Beispielsprache}
\label{fig:typeChecker-grammar}
\end{figure}



\noindent
Die genaue Syntax der Sprache \textsc{Ttl} wird durch die in Abbildung \ref{fig:typeChecker-grammar}
gezeigte Grammatik definiert.  Diese Grammatik verwendet neben den Zeichen-Reihen, die in doppelten
Anf\"uhrungs-Zeichen gesetzt sind, die folgenden Terminale:
\begin{enumerate}
\item \textsc{Function} bezeichnet entweder einen Typ-Konstruktor wie \texttt{list}, eine Variable
      wie \texttt{x} oder \texttt{y} oder einen Funktionsnamen wie \texttt{concat}.  Syntaktisch
      werden Typ-Konstruktoren, Variablen und Funktionsnamen dadurch erkannt, dass sie mit
      einem Kleinbuchstaben beginnen. 
\item \textsc{Parameter} bezeichnet einen Typ-Parameter wie z.B.~das $X$ in $\texttt{list}(X)$.
      Diese beginnen immer mit einem Gro{\ss}buchstaben.
\end{enumerate}
Bevor wir einen Algorithmus zur Typ-\"Uberpr\"ufung vorstellen k\"onnen ist es erforderlich,
einige grundlegende Begriffe wie 
den Begriff der Substitution und die Anwendung von Substitutionen auf Typen zu diskutieren.

\section{Grundlegende Begriffe}
Als erstes definieren wir den Begriff der Signatur eines Funktions-Zeichens.
Die \emph{Signatur} eines Funktionszeichens legt fest, 
\begin{itemize}
\item welchen Typ die Argumente der Funktion haben und
\item von welchem Typ das Ergebnis der Funktion ist.
\end{itemize}
Wir geben die Signatur einer Funktion $f$ in der Form
\\[0.2cm]
\hspace*{1.3cm}
$f: \sigma_1 \times \cdots \times \sigma_n \rightarrow \varrho$
\\[0.2cm]
an.  Damit spezifizieren wir:
\begin{enumerate}
\item Die Funktion $f$ erwartet $n$ Argumente.
\item Das $i$-te Argument hat den Typ $\sigma_i$.  
\item Das von der Funktion berechnete Ergebnis hat den Typ $\varrho$.
\end{enumerate}
Als n\"achstes definieren wir, was wir unter einem Typ verstehen wollen.
Anschaulich sind das Ausdr\"ucke wie 
\\[0.2cm]
\hspace*{1.3cm}
$\textsl{map}(K,V)$, \quad $\textsl{double}$, \quad oder \quad $\textsl{list}(\textsl{int})$.
\\[0.2cm]
Formal werden Typen aus Typ-Parametern und Typ-Konstruktoren aufgebaut.
In dem obigen Beispiel sind \textsl{map}, \textsl{double}, \textsl{list} und \textsl{int}
Typ-Konstruktoren, w\"ahrend $K$ und $V$ Typ-Parameter sind.
Wir nehmen an, dass einerseits eine Menge $\mathbb{P}$ von Typ-Parametern und
andererseits eine Menge von Typ-Konstruktoren $\mathbb{K}$ gegeben sind.  In dem letzten Beispiel
k\"onnten wir
\\[0.2cm]
\hspace*{1.3cm}
$\mathbb{K} = \bigl\{ \textsl{map}, \textsl{list}, \textsl{int}, \textsl{double}\bigr\}$
\quad und \quad
$\mathbb{P} = \bigl\{ K, V \bigr\}$
\\[0.2cm]
setzen.  Zus\"atzlich muss noch eine Funktion
\\[0.2cm]
\hspace*{1.3cm}
$\textsl{arity}: \mathbb{K} \rightarrow \mathbb{N}$
\\[0.2cm]
gegeben sein, die f\"ur jeden Typ-Konstruktor festlegt, wieviel Argumente er erwartet.
In dem obigen Beispiel h\"atten wir
\\[0.2cm]
\hspace*{1.3cm}
$\textsl{arity}(\textsl{map}) = 2$, \quad
$\textsl{arity}(\textsl{list}) = 1$, \quad
$\textsl{arity}(\textsl{int}) = 0$ \quad und \quad
$\textsl{arity}(\textsl{double}) = 0$.
\\[0.2cm]
Dann wird die Menge $\mathcal{T}$ der Typen induktiv definiert:
\begin{enumerate}
\item Jeder Typ-Parameter $X$ ist ein Typ:
      \\[0.2cm]
      \hspace*{1.3cm} $X \in \mathbb{P} \Rightarrow X \in \mathcal{T}$.
\item Ist $c$ ein Typ-Konstruktor mit $\textsl{arity}(c) = 0$, so ist auch $c$ ein Typ:
      \\[0.2cm]
      \hspace*{1.3cm}
      $c \in \mathbb{K} \wedge \textsl{arity}(c) = 0 \Rightarrow c \in \mathcal{T}$.
\item Ist $f$ ein $n$-stelliger Typ-Konstruktor und sind $\tau_1$, $\cdots$, $\tau_n$
      Typen, so ist auch $f(\tau_1,\cdots,\tau_n)$ ein Typ:
      \\[0.2cm]
      \hspace*{1.3cm}
      $f \in \mathbb{K} \wedge \textsl{arity}(f) = n \wedge n > 0 \wedge
       \tau_1 \in \mathcal{T} \wedge \cdots \wedge \tau_n \in \mathcal{T} \Rightarrow
       f(\tau_1, \cdots, \tau_n) \in \mathcal{T}$.
\end{enumerate}

\examplesEng
If the type constructors \textsl{map}, \textsl{int}, \textsl{list}, and \textsl{double}
have the arities given above and if, furthermore, $K$ and $V$ are type parameters, then
the following are types:
\begin{enumerate}
\item \textsl{int},
\item \textsl{double},
\item $\textsl{list}(\textsl{double})$, 
\item $\textsl{list}(V)$, 
\item $\textsl{map}(K,\textsl{list}(\textsl{double}))$.  \eox
\end{enumerate}  


\begin{Definition}[Parameter Substitution]
{\em
    A \emph{parameter substitution} is a finite set of pairs of the form \\[0.2cm]
    \hspace*{1.3cm} $\sigma = \bigl\{ \langle X_1, \tau_1 \rangle, \cdots, \langle X_n, \tau_n \rangle \bigr\}$ \\[0.2cm]
    such that
    \begin{enumerate}
    \item $X_i$ is a type parameter for all $i \in \{1, \cdots, n\}$,
    \item $\tau_i$ is a type for all $i \in \{1, \cdots, n\}$,
    \item the type parameters occurring in $\sigma$ are pairwise distinct, that is we have
          \\[0.2cm]
          \hspace*{1.3cm}
          $i\not=j \rightarrow X_i \not= X_j$ \quad for all $i,j  \in \{1, \cdots, n\}$.
    \end{enumerate}
    If $\sigma = \bigl\{ \langle X_1, \tau_1 \rangle, \cdots, \langle X_n, \tau_n \rangle
    \bigr\}$ is a  parameter substitution, then $\sigma$ is written as  \\[0.2cm]
    \hspace*{1.3cm} 
    $\sigma = \bigl[ X_1 \mapsto \tau_1, \cdots, X_n \mapsto \tau_n \bigr]$.  
    \\[0.2cm]
    In this case, the  \emph{domain} of $\sigma$ is defined as 
    \\[0.2cm]
    \hspace*{1.3cm}
    $\textsl{dom}(\sigma) = \{ X_1, \cdots, X_n\}$.
    \\[0.2cm]
    The set of all parameter substitutions is denoted as \textsc{Subst}.
    In the following, \emph{parameter substitutions} are just called
    \emph{substitutions}.  \qed
}  
\end{Definition}

\noindent
Substitutions can be \emph{applied} to types.
If $\tau$ is a type and 
$\vartheta = \bigl[ X_1 \mapsto \tau_1, \cdots, X_n \mapsto \tau_n \bigr]$ 
is a substitution, then $\tau\vartheta$ is the type
that we get if we replace all occurrences of $X_i$ in $\tau$ by $\tau_i$,
The formal definition follows. 

\begin{Definition}[Application of a  Substitution]
{\em \hspace*{\fill} \\
If  $\tau$ is a type and  
$\vartheta = \bigl[ X_1 \mapsto \tau_1, \cdots, X_n \mapsto \tau_n \bigr]$ is a
substitution, then the \emph{application} of $\vartheta$ to $\tau$   
(written $\tau\vartheta$) is defined by  induction on  $\tau$: 
\begin{enumerate}
\item If $\tau$ is a  type parameter, there are two cases:
  \begin{enumerate}
  \item $\tau = X_i$ for some $i\in\{1,\cdots,n\}$.  Then
        \\[0.2cm]
        \hspace*{1.3cm}
        $X_i\vartheta := \tau_i$.
  \item $\tau = Y$ where $Y$ is a type parameter such that $Y \not\in \{X_1,\cdots,X_n\}$. 
        Then
        \\[0.2cm]
        \hspace*{1.3cm}
        $Y\vartheta := Y$.
  \end{enumerate}
\item Otherwise $\tau$ must have the form $\tau = f(\sigma_1,\cdots,\sigma_m)$.
      Then $\tau\vartheta$ is defined as \\[0.2cm] 
      \hspace*{1.3cm} 
      $f(\sigma_1, \cdots, \sigma_m)\vartheta := f(\sigma_1\vartheta, \cdots,
      \sigma_m\vartheta)$.   \qed
\end{enumerate}
}
\end{Definition}

\examplesEng
Define the substitution $\vartheta$ as \\[0.2cm]
\hspace*{1.3cm} 
$\vartheta := \big[ X_1 \mapsto \mathtt{double},\; X_2 \mapsto \mathtt{list(int)} \big]$. \\[0.2cm]
Then we have the following:
\begin{enumerate}
\item $X_3\vartheta = X_3$,
\item $\mathtt{list}(X_2)\vartheta = \mathtt{list}\bigl(\mathtt{list(int)}\bigr)$,
\item $\mathtt{map}(X_1,\mathtt{set}(X_2))\vartheta = 
       \mathtt{map}\bigl(\mathtt{double},\mathtt{set}(\mathtt{list}(\mathtt{int}))\bigr)$.
\end{enumerate}


\noindent
Next, we show how substitutions can be composed.
\begin{Definition}[Composition of Substitutions] 
{\em
    If \\[0.2cm]
    \hspace*{1.3cm}  
    $\vartheta = \big[ X_1 \mapsto \sigma_1, \cdots, X_m \mapsto \sigma_m \big]$ \quad and \quad
    $\eta = \big[ Y_1 \mapsto \tau_1, \cdots, Y_n \mapsto \tau_n \big]$ 
    \\[0.2cm]
    are substitutions such that $\textsl{dom}(\vartheta) \cap \textsl{dom}(\eta) = \{\}$,
    then we define the \emph{composition} $\vartheta\eta$ of $\vartheta$ and $\eta$ as \\[0.2cm]
    \hspace*{1.3cm} $\vartheta\eta := 
    \big[ X_1 \mapsto \sigma_1\eta, \cdots, X_m \mapsto \sigma_m\eta,\; 
          Y_1 \mapsto \tau_1, \cdots, Y_n \mapsto \tau_n \big]$.
    \qed
}
\end{Definition}

\exampleEng
Define \\[0.2cm]
\hspace*{1.3cm} 
$\vartheta := \big[ X_1 \mapsto \mathtt{double},\; X_2 \mapsto \mathtt{list(X_3)} \big]$
\quad and 
\\[0.2cm]
\hspace*{1.3cm}
$\eta := \big[ X_3 \mapsto \texttt{map}(\mathtt{int},\mathtt{double}),\; X_4 \mapsto \mathtt{char} \big]$. 
\\[0.2cm]
Then we have \\[0.2cm]
\hspace*{1.3cm} 
$\vartheta\eta = \big[ X_1 \mapsto \mathtt{double},\; 
                       X_2 \mapsto \mathtt{list}(\mathtt{map(int, double)}),\;
 X_3 \mapsto \mathtt{map(int, double)},\; X_4 \mapsto \mathtt{char} \big]
$.
\eox
\vspace{0.3cm}

\exerciseEng
How would we have to change the definition of the composition of $\vartheta$ and
$\eta$ if we drop the condition $\textsl{dom}(\vartheta) \cap \textsl{dom}(\eta) = \{\}$?
\vspace{0.3cm}

The following theorem is a consequence of the previous definition.
\begin{Theorem}[Associativity of Composition]\label{satz:komposition}  \hspace*{\fill} \\
{\em
    If $\tau$ is a type and $\vartheta$ and $\eta$ are substitutions such that 
    $\textsl{dom}(\vartheta) \cap \textsl{dom}(\eta) = \{\}$ holds, then we have \\[0.2cm]
    \hspace*{1.3cm} $(\tau \vartheta)\eta = \tau (\vartheta\eta)$.
    \hspace*{\fill} $\Box$
}
\end{Theorem}
This theorem is proved by an easy induction on $\tau$.
\vspace{0.3cm}

\noindent
Next, we formalize the notion of a term $t$ being of type $\tau$.
\begin{Definition}[$t:\tau$]  \lb
{\em
  For a term  $t$ and a type $\tau$ the relation $t: \tau$ 
  (read: $t$ has type $\tau$) is defined by induction on  $t$.
  \begin{enumerate}
  \item If $c$ is a nullary function symbol with signature $c: \tau$ and if
        $\vartheta$ is any  parameter substitution, then
        \\[0.2cm]
        \hspace*{1.3cm} $c: \tau\vartheta$.
  \item Assume that
        \begin{enumerate}
        \item  $f$ is an  $n$-ary function symbol such that $n > 0$.
        \item  $f$ has the signature
               \\[0.2cm]
               \hspace*{1.3cm}
               $f: \sigma_1 \times \cdots \times \sigma_n \rightarrow \varrho$.
        \item $\vartheta$ is a substitution such that we have $t_i:\sigma_i\vartheta$.
        \end{enumerate}
        Then  $f(t_1, \cdots, t_n): \varrho\vartheta$.
  \end{enumerate}
}  
\end{Definition}

\examplesEng
The following examples assume that the types and signatures are given as in Figure
 \ref{fig:types.ttl} on page \pageref{fig:types.ttl}.
\begin{enumerate}
\item We have $\textsl{nil}: \mathtt{list(int)}$, 
      because $\textsl{nil}$ has the signature
      \\[0.2cm]
      \hspace*{1.3cm}
      $\textsl{nil}: \mathtt{list(X)}$. 
      \\[0.2cm]
      Therefore, defining
      \\[0.2cm]
      \hspace*{1.3cm}
      $\vartheta = [ X \mapsto \mathtt{int} ]$
      \\[0.2cm]
      gives $\mathtt{list}(X)\vartheta = \mathtt{list(int)}$ showing the claim.
\item Next, we show 
      \\[0.2cm]
      \hspace*{1.3cm}
      $\textsl{concat}(\textsl{nil}, \textsl{nil}): \mathtt{list(int)}$.
      \\[0.2cm]
      The signature of $\textsl{concat}$ is
      \\[0.2cm]
      \hspace*{1.3cm}
      $\textsl{concat}: \textsl{list}(T) \times \textsl{list}(T) \rightarrow \textsl{list}(T)$. 
      \\[0.2cm]
      Define
      \\[0.2cm]
      \hspace*{1.3cm}
      $\vartheta = [ T \mapsto \mathtt{int} ]$.
      \\[0.2cm]
      We have already seen that
      \\[0.2cm]
      \hspace*{1.3cm}
      $\textsl{nil}: \mathtt{list(int)}$
      \\[0.2cm]
      holds.  Because of $\mathtt{list(T)}\vartheta = \mathtt{list(int)}$ this shows the claim.  \eox
\end{enumerate}


\section{A Type Checking Algorithm}
Assume a  term $t = f(t_1,\cdots,t_n)$ and a type $\tau$ is given and we want to check
whether $t:\tau$ holds.  Furthermore, assume that the signature of  $f$ is given as
\[ f: \sigma_1 \times \cdots \times \sigma_n \rightarrow \varrho. \]
According to the definition of  $t:\tau$ the term $t$ has type $\tau$ iff there is a
parameter substitution $\vartheta$ such that $\varrho \vartheta = \tau$
and  $t_i:\sigma_i \vartheta$. Therefore
\[ 
           f(t_1,\cdots,t_n): \tau \;\Leftrightarrow\; 
           \exists \vartheta \in \textsc{Subst}: \bigl(\varrho \vartheta = \tau \wedge 
           \forall i \in \{1,\cdots,n\}: t_i : \sigma_i \vartheta \bigr)
\]
The problem of type checking is to compute the substitution $\vartheta$ or to show that
$\vartheta$ can not exist.
 

\examplesEng
We discuss the previous examples (that is $\textsl{nil}: \mathtt{list(int)}$ and
$\textsl{concat}(\textsl{nil}, \textsl{nil}): \mathtt{list(int)}$) 
again using the formula
\\[0.2cm]
\hspace*{1.3cm}
$f(t_1,\cdots,t_n): \tau \;\Leftrightarrow\; 
 \exists \vartheta \in \textsc{Subst}: \bigl(\varrho \vartheta = \tau \wedge 
 \forall i \in \{1,\cdots,n\}: t_i : \sigma_i \vartheta \bigr)
$.
\begin{enumerate}
\item Lets prove $\textsl{nil}: \mathtt{list(int)}$ again.  We have the signature
      \[ \textsl{nil}: \mathtt{list(X)}. \]
      Therefore, we have
      \[ \textsl{nil}: \mathtt{list(int)} \;\Leftrightarrow\; 
         \exists \vartheta \in \textsc{Subst}: \bigl(\mathtt{list}(X) \vartheta = \mathtt{list(int)}  \bigr).
      \]
      Obviously, we can define 
      \[ \vartheta = [ X \mapsto \mathtt{int} ]. \]
      Then, the equation $\mathtt{list}(X) \vartheta = \mathtt{list(int)}$ is true and we
      conclude that $\textsl{nil}: \mathtt{list(int)}$ holds. 
\item Next, we prove $\textsl{concat}(\textsl{nil}, \textsl{nil}): \mathtt{list(int)}$.
      The signature of \textsl{concat} is given as
      \[ \textsl{concat}: \textsl{list}(T) \times \textsl{list}(T) \rightarrow \textsl{list}(T). \]
      Therefore, our claim is equivalent to
      \\[0.2cm]
      \hspace*{1.3cm}
      $
      \begin{array}{cl}
                        & \textsl{concat}(\textsl{nil},\textsl{nil}): \mathtt{list(int)} \\[0.2cm]
       \Leftrightarrow  & \exists \vartheta \in \textsc{Subst}: \bigl(\mathtt{list}(T) \vartheta = \mathtt{list(int)}
                          \wedge \textsl{nil} : \textsl{list}(T) \vartheta \wedge \textsl{nil} : \textsl{list}(T)
                          \vartheta \bigr).
      \end{array}
      $
      \\[0.2cm]
      The equation $\mathtt{list}(T) \vartheta = \mathtt{list(int)}$ is solved by the substitution
      \[ \vartheta = [ T \mapsto \mathtt{int} ] \]
      and therefore the correctness of 
      \[ \textsl{concat}(\textsl{nil},\textsl{nil}): \mathtt{list(int)} \]
      is reduced to the correctness of
      \[ \textsl{nil} : \textsl{list}(T) \vartheta,  \]      
      As $\textsl{list}(T) \vartheta = \textsl{list}(\textsl{int})$ and 
      $\textsl{nil} : \textsl{list(int})$ holds, the proof is complete. \qed
\end{enumerate}


The previous examples show that the correctness of an expression
$t: \tau$ can be reduced to the solution of a set of  \emph{type equations} of the form
$\varrho \vartheta = \tau$.
We formalize this observation by defining a function 
\[ \textsl{typeEqs}: \mathtt{Term} \times \mathtt{Type} \rightarrow \mathtt{set}(\mathtt{Equation}). \]
For a term $t$ and a  type $\tau$ 
\[ \textsl{typeEqs}(t, \tau) \]
yields a set of type equations.  These type equations will be solvable if and only if
 $t: \tau$ is correct.
The function $\textsl{typeEqs}$ is defined as follows:
\\[0.2cm]
\hspace*{1.3cm}
$\bigl(f: \sigma_1 \times \cdots \times \sigma_n \rightarrow \varrho\bigr) \;\Longrightarrow\;
 \textsl{typeEqs}\bigl( f(t_1,\cdots,t_n), \tau \bigr) := 
   \{ \varrho = \tau \} \cup \bigcup\limits_{i=1}^n \textsl{typeEqs}(t_i, \sigma_i)$.
\\[0.2cm]
This definition has to be read as follows:  If the function  $f$ has the signature
$f: \sigma_1 \times \cdots \times \sigma_n \rightarrow \varrho$, 
then  $f(t_1, \cdots, t_n): \tau$ is true if and only if the set of \emph{type equations}
\[ \{ \varrho = \tau \} \cup \bigcup\limits_{i=1}^n \textsl{typeEqs}(t_i, \sigma_i) \]
has a solution $\vartheta$.  We formalize the notion of a \emph{type equation} next.


\begin{Definition}[Type Equation]
{\em
If $\sigma$ and $\tau$ are types, then the expression
\\[0.2cm]
\hspace*{1.3cm}
$\sigma \doteq \tau$
\\[0.2cm]
is called a \emph{type equation}.  A \emph{system of type equations} is a set of type
equations. A substitution $\vartheta$ \emph{solves} a type equation
 $\sigma \doteq \tau$ if and only if $\sigma\vartheta = \tau\vartheta$.  A substitution
 $\vartheta$ solves a system of type equations $E$ iff it solves every type equation in $E$.
}
\qed
\end{Definition}

In order to implement type checking we still need an algorithm to solve systems of type equations.
The crucial observation is that type equations can be solved by unification.  We do not
need the general form of unification because if we have a type equation of the form
 $\varrho = \tau$ then only the left hand side $\varrho$ will contain type parameters.
We use the algorithm given by  Martelli and Montanari \cite{martelliMontanari:82}.
This algorithm works with pairs of the form
\\[0.2cm]
\hspace*{1.3cm}
$\langle E, \vartheta \rangle$
\\[0.2cm]
where $E$ is a set of type equations and $\vartheta$ is a substitution. 
We start with the pair
\\[0.2cm]
\hspace*{1.3cm}
$\langle E, [] \rangle$. 
\\[0.2cm]
Here, the substitution is empty, while  $E$ is the set of type equations we want to solve.
These pairs will be gradually transformed until we arrive a a pair of the form
\\[0.2cm]
\hspace*{1.3cm}
$\langle \{\}, \vartheta \rangle$
\\[0.2cm]
such that $\vartheta$ is the solution of the system of type equations $E$.
We use the following rules to rewrite the pairs.
\begin{enumerate}
\item If $X$ is a type parameter, then 
      \\[0.2cm]
      \hspace*{1.3cm}
      $\Big\langle E \cup \big\{ X \doteq \tau \big\}, \vartheta \Big\rangle \quad\leadsto\quad 
       \Big\langle E[X \mapsto \tau], \vartheta\big[ X \mapsto \tau \big] \Big\rangle$.
      \\[0.2cm]
      This works as follows:  If $E$ contains a type equation of the form $X \doteq \tau$      , 
      then we can remove this type equation from $E$ if we incorporate this equation
      in the substitution $\vartheta$ by transforming $\vartheta$ into the new substitution
      $\vartheta\big[ X \mapsto \tau \big]$.  Of course, we also have to apply the
      substitution $[X \mapsto \tau]$ to the remaining type equations in $E$.
\item If $f$ is an $n$-ary type constructor we have the rule
      \\[0.2cm]
      \hspace*{1.3cm}
      $\Big\langle E \cup \big\{ f(\sigma_1,\cdots,\sigma_n) \doteq f(\tau_1,\cdots,\tau_n) \big\}, 
        \vartheta \Big\rangle 
        \;\leadsto\; 
        \Big\langle E \cup \big\{ \sigma_1 \doteq \tau_1, \cdots, \sigma_n \doteq \tau_n\}, 
        \vartheta \Big\rangle$.
      \\[0.2cm]
      Therefore, a type equation of the form $f(\sigma_1,\cdots,\sigma_n) \doteq f(\tau_1,\cdots,\tau_n)$
      is replaced by the  $n$ type equations  
      $\sigma_1 \doteq \tau_1$, $\cdots$, $\sigma_n \doteq \tau_n$.

      A special case of this occurs if $n=0$.  It reads
      \\[0.2cm]
      \hspace*{1.3cm}
      $\Big\langle E \cup \big\{ c \doteq c \big\}, \vartheta \Big\rangle \;\leadsto\; 
       \Big\langle E, \vartheta \Big\rangle$.
      \\[0.2cm]
      Here $c$ is a nullary type constructor.   This rule just says that trivial 
      type equations can be dropped.
\item A system of type equations of the form  $E \cup \big\{ f(\sigma_1,\cdots,\sigma_m) \doteq g(\tau_1,\cdots,\tau_n) \big\}$
      has no solution if $f$ and $g$ are different.   Therefore, we have
      \\[0.2cm]
      \hspace*{1.3cm}
      $\Big\langle E \cup \big\{ f(\sigma_1,\cdots,\sigma_m) \doteq g(\tau_1,\cdots,\tau_n) \big\}, 
       \vartheta \Big\rangle 
         \;\leadsto\; \Omega$  \quad if $f \not= g$.
      \\[0.2cm]
      Here $\Omega$ denotes unsolvability.  If the solution of $\textsl{typeEqs}(t,\tau)$ yields $\Omega$,
      then the set of type equations $E$ can not be solved and  $t:\tau$ 
      is wrong.  \eox
\end{enumerate}

\exampleEng
We demonstrate the algorithm by solving the type equation \\[0.1cm]
\hspace*{1.3cm}  $\texttt{map}( X_2, X_3) \doteq \texttt{map}(\texttt{char}, \mathtt{list}(\texttt{int}))$.  
\\[0.1cm]
We proceed as follows:
$$
\begin{array}{ll}
          &  \big\langle \big\{ \texttt{map}( X_2, X_3) \doteq \texttt{map}(\texttt{char}, \texttt{list}(\texttt{int}))  \big\}, \big[ \big] \big\rangle \\[0.2cm]
 \leadsto &  \big\langle \big\{ X_2 \doteq \mathtt{char},\;  X_3 \doteq\texttt{list}(\texttt{int}) \big\}, \big[ \big] \big\rangle \\[0.2cm]
 \leadsto &  \big\langle \big\{ X_3 \doteq \texttt{list}(\texttt{int}) \big\}, \big[ X_2 \mapsto \mathtt{char} \big] \big\rangle \\[0.2cm]
 \leadsto &  \big\langle \big\{\big\}, \big[ X_2 \mapsto \texttt{char},\; X_3 \mapsto \texttt{list}(\texttt{int}) \big] \big\rangle \\[0.2cm]
\end{array}
$$
In this case, the algorithm is successful and the resulting  substitution  \\[0.1cm]
\hspace*{1.3cm} $\big[ X_2 \mapsto \texttt{char},\; X_3 \mapsto \texttt{list}(\texttt{int}) \big]$ \\[0.1cm]
is a solution of the type equation given above.  
\pagebreak

%
\section{Exkurs: Der Klassen-Generator \texttt{EP}}
Die meisten Parser haben die Aufgabe, einen Parse-Baum zu erstellen.  In diesem Parsebaum wird jeder
Knoten durch ein Objekt einer Klasse dargestellt.  Die Klassen dieser Objekte orientieren sich
typischerweise an den Symbolen der Grammatik, so dass in der Regel f\"ur jedes Symbol auch eine Klasse
existiert.  Das f\"uhrt dazu, dass wir bei der Entwicklung eines solchen Parses eine gro{\ss}e Zahl von
\textsl{Java}-Klassen definieren m\"ussen.  Das ist zwar nicht weiter schwer, aber doch mit einem
hohen Schreibaufwand verbunden.  Um hier die Arbeit abzuk\"urzen, k\"onnen wir einen
\emph{Klassen-Generator} einsetzen, die die ben\"otigten Klassen aus einer Spezifikation erzeugt.  Da
diese Spezifikation meist wesentlich k\"urzer ist als die \textsl{Java}-Klassen, k\"onnen wir dadurch 
einiges an Schreibarbeit einsparen.  In diesem Abschnitt m\"ochte ich Ihnen ein solches System
vorstellen.  Es handelt sich um das System EP, dass Sie unter der Adresse
\\[0.2cm]
\hspace*{1.3cm}
\texttt{http://www.ba-stuttgart.de/stroetmann/EP/ep.tar.gz}
\\[0.2cm]
im Netz finden.

\subsection{Installation}
Wir beschreiben zun\"achst die Installation dieses Systems.  Die nachfolgende Anleitung geht
von einem Linux oder Unix-System aus.  Wir nehmen an, dass der Klassen-Generator in
dem Ordner \texttt{Software}, der sich im Home-Verzeichnis des Benutzers befindet, installiert
werden soll.  Die Installation besteht dann aus folgenden Schritten:
\begin{enumerate}
\item Zun\"achst schieben wir die Datei ``\texttt{ep.tar.gz}'' 
      in das Verzeichnis \texttt{\symbol{126}/Software}: \\[0.2cm]
      \hspace*{1.3cm} \texttt{mv ep.tar.gz \symbol{126}/Software}
\item Nun wechseln wir in dieses Verzeichnis: \\[0.2cm]
      \hspace*{1.3cm} \texttt{cd \symbol{126}/Software}
\item Dort dekomprimieren wir die Datei \texttt{ep.tar.gz} mit dem Befehl \\[0.2cm]
      \hspace*{1.3cm} \texttt{gunzip ep.tar.gz} \\[0.2cm]
      Dabei entsteht die Datei \texttt{ep.tar}.
\item Die Datei \texttt{ep.tar} wird nun entpackt: \\[0.2cm]
      \hspace*{1.3cm} \texttt{untar xf ep.tar} \\[0.2cm]
      Durch diesen Befehl wird im Verzeichnis \texttt{Software} ein Unterverzeichnis mit dem Namen 
       \texttt{EP} angelegt.  Dieses Unterverzeichnis enth\"alt drei Verzeichnisse:
      \begin{enumerate}
      \item \texttt{src} enth\"alt die \textsl{Java}-Quellen zusammen mit den Klassen-Dateien.
            Falls die Version Ihrer \textsl{Java}-Umgebung nicht mit der Version \"ubereinstimmt,
            f\"ur die die Klassen-Dateien erzeugt wurden, m\"ussen Sie in diesem Verzeichnis mit dem Befehl
            \\[0.2cm]
            \hspace*{1.3cm}
            \texttt{javac *.java}
            \\[0.2cm]
            alle \textsl{Java}-Quelldateien neu \"ubersetzen.
      \item \texttt{doc} enth\"alt die Dokumentation.
      \item \texttt{examples} enth\"alt die Beispiel-Datei ``\texttt{expr.ep}''.
            In dieser Datei werden Klassen spezifiziert, die arithmetische Ausdr\"ucke beschreiben.
      \end{enumerate}
      Zus\"atzlich enth\"alt das Verzeichnis \texttt{EP} noch die Datei \texttt{antlr-2.7.5.jar}.
      Hierbei handelt es sich um die Version des Parser-Generator \textsc{Antlr}
      \cite{parr95antlr},
      die ich bei der Erstellung des Parsers verwendet habe.
\item Um EP verwenden zu k\"onnen, m\"ussen Sie die Datei \texttt{antlr-2.7.5.jar} und das
      Verzeichnis \texttt{src} in den \texttt{CLASSPATH} aufnehmen.  Das geht in einem
      \textsl{Unix}-basiertem Betriebssystem beispielsweise,
      indem Sie in der Datei \texttt{\symbol{126}/.bashrc} folgende Zeile hinzuf\"ugen:
      \\[0.2cm]
      \texttt{export
        CLASSPATH=\symbol{36}CLASSPATH:\symbol{126}/Software/EP/antlr-2.7.5.jar:\symbol{126}/Software/EP/src} 
\item Um die Installation zu testen, wechseln Sie in das Verzeichnis ``\texttt{examples}'' and
      starten dort EP, indem Sie das Kommando 
      \\[0.2cm]
      \hspace*{1.3cm} \texttt{java EP expr} 
      \\[0.2cm]
      ausf\"uhren.  
      Dieses Kommando sollte eine Reihe von \textsl{Java}-Dateien erzeugen, die dann durch den Befehl
      \\[0.2cm]
      \hspace*{1.3cm} \texttt{javac *.java} \\[0.2cm]
      \"ubersetzt werden k\"onnen.
      Anschlie{\ss}end k\"onnen wir das Kommando \\[0.2cm]
      \hspace*{1.3cm} \texttt{java Main} \\[0.2cm]
      ausf\"uhren.  Dieses Kommando sollte das folgende Ergebnis produzieren: \\[0.2cm]
      \hspace*{1.3cm} \texttt{Product(Variable(x), Variable(x))/dx = } \\
      \hspace*{2.3cm} \texttt{Sum(Product(Number(1.0), Variable(x)),} \\
      \hspace*{3.04cm}     \texttt{Product(Variable(x), Number(1.0)))}. 
\end{enumerate}

\section{Eine Beispiel-Datei}
Abbildung \ref{fig:expr.ep} zeigt eine EP-Spezifikation f\"ur Klassen, die arithmetische Ausdr\"ucke
spezifizieren.   Die eigentliche Spezifikation der Klassen findet sich in den Zeilen 1 -- 6.
Hier wird eine abstrakte Klasse \texttt{Expr} definiert, von der insgesamt 6 Klassen abgeleitet
werden.  Als ein Beispiel betrachten wir die Klasse \texttt{Sum}.  Der Ausdruck
\\[0.2cm]
\hspace*{1.3cm}
\texttt{Sum(Expr lhs, Expr rhs)}
\\[0.2cm]
spezifiziert, dass die Klasse \texttt{Sum} zwei Member-Variablen enth\"alt, die beide den Typ
\texttt{Expr} haben.  Weiterhin werden die Namen dieser Member-Variablen definiert.  Der Name der
ersten Member-Variable ist sp\"ater \texttt{mLhs}, die zweite Member-Variable hei{\ss}t \texttt{mRhs}.
Dem in der Spezifikation angegebenen Namen wird also ein ``\texttt{m}'' vorangestellt und der darauf
folgende Buchstabe wird gro{\ss} geschrieben.

\begin{figure}[thb]
  \centering
\begin{Verbatim}[ frame         = lines, 
                  framesep      = 0.3cm, 
                  labelposition = bottomline,
                  numbers       = left,
                  numbersep     = -0.2cm,
                  xleftmargin   = 0.0cm,
                  xrightmargin  = 0.0cm,
                ]
    Expr = Number(Double value)
         + Variable(String name)
         + Sum(Expr lhs, Expr rhs) 
         + Difference(Expr lhs, Expr rhs)
         + Product(Expr lhs, Expr rhs)     
         + Quotient(Expr lhs, Expr rhs);
    
    diff: Expr * String -> Expr;

    Number(value).diff(x) = Number(0.0);

    v.equals(x) -> Variable(v).diff(x) = Number(1.0);
    Variable(v).diff(x) = Number(0.0);

    Sum(l, r).diff(x) = Sum(l.diff(x), r.diff(x));
    Difference(l, r).diff(x) = Difference(l.diff(x), r.diff(x));
    Product(l, r).diff(x) = 
        Sum(Product(l.diff(x), r), Product(l, r.diff(x)));
    Quotient(l, r).diff(x) = 
        Quotient( Difference(Product(l.diff(x), r), Product(l, r.diff(x))), 
                  Product(r, r) );
\end{Verbatim} 
\vspace*{-0.3cm}
  \caption{Die Spezifikation arithmetischer Ausdr\"ucke.}
  \label{fig:expr.ep}
\end{figure} %\$

\begin{figure}[thb]
  \centering
\begin{Verbatim}[ frame         = lines, 
                  framesep      = 0.3cm, 
                  labelposition = bottomline,
                  numbers       = left,
                  numbersep     = -0.2cm,
                  xleftmargin   = 0.0cm,
                  xrightmargin  = 0.0cm,
                ]
    public class Sum extends Expr {
        private Expr mLhs;
        private Expr mRhs;
    
        public Sum(Expr lhs, Expr rhs) {
            mLhs = lhs;
            mRhs = rhs;
        }
        public Expr diff(String x) {
            return new Sum(mLhs.diff(x), mRhs.diff(x));
        }
        public Boolean equals(Expr rhs) {
            if (!(rhs instanceof Sum)) {
                return false;
            }
            Sum r = (Sum) rhs;
            if(!mLhs.equals(r.mLhs)) {
                return false;
            }
            if(!mRhs.equals(r.mRhs)) {
                return false;
            }
            return true;
        }
        public Expr getLhs() {
            return mLhs;
        }
        public Expr getRhs() {
            return mRhs;
        }
        public String toString() {
            return "Sum(" + mLhs.toString() + ", " + mRhs.toString() + ")";
        }
    }
\end{Verbatim} 
\vspace*{-0.3cm}
  \caption{Die Klasse \texttt{Sum}.}
  \label{fig:Sum.java.ep}
\end{figure} %\$

Neben der Spezifikation der Klassen enth\"alt die in Abbildung \ref{fig:expr.ep} gezeigte Datei noch
die Spezifikation der Signatur einer Methode: In Zeile 8 wird die Methode \texttt{diff}
spezifiziert.  Das implizite Argument dieser Methode ist vom Typ \texttt{Expr}, denn es handelt sich
ja um eine Methode der Klasse \texttt{Expr}, das erste Argument ist vom Type \texttt{String} und das Ergebnis
ist vom Type \texttt{Expr}.  Die nachfolgenden bedingten Gleichungen zeigen, wie die Methode
$\textsl{diff}()$, die das symbolische Differenzieren beschreibt, spezifiziert werden kann.   Aus
dieser Spezifikation erzeugt \textsc{EP} nun die in Zeile 1 -- 6 spezifizierten Klassen.  Wir
betrachten exemplarisch die Klasse \texttt{Sum}, die in Abbildung \ref{fig:Sum.java.ep} gezeigt ist.
Wir diskutieren den erzeugten \textsl{Java}-Code jetzt im Detail.
\begin{enumerate}
\item In Zeile 1 sehen wir, dass die Klasse \texttt{Sum} von der Klasse \texttt{Expr}
      abgeleitet wird.
\item In Zeile 2 und 3 werden die Member-Variablen \texttt{mLhs} und \texttt{mRhs} definiert.
\item In Zeile 5 -- 8 finden wir einen Konstruktor, der die Member-Variablen initialisiert.
\item In Zeile 9 -- 11 finden wir die Implementierung der Methode $\textsl{diff}$.  Diese Zeile
      setzt die Zeile 15 aus der Abbildung \ref{fig:expr.ep} um.
\item Des weiteren enth\"alt die Klasse \texttt{Sum} noch eine Methode $\texttt{equals}()$, die
      automatisch erzeugt wird.
\item Au{\ss}erdem werden noch Getter-Methoden f\"ur die Member-Variablen in den Zeilen 25 -- 30
      erzeugt.
\item Schlie{\ss}lich zeigt Zeile 31 -- 33 eine automatisch erzeugte \texttt{toString}-Methode.
\end{enumerate}
Um die von \textsc{EP} erzeugten Klassen zu testen, ben\"otigen wir noch eine Methode
$\texttt{main}()$.  Abbildung \ref{fig:Main.java} zeigt die Klasse \texttt{Main}, die eine
solche Methode enth\"alt.  \"Ubersetzen wir diese Klasse und f\"uhren wir das erzeugte Programm
aus, so erhalten wir die folgende Ausgabe:
\\[0.2cm]
\hspace*{1.3cm}
\texttt{Sum(Product(Number(1.0), Variable(x)), Product(Variable(x), Number(1.0)))}
\\[0.2cm]


\begin{figure}[!ht]
\centering
\begin{Verbatim}[ frame         = lines, 
                  framesep      = 0.3cm, 
                  labelposition = bottomline,
                  numbers       = left,
                  numbersep     = -0.2cm,
                  xleftmargin   = 0.8cm,
                  xrightmargin  = 0.8cm,
                ]
    public class Main {
        public static void main(String[] args) {
            Expr x      = new Variable("x");
            Expr square = new Product(x, x);
            System.out.println(square.diff("x"));
        }
    }
\end{Verbatim}
\vspace*{-0.3cm}
\caption{Eine Klasse zum Testen der von EP erzeugten Klassen.}
\label{fig:Main.java}
\end{figure}


\begin{figure}[!ht]
\centering
\begin{Verbatim}[ frame         = lines, 
                  framesep      = 0.3cm, 
                  labelposition = bottomline,
                  numbers       = left,
                  numbersep     = -0.2cm,
                  xleftmargin   = 0.0cm,
                  xrightmargin  = 0.0cm,
                ]
    import java_cup.runtime.*;
          
    %%
       
    %char
    %line
    %column
    %cup
       
    %{   
        private Symbol symbol(int type) {
            return new Symbol(type, yychar, yychar + yylength());
        }
        
        private Symbol symbol(int type, Object value) {
            return new Symbol(type, yychar, yychar + yylength(), value);
        }
    %}
       
    %%
       
    "+"          { return symbol( sym.PLUS   ); }
    "-"          { return symbol( sym.MINUS  ); }
    "*"          { return symbol( sym.TIMES  ); }
    "/"          { return symbol( sym.DIVIDE ); }
    "("          { return symbol( sym.LPAREN ); }
    ")"          { return symbol( sym.RPAREN ); }
    
    [1-9][0-9]*  { return symbol(sym.NUMBER,     new Integer( yytext())); }
    [a-zA-Z]+    { return symbol(sym.IDENTIFIER, new Variable(yytext())); }
    [ \t\n]      { /* skip white space */ }   
\end{Verbatim}
\vspace*{-0.3cm}
\caption{\textsl{JFlex}-Spezifikation des Scanners}
\label{fig:differentiator.jflex}
\end{figure}
\noindent
Wir zeigen nun, wie wir alle  drei Werkzeuge
\textsl{JFlex}, \textsl{JavaCup} und \textsc{Ep} gemeinsam verwenden k\"onnen um ein
Programm zum symbolischen Differenzieren erstellen zu k\"onnen.  
\begin{enumerate}
\item Abbildung \ref{fig:differentiator.jflex} zeigt die Spezifikation des Scanners.
      Gegen\"uber dem in Abbildung \ref{fig:calc.jflex} gezeigten Scanner kommt hier die Zeile 30
      hinzu, in der Variablen erkannt werden.
\item Abbildung \ref{fig:differentiator.cup} zeigt die Spezifikation der Grammatik.
      Der wesentliche Unterschied zu der in Abbildung \ref{fig:calc.cup} gezeigten Grammatik besteht
      darin, dass mit der syntaktische Variable \textsl{expr} nun nicht mehr ein Objekt vom Typ
      \texttt{Integer} assoziiert ist, sondern stattdessen ein Objekt der Klasse \texttt{Expr}.
      Dieses Objekt repr\"asentiert einen Syntax-Baum.
\item Abbildung \ref{fig:expr.ep} aus dem letzten Abschnitt zeigt die Definition der abstrakten Klasse
      \texttt{Expr} und der davon abgeleiteten Klassen mit Hilfe einer \textsc{Ep}-Spezifikation.
\item Abbildung \ref{fig:Differentiator.java} zeigt die Definition der Klasse
      Differentiator, in der die Methode $\textsl{main}()$ enthalten ist.
  
      Hier muss die Zeile 7 n\"aher erl\"autert werden:
      Die Methode $\textsl{parse}()$ gibt ein Objekt vom Typ \texttt{Symbol} zur\"uck.  Die
      Klasse \texttt{Symbol} ist in dem Paket 
      \texttt{java\_cup.runtime} definiert und enth\"alt eine als \texttt{public} deklarierte
      Member-Variable \texttt{value}.  Diese Variable enth\"alt den Wert, welcher dem von der Methode
      $\textsl{parse}()$ erkannten Nichtterminal \textsl{expr} zugeordnet ist.  Dieser Wert ist ein
      Element der Klasse \texttt{Expr}, der in der Klasse \texttt{Symbol} aber als \texttt{Object} 
      deklariert ist.  Daher m\"ussen wir diesen Wert erst durch den Cast in Zeile 8 in den Typ 
      \texttt{Expr} \"uberf\"uhren, damit wird dann in Zeile 9 die Methode $\textsl{diff}()$  Klasse
      \texttt{Expr} aufrufen d\"urfen.
\end{enumerate}


\begin{figure}[!ht]
\centering
\begin{Verbatim}[ frame         = lines, 
                  framesep      = 0.3cm, 
                  labelposition = bottomline,
                  numbers       = left,
                  numbersep     = -0.2cm,
                  xleftmargin   = 0.0cm,
                  xrightmargin  = 0.0cm,
                ]
    import java_cup.runtime.*;
    
    terminal           PLUS, MINUS, TIMES, DIVIDE, LPAREN, RPAREN;
    terminal Integer   NUMBER;
    terminal Variable  IDENTIFIER;
    
    nonterminal Expr   expr;
    
    precedence left    PLUS, MINUS;
    precedence left    TIMES, DIVIDE;
    
    /* The grammar */
    expr    ::= expr:e1 PLUS   expr:e2 {: RESULT = new Sum(        e1, e2 ); :} 
             |  expr:e1 MINUS  expr:e2 {: RESULT = new Difference( e1, e2 ); :}
             |  expr:e1 TIMES expr:e2  {: RESULT = new Product(    e1, e2 ); :}
             |  expr:e1 DIVIDE expr:e2 {: RESULT = new Quotient(   e1, e2 ); :}
             |  NUMBER:n               {: RESULT = new Number(     n      ); :} 
             |  IDENTIFIER:v           {: RESULT = v;                        :} 
             |  LPAREN expr:e RPAREN   {: RESULT = e;                        :}
             ;
\end{Verbatim}
\vspace*{-0.3cm}
\caption{Grammatik f\"ur symbolische Ausdr\"ucke}
\label{fig:differentiator.cup}
\end{figure}


\begin{figure}[!ht]
\centering
\begin{Verbatim}[ frame         = lines, 
                  framesep      = 0.3cm, 
                  labelposition = bottomline,
                  numbers       = left,
                  numbersep     = -0.2cm,
                  xleftmargin   = 0.8cm,
                  xrightmargin  = 0.8cm,
                ]
    import java_cup.runtime.*;
    
    public class Differentiator {
        public static void main(String[] args) {
            try { 
                parser p = new parser(new Yylex(System.in)); 
                Symbol s = p.parse(); 
                Expr   e = (Expr) s.value;
                Expr   d = e.diff("x");
                System.out.println("d/dx " + e + " = " + d);
            } catch (Exception e) {}
        }
    }
\end{Verbatim}
\vspace*{-0.3cm}
\caption{Die Klasse \texttt{Differentiator}}
\label{fig:Differentiator.java}
\end{figure}

%%% Local Variables: 
%%% mode: latex
%%% TeX-master: "formale-sprachen"
%%% End: 


\section{Implementierung eines Typ-Checkers f\"ur \textsc{TTL}}
Zur Illustration der dargestellten Theorie implementieren wir einen Typ-Checker f\"ur \textsc{TTL}.
Die Grammatik hatten wir ja bereits in Abbildung \ref{fig:typeChecker-grammar} auf Seite
\pageref{fig:typeChecker-grammar} pr\"asentiert.  Die Abbildungen 
\ref{fig:typeChecker.cup-1}, \ref{fig:typeChecker.cup-2} und \ref{fig:typeChecker.cup-3} 
auf den folgenden Seiten zeigen die \textsl{JavaCup}-Spezifikation eines Parsers f\"ur diese Grammatik.
Der zugeh\"orige Scanner ist in Abbildungen  \ref{fig:typeChecker.jflex} auf Seite
\pageref{fig:typeChecker.jflex} gezeigt.  Der Scanner unterscheidet in den Zeilen 34 und 35 zwischen
Namen, die mit einem gro{\ss}en Buchstaben beginnen und solchen Namen, die mit einem kleinen Buchstaben
beginnen.  Erstere bezeichnen Typ-Parameter, letztere bezeichnen sowohl Funktionen als auch Typ-Konstruktoren.
Der von \textsl{JavaCup} generierte Parser baut einen abstrakten
Syntax-Baum auf.  Die zugeh\"origen Klassen wurden mit Hilfe des im vorhergehenden Abschnitts diskutierten
Klassen-Generators 
\textsc{EP} aus der in Abbildung \ref{fig:typeExpr.ep} gezeigten Spezifikation erzeugt.


\begin{figure}[!ht]
\centering
\begin{Verbatim}[ frame         = lines, 
                  framesep      = 0.3cm, 
                  labelposition = bottomline,
                  numbers       = left,
                  numbersep     = -0.2cm,
                  xleftmargin   = 0.0cm,
                  xrightmargin  = 0.0cm,
                ]
    import java_cup.runtime.*;
    import java.util.*;
    
    terminal           TYPE, SIGNATURE, LEFT_PAR, RIGHT_PAR;
    terminal           COMMA, COLON, SEMICOLON, ASSIGN, ARROW, PLUS, TIMES;
    terminal String    FUNCTION, PARAMETER;
    
    nonterminal Program         program;
    nonterminal Term            term;
    nonterminal List<Term>      termList;
    nonterminal List<Parameter>  varList;
    nonterminal Type            type;
    nonterminal List<Type>      typeList;
    nonterminal List<Type>      typeSum;
    nonterminal TypeDef         typeDef;
    nonterminal List<TypeDef>   typeDefList;
    nonterminal Signature       signature;
    nonterminal List<Signature> signatures;
    nonterminal List<Type>      argTypes;
    nonterminal TypedTerm       typedTerm;
    nonterminal List<TypedTerm> typedTerms;
    
    program     ::= typeDefList:typDefs signatures:signList typedTerms:termList 
                    {: RESULT = new Program(typDefs, signList, termList); :}
                 ;
    typeDefList ::= typeDefList:l typeDef:t {: l.add(t); RESULT = l; :}
                 |  typeDef:t               
                    {: List<TypeDef> l = new LinkedList<TypeDef>(); 
                       l.add(t); RESULT = l; 
                    :}
                 ;
    typeDef     ::= TYPE FUNCTION:f ASSIGN typeSum:s SEMICOLON
                    {: RESULT = new SimpleTypeDef(f, s); :}
                 |  TYPE FUNCTION:f LEFT_PAR varList:a RIGHT_PAR ASSIGN 
                         typeSum:s SEMICOLON 
                    {: RESULT = new ParamTypeDef(f, a, s); :}
                 ;
    type        ::= FUNCTION:f LEFT_PAR typeList:t RIGHT_PAR 
                                {: RESULT = new CompositeType(f, t); :}
                 |  FUNCTION:f  {: RESULT = new CompositeType(f);    :}
                 |  PARAMETER:v {: RESULT = new Parameter(v);        :}
                 ;    
    typeList    ::= typeList:l COMMA type:t {: l.add(t); RESULT = l; :}
                 |  type:t {: List<Type> l = new LinkedList<Type>(); 
                              l.add(t); 
                              RESULT = l; 
                           :}   
                 ;   
\end{Verbatim}
\vspace*{-0.3cm}
\caption{\textsl{JavaCup}-Spezifikation der \textsc{Ttl}-Grammatik, 1. Teil}
\label{fig:typeChecker.cup-1}
\end{figure}

\begin{figure}[!ht]
\centering
\begin{Verbatim}[ frame         = lines, 
                  framesep      = 0.3cm, 
                  labelposition = bottomline,
                  numbers       = left,
                  numbersep     = -0.2cm,
                  xleftmargin   = 0.8cm,
                  xrightmargin  = 0.8cm,
                ]
    typeSum     ::= typeSum:l PLUS type:t {: l.add(t); RESULT = l; :}   
                 |  type:t 
                    {: 
                       List<Type> l = new LinkedList<Type>(); 
                       l.add(t);  
                       RESULT = l; 
                    :}   
                 ;
    signature   ::= SIGNATURE FUNCTION:f COLON argTypes:a 
                              ARROW type:t SEMICOLON
                    {: RESULT = new Signature(f, a, t); :}
                 |  SIGNATURE FUNCTION:f COLON type:t SEMICOLON
                    {: List a = new LinkedList<Type>(); 
                       RESULT = new Signature(f, a, t); 
                    :}
                 ;
    signatures  ::= signatures:l signature:s {: l.add(s); RESULT = l; :}
                 |  signature:s 
                    {: List<Signature> l = new LinkedList(); 
                       l.add(s); 
                       RESULT = l; 
                    :}
                 ;    
    argTypes    ::= argTypes:l TIMES type:t {: l.add(t); RESULT = l; :}
                 |  type:t 
                    {: List<Type> l = new LinkedList(); 
                       l.add(t); RESULT = l; 
                    :}
                 ;
    varList     ::= varList:l COMMA PARAMETER:v 
                                {: l.add(new Parameter(v)); RESULT = l; :}
                 |  PARAMETER:v {: List<Parameter> l = new LinkedList(); 
                                   l.add(new Parameter(v)); 
                                   RESULT = l; 
                                :}
                 ;    
    term        ::= FUNCTION:f LEFT_PAR termList:l RIGHT_PAR 
                               {: RESULT = new Term(f, l); :}
                 |  FUNCTION:f {: List<Term> l = new LinkedList<Term>(); 
                                  RESULT = new Term(f, l); 
                               :}
                 ;
    termList    ::= termList:l COMMA term:t 
                           {: l.add(t); RESULT = l;            :}
                 |  term:t {: List<Term> l = new LinkedList(); 
                              l.add(t); RESULT = l; 
                           :}
                 ;
\end{Verbatim}
\vspace*{-0.3cm}
\caption{\textsl{JavaCup}-Spezifikation der \textsc{Ttl}-Grammatik, 2. Teil}
\label{fig:typeChecker.cup-2}
\end{figure}

\begin{figure}[!ht]
\centering
\begin{Verbatim}[ frame         = lines, 
                  framesep      = 0.3cm, 
                  labelposition = bottomline,
                  numbers       = left,
                  numbersep     = -0.2cm,
                  xleftmargin   = 0.0cm,
                  xrightmargin  = 0.0cm,
                ]
    typedTerm   ::= term:t COLON type:s SEMICOLON 
                    {: RESULT = new TypedTerm(t, s); :}
                 ;
    
    typedTerms  ::= typedTerms:l typedTerm:t 
                    {: l.add(t); RESULT = l; :}
                 |  typedTerm:t 
                    {: List<TypedTerm> l = new LinkedList<TypedTerm>(); 
                       l.add(t); 
                       RESULT = l; 
                    :}
                 ;
\end{Verbatim}
\vspace*{-0.3cm}
\caption{\textsl{JavaCup}-Spezifikation der \textsc{Ttl}-Grammatik, 3. Teil}
\label{fig:typeChecker.cup-3}
\end{figure}

\begin{figure}[!ht]
\centering
\begin{Verbatim}[ frame         = lines, 
                  framesep      = 0.3cm, 
                  labelposition = bottomline,
                  numbers       = left,
                  numbersep     = -0.2cm,
                  xleftmargin   = 0.0cm,
                  xrightmargin  = 0.0cm,
                ]
    Program = Program(List<TypeDef>   typeDefs, 
                      List<Signature> signatures, 
                      List<TypedTerm> typedTerms);
    
    TypeDef = SimpleTypeDef(String name, List<Type> typeSum)
            + ParamTypeDef(String       name, 
                           List<String> parameters, 
                           List<Type>   typeSum);
    
    Type = Parameter(String name) 
         + CompositeType(String name, List<Type> argTypes);
    
    substitute: Type * Parameter * Type -> Type;
    
    Signature = Signature(String name, List<Type> argList, Type result);
    
    Term = Term(String function, List<Term> termList);
    
    typeEqs: Term * Type * Map<String, Signature> -> List<Equation>;
    
    TypedTerm = TypedTerm(Term term, Type type);
    
    Substitution = Substitution(List<Parameter> variables, List<Type> types);
    
    Equation = Equation(Type lhs, Type rhs);
    
    substitute: Equation * Parameter * Term -> Equation;
    
    MartelliMontanari = 
        MartelliMontanari(List<Equation> equations, Substitution theta);
\end{Verbatim}
\vspace*{-0.3cm}
\caption{Definition der ben\"otigten \textsl{Java}-Klassen mit Hilfe von \textsc{EP}.}
\label{fig:typeExpr.ep}
\end{figure}


\begin{figure}[!ht]
\centering
\begin{Verbatim}[ frame         = lines, 
                  framesep      = 0.3cm, 
                  labelposition = bottomline,
                  numbers       = left,
                  numbersep     = -0.2cm,
                  xleftmargin   = 0.0cm,
                  xrightmargin  = 0.0cm,
                ]
    import java_cup.runtime.*;
          
    %%
       
    %char
    %line
    %column
    %cup
       
    %{   
        private Symbol symbol(int type) {
            return new Symbol(type, yychar, yychar + yylength());
        }
        
        private Symbol symbol(int type, Object value) {
            return new Symbol(type, yychar, yychar + yylength(), value);
        }
    %}
       
    %%
       
    "+"                { return symbol( sym.PLUS      ); } 
    "*"                { return symbol( sym.TIMES     ); } 
    "("                { return symbol( sym.LEFT_PAR  ); } 
    ")"                { return symbol( sym.RIGHT_PAR ); }
    ","                { return symbol( sym.COMMA     ); }
    ":"                { return symbol( sym.COLON     ); }
    ";"                { return symbol( sym.SEMICOLON ); }
    ":="               { return symbol( sym.ASSIGN    ); }
    "->"               { return symbol( sym.ARROW     ); }
    "type"             { return symbol( sym.TYPE      ); }
    "signature"        { return symbol( sym.SIGNATURE ); }
    
    [a-z][a-zA-Z_0-9]* { return symbol(sym.FUNCTION,  yytext()); }
    [A-Z][a-zA-Z_0-0]* { return symbol(sym.PARAMETER, yytext()); }
    
    [ \t\n]            { /* skip white space */ }   
    "//" [^\n]*        { /* skip comments    */ }   
    
    [^] { throw new Error("Illegal character '" + yytext() + 
                          "' at line " + yyline + ", column " + yycolumn); }
\end{Verbatim}
\vspace*{-0.3cm}
\caption{\textsl{JFlex}-Spezifikation des Scanners}
\label{fig:typeChecker.jflex}
\end{figure}


\begin{figure}[!ht]
\centering
\begin{Verbatim}[ frame         = lines, 
                  framesep      = 0.3cm, 
                  labelposition = bottomline,
                  numbers       = left,
                  numbersep     = -0.2cm,
                  xleftmargin   = 0.0cm,
                  xrightmargin  = 0.0cm,
                ]
    public Substitution solve() {
        while (mEquations.size() != 0) {
            Equation eq = mEquations.remove(0);
            Type lhs = eq.getLhs();
            Type rhs = eq.getRhs();
            if (lhs instanceof Parameter) {
                Parameter var = (Parameter) lhs;
                List<Equation> newEquations = new LinkedList<Equation>();
                for (Equation equation: mEquations) {
                    Equation neq = equation.substitute(var, rhs);
                    newEquations.add(neq);
                }
                mEquations = newEquations;
                mTheta     = mTheta.substitute(var, rhs);
            } else if (lhs instanceof CompositeType) {
                CompositeType compLhs = (CompositeType) lhs;
                CompositeType compRhs = (CompositeType) rhs;
                if (!compLhs.getName().equals(compRhs.getName())) {
                    // different type constructors, no solution
                    System.err.println("Error: different type constructors\n");
                    return null;
                }
                List<Type> lhsArgs = compLhs.getArgTypes();
                List<Type> rhsArgs = compRhs.getArgTypes();
                for (int i = 0; i < lhsArgs.size(); ++i) {
                    Type sigmaLhs = lhsArgs.get(i);
                    Type sigmaRhs = rhsArgs.get(i);
                    mEquations.add(0, new Equation(sigmaLhs, sigmaRhs));
                }
            } 
        }
        return mTheta;
    }
\end{Verbatim}
\vspace*{-0.3cm}
\caption{Die Methode $\textsl{solve}()$ aus der Klasse \texttt{MartelliMontanari}.}
\label{fig:MartelliMontanari:solve.java}
\end{figure}

Die Klasse \texttt{MartelliMontanari} enth\"alt die in Abbildung \ref{fig:MartelliMontanari:solve.java}
gezeigte Methode $\textsl{solve}()$, mit der sich ein syntaktisches Gleichungs-System l\"osen l\"asst.
Wir diskutieren diese Methode jetzt im Detail.
\begin{enumerate}
\item Am Anfang w\"ahlen wir in  Zeilen 3 willk\"urlich die erste syntaktische Gleichung aus der Menge
      \texttt{mEquations} der zu l\"osenden syntaktischen Gleichungen aus.  
      
      Das weitere Vorgehen richtet sich dann nach der Art der Gleichung.
\item In den Zeilen 6 -- 14 behandeln wir den Fall, dass auf der linken Seite der syntaktischen Gleichung
      ein Typ-Parameter steht.  In diesem Fall formen wir das syntaktische Gleichungs-System nach
      der folgenden Regel um:
      \\[0.2cm]
      \hspace*{1.3cm}
      $\Big\langle E \cup \big\{ X \doteq \tau \big\}, \vartheta \Big\rangle \quad\leadsto\quad 
       \Big\langle E[X \mapsto \tau], \vartheta\big[ X \mapsto \tau \big] \Big\rangle$. 
      \\[0.2cm]
      Der Typ-Parameter, der in der obigen Formel mit \texttt{X} bezeichnet wird, tr\"agt im Programm 
      den Namen \texttt{var} und der Typ $\tau$ wird im Programm mit \texttt{rhs} bezeichnet.
\item In Zeile 15 betrachten wir den Fall, dass auf der linken Seite der syntaktischen Gleichung
      ein zusammengesetzter Typ steht.  Wenn die Typ-Gleichung l\"osbar sein soll, muss dann auch auf der
      rechten Seite ein zusammengesetzter Typ stehen.  Dies wird in  Zeile 16 \"uberpr\"uft.

      Der restliche Code besch\"aftigt sich nun mit dem Fall, dass auf beiden Seiten der syntaktischen
      Gleichung ein zusammengesetzter Typ steht.
\item Die Zeilen 22 -- 25 behandeln den Fall, dass die Typ-Konstruktoren auf beiden Seiten verschieden sind,
      es wird also der Fall
      \\[0.2cm]
      \hspace*{1.3cm}
      $\Big\langle E \cup \big\{ f(\sigma_1,\cdots,\sigma_m) \doteq g(\tau_1,\cdots,\tau_n) \big\}, 
       \vartheta \Big\rangle 
       \;\leadsto\; \Omega$. 
      \\[0.2cm]
      behandelt.  Die Methode liefert in diesem Fall statt einer Substitution den Wert \texttt{null}
      zur\"uck.
\item Die Zeilen 27 -- 32 behandeln schlie{\ss}lich den Fall, in dem wir das Gleichungs-System nach der Regel
      \\[0.2cm]
      \hspace*{1.3cm}
      $\Big\langle E \cup \big\{ f(\sigma_1,\cdots,\sigma_n) \doteq f(\tau_1,\cdots,\tau_n) \big\}, 
       \vartheta \Big\rangle 
       \;\leadsto\; 
       \Big\langle E \cup \big\{ \sigma_1 \doteq \tau_1, \cdots, \sigma_n \doteq \tau_n\}, 
       \vartheta \Big\rangle$.
      \\[0.2cm]
      umformen.
\end{enumerate}

\begin{figure}[!ht]
\centering
\begin{Verbatim}[ frame         = lines, 
                  framesep      = 0.3cm, 
                  labelposition = bottomline,
                  numbers       = left,
                  numbersep     = -0.2cm,
                  xleftmargin   = 0.0cm,
                  xrightmargin  = 0.0cm,
                ]
    public List<Equation> typeEqs(Type tau, Map<String, Signature> map) {
        Signature sign = map.get(mFunction);
        if (sign == null) {
            System.err.println("The function " + mFunction + 
                               " has not been declared!");
            throw new Error("Undeclared function in " + myString());
        }
        List<Type> argTypes = sign.getArgList();
        if (argTypes.size() != mTermList.size()) {
            System.err.println("Wrong number of parameters for function " + 
                               mFunction);
            System.err.println("expected: " + argTypes.size());
            System.err.println("found:    " + mTermList.size());
            throw new Error("Wrong number of parameters in " + myString());
        }
        List<Equation> result = new LinkedList<Equation>();
        Equation eq = new Equation(sign.getResult(), tau);
        result.add(eq);
        for (int i = 0; i < mTermList.size(); ++i) {
            Term argI   = mTermList.get(i);
            Type sigmaI = argTypes.get(i);
            result.addAll( argI.typeEqs(sigmaI, map) );
        }
        return result;
    }
\end{Verbatim}
\vspace*{-0.3cm}
\caption{Berechnung der Typ-Gleichungen}
\label{fig:Term:typeEqs.java}
\end{figure}

Als letztes diskutieren wir die Berechnung der Typ-Gleichungen, die in der Methode $\textsl{typeEqs}()$ der
Klasse \texttt{Term} implementiert ist.  
Ein Term $f(s_1, \cdots, s_n)$ hat genau dann den Typ $\tau$, falls die Funktion $f$
eine Signatur
\\[0.2cm]
\hspace*{1.3cm}
$f: \sigma_1 \times \cdots \times \sigma_n \rightarrow \varrho$
\\[0.2cm]
hat und es dar\"uber hinaus eine Parameter-Substitution $\vartheta$ gibt, so dass
$\tau = \varrho\vartheta$ gilt und weiterhin die Terme $s_i$ vom Typ $\sigma_i\vartheta$ sind:
\\[0.2cm]
\hspace*{1.3cm}
$f(s_1,\cdots,s_n):\tau \;\Leftrightarrow\; 
 \exists \vartheta\in\textsc{Subst}: \varrho\vartheta = \tau \wedge 
 s_1:\sigma_1\vartheta \wedge \cdots \wedge s_n:\sigma_n\vartheta$.
\begin{enumerate}
\item Um die obige Definition von $t:\tau$ umzusetzen, suchen wir f\"ur einen Term der Form
      $f(s_1,\cdots,s_n)$ zun\"achst in Zeile 2 nach der Signatur des Funktions-Zeichens $f$.
      Diese Signaturen sind in der Symboltabelle \texttt{map} hinterlegt.
      Findet sich f\"ur das Funktions-Zeichen keine Signatur, so liegt offenbar ein Fehler vor, 
      der in Zeile 4 ausgegeben wird.
\item Ansonsten speichern wir in Zeile 8 die Argument-Typen $\sigma_1, \cdots, \sigma_n$ in der Variablen
      \texttt{argTypes}.  Die Signatur von $f$ muss nat\"urlich genau so viele Argument-Typen
      haben, wie das Funktions-Zeichen $f$ Argumente hat.  Dies wird in Zeile  9 \"uberpr\"uft.
\item Falls bis hierhin keine Probleme aufgetreten sind, erzeugen wir nun die Typ-Gleichungen nach der Formel
      \\[0.2cm]
      \hspace*{1.3cm}
      $\bigl(f: \sigma_1 \times \cdots \times \sigma_n \rightarrow \varrho\bigr) \;\Longrightarrow\;
      \textsl{typeEqs}\bigl( f(t_1,\cdots,t_n), \tau \bigr) := 
      \{ \varrho = \tau \} \cup \bigcup\limits_{i=1}^n \textsl{typeEqs}(t_i, \sigma_i)$.
      \\[0.2cm]
      Dazu erstellen wir in Zeile 17 zu\"achst die Typ-Gleichungen
      \\[0.2cm]      
      \hspace*{1.3cm}
      $\varrho \doteq \tau$
      \\[0.2cm]
      und berechnen dann in der Schleife in den Zeilen 19 -- 22 rekursiv die Typ-Gleichungen, die sich aus
      den Forderungen 
      \\[0.2cm]
      \hspace*{1.3cm}
      $s_i:\sigma_i \quad \mbox{f\"ur} \; i=1,\cdots,n$
      \\[0.2cm]
      ergeben.  Als Ergebnis erhalten wir eine Menge von Typ-Gleichungen, die genau
      dann l\"osbar sind, wenn der Term $f(s_1,\cdots,s_n)$ den Typ $\tau$ hat.
\end{enumerate}

\begin{figure}[!ht]
\centering
\begin{Verbatim}[ frame         = lines, 
                  framesep      = 0.3cm, 
                  firstnumber   = 1,
                  labelposition = bottomline,
                  numbers       = left,
                  numbersep     = -0.2cm,
                  xleftmargin   = -0.3cm,
                  xrightmargin  = 0.0cm,
                ]
    public Program(List<TypeDef>   typeDefs, 
                   List<Signature> signatures, 
                   List<TypedTerm> typedTerms) 
    {
        mTypeDefs   = typeDefs;
        mSignatures = signatures;
        mTypedTerms = typedTerms;
        mSignatureMap = new TreeMap<String, Signature>();
        for (Signature s: mSignatures) {
            mSignatureMap.put(s.getName(), s);
        }
        for (TypeDef td: mTypeDefs) {
            if (td instanceof SimpleTypeDef) {
                SimpleTypeDef std = (SimpleTypeDef) td;
                Type rho = new CompositeType(std.getName(), new LinkedList<Type>());
                for (Type tau: std.getTypeSum()) {
                    CompositeType c = (CompositeType) tau;
                    String name = c.getName();
                    Signature s = new Signature(name, c.getArgTypes(), rho);
                    mSignatureMap.put(name, s);
                }
            } else {
                ParamTypeDef ctd = (ParamTypeDef) td;
                List<Type> paramList = new LinkedList<Type>();
                for (Parameter v : ctd.getParameters()) {
                    paramList.add(v);
                }
                Type rho = new CompositeType(ctd.getName(), paramList);
                for (Type tau: ctd.getTypeSum()) {
                    CompositeType c = (CompositeType) tau;
                    String name = c.getName();
                    Signature s = new Signature(name, c.getArgTypes(), rho);
                    mSignatureMap.put(name, s);
                }
            }
        }
    }
\end{Verbatim}
\vspace*{-0.3cm}
\caption{Der Konstruktor der Klasse \texttt{Program}.}
\label{fig:Program.Program.java}
\end{figure}
\vspace*{0.3cm}


\noindent
Als letztes diskutieren wir die Klasse \texttt{Program}.  Abbildung \ref{fig:Program.Program.java}
zeigt den Konstruktor dieser Klasse.  Dieser Konstruktor bekommt als Argumente 
\begin{itemize}
\item eine Liste von Typ-Definitionen \texttt{typeDefs},
\item eine Liste von Signaturen \texttt{signatures} und
\item eine Liste von getypten Termen \texttt{typedTerms}, deren Typ-Korrektheit
      \"uberpr\"uft werden soll.
\end{itemize}
Die wesentliche Aufgabe des Konstruktors besteht darin,  die Member-Variable
\texttt{mSignatureMap} zu initialisieren.  Hierbei handelt es sich um eine Symboltabelle,
in der zu jedem Funktions-Namen die zugeh\"orige Signatur abgelegt ist.
Dazu werden zun\"achst in der Schleife in Zeile 9 -- 11 alle
Signaturen aus der als Argument \"ubergebenen Liste \texttt{signatures} in dieser Liste eingetragen.
Dies alleine ist allerdings nicht ausreichend, denn durch die Typ-Definitionen werden implizit noch
weitere Funktions-Zeichen deklariert.  Beispielsweise werden durch die Typ-Definition
\\[0.2cm]
\hspace*{1.3cm}
\texttt{type \ list(X) := nil + cons(X, list(X));}
\\[0.2cm]
implizit die Funktionszeichen \textsl{nil} und \textsl{cons} definiert.  Diese Funktionszeichen
haben die folgenden Signaturen:
\begin{enumerate}
\item \texttt{nil: List(T)},
\item \texttt{cons: T * List(T) -> List(T)}.
\end{enumerate}
Die Aufgabe des Konstruktors der Klasse \texttt{Program} besteht also darin, aus den in dem Argument
\texttt{typeDefs} enthaltenen Typ-Definitionen die Signaturen der implizit deklarierten
Funktionszeichen zu generieren.  Dazu iteriert die Schleife in Zeile 12 zun\"achst \"uber alle
Typ-Definitionen.   Es gibt zwei Arten von Typ-Definitionen:  Typ-Definitionen, bei denen der
erzeugte Typ noch von einem oder mehreren Parametern abh\"angt, wie dies z.B.~bei der Typ-Definition
von \texttt{List(T)} der Fall ist, oder solche Typ-Definitionen, bei denen der erzeugte Typ keine
Parameter hat.  Letztere Typ-Definitionen werden durch die Klasse \texttt{SimpleTypeDef}
dargestellt, erstere durch die Klasse \texttt{CompositeType}.  Diese beiden F\"alle werden durch die
\texttt{if}-Abfrage in Zeile 13 unterschieden.
\begin{enumerate}
\item Falls die untersuchte Typ-Definition \texttt{td} eine einfache Typ-Definition der Form
      \[ \tau := f_1(\sigma_1^{(1)}, \cdots, \sigma_{n_1}^{(1)}) + \cdots
                 f_k(\sigma_1^{(k)}, \cdots, \sigma_{n_k}^{(k)})
      \]
      ist, so wird der Typ $\tau$ von den Funktionen $f_1$, $\cdots$, $f_k$ erzeugt.  Die $i$-te
      Funktion $f_i$ hat die Signatur
      \[ f_i: \sigma_1^{(i)} \times \cdots \times \sigma_{n_i}^{(i)} \rightarrow \tau.  \]
      Diese Signatur wird in Zeile 19 aus dem Ergebnis-Typ $\tau$ und den Argument-Typen 
      \texttt{c.getArgTypes()} der Funktion $f_i$ zusammengebaut.  Die so konstruierte Signatur
      wird dann in der Symboltabelle \texttt{mSignatureMap} abgelegt.
\item Falls es sich bei der untersuchten Typ-Definition um die Definition eines parametrisierten
      Typs handelt, muss darauf geachtet werden, dass der Ergebnis-Typ der Signatur der Funktionen
      $f_i$ auch von diesen Parametern abh\"angt.  Zu diesem Zweck wird in Zeile 24 zun\"achst eine neue
      Parameter-Liste \texttt{paramList} angelegt, in die dann in der Schleife in Zeile 25 --- 27
      die Parameter des Typs hineinkopiert werden.  Der Rest dieses Falls ist analog zum ersten Fall.
\end{enumerate}
Neben dem Konstruktor enth\"alt die Klasse \texttt{Program} noch die Methode \texttt{typeCheck()},
welche die Typ\"uberpr\"ufung ausf\"uhrt.  Die Implementierung dieser Methode ist in der Abbildung
\ref{fig:Program.typeCheck} gezeigt.  Diese Methode iteriert in der Schleife, die in Zeile 2
beginnt, \"uber alle getypten Terme $t: \tau$, die dem Konstruktor der Klasse \texttt{Program}
\"ubergeben wurden.   Dazu wird in Zeile 6 zu jedem Term $t$ und Typ $\tau$ die Menge der
Typ-Gleichungen $\textsl{typeEqs}(t,\tau)$ berechnet.  Dazu wird der Methode \texttt{typeEqs()}, die
die Funktion $\textsl{typeEqs}()$ implementiert, noch die vom Konstruktor berechnete Symboltabelle
\"ubergeben.  In dieser Symboltabelle sind die Signaturen der verschiedenen Funktions-Zeichen
abgespeichert.   Die berechneten Typ-Gleichungen werden anschlie{\ss}end mit Hilfe der Methode
\texttt{solve()}  der Klasse \texttt{MartelliMontanari} gel\"ost.  Falls die Typ-Gleichungen
gel\"ost werden konnten, hat der Term $t$ tats\"achlich den Typ $\tau$.  Andernfalls wird eine
Fehlermeldung ausgegeben.


\begin{figure}[!ht]
\centering
\begin{Verbatim}[ frame         = lines, 
                  framesep      = 0.3cm, 
                  firstnumber   = 1,
                  labelposition = bottomline,
                  numbers       = left,
                  numbersep     = -0.2cm,
                  xleftmargin   = 0.8cm,
                  xrightmargin  = 0.8cm,
                ]
    public void typeCheck() {
        for (TypedTerm tt: mTypedTerms) {
            System.out.println("\nChecking " + tt.myString());
            Term t   = tt.getTerm();
            Type tau = tt.getType();
            List<Equation> typeEquations = t.typeEqs(tau, mSignatureMap);
            MartelliMontanari mm = new MartelliMontanari(typeEquations);
            Substitution theta = mm.solve();
            if (theta != null) {
                System.out.println(tt.myString() + " has been verified!");
                System.out.println(theta);
            } else {
                System.out.println(tt.myString() + " type ERROR!!!");
            }
        }
    }
\end{Verbatim}
\vspace*{-0.3cm}
\caption{Die Methode \textsl{typeCheck()} in der Klasse \texttt{Program}.}
\label{fig:Program.typeCheck}
\end{figure}


\section{Inklusions-Polymorphismus}
Interpretieren wir Typen als Mengen von Werten, so stellen wir fest, dass es bei Typen eine
Inklusions-Hierarchy gibt.  In der Sprache \textsl{Java} hat jeder Wert vom Typ \texttt{String} auch
gleichzeitig den Typ \texttt{Object}, es gilt also
\\[0.2cm]
\hspace*{1.3cm}
$\texttt{String} \subseteq \texttt{Object}$.
\\[0.2cm]
Allgemein gilt: Ist $e$ ein Ausdruck, der den Typ $A$ hat und ist $A$ weiter eine Klasse, die von der
Klasse $B$ abgeleitet ist, so kann der Ausdruck $e$ \"uberall dort verwendet werden, wo ein Ausdruck vom Typ
$B$ ben\"otigt wird.  Damit kann der Ausdruck $e$ sowohl als ein $A$, als auch als ein $B$ verwendet werden:
$e$ kann also \emph{polymorph} verwendet werden.  Im Gegensatz zu dem \emph{parametrischen Polymorphismus},
den wir bisher betrachtet haben, sprechen wir hier von \emph{Inklusions-Polymorphismus}.  Die Behandlung von
Inklusions-Polymorphismus ist besonders dann interessant, wenn dieser zusammen mit parametrischen
Polymorphismus auftritt.  Wir zeigen exemplarisch ein Beispiel in Abbildung \ref{fig:TypeSurprise.java}.
\begin{enumerate}
\item In Zeile 4 wird hier zun\"achst ein Feld \texttt{x} von Strings angelegt.
\item In Zeile 5 definieren wir ein Feld \texttt{y} von Objekten und initialisieren dieses Feld mit
      dem bereits erzeugten Feld \texttt{x}.  Da jeder String zugleich auch ein Objekt ist,
      ist dies m\"oglich.
\item In Zeile 6 weisen wir dem zweiten Element des Feldes \texttt{y} die Zahl 2 zu.
      Da \texttt{y} ein Feld von Objekten ist und eine Zahl ebenfalls als Objekt angesehen werden kann,
      sollte auch dies kein Problem sein.
\end{enumerate}
In der Tat l\"asst sich das Programm fehlerfrei \"ubersetzen.  Bei der Ausf\"uhrung wird aber eine Ausnahme
ausgel\"ost.  Der Grund ist, dass mit der Zuweisung in Zeile 5 die Variable \texttt{y} eine Referenz auf das
Feld von Strings ist, das in Zeile 4 angelegt worden ist.  Wenn wir nun versuchen, in dieses Feld eine Zahl
einzuf\"ugen, dann gibt es eine \texttt{ArrayStoreException}.  Dieses Beispiel zeigt, dass die Sprache
\textsl{Java} nicht statisch auf Typ-Sicherheit gepr\"uft werden kann.  Es zeigt auch, dass die Behandlung der
Kombiation von Inklusions-Polymorphismus und parametrischen Polymorphismus deutlich komplexer ist, als die
Behandlung von parametrischem Polymorphismus alleine.

\begin{figure}[!ht]
\centering
\begin{Verbatim}[ frame         = lines, 
                  framesep      = 0.3cm, 
                  firstnumber   = 1,
                  labelposition = bottomline,
                  numbers       = left,
                  numbersep     = -0.2cm,
                  xleftmargin   = 0.8cm,
                  xrightmargin  = 0.8cm,
                ]
    public class TypeSurprise {
    
        public static void main(String[] args) {
            String[] x = { "a", "b", "c" };
            Object[] y = x;
            y[1] = new Integer(2);
            System.out.println(x[1]);
        }
    }
\end{Verbatim}
\vspace*{-0.3cm}
\caption{Ein Typ-Problem in \textsl{Java}.}
\label{fig:TypeSurprise.java}
\end{figure}
In dem Papier ``\emph{Type Inference for Java 5}''  von Mazurak und Zdancewic wird das Typ-System
der Sprache \textsl{Java} n\"aher untersucht.  Die Autoren kommen zu dem Schluss, dass die Frage, ob
ein gegebenes \textsl{Java}-Programm korrekt getypt ist, unentscheidbar ist.  Es ist daher auch
nicht verwunderlich,  dass sich der \textsl{Java}-Compiler bei manchen Programmen mit einem
Stack-Overflow verabschiedet, der seine Ursache im Typ-Checker des Compilers hat.
Des Weiteren ist der folgende Satz aus der ``\textsl{Java Language Specifiaction}'', also
der verbindlichen Definition der Sprache \textsl{Java}, aufschlussreich:
\vspace*{0.3cm}

\begin{minipage}[c]{0.9\linewidth}
{\em
The type inference algorithm should be viewed as a heuristic, designed to perform well in
practice. If it fails to infer the desired result, explicit type parameters may be used
instead.   
}  
\end{minipage}

\vspace*{0.3cm}
\noindent
Dies zeigt, dass die Kombination von parametrischem Polymorphismus und Inklusions-Polymor\-phismus zu
sehr komplexen Problemen f\"uhren kann, die wir aber aus Zeitgr\"unden nicht tiefer betrachten k\"onnen.



%%% Local Variables: 
%%% mode: latex
%%% TeX-master: "formale-sprachen"
%%% End: 
