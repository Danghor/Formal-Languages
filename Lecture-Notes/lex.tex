\chapter{Die Werkzeuge \textsl{Flex}\/ und \textsl{JFlex}}
Ein \emph{Scanner} ist ein Werkzeug, das einen gegebenen Text in Gruppen einzelner
\emph{Token} aufspaltet.  Beispielsweise spaltet der Scanner, der f\"ur einen
\texttt{C}-Compiler eingesetzt wird, den Programmtext in die folgenden Token auf:
\begin{enumerate}
\item Schl\"usselw\"orter wie ``\texttt{if}'', ``\texttt{while}'', etc.
\item Operator-Symbole wie ``\texttt{+}'', ``\texttt{+=}'', ``\texttt{<}'',
      ``\texttt{<=}'', etc.
\item Konstanten, wobei es in der Sprache \texttt{C} drei Arten von Konstanten gibt:
      \begin{enumerate}
      \item Zahlen, beispielsweise ``\texttt{123}'' oder ``\texttt{1.23e2}'',
      \item Strings, beispielsweise \texttt{\symbol{34}hallo\symbol{34}},
      \item einzelne Buchstaben, beispielsweise \texttt{\symbol{39}a\symbol{39}}.
      \end{enumerate}
\item Namen, die als Bezeichner f\"ur Variablen, Funktionen, oder Typ-Definitionen
      fungieren.
\item Kommentare
\item Sogenannte \emph{White-Space-Zeichen}.  Hierzu geh\"oren Leerzeichen, horizontale und
      vertikale Tabulatoren, Zeilenumbr\"uche und Seitenvorsch\"ube.
\end{enumerate}
\textsl{Flex}\/ ist ein sogenannter Scanner-Generator, also ein Werkzeug, das aus einer
Spezifikation verschiedener Token automatisch einen Scanner generiert.  Die einzelnen
Token werden dabei durch regul\"are Ausdr\"ucke definiert.  

In n\"achsten Abschnitt besprechen wir die Struktur einer \textsl{Flex}-Eingabe-Datei und
zeigen wie \textsl{Flex}\/ aufgerufen wird.  Anschlie{\ss}end zeigen wir, wie regul\"are Ausdr\"ucke
in der Eingabe-Sprache von \textsl{Flex}\/ spezifiziert werden k\"onnen.  Das Kapitel wird
durch ein Beispiel abgerundet, bei dem wir mit Hilfe von \textsl{Flex}\/ ein Programm
erzeugen, mit dessen Hilfe die Ergebnisse einer Klausur ausgewertet werden k\"onnen.

Die von \textsl{Flex}\/ erzeugten Scanner sind \texttt{C}-Programme.  F\"ur das
\textsl{Java}-Umfeld gibt es ein \"Aquivalent unter dem Namen \textsl{JFlex}.

\section{Das Werkzeug \textsl{Flex}}
Wir beginnen mit einem einfachen Beispiel und diskutieren anschlie{\ss}end einige Fallstricke
von \textsl{Flex}, \"uber die Anf\"anger h\"aufig stolpern.
\subsection{Ein einfaches Beispiel}
Eine Eingabe-Datei f\"ur \textsl{Flex}\/ besteht aus vier Abschnitten, die aufeinander folgen.
\begin{enumerate}
\item Der \emph{Deklarations-Abschnitt} enth\"alt die Deklarationen von Variablen und Hilfsfunktionen.
      Zus\"atzlich kann dieser Teil auch Kommentare und \texttt{include}-Anweisungen
      enthalten.  Der Deklarations-Abschnitt wird durch den String ``\texttt{\symbol{37}\{}''
      eingeleitet und durch den String ``\texttt{\symbol{37}\}}'' ausgeleitet.
      In dem in Abbildung \ref{fig:change.l} gezeigten Beispiel erstreckt sich der
      Deklarations-Abschnitt von Zeile 1 bis Zeile 7.

      Die von \textsl{Flex}\/ erzeugten Scanner sind \texttt{C}-Programme.  Bei der Sprache
      \texttt{C} ist es erforderlich, dass Funktionen und Variablen vor ihrer Benutzung
      deklariert werden.  Daher werden Variablen, die sp\"ater im Regel-Abschnitt benutzt
      werden sollen, im Deklarations-Abschnitt deklariert.

      Der Deklarations-Abschnitt ist optional: Falls im Regel-Abschnitt keine Variablen
      verwendet werden, dann kann der Deklarations-Abschnitt auch entfallen.
\item Der \emph{Definitions-Abschnitt} enth\"alt die Definitionen von sogenannten \emph{regul\"aren Definitionen}.
      Hierbei handelt es sich um Abk\"urzungen f\"ur komplexere regul\"are Ausdr\"ucke.
      Der Definitions-Abschnitt beginnt hinter dem Deklarationsteil und erstreckt sich bis zum
      ersten Auftreten des Strings ``\texttt{\symbol{37}\symbol{37}}''.
      Der Definitions-Abschnitt kann leer sein.  In dem in Abbildung \ref{fig:change.l}
      gezeigten Beispiel ist dies der Fall, denn dort folgt der String
      ``\texttt{\symbol{37}\symbol{37}}'' unmittelbar auf den String ``\texttt{\symbol{37}\}}''.
\item Der \emph{Regel-Abschnitt} folgt auf den Definitions-Abschnitt und besteht aus Regeln
      der Form
      \\[0.2cm]
      \hspace*{1.3cm} \textsl{regexp} \texttt{\{}\textsl{cmds}\texttt{\}}
      \\[0.2cm]
      Hier steht \textsl{regexp}\/ f\"ur einen regul\"aren Ausdruck. Dieser muss am Zeilenanfang
      stehen.  Jedesmal, wenn der generierte Scanner in seiner Eingabe einen String
      erkennt, der diesem regul\"aren Ausdruck entspricht, werden die in \textsl{cmds}\/
      angegebenen \emph{Aktionen} ausgef\"uhrt.  Genauer handelt es sich bei \texttt{cmds} um eine
      oder mehrere Anweisungen der Sprache \texttt{C}.

      Der Regel-Abschnitt wird durch ein Auftreten des Strings
      ``\texttt{\symbol{37}\symbol{37}}''
      beendet.  In dem in Abbildung \ref{fig:change.l}
      gezeigten Beispiel erstreckt sich der Regel-Abschnitt von Zeile 9 bis Zeile 11.
\item Der \emph{Programm-Abschnitt} enth\"alt die Definition der Funktion $\texttt{main}()$,
      sowie eventuell die Definition weiterer Hilfsfunktionen.
      In Abbildung \ref{fig:change.l} erstreckt sich der Programm-Abschnitt von Zeile 13
      bis Zeile 18.
\end{enumerate}

\begin{figure}[!h]
\centering
\begin{Verbatim}[ frame         = lines, 
                  framesep      = 0.3cm, 
                  labelposition = bottomline,
                  numbers       = left,
                  numbersep     = -0.2cm,
                  xleftmargin   = 0.8cm,
                  xrightmargin  = 0.8cm,
                ]
    %{    
    /* This trivial lexer replaces occurrences of the 
       string "Stephan" with "Stefan". 
    */
    #include <stdio.h>
    int numberChanges = 0; // number of occurrences changed     
    %}
    %%     

    Stephan  { printf("Stefan"); ++numberChanges; }
    .        { printf("%s", yytext);              }
    %%

    int main() {
        yylex();
        printf("\nNumber of changes: %d\n", numberChanges);
        return 0;
    }
\end{Verbatim}
\vspace*{-0.3cm}
\caption{Eine einfache \textsl{Flex}-Spezifikation}
\label{fig:change.l}
\end{figure}

Wir diskutieren jetzt das in Abbildung \ref{fig:change.l} gezeigte Beispiel im Detail.
In diesem Beispiel wird ein Scanner spezifiziert, dessen Aufgabe es ist, alle Auftreten des
Strings ``\texttt{Stephan}'' durch ``\texttt{Stefan}'' zu ersetzen.  Zus\"atzlich
gibt der Scanner am Ende aus, wieviele Ersetzungen tats\"achlich durchgef\"uhrt worden sind.
\begin{enumerate}
\item Im Deklarations-Abschnitt binden wir die Datei ``\texttt{stdio.h}'' ein, damit wir sp\"ater
      die Funktion $\texttt{printf}()$ benutzen k\"onnen.
      Zus\"atzlich deklarieren wir die Variable \texttt{numberChanges} und initialisieren
      sie mit dem Wert 0.  In dieser Variable z\"ahlen wir die Anzahl der durchgef\"uhrten
      Ersetzungen.  
\item Da die in diesem Beispiel verwendeten regul\"aren Ausdr\"ucke trivial sind,
      enth\"alt das Beispiel keinen Definitions-Abschnitt.
\item Das Beispiel enth\"alt in Zeile 10 und 11 jeweils eine Regel:
      \begin{enumerate}
      \item In Zeile 10 ist der regul\"are Ausdruck durch den String ``\texttt{Stephan}''
            gegeben.  Da dieser Ausdruck keinerlei Operatoren enth\"alt, besteht die dadurch
            spezifizierte Sprache genau aus dem String ``\texttt{Stephan}''.  Die
            ausgef\"uhrte Aktion besteht aus zwei Kommandos:
            \begin{enumerate}
            \item Zun\"achst geben wir den String ``\texttt{Stefan}'' aus, denn wir wollen
                  den String ``\texttt{Stephan}'' durch den String ``\texttt{Stefan}'' 
                  ersetzen.
            \item Anschlie{\ss}end wird der Z\"ahler \texttt{numberChanges} inkrementiert,
                  denn wir haben ja nun eine Ersetzung durchgef\"uhrt.
            \end{enumerate}
      \item In Zeile 11 besteht der regul\"are Ausdruck nur aus dem Punkt ``\texttt{.}''.
            Dieser regul\"are Ausdruck spezifiziert ein beliebiges Zeichen, das von dem
            Zeilenumbruch ``\texttt{\symbol{92}n}'' verschieden ist.  Dieses Zeichen
            geben wir mit dem Befehl 
            \\[0.2cm]
            \hspace*{1.3cm}
            \texttt{printf(\symbol{34}\symbol{37}s\symbol{34}, yytext);}
            \\[0.2cm]
            aus.  Hier benutzen wir die in \textsl{Flex}\/ vordefinierte Variable
            ``\texttt{yytext}'', die genau den Text enth\"alt, der mit dem regul\"aren Ausdruck
            erkannt worden ist.  In diesem Fall besteht dieser Text immer aus genau einem
            Zeichen.

            An dieser Stelle fragen Sie sich vielleicht, was passiert, wenn der Scanner
            auf einen Zeilenumbruch st\"o{\ss}t.  Ein Zeilenumbruch w\"urde weder von der ersten
            noch von der zweiten Regel erfasst.  Die Konvention bei \textsl{Flex}\/ ist,
            das alle Zeichen, die von keiner Regel erfasst werden, unver\"andert ausgegeben
            werden.   Aus diesem Grunde h\"atten wir die zweite Regel ebenfalls weglassen k\"onnen.
      \end{enumerate}
      Die regul\"aren Ausdr\"ucke der beiden Regeln \"uberlappen sich, denn beispielsweise kann
      der erste Buchstaben von ``\texttt{Stephan}'' auch durch den regul\"aren Ausdruck
      ``\texttt{.}'' erkannt werden.  Damit k\"onnten im Prinzip die Aktionen
      beider Regeln ausgef\"uhrt werden.  Falls zwei verschiedene Regeln angewendet werden k\"onnen,
      geht \textsl{Flex}\/ nach folgender  Konvention vor:
      \begin{enumerate}
      \item Zun\"achst gewinnt die Regel, die auf den l\"angeren Text passt.
            In dem Beispiel wird also bei jedem Auftreten von ``\texttt{Stephan}''
            die erste Regel angewendet, denn diese passt auf den gesamten Text, w\"ahrend
            die zweite Regel nur auf einen einzigen Buchstaben passt.
      \item Ist der Text f\"ur zwei Regeln gleich lang, so entscheidet die Reihenfolge,
            in der die Regeln in der \textsl{Flex}-Spezifikation auftreten: Die Regel, die fr\"uher
            auftritt, gewinnt.
      \end{enumerate}
\item Im Programm-Abschnitt definieren wir die Funktion $\texttt{main}()$.  Diese besteht
      im wesentlichen aus dem Aufruf der Funktion $\texttt{yylex}()$, die von
      \textsl{Flex}\/ automatisch erzeugt wird.  Diese Funktion startet den erzeugten Lexer.
      Dieser Lexer liest seine Eingabe Zeichen f\"ur Zeichen ein und \"uberpr\"uft nach jedem
      gelesenen Zeichen, ob eine der im Regel-Abschnitt angegebenen Regeln anwendbar ist
      und f\"uhrt gegebenenfalls die Aktionen dieser Regel aus.
      Nach dem Aufruf von $\texttt{yylex}()$  geben wir noch die 
      Anzahl der durchgef\"uhrten Ersetzungen aus.
\end{enumerate}


\subsection{\"Ubersetzung des Beispiels}
Wir speichern die in Abbildung \ref{fig:change.l} gezeigte \textsl{Flex}-Spezifikation in einer
Datei mit dem Namen ``\texttt{change.l}'' und rufen anschlie{\ss}end \textsl{Flex}\/ mit dem Befehl
\\[0.2cm]
\hspace*{1.3cm} \texttt{flex change.l}
\\[0.2cm]
auf.  Dieser Aufruf erzeugt das \texttt{C}-Programm ``\texttt{lex.yy.c}'', das den generierten
Scanner enth\"alt.  Dieses Programm \"ubersetzen wir mit dem Befehl
\\[0.2cm]
\hspace*{1.3cm} \texttt{gcc -c lex.yy.c -o lex.yy.o}
\\[0.2cm]
in die Objekt-Datei ``\texttt{lex.yy.o}''.  Um daraus ein ausf\"uhrbares Programm zu erzeugen, binden
wir die \textsl{Flex}-Bibliothek ``\texttt{fl}'' (\underline{f}lex \underline{l}ibrary) mit dem
Befehl
\\[0.2cm]
\hspace*{1.3cm} \texttt{gcc -o change lex.yy.o -lfl}
\\[0.2cm]
ein.  Dabei entsteht die ausf\"uhrbare Datei ``\texttt{change}''.  Haben wir eine weitere Datei
``\texttt{test.txt}'', so k\"onnen wir den erzeugten Scanner durch den Aufruf
\\[0.2cm]
\hspace*{1.3cm} \texttt{./change < test.txt}
\\[0.2cm]
testen.  Bei diesem Aufruf liest der erzeugte Scanner seine Eingabe von der Datei ``\texttt{test.txt}''
und gibt die Ergebnisse am Bildschirm aus.

\subsection{Fallstricke}
Bei der Erstellung von \textsl{Flex}-Spezifikationen ist auf die folgenden Punkte besonders zu
achten. 
\begin{enumerate}
\item Die Strings ``\texttt{\symbol{37}\symbol{37}}'', mit denen der Definitions-Abschnitt von dem 
      Regel-Abschnitt und der Regel-Abschnitt von dem Programm-Abschnitt getrennt werden, m\"ussen am
      Anfang einer ansonsten leeren Zeile stehen.
\item Bei den Regeln muss der regul\"are Ausdruck am Anfang der Zeile stehen.
\item Die Kommandos, die auf eine Regel folgen, m\"ussen in derselben Zeile beginnen wie der
      zugeh\"orige regul\"are Ausdruck.  Es ist sinnvoll, diese Kommandos immer in
      geschweifte Klammern einzuschlie{\ss}en.  Dann m\"ussen wir lediglich darauf achten, dass
      die \"offnende Klammer ``\texttt{\{}'' in derselben Zeile steht wie der zugeh\"orige 
      regul\"are Ausdruck.  
\end{enumerate}


\subsection{Regul\"are Ausdr\"ucke in \textsl{Flex}}
Im letzten Kapitel haben wir regul\"are Ausdr\"ucke mit einer minimalen Syntax definiert.
Dies ist n\"utzlich, wenn wir sp\"ater die \"Aquivalenz von den durch regul\"aren Ausdr\"ucken spezifizierten
Sprachen mit den Sprachen, die von endlichen Automaten erkannt werden k\"onnen,  beweisen wollen.
F\"ur die Praxis ist eine reichhaltigere Syntax w\"unschenswert.  Daher bietet die Eingabe-Sprache von
\textsl{Flex}\/ eine Reihe von Abk\"urzungen an, mit der komplexe regul\"are Ausdr\"ucke kompakter
beschrieben werden k\"onnen.  Den regul\"aren Ausdr\"ucken von \texttt{Flex} liegt das
\textsc{Ascii}-Alphabet zu Grunde, wobei zwischen den Zeichen, die als Operatoren dienen k\"onnen,
und den restlichen Zeichen unterschieden wird.  Die Menge \textsl{OpSyms}\/ der Operator-Symbole 
ist wie folgt definiert:
\\[0.2cm]
\hspace*{0.1cm}
$\textsl{OpSyms} := $ \\[0.1cm]
\hspace*{0.8cm} 
$\{$ ``\texttt{.}'', 
``\texttt{*}'', 
``\texttt{+}'', 
``\texttt{?}'', 
``\texttt{|}'', 
``\texttt{(}'', 
``\texttt{)}'', 
``\texttt{[}'', 
``\texttt{]}'', 
``\texttt{\{}'', 
``\texttt{\}}'', 
``\texttt{<}'', 
``\texttt{>}'', 
``\texttt{/}'', 
``\texttt{\symbol{92}}'', 
``\texttt{\symbol{94}}'', 
``\texttt{\symbol{36}}'', 
``\texttt{\symbol{34}}'' $\}$
\\[0.2cm]
Damit k\"onnen wir nun die Menge $\textsl{Regexp}\/$ der von \textsl{Flex}\/ unterst\"utzen regul\"aren
Ausdr\"ucke induktiv definieren. 
\begin{enumerate}
\item $c \in \textsl{Regexp}$ \quad falls $c \in \Sigma_{\textsc{\scriptsize Ascii}} \backslash \textsl{OpSyms}$

      Alle Buchstaben $c$ aus dem \textsc{Ascii}-Alphabet, die keine Operator-Symbol sind, k\"onnen
      als regul\"are Ausdr\"ucke, verwendet werden.  Diese Ausdr\"ucke spezifizieren genau diesen Buchstaben.
      
\item $\texttt{\symbol{92}}x \in \textsl{Regexp}$ \quad 
      falls $x \in \{ \texttt{a}, \texttt{b}, \texttt{f}, \texttt{n}, \texttt{r}, \texttt{t}, \texttt{v} \}$ 

      Die Syntax $\texttt{\symbol{92}}x$ erm\"oglicht es, Steuerzeichen 
      zu spezifizieren.  Im Einzelnen gilt:
      \begin{enumerate}
      \item \texttt{\symbol{92}a} entspricht dem Steuerzeichen \texttt{Ctrl-G} (\emph{alert}).
      \item \texttt{\symbol{92}b} entspricht dem Steuerzeichen \texttt{Ctrl-H} (\emph{backspace}).
      \item \texttt{\symbol{92}f} entspricht dem Steuerzeichen \texttt{Ctrl-L} (\emph{form feed}).
      \item \texttt{\symbol{92}n} entspricht dem Steuerzeichen \texttt{Ctrl-J} (\emph{newline}).
      \item \texttt{\symbol{92}r} entspricht dem Steuerzeichen \texttt{Ctrl-M} (\emph{carriage return}).
      \item \texttt{\symbol{92}t} entspricht dem Steuerzeichen \texttt{Ctrl-I} (\emph{tabulator}).
      \item \texttt{\symbol{92}v} entspricht dem Steuerzeichen \texttt{Ctrl-K} (\emph{vertical tabulator}).
      \end{enumerate}
\item $\texttt{\symbol{92}}abc \in \textsl{Regexp}$ \quad 
      falls $a,b,c \in \{ \texttt{0}, \cdots \texttt{7} \}$ 

      Bei der Syntax $\texttt{\symbol{92}}abc$ sind $a$, $b$ und $c$ oktale Ziffern und $abc$
      muss als Zahl im Oktal-System interpretierbar sein.
      Dann wird durch  $\texttt{\symbol{92}}abc$ das Zeichen spezifiziert, das im
      \textsc{Ascii}-Code an der durch die Oktalzahl $abc$ spezifizierten Stelle steht.
\item $\texttt{\symbol{92}}o \in \textsl{Regexp}$ \quad falls $o \in \textsl{OpSyms}$ 
      
      Die Operator-Symbole k\"onnen durch Voranstellen eines Backslashs spezifiziert werden.
\item $r_1r_2 \in \textsl{Regexp}$ \quad falls $r_1,r_2 \in \textsl{Regexp}$

      Die Konkatenation zweier regul\"arer Ausdr\"ucke wird in \textsl{Flex}\/ ohne den Infix-Operator
      ``$\cdot$'' geschrieben.
\item $r_1\texttt{|}r_2 \in \textsl{Regexp}$ \quad falls $r_1,r_2 \in \textsl{Regexp}$

      F\"ur die Addition zweier regul\"arer Ausdr\"ucke wird in \textsl{Flex}\/ an Stelle des Infix-Operators
      ``$+$'' der Operator ``\texttt{|}'' verwendet.
\item $r\texttt{*} \in \textsl{Regexp}$ \quad falls $r \in \textsl{Regexp}$

      Der Postfix-Operator ``\texttt{*}'' bezeichnet den Kleene-Abschluss.
\item $r\texttt{+} \in \textsl{Regexp}$ \quad falls $r \in \textsl{Regexp}$

      Der Ausdruck ``$r\texttt{+}$'' ist eine Variante des Kleene-Abschlusses, bei der
      gefordert wird, dass $r$ mindestens einmal auftritt.  Daher gilt die folgende \"Aquivalenz:
      \\[0.2cm]
      \hspace*{1.3cm}
      $r\texttt{+} \doteq rr*$.
\item $r\texttt{?} \in \textsl{Regexp}$ \quad falls $r \in \textsl{Regexp}$

      Der Ausdruck ``$r\texttt{?}$'' legt fest, dass $r$ einmal oder keinmal auftritt.
      Es gilt die folgende \"Aquivalenz:
      \\[0.2cm]
      \hspace*{1.3cm}
      $r\texttt{?} \doteq r|\varepsilon$.
      \\[0.2cm]
      Hier ist allerdings zu beachten, dass der Ausdruck ``$\varepsilon$'' 
      von \textsl{Flex} nicht unterst\"utzt wird.
\item $r\texttt{\{}n\texttt{\}} \in \textsl{Regexp}$ \quad falls $n \in \mathbb{N}$

      Der Ausdruck ``$r\texttt{\{}n\texttt{\}}$'' legt fest, dass $r$ genau $n$ mal auftritt.
      Der regul\"are Ausdruck ``\texttt{a\{4\}}'' beschreibt also den String ``\texttt{aaaa}''.
\item $\texttt{\symbol{94}}r$ \quad falls $r \in \textsl{Regexp}$

      Der Ausdruck $\texttt{\symbol{94}}r$ legt fest, dass der regul\"are Ausdruck $r$
      am Anfang einer Zeile stehen muss.  
\item $r\texttt{\symbol{36}}$ \quad falls $r \in \textsl{Regexp}$

      Der Ausdruck $r\texttt{\symbol{36}}$ legt fest, dass der regul\"are Ausdruck $r$
      am Ende einer Zeile stehen muss.  
\item $r_1\texttt{/}r_2$ \quad falls $r_1, r_2 \in \textsl{Regexp}$
  
      Der Ausdruck $r_1\texttt{/}r_2$ legt fest, dass auf den durch $r_1$ spezifizierten Text
      ein Text folgen muss, der der Spezifikation $r_2$ gen\"ugt.  Im Unterschied zur einfachen
      Konkatenation von $r_1$ und $r_2$ wird durch den $r_1\texttt{/}r_2$ aber derselbe Text
      spezifiziert, der durch $r_1$ spezifiziert wird.  Der Operator ``\texttt{/}'' liefert
      also nur eine zus\"atzliche Bedingung, die f\"ur eine erfolgreiche Erkennung des regul\"aren
      Ausdrucks erf\"ullt sein muss.  Die Variable ``\texttt{yytext}'', in der der erkannte Text
      aufgesammelt wird, bekommt nur den Text zugewiesen, der dem regul\"aren Ausdruck $r_1$
      entspricht.  Der Text, der dem regul\"aren Ausdruck $r_2$ entspricht, wird dann von der
      n\"achsten Regel gematcht.  In der angels\"achsischen Literatur wird $r_2$ als
      \emph{trailing context} bezeichnet.
\item $(r) \in \textsl{Regexp}$ \quad falls $r \in \textsl{Regexp}$

      Genau wie im letzten Kapitel auch k\"onnen regul\"are Ausdr\"ucke geklammert werden.
      F\"ur die Pr\"azedenzen der Operatoren gilt:
      Die Postfix-Operatoren ``\texttt{*}'', ``\texttt{?}'', ``\texttt{+}'' und ``\texttt{\{}$n$\texttt{\}}''
      binden am st\"arksten, der Operator ``\texttt{|}'' bindet am schw\"achsten.
\end{enumerate}
Die Spezifikation der regul\"aren Ausdr\"ucke ist noch nicht vollst\"andig, denn es gibt in \textsl{Flex}
noch die M\"oglichkeit, sogenannte \emph{Bereiche} zu spezifizieren.  Ein \emph{Bereich}
spezifiziert eine Menge von Buchstaben in kompakter Weise.  Dazu werden die eckigen
Klammern benutzt.  Beispielsweise lassen sich die Vokale durch den regul\"aren Ausdruck
\\[0.2cm]
\hspace*{1.3cm}
\texttt{[aeiou]}
\\[0.2cm]
spezifizieren.  Dieser Ausdruck ist als Abk\"urzung zu verstehen, es gilt:
\\[0.2cm]
\hspace*{1.3cm}
$\texttt{[aeiou]} \doteq \texttt{a|e|i|o|u}$
\\[0.2cm]
Die Menge aller kleinen lateinischen Buchstaben l\"asst sich durch
\\[0.2cm]
\hspace*{1.3cm}
\texttt{[a-z]}
\\[0.2cm]
spezifizieren, es gilt also
\\[0.2cm]
\hspace*{1.3cm}
$\texttt{[a-z]} \doteq \texttt{a|b|c|$\cdots$|x|y|z}$.
\\[0.2cm]
Die Menge aller lateinischen Buchstaben zusammen mit dem Unterstrich
kann durch
\\[0.2cm]
\hspace*{1.3cm}
\texttt{[a-zA-Z\_]}
\\[0.2cm]
beschrieben werden.  \textsl{Flex} gestattet auch, das Komplement einer solchen Menge zu bilden.
Dazu ist es lediglich erforderlich, nach der \"offnenden eckigen Klammer das Zeichen
``\texttt{\symbol{94}}'' zu verwenden.  Beispielsweise beschreibt der Ausdruck
\\[0.2cm]
\hspace*{1.3cm}
\texttt{[\symbol{94}0-9]}
\\[0.2cm]
alle \textsc{Ascii}-Zeichen, die keine Ziffern sind.  Innerhalb von Bereichen verlieren 
die meisten Operator-Symbole ihre Sonderbedeutung und k\"onnen ohne Backslash geschrieben
werden.  Innerhalb eines Bereiches z\"ahlen nur die Symbole
\\[0.2cm]
\hspace*{1.3cm}
``\texttt{-}'', ``\texttt{\symbol{94}}'', ``\texttt{]}'' und ``\texttt{\symbol{92}}''
\\[0.2cm]
als Operator-Symbole.  Damit erkennt der Bereich
\\[0.2cm]
\hspace*{1.3cm}
\texttt{[+?*]}
\\[0.2cm]
also genau die Postfix-Operatoren ``\texttt{+}'', ``\texttt{?}'' und ``\texttt{*}''.

\examples
Um die Diskussion anschaulicher zu machen, pr\"asentieren wir einige Beispiele regul\"arer Ausdr\"ucke.
\begin{enumerate}
\item \texttt{[a-zA-Z][a-zA-Z0-9\_]*}

      Dieser regul\"are Ausdruck spezifiziert die Worte, die aus lateinischen Buchstaben, Ziffern und
      dem Unterstrich ``\texttt{\_}'' bestehen und die au{\ss}erdem mit einem lateinischen 
      Buchstaben beginnen.
\item \texttt{\symbol{92}/\symbol{92}/.*}

      Hier wird ein \texttt{C}-Kommentar beschrieben, der sich bis zum Zeilenende erstreckt.
\item \texttt{0|[1-9][0-9]*}

      Dieser Ausdruck beschreibt nat\"urliche Zahlen.  Hier ist es wichtig darauf zu achten,
      dass eine nat\"urliche Zahl nur dann mit der Ziffer \texttt{0} beginnt, wenn es sich um die
      Zahl \texttt{0} handelt.
\end{enumerate}

\subsection{Ein Beispiel}
In diesem Abschnitt diskutieren wir eine Anwendung von \textsl{Flex}.  Es geht dabei um die
Auswertung von Klausuren.   Bei der Korrektur einer Klausur lege ich eine Datei an, die das in dem
in Abbildung \ref{fig:ergebnis} beispielhaft gezeigte Format besitzt.

\begin{figure}[!h]
\centering
\begin{Verbatim}[ frame         = lines, 
                  framesep      = 0.3cm, 
                  labelposition = bottomline,
                  numbers       = left,
                  numbersep     = -0.2cm,
                  xleftmargin   = 0.8cm,
                  xrightmargin  = 0.8cm,
                ]
    Klausur: Algorithmen und Datenstrukturen
    Kurs:    TIT07AIX
    
    Aufgaben:            1. 2. 3. 4. 5. 6.
    Max M\"uller:          9 12 10  6  6  0
    Dietmar Dumpfbacke:  4  4  2  0  -  -
    Susi Sorglos:        9 12 12  9  9  6
\end{Verbatim}
\vspace*{-0.3cm}
\caption{Klausurergebnisse}
\label{fig:ergebnis}
\end{figure}

\begin{enumerate}
\item Die erste Zeile enth\"alt nach dem Schl\"usselwort \texttt{Klausur} den Titel der Klausur.
\item Die zweite Zeile gibt den Kurs an.
\item Die dritte Zeile ist leer.
\item Die vierte Zeile gibt die Nummern der einzelnen Aufgaben an.
\item Danach folgt eine Tabelle.  Jede Zeile dieser Tabelle listet die Punkte auf,
      die ein Student erzielt hat.  Der Name des Studenten wird dabei am Zeilenanfang angegeben.
      Auf den Namen folgt ein Doppelpunkt und daran schlie{\ss}en sich dann Zahlen an, die angeben,
      wieviele Punkte bei den einzelnen Aufgaben erzielt wurden.  Wurde eine Aufgabe nicht
      bearbeitet, so steht in der entsprechenden Spalte ein Bindestrich ``\texttt{-}''.
\end{enumerate}
Das \textsl{Flex}-Programm, das wir entwickeln werden, berechnet zun\"achst die Summe
\texttt{sumPoints} aller Punkte, die ein Student erzielt hat.  Aus dieser Summe wird dann nach der
Formel 
\\[0.2cm]
\hspace*{1.3cm}
$\texttt{note} = 7 - 6 \cdot \bruch{\;\texttt{sumPoints}\;}{\texttt{maxPoints}}$
\\[0.2cm]
die Note errechnet, wobei die Variable \texttt{maxPoints} die  Punktzahl
angibt, die f\"ur die Note 1,0 ben\"otigt wird.  Diese Zahl ist ein Argument, das dem Programm
beim Start \"ubergeben wird. 

\begin{figure}[!h]
\centering
\begin{Verbatim}[ frame         = lines, 
                  framesep      = 0.3cm, 
                  labelposition = bottomline,
                  numbers       = left,
                  numbersep     = -0.2cm,
                  xleftmargin   = 0.8cm,
                  xrightmargin  = 0.8cm,
                ]
    %{
    /* This lexer computes grades. */
    #include <stdlib.h>
    #include <stdio.h>
    int   sumPoints, maxPoints;
    int   lineNumber = 1;
    void  errorMsg();
    float note();
    %}    
    ZAHL   0|[1-9][0-9]*
    NAME   [A-Za-z\"o\"a\"u\"O\"A\"U{\ss}]+[ ][A-Za-z\"o\"a\"u\"O\"A\"U{\ss}]+
    %% 
    [A-Za-z]+:.*\n { ++lineNumber;               }
    {NAME}/:       { printf("%s", yytext); 
                     sumPoints = 0;              }
    :[ \t]+        { printf("%s", yytext);       }
    {ZAHL}         { sumPoints += atoi(yytext);  }
    -              { /* skip hyphens     */      }
    [ \t]          { /* skip white space */      }
    ^[ \t]*\n      { ++lineNumber;               }
    \n             { printf(" %3.1f\n", note()); 
                     ++lineNumber;
                   }
    .              { errorMsg();                 }
    %%    
    float note() {
        return 7.0 - 6.0 * sumPoints / maxPoints;
    }
    void errorMsg() {
        printf("invalid character '%s' at line %d\n", yytext, lineNumber); 
    }
    int main(int argc, char* argv[]) {
        maxPoints = atoi(argv[1]);
        yylex();
        return 0;
    }
\end{Verbatim}
\vspace*{-0.3cm}
\caption{Ein \textsl{Flex}-Programm zur Berechnung von Noten}
\label{fig:noten.l}
\end{figure}

Abbildung \ref{fig:noten.l} zeigt ein \textsl{Flex}-Programm, das die Aufgabe der Notenberechnung
l\"ost.  Wir diskutieren dieses Programm jetzt Zeile f\"ur Zeile.
\begin{enumerate}
\item In dem Deklarations-Abschnitt, der sich von Zeile 1 bis Zeile 9 erstreckt, binden wir
      zun\"achst die Header-Dateien ``\texttt{stdlib.h}'' und ``\texttt{stdio.h}'' ein.  
      Anschlie{\ss}end deklarieren wir Variablen und Funktionen:
      \begin{enumerate}
      \item Die Variable \texttt{sumPoints} speichert die Summe aller Punkte, die ein Student 
            in der Klausur erreicht hat und \texttt{maxPoints} speichert die Anzahl der 
            Punkte, bei der die Note 1,0 erreicht wird.
      \item In der Variablen \texttt{lineNumber} speichern wir die Zeilennummer.
            Dies ist n\"utzlich um sp\"ater aussagekr\"aftige Fehlermeldungen f\"ur den Fall geben
            zu k\"onnen, wenn in der Eingabedatei etwas unerwartetes gefunden wird.
      \item Die in Zeile 7 deklarierte Funktion $\texttt{errorMsg}()$ kann verwendet werden
            um Fehlermeldungen auszugeben.
            Diese Funktion wird im Programm-Abschnitt in den Zeilen 29 -- 31 definiert.  
      \item Die in Zeile 8 deklarierte Funktion $\texttt{note}()$ dient der Berechnung von Noten.
            Diese Funktion wird in den Zeilen 26 -- 28 definiert.
      \end{enumerate}
\item In dem Definitions-Abschnitt werden zwei Abk\"urzungen definiert:
      \begin{enumerate}
      \item Zeile 10 enth\"alt die Definition von \texttt{ZAHL}.  Mit dieser Definition k\"onnen
            wir sp\"ater anstelle des regul\"aren Ausdrucks
            \\[0.2cm]
            \hspace*{1.3cm}
            \texttt{0|[1-9][0-9]*}
            \\[0.2cm]
            k\"urzer ``\texttt{\{ZAHL\}}'' schreiben.  Beachten Sie, dass der Name einer Abk\"urzung
            bei der Verwendung der Abk\"urzung in geschweiften Klammern eingefasst werden muss.
      \item Zeile 11 enth\"alt die Definition von \texttt{NAME}.  In dem regul\"aren Ausdruck
            \\[0.2cm]
            \hspace*{1.3cm}
            \texttt{[A-Za-z\"o\"a\"u\"O\"A\"U{\ss}]+[ ][A-Za-z\"o\"a\"u\"O\"A\"U{\ss}]+}
            \\[0.2cm]
            wird festgelegt, dass ein Name gro{\ss}en und kleinen lateinischen
            Buchstaben sowie Umlauten besteht und das Vor- und Nachname durch ein
            Leerzeichen getrennt werden.
      \end{enumerate}
\item Der Regel-Abschnitt erstreckt sich von Zeile 13 -- 24.
      \begin{enumerate}
      \item Die Regel in Zeile 13 dient dazu, die beiden Kopfzeilen
            der zu verarbeitenden Datei zu lesen.  Diese Zeilen bestehen jeweils aus einem Wort,
            auf das ein Dopplepunkt folgt.  Dahinter steht beliebiger Text, der mit einem
            Zeilenumbruch endet.  Da wir die Kopfzeilen nicht weiter verarbeiten wollen,
            inkrementieren wir lediglich den Z\"ahler \texttt{lineNumber}, denn wir haben ja gerade
            einen Zeilenumbruch gelesen.
      \item Die Regel in Zeile 14 liest den Namen eines Studenten, dem ein Doppelpunkt
            folgen muss.  Da wir den Doppelpunkt mit dem Operator ``\texttt{/}''von dem Namen
            abtrennen, ist der Doppelpunkt nicht Bestandteil des von dieser Regel gelesenen Textes.
            Dadurch k\"onnen wir den Doppelpunkt in der n\"achsten Regel noch benutzen.

            Wenn wir einen Namen gelesen haben, geben wir diesen mit Hilfe eines
            \texttt{printf}-Befehls aus und setzen anschlie{\ss}end die Variable \texttt{sumPoints}
            auf 0.  Dies ist erforderlich, weil diese Variable ja vorher noch die Punkte eines
            anderen Studenten enthalten k\"onnte.
      \item Die n\"achste Regel in Zeile 16 liest die Leerzeichen und Tabulatoren ein, die auf 
            den Doppelpunkt folgen und gibt diese aus.  Dadurch erreichen wir, dass die Ausgabe
            der Noten genauso formatiert wird wie die Eingabe-Datei.
      \item Die Regel in Zeile 17 dient dazu, die Punkte, die der Student bei einer Aufgabe 
            erreicht hat, einzulesen.  Da die Zahl zun\"achst nur als String zur Verf\"ugung steht,
            m\"ussen wir diesen String mit Hilfe der Bibliotheks-Funktion $\texttt{atoi()}$ in eine Zahl
            umwandeln.  Anschlie{\ss}end wird diese Zahl dann zu der Summe der Punkte hinzuaddiert.

            Die Verwendung der Funktion $\texttt{atoi}()$ ist \"ubrigens der Grund, weshalb wir
            die Header-Datei ``\texttt{stdlib.h}'' einbinden m\"ussen.
      \item F\"ur nicht bearbeitete Aufgaben enth\"alt die Eingabe-Datei einen Bindestrich
            ``\texttt{-}''.  Diese Bindestriche werden durch die Regel in Zeile 18
            eingelesen und ignoriert.  Daher ist das Kommando dieser Regel (bis auf den Kommentar)
            leer.
      \item In der gleichen Weise \"uberlesen wir mit der Regel in Zeile 19
            Leerzeichen und Tabulatoren, die nicht auf einen Doppelpunkt folgen.
      \item Die Regel in Zeile 20 dient dazu Zeilen einzulesen, die nur aus Leerzeichen
            und Tabulatoren bestehen.  Hier muss lediglich der Z\"ahler ``\texttt{lineNumber}''
            inkrementiert werden.
      \item Wenn wir einen einzelnen Zeilenumbruch lesen, dann muss dieser von einer Zeile
            stammen, die die Punkte eines Studenten auflistet.  In diesem Fall berechnen wir mit
            der Regel in Zeile 21 die erzielte Note und geben sie mit einer Stelle hinter
            dem Komma aus.  Zus\"atzlich wird wieder der Z\"ahler ``\texttt{lineNumber}''
            inkrementiert. 
      \item Die bis hierhin vorgestellten Regeln erm\"oglichen es, eine syntaktisch korrekte 
            Eingabe-Datei zu verarbeiten.  F\"ur den Fall, dass die Eingabe-Datei Syntaxfehler
            enth\"alt, ist es sinnvoll, eine Fehlermeldung auszugeben, denn sonst k\"onnte es passieren,
            dass auf Grund eines einfachen Tippfehlers eine falsche Note berechnet wird.
            Daher enth\"alt Zeile 24 eine Default-Regel, die immer dann greift, wenn keine der
            anderen Regeln zum Zuge gekommen ist.  Diese Regel liest ein einzelnes Zeichen und
            gibt eine Fehlermeldung aus.  Diese Fehlermeldung enth\"alt das gelesene Zeichen sowie
            die Zeilennummer, in der dieses Zeichen gefunden wurde. 
      \end{enumerate}
\item Der Programm-Abschnitt erstreckt sich von Zeile 26 bis zum Ende der Datei.
      Bemerkenswert ist hier lediglich die Implementierung der Funktion $\texttt{main}()$.
      Um die Notenberechnung durchf\"uhren zu k\"onnen ist es erforderlich, dass wir dem Programm
      die maximale erreichbare Punktzahl mitteilen, denn diese Punktzahl geht nicht aus der
      Tabelle hervor.  Wir \"ubergeben diese Zahl der Funktion $\texttt{main}()$ als erstes Argument.
      Dieses Argument liegt in \texttt{argv[1]} zun\"achst als String vor.  Wir konvertieren
      es mit Hilfe der Funktion $\texttt{atoi}()$ in eine Zahl, die wir in der globalen
      Variablen \texttt{maxPoints} abspeichern.
\end{enumerate}
Das obige Beispiel l\"ost eine vergleichsweise einfache Aufgabe.  Nat\"urlich k\"onnte diese Aufgabe
auch durch ein ganz normales \texttt{C}-Programm gel\"ost werden.  Dieses
\texttt{C}-Programm w\"are allerdings l\"anger als das oben gezeigte \textsl{Flex}-Programm
und es w\"are in jedem Fall deutlich schwieriger zu verstehen, denn regul\"are Ausdr\"ucke sind eine
sehr pr\"agnante M\"oglichkeit um die Syntax einzelner Token zu beschreiben.

\subsection{Start-Zust\"ande}
Viele syntaktische Konstrukte lassen sich zwar im Prinzip mit regul\"aren Ausdr\"ucken beschreiben,
aber die Ausdr\"ucke, die ben\"otigt werden, sind sehr un\"ubersichtlich.  Ein gutes Beispiel
hierf\"ur ist der regul\"are Ausdruck zur Spezifikation von mehrzeilige
\texttt{C}-Kommentaren, also Kommentaren der Form
\\[0.2cm]
\hspace*{1.3cm}
\texttt{/*} $\cdots$ \texttt{*/}
\\[0.2cm]
Der regul\"are Ausdruck, der diese Art von Kommentaren spezifiziert, ist wie folgt:
%\/\*([^*]|(\*+[^*/]))*\*+\/ 
\begin{equation}
  \label{eq:multiline-comment}
\texttt{\symbol{92}/\symbol{92}*([\symbol{94}*]|\symbol{92}*+[\symbol{94}*/])*\symbol{92}*+\symbol{92}/}  
\end{equation}
Zun\"achst ist dieser Ausdruck schwer zu lesen.  Das liegt vor allem daran,
dass die Operator-Sysmbole ``\texttt{/}'' und ``\texttt{*}'' durch einen Backslash
gesch\"utzt werden m\"ussen.  Aber auch die Logik, die hinter diesem Ausdruck steht, ist nicht
ganz einfach.  Wir analysieren die einzelnen Komponenten dieses Ausdrucks:
\begin{enumerate}
\item \texttt{\symbol{92}/\symbol{92}*} 

      Hierdurch wird der String ``\texttt{/*}'', der den Kommentar einleitet,  spezifiziert.
\item \texttt{([\symbol{94}*]|\symbol{92}*+[\symbol{94}*/])*}
  
      Dieser Teil spezifiziert alle Zeichen, die zwischen dem \"offnenden String ``\texttt{/*}''
      und einem schlie{\ss}enden String der Form \texttt{*$\cdots$*/} liegen.  Wie m\"ussen
      sicherstellen, dass dieser Teil die Zeichenreihe ``\texttt{*/}'' nicht enth\"alt, denn
      sonst w\"urden wir in einer Zeile der Form
      \\[0.2cm]
      \hspace*{1.3cm} \texttt{/* first */ ++n; /* second */}
      \\[0.2cm]
      den Befehl ``\texttt{++n;}'' f\"ur einen Teil des Kommentars halten.  Der erste Teil des
      obigen regul\"aren Ausdrucks ``\texttt{[\symbol{94}*]}'' steht f\"ur ein beliebiges
      von ``\texttt{*}'' verschiedenes Zeichen.  Denn solange wir kein ``\texttt{*}'' lesen,
      kann der Text auch kein ``\texttt{*/}'' enthalten.  Das Problem ist, dass das Innere
      eines Kommentars aber durchaus das Zeichen ``\texttt{*}'' enthalten kann, es darf
      nur kein ``\texttt{/}'' folgen.  Daher spezifiziert die Alternative
      ``\texttt{\symbol{92}*+[\symbol{94}*/]}'' einen String, der aus beliebig vielen 
      ``\texttt{*}''-Zeichen besteht, auf die dann aber noch ein Zeichen folgen muss, dass
      sowohl von ``\texttt{/}'' als auch von ``\texttt{*}'' verschieden ist.
      
      Der Ausdruck ``\texttt{[\symbol{94}*]|\symbol{92}*+[\symbol{94}*/]}''
      spezifiziert jetzt also entweder ein Zeichen, das von ``\texttt{*}'' verschieden
      ist, oder aber eine Folge von ``\texttt{*}''-Zeichen, auf die dann noch ein von
      ``\texttt{/}'' verschiedenes Zeichen folgt.  Da solche Folgen beliebig oft vorkommen
      k\"onnen, wird der ganze Ausdruck in Klammern eingefasst und mit dem Quantor
      ``\texttt{*}'' dekoriert.   
\item \texttt{\symbol{92}*+\symbol{92}/}

      Dieser regul\"are Ausdruck spezifiziert das Ende des Kommentars.  Es kann aus einer 
      beliebigen positiven Anzahl von ``\texttt{*}''-Zeichen bestehen, auf die dann noch
      ein ``\texttt{/}'' folgt.  Wenn wir hier nur den Ausdruck
      ``\texttt{\symbol{92}*\symbol{92}/}'' verwenden w\"urden, dann k\"onnten wir Kommentare der
      Form 
      \\[0.2cm]
      \hspace*{1.3cm}
      \texttt{/*** blah ***/}
      \\[0.2cm]
      nicht mehr erkennen, denn der unter 2. diskutierte regul\"are Ausdruck akzeptiert
      nur Folgen von ``\texttt{*}'', auf die kein ``\texttt{*}'' folgt.
\end{enumerate}


\begin{figure}[!h]
\centering
\begin{Verbatim}[ frame         = lines, 
                  framesep      = 0.3cm, 
                  labelposition = bottomline,
                  numbers       = left,
                  numbersep     = -0.2cm,
                  xleftmargin   = 0.8cm,
                  xrightmargin  = 0.8cm,
                ]
    %{
      /* remove C comments from a file */
    %}
    
    %%
    \/\*([^*]|\*+[^*/])*\*+\/ { /* skip multi  line comments */ }
    \/\/.*                    { /* skip single line comments */ }
    
    %%
    int main() {
        yylex();
    }
\end{Verbatim}
\vspace*{-0.3cm}
\caption{Entfernung von Kommentaren aus einem C-Programm.}
\label{fig:decoment.l}
\end{figure}

\noindent
Abbildung \ref{fig:decoment.l} zeigt ein \textsl{Flex}-Programm, das aus einem
\texttt{C}-Programm sowohl einzeilige Kommentare der Form ``\texttt{//} $\cdots$''
als auch mehrzeilige Kommentare der Form ``\texttt{/*} $\cdots$ \texttt{*/} entfernt.
Das Programm an sich ist zwar recht kurz, aber der regul\"are Ausdruck zur Erkennung
mehrzeiliger Kommentare ist sehr kompliziert und damit nur schwer zu verstehen.
Um in \"ahnlichen F\"allen ein lesbareres Programm schreiben zu k\"onnen, gibt es in
\textsl{Flex}\/ sogenannte \emph{Start-Zust\"ande} (engl.~\emph{start conditions}), mit deren
Hilfe komplizierte regul\"are Ausdr\"ucke vermieden werden k\"onnen.
Wir diskutieren diese Zust\"ande anhand eines Beispiels: Wir wollen eine
\textsc{Html}-Datei in eine Text-Datei konvertieren.  Die \textsl{Flex}-Spezifikation, die
in Abbildung \ref{fig:html2txt} gezeigt wird, f\"uhrt dazu die folgenden Aktionen durch:
\begin{enumerate}
\item Zun\"achst wird der Kopf der \textsc{Html}-Datei, der in den Tags ``\texttt{<head>}''
      und ``\texttt{</head>}'' eingeschlossen ist, entfernt.
\item Die Skripte, die in der \textsc{Html}-Datei enthalten sind, werden ebenfalls
      entfernt.
\item Au{\ss}erdem werden die \textsc{Html}-Tags entfernt.
\end{enumerate}

\begin{figure}[!h]
\centering
\begin{Verbatim}[ frame         = lines, 
                  framesep      = 0.3cm, 
                  labelposition = bottomline,
                  numbers       = left,
                  numbersep     = -0.2cm,
                  xleftmargin   = 0.8cm,
                  xrightmargin  = 0.8cm,
                ]
    %x header script
    %%
    "<head>"            { BEGIN(header);           }
    "<script"[^>\n]+">" { BEGIN(script);           }
    "<"[^>\n]+">"       { /* skip html tags */     }
    &nbsp;              { printf(" ");             }
    
    <header>"</head>"   { BEGIN(INITIAL);          }
    <header>.|\n        { /* skip anything else */ }
    
    <script>"</script>" { BEGIN(INITIAL); }
    <script>.|\n        { /* skip anything else */ }
    %%
    int main() { yylex(); }
\end{Verbatim}
\vspace*{-0.3cm}
\caption{\textsl{Flex}-Spezifikation zur Transformation von \textsc{Html} in Text.}
\label{fig:html2txt}
\end{figure}
\noindent
Wir diskutieren jetzt die \textsl{Flex}-Spezifikation aus Abbildung \ref{fig:html2txt} im
Detail.
\begin{enumerate}
\item Im Definitions-Teil deklarieren wir in Zeile 1 die beiden \textsl{Zust\"ande}\/
      \texttt{header} und \texttt{script} als \emph{exklusive} Start-Zust\"ande.
      Die allgemeine Syntax einer solchen Deklaration ist wie folgt:
      \begin{enumerate}
      \item Am Zeilen-Anfang einer Zustands-Deklaration steht der String ``\texttt{\symbol{37}x}''
            oder ``\texttt{\symbol{37}s}''.  
            Der String ``\texttt{\symbol{37}x}'' spezifiziert \emph{exklusive} Zust\"ande,
            der String ``\texttt{\symbol{37}s}'' spezifiziert \emph{inklusive} Zust\"ande.
            Den Unterschied zwischen diesen beiden Zustandsarten erkl\"aren wir sp\"ater.
      \item Darauf folgt eine Liste der Namen der deklarierten Zust\"ande.  Die Namen werden
            durch Leerzeichen getrennt.
      \end{enumerate}
\item In Zeile 3 haben wir den String ``\texttt{<head>}'' in doppelte Hochkommata
      eingeschlossen.  Dadurch verlieren die Operator-Symbole ``\texttt{<}'' ihre
      Bedeutung.  Dies ist eine allgemeine M\"oglichkeit, um Operator-Symbole in
      \textsl{Flex}\/ spezifizieren zu k\"onnen.  Wollen wir beispielsweise den String
      ``\texttt{a*}''  w\"ortlich erkennen, so k\"onnen wir an Stelle von ``\texttt{a\symbol{92}*}''
      auch klarer
      \\[0.2cm]
      \hspace*{1.3cm}
      \texttt{\symbol{34}a*\symbol{34}}
      \\[0.2cm]
      schreiben.
      
      Wird der String ``\texttt{<head>}'' erkannt, so wird in Zeile 3 die Aktion 
      ``\texttt{BEGIN(header)}'' ausgef\"uhrt.  Damit wechselt der Scanner aus dem
      Default-Zustand ``\texttt{INITIAL}'', in dem der Scanner startet,
      in den oben deklarierten Zustand \texttt{header}.
      Da dieser Zustand als \emph{exklusiver} Zustand deklariert worden ist, k\"onnen jetzt
      nur noch solche Regeln angewendet werden, die mit dem Prefix ``\texttt{<header>}''
      beginnen.  W\"are der Zustand als \emph{inklusiver} Zustand deklariert worden, so k\"onnten auch solche 
      Regeln verwendet werden, die nicht mit einem Zustand markiert sind.  Solche Regeln
      sind implizit mit dem Zustand ``\texttt{<INITIAL>}'' markiert.
      
      Die Regeln, die mit dem Zustand ``\texttt{header}'' markiert sind, finden wir weiter
      unten in den Zeilen 8 und 9.
\item In Zeile 4 wechseln wir entsprechend in den Zustand ``\texttt{script}'' wenn
      wir ein \"offnendes \texttt{Script}-Tag sehen.
\item In Zeile 5 werden alle restlichen Tags gelesen.  Da die Aktion hier leer ist,
      werden diese Tags entfernt.
\item In Zeile 6 ersetzen wir die Zeichenreihe ``\texttt{\&nbsp;}'' durch ein Blank.
      Wollten wir das Skript wirklich einsetzen, so h\"atten wir hier noch analoge
      Zeilen zur Verarbeitung von Umlauten und Sonderzeichen.
\item Zeile 8 beginnt mit der Zustands-Spezifikation ``\texttt{header}''.  Daher ist
      diese Regel nur dann aktiv, wenn der Scanner in dem Zustand ``\texttt{head}'' ist.
      Diese Regel sucht nach dem schlie{\ss}enden Tag ``\texttt{</head>}''.  Wird dieses Tag
      gefunden, so wechselt der Scanner zur\"uck in den Default-Zustand \texttt{INITIAL},
      in dem nur die Regeln verwendet werden, die nicht mit einem Zustand markiert sind.
\item Zeile 9 enth\"alt ebenfalls eine Regel, die nur im Zustand ``\texttt{header}''
      ausgef\"uhrt wird.  Diese Regel liest ein beliebiges Zeichen, welches nicht weiter 
      verarbeitet wird und daher im Endeffekt verworfen wird.
\item Die Zeilen 11 und 12 enthalten entsprechende Regeln f\"ur den Zustand
      ``\texttt{script}''.
\end{enumerate}
Ist der Automat im Default-Zustand ``\texttt{INITIAL}'' und liest ein Zeichen, das nicht
durch die obigen Regeln verarbeitet wird, so wird dieses Zeichen unver\"andert ausgegeben.
Dadurch erreichen wir, dass der in der HTML-Datei enthaltene Text ausgegeben wird.

\exercise
Implementieren Sie eine \textsl{Flex}-Spezifikation, die aus einem
\texttt{C}-Programm alle Kommentare entfernt.  Das Programm soll mit Hilfe von Zust\"anden arbeiten.


\subsection{Zusammenfassung}
\textsl{Flex}\/ ist ein m\"achtiges Werkzeug um effiziente Scanner zu erzeugen,
das in den Werkzeugkasten jedes Informatikers geh\"ort.  Die obige Darstellung ist eher als
Appetitanreger gedacht, denn f\"ur eine ausf\"uhrliche Darstellung fehlt uns jetzt die Zeit.
Wir werden das Thema \textsl{Flex}\/ allerdings sp\"ater noch einmal aufgreifen und zwar dann, wenn wir den
Parser-Generator \textsl{Bison}\/ diskutieren.

Falls Sie sp\"ater einmal \textsl{Flex}\/ selber einsetzen wollen, dann finden Sie weitere Informationen
in dem Buch von Levine, Mason und Brown \cite{levine92}, sowie in den Manual-Seiten und 
im Unix-Info-System.

\paragraph{Historisches}
Das Werkzeug \textsl{Lex}\/, das der Vorl\"aufer von \textsl{Flex}\/ ist, wurde 1975 vom Michael
E.~Lesk in den Bell Laboratories entwickelt \cite{lesk:1975}.
\pagebreak



%%% Local Variables: 
%%% mode: latex
%%% TeX-master: "formale-sprachen"
%%% End: 
