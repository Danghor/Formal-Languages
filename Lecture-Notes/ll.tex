\chapter{LL(k)-Sprachen}
In diesem Kapitel werden wir die Theorie vorstellen,  die 
Top-Down Parser-Generatoren wie beispielsweise \textsc{Antlr} zu Grunde liegt.  Es handelt
sich dabei um die Theorie der LL($k$)-Sprachen. 
Dabei steht das erste $L$ daf\"ur, dass der Parser die Eingabe von \underline{l}inks nach rechts parst,
das zweite $L$ steht daf\"ur, dass der Parser versucht, eine \underline{L}inks-Ableitung des
zu parsenden Wortes zu berechnen. Eine Links-Ableitung ist dabei eine Ableitung, bei der
immer die linkeste Variable ersetzt wird.
 Die Zahl $k$ in LL($k$) bedeutet, dass der Parser anhand der n\"achsten $k$ Token
entscheidet, welche Regel verwendet wird.  Falls beispielsweise $k=1$ ist, wird also nur das n\"achste 
Token zur Entscheidung herangezogen.  Dieses Token bezeichnen wir dann als \emph{Lookahead-Token}.
Wir betrachten zun\"achst den Fall $k=1$. 
Bevor wir die Theorie der LL(1)-Parser darstellen, machen wir auf zwei Probleme
aufmerksam, die wir bei der Erstellung von Top-Down-Parsern l\"osen m\"ussen.
\begin{enumerate}
\item Das erste Problem ist die Links-Rekursion, die beispielsweise in den folgenden
      Regeln zur Beschreibung arithmetischer Ausdr\"ucke auftritt:
      \begin{eqnarray*}
        \textsl{expr}    & \rightarrow & \;\textsl{expr} \quoted{+} \textsl{product}  \\
                         & \mid        & \;\textsl{expr} \quoted{-} \textsl{product}  \\
                         & \mid        & \;\textsl{product}                           
      \end{eqnarray*}
      Wir hatten im Abschnitt \ref{links-rekursion} gezeigt, wie die Links-Rekursion aus
      einer Grammatik eliminiert werden kann. 
      
      Zus\"atzlich haben wir gesehen, dass es bei der Verwendung von \textsc{Ebnf}-Grammatiken 
      oft in nat\"urlicher Weise m\"oglich ist, die Links-Rekursion durch die Verwendung der
      Postfix-Operatoren ``\texttt{*}'' und ``\texttt{+}'' zu vermeiden.  Beispielsweise
      k\"onnen wir die obigen Regeln f\"ur arithmetische Ausdr\"ucke zu der \textsc{Ebnf}-Regel
      \begin{eqnarray*} 
        \textsl{expr}    & \rightarrow & \textsl{product}\;\; \bigl((\quoted{+}|\quoted{-}) \;\;\textsl{product}\bigr)^* 
      \end{eqnarray*}
      umschreiben.
\item Das zweite Problem erkennen wir, wenn wir die folgende Grammatik-Regel f\"ur 
      Gleichungen und Ungleichungen betrachten:
      \begin{eqnarray*}
        \textsl{boolExpr} & \rightarrow & \textsl{expr} \quoted{==}    \textsl{expr} \\
                          & \mid        & \textsl{expr} \quoted{<}\;\, \textsl{expr} 
      \end{eqnarray*}
      Diese Grammatik-Regeln sind nicht links-rekursiv, aber
      f\"ur einen Top-Down-Parser, der mit nur einem Token Look-Ahead
      auskommen soll, ist die Frage, welche der beiden Regeln zum Parsen verwendet
      werden soll, offenbar nicht zu beantworten.  Wir stellen gleich ein Verfahren vor, mit dem sich
      Grammatiken so transformieren lassen, dass dieses Problem verschwindet.
\end{enumerate}

\section{Links-Faktorisierung}
Ist $\textsl{a}$ ein Nicht-Terminal
und gibt es zwei verschiedene Regeln, mit denen $\textsl{a}$ abgeleitet werden kann, beispielsweise
\\[0.2cm]
\hspace*{1.3cm}
$\textsl{a} \rightarrow \beta$ \quad und \quad $\textsl{a} \rightarrow \gamma$,
\\[0.2cm]
so muss es bei der Verwendung eines LL(1)-Parsers m\"oglich sein, anhand des Look-Ahead-Tokens zu
erkennen, welche Regel benutzt werden soll.  In der Praxis gibt es h\"aufig Situationen, wo
diese Voraussetzung nicht erf\"ullt ist.  Wir haben oben bereits ein solches Beispiel
gesehen.  Um das Beispiel zu vervollst\"andigen, ben\"otigen wir noch Regeln zur Ableitung von
\textsl{expr}.  Abbildung \ref{fig:Expr3} zeigt eine vollst\"andige Grammatik, mit der
Gleichungen und Ungleichungen beschrieben werden k\"onnen, die aus arithmetischen Ausdr\"ucken
aufgebaut sind.


\begin{figure}[htbp]
  \begin{center}    
  \framebox{
  \framebox{
  \begin{minipage}[t]{9cm}

  \begin{eqnarray*}
  \textsl{boolExpr}    & \rightarrow & \textsl{expr} \quoted{==}    \textsl{expr} \\
                       & \mid        & \textsl{expr} \quoted{!=}\;\, \textsl{expr} \\
                       & \mid        & \textsl{expr} \quoted{<=}\;\, \textsl{expr} \\
                       & \mid        & \textsl{expr} \quoted{>=}\;\, \textsl{expr} \\
                       & \mid        & \textsl{expr} \quoted{>}\;\, \textsl{expr} \\
                       & \mid        & \textsl{expr} \quoted{<}\;\, \textsl{expr} \\[0.2cm]
  \textsl{expr}        & \rightarrow & \;\textsl{product}\;\;\textsl{exprRest}            \\[0.2cm]
  \textsl{exprRest}    & \rightarrow & \quoted{+} \textsl{product}\;\;\textsl{exprRest}   \\
                       & \mid        & \quoted{-} \textsl{product}\;\;\textsl{exprRest}   \\
                       & \mid        & \;\varepsilon                                      \\[0.2cm]
  \textsl{product}     & \rightarrow & \;\textsl{factor}\;\;\textsl{productRest}          \\[0.2cm]
  \textsl{productRest} & \rightarrow & \quoted{*} \textsl{factor}\;\;\textsl{productRest} \\
                       & \mid        & \quoted{/} \textsl{factor}\;\;\textsl{productRest} \\
                       & \mid        & \;\varepsilon                                      \\[0.2cm]
  \textsl{factor}      & \rightarrow & \quoted{(} \textsl{expr} \quoted{)}                \\
                       & \mid        & \;\textsc{Number}                                  \\
                       & \mid        & \;\textsc{Identifier} 
  \end{eqnarray*}
  \vspace*{-0.5cm}

  \end{minipage}}}
  \end{center}
  \caption{Grammatik ohne Links-Rekursion f\"ur Gleichungen und Ungleichungen.}
  \label{fig:Expr3}
\end{figure}

\noindent
Es ist nicht m\"oglich einen LL(1)-Parser zu implementieren, der Boole'sche Ausdr\"ucke
mit Hilfe dieser Regeln  erkennen kann, denn alle Regeln f\"ur \textsl{boolExpr} beginnen
mit \textsl{expr}.  Wir k\"onnen die Grammatik aber durch \emph{Links-Faktorisierung}
(Englisch: \emph{left factoring})  
so umschreiben, dass ein Token als Look-Ahead ausreicht, indem wir den Teil aus den beiden 
Grammatik-Regeln ausklammern, der am Anfang der beiden Regeln identisch ist.  In dem
obigen Beispiel f\"uhren wir dann f\"ur den verbleibenden Rest das neue Nicht-Terminal
\textsl{boolExprRest} ein und erhalten so die Regeln
\begin{eqnarray*}
\textsl{boolExpr}     & \rightarrow & \;\textsl{expr} \;\;\textsl{boolExprRest} \\[0.2cm]
\textsl{boolExprRest} & \rightarrow & \quoted{==}    \textsl{expr} \quad \mid \quad \quoted{!=} \textsl{expr}
                                      \quad \mid \quad \quoted{<=} \textsl{expr}                                  \\
                      & \mid        & \quoted{>=}\;\, \textsl{expr} \quad \mid \quad \quoted{<} \textsl{expr} 
                                      \quad \mid \quad \quoted{>} \textsl{expr}. 
\end{eqnarray*}
Mit diesen Regeln reicht ein Token als Look-Ahead aus, denn die verschiedenen Alternativen f\"ur 
\mbox{\textsl{boolExprRest}} unterscheiden sich in dem ersten Token des Rumpfs der Regel.
Verwenden wir statt einer einfachen Grammatik eine \textsc{Ebnf}-Grammatik, so lassen sich die 
obigen Regeln k\"urzer und klarer in der Form
\begin{eqnarray*}
\textsl{boolExpr} & \rightarrow & \textsl{expr}\;\;
                                  (\quoted{==}\mid \quoted{!=}\mid \quoted{<=} \mid \quoted{>=} \mid \quoted{<})  
                                  \;\;\textsl{expr} 
\end{eqnarray*}
darstellen.

Um den allgemeinen Fall der Links-Faktorisierung diskutieren zu k\"onnen, nehmen wir an, dass
$\textsl{a}$ ein Nicht-Terminal ist, das durch insgesamt $m + n$ Regeln definiert wird, wobei der
Rumpf der ersten $m$ Regeln immer mit $\alpha$ anf\"angt, wobei $\alpha$ ein String aus
Terminalen und syntaktischen Variablen ist.  Die Regeln haben also die folgende Form:
\begin{eqnarray*}
  \textsl{a} & \rightarrow & \alpha \;\; \beta_1 \\
    & \mid        & \alpha \;\; \beta_2 \\
    & \vdots      & \vdots              \\
    & \mid        & \alpha \;\; \beta_m \\
    & \mid        & \gamma_1            \\
    & \vdots      & \vdots              \\
    & \mid        & \gamma_n            
\end{eqnarray*}
Bei dieser Darstellung sei vorausgesetzt, dass die Strings $\beta_1$, $\cdots$, $\beta_m$
keinen  Pr\"afix haben, der allen $\beta_i$ gemeinsam ist und dass $\alpha$ kein Pr\"afix einer der Strings
$\gamma_1, \cdots, \gamma_n$ ist.  Bei der Links-Faktorisierung dieser Regeln klammern wir einerseits den
gemeinsamen Pr\"afix $\alpha$ aus und f\"uhren andererseits eine neue syntaktische Variable $\textsl{b}$ ein, die
den auf $\alpha$ folgenden Rest bezeichnet.  Wir erhalten dann die folgenden Regeln:
\[
\begin{array}{lclclcl}
  \textsl{a} & \rightarrow & \alpha \;\; \textsl{b} & \qquad & \textsl{b} & \rightarrow & \beta_1   \\
    & \mid        & \gamma_1      &        &   & \mid        & \beta_2   \\
    & \vdots      & \vdots        &        &   & \vdots      & \vdots    \\
    & \mid        & \gamma_n      &        &   & \mid        & \beta_m   \\[0.2cm]
\end{array}
\]
Um alle gemeinsamen Pr\"afixe auszuklammern muss dieses Verfahren unter 
Umst\"anden mehrfach durchgef\"uhrt werden.  Die n\"achste Aufgabe gibt daf\"ur ein Beispiel.

\exercise
Geben Sie eine Links-Faktorisierung f\"ur die folgenden Grammatik-Regeln an.
\begin{eqnarray*}
  \textsl{a} & \rightarrow & \quoted{A} \quoted{B} \textsl{u} \quoted{D} \\
             & \mid        & \quoted{A} \textsl{v} \quoted{B} \quoted{D} \\
             & \mid        & \quoted{A} \quoted{B} \textsl{w}            \\
             & \mid        & \quoted{X} \textsl{u}                       \\
             & \mid        & \quoted{X} \textsl{v}                       
\end{eqnarray*}

\solution
Zun\"achst eliminieren wir das gemeinsame Pr\"afix \qote{A} und f\"uhren dazu die neue
syntaktische Variable $\textsl{b}$ ein.  Wir erhalten:
\begin{eqnarray*}
  \textsl{a} & \rightarrow & \quoted{A} \textsl{b}             \\
    & \mid        & \quoted{X} \textsl{u}             \\
    & \mid        & \quoted{X} \textsl{v}             \\[0.2cm]
  \textsl{b} & \rightarrow & \quoted{B} \textsl{u} \quoted{D}  \\
    & \mid        & \;\textsl{v} \quoted{B} \quoted{D}\\
    & \mid        & \quoted{B} \textsl{w}             
\end{eqnarray*}
Nun eliminieren wir das Pr\"afix \qote{X} aus beiden letzten Regeln f\"ur $\textsl{a}$.  Wir f\"uhren
dazu die neue syntaktische Variable $\textsl{c}$ ein. Dann erhalten wir:
\begin{eqnarray*}
  \textsl{a} & \rightarrow & \quoted{A} \textsl{b}              \\
    & \mid        & \quoted{X} \textsl{c}              \\[0.2cm]
  \textsl{c} & \rightarrow & \textsl{u}                         \\
    & \mid        & \textsl{v}                         \\[0.2cm]
  \textsl{b} & \rightarrow & \quoted{B} \textsl{u} \quoted{D}   \\
    & \mid        & \;\textsl{v} \quoted{B} \quoted{D} \\
    & \mid        & \quoted{B} \textsl{w}              
\end{eqnarray*}
Als letztes eliminieren wir das Pr\"afix \qote{B}, das in zwei der Regeln f\"ur die
syntaktische Variable $\textsl{b}$ auftritt.  Wir nennen die neu eingef\"uhrte Variable $\textsl{d}$ und erhalten:
\begin{eqnarray*}
  \textsl{a} & \rightarrow & \quoted{A} \textsl{b}              \\
    & \mid        & \quoted{X} \textsl{c}              \\[0.2cm]
  \textsl{c} & \rightarrow & \textsl{u}                         \\
    & \mid        & \textsl{v}                         \\[0.2cm]
  \textsl{b} & \rightarrow & \quoted{B} \textsl{d}              \\
    & \mid        & \;\textsl{v} \quoted{B} \quoted{D} \\[0.2cm]
  \textsl{d} & \rightarrow & \textsl{u} \quoted{D}              \\
    & \mid        & \textsl{w}              \hspace*{11cm} _\Box
\end{eqnarray*}
\vspace*{0.3cm}

\noindent
\textbf{Remark}: The parser generator \textsc{Antlr} performs left-factorisation automatically.


\section{\textsl{First} und \textsl{Follow}}
Nicht f\"ur jede links-faktorisierte Grammatik l\"asst sich ein LL(1)-Parser bauen.  Betrachten
wir die folgenden Regeln:
\begin{eqnarray*}
  \textsl{a} & \rightarrow & \textsl{b} \; \mid\; \textsl{c} \\
  \textsl{b} & \rightarrow & \quoted{A} \textsl{u}  \\
  \textsl{c} & \rightarrow & \quoted{A} \textsl{v}  
\end{eqnarray*}
Will der Parser ein $\textsl{a}$ parsen und ist das n\"achste Token ein \qote{A}, so ist nicht klar,
ob der Parser als n\"achstes die Regel
\\[0.2cm]
\hspace*{1.3cm}
$\textsl{a} \rightarrow \textsl{b}$ \quad oder \quad $\textsl{a} \rightarrow \textsl{c}$
\\[0.2cm]
verwenden soll.  F\"ur die obige Grammatik l\"asst sich daher kein LL(1)-Parser implementieren.
Zur Entscheidung, ob sich f\"ur eine gegebene Grammatik ein LL(1)-Parser implementieren l\"asst,
ben\"otigen wir die Funktionen $\textsl{First}()$ und $\textsl{Follow}()$, die wir gleich
definieren werden.  Um diese Funktionen implementieren zu k\"onnen, definieren wir vorher  den
Begriff einer $\varepsilon$-erzeugenden syntaktischen Variablen.

\begin{Definition}[$\varepsilon$-erzeugend]
Es sei $G = \langle V, T, R, S \rangle$ eine kontextfreie Grammatik und $\textsl{a}$ sei eine
syntaktische Variable, also $\textsl{a} \in V$.  Die Variable $\textsl{a}$ hei{\ss}t 
\emph{$\varepsilon$-erzeugend} genau dann, wenn
\\[0.2cm]
\hspace*{1.3cm}
$\textsl{a} \Rightarrow^* \varepsilon$
\\[0.2cm]
gilt, also dann, wenn sich aus der Variablen $\textsl{a}$ das leere Wort ableiten l\"asst. 
Wir schreiben $\textsl{nullable}(\textsl{a})$ wenn die Variable $\textsl{a}$ als $\varepsilon$-erzeugend
nachgewiesen ist.
\eox
\end{Definition}

\examples
\begin{enumerate}
\item Bei der in Abbildung \ref{fig:Expr3} auf Seite \pageref{fig:Expr3} gezeigten Grammatik
      sind offenbar die Variablen \textsl{exprRest} und \textsl{productRest} $\varepsilon$-erzeugend.
\item Wir betrachten nun ein weniger offensichtliches Beispiel.  Die Grammatik $G$
      enthalte die folgenden Regeln:
      \\[0.2cm]
      \hspace*{1.3cm}
      $S \rightarrow \textsl{a} \; \textsl{b} \; \textsl{c}$
      \\[0.2cm]
      \hspace*{1.3cm}
      $\textsl{a} \rightarrow \quoted{X} \textsl{b} \mid \textsl{a} \quoted{Y} \mid \textsl{b}\;\textsl{c}$
      \\[0.2cm]
      \hspace*{1.3cm}
      $\textsl{b} \rightarrow \quoted{X} \textsl{b} \mid \textsl{a} \quoted{Y} \mid \textsl{c}\;\textsl{c}$
      \\[0.2cm]
      \hspace*{1.3cm}
      $\textsl{c} \rightarrow \textsl{a}\;\textsl{b}\; \textsl{c} \mid \varepsilon$
      \\[0.2cm]
      Zun\"achst ist offenbar die Variable $\textsl{c}$ $\varepsilon$-erzeugend.  Dann sehen wir,
      dass aufgrund der Regel $\textsl{b} \rightarrow \textsl{c} \;\textsl{c}$ auch $\textsl{b}$ $\varepsilon$-erzeugend ist
      und daraus folgt wegen der Regel $\textsl{a} \rightarrow \textsl{b}\;\textsl{c}$, dass auch $\textsl{a}$
      $\varepsilon$-erzeugend ist.  Schlie{\ss}lich erkennen wir $S$ als $\varepsilon$-erzeugend,
      denn die erste Regel lautet
      \\[0.2cm]
      \hspace*{1.3cm}
      $S \rightarrow \textsl{a} \; \textsl{b} \; \textsl{c}$
      \\[0.2cm]
      und hier sind alle Variablen auf der rechten Seite der Regel bereits als
      $\varepsilon$-erzeugende Variablen nachgewiesen worden.
\end{enumerate}
 
\begin{Definition}[$\textsl{First}()$]
Es sei $G = \langle V, T, R, S \rangle$ eine kontextfreie Grammatik und $\textsl{a} \in V$.
Dann definieren wir $\textsl{First}(\textsl{a})$ als die Menge aller der Token $t$, mit denen ein
von $\textsl{a}$ abgeleitetes Wort beginnen kann:
\\[0.2cm]
\hspace*{1.3cm}
$\textsl{First}(\textsl{a}) := \{ t \in T \mid \exists \gamma \in (V \cup T)^*: \textsl{a} \Rightarrow^* t\gamma \}$.
\\[0.2cm]
Die Definition der Funktion $\textsl{First}()$ kann wie folgt auf Strings aus $(V \cup T)^*$ 
erweitert werden: 
\begin{enumerate}
\item $\textsl{First}(\varepsilon) = \{\}$.
\item $\textsl{First}(t \beta) = \{ t \}$ \quad if $t \in T$.
\item $\textsl{First}(\textsl{a} \beta) = \left\{
       \begin{array}[c]{ll}
         \textsl{First}(\textsl{a}) \cup \textsl{First}(\beta) & \mbox{if $\textsl{a} \Rightarrow^* \varepsilon$;} \\
         \textsl{First}(\textsl{a})                            & \mbox{otherwise.}
       \end{array}
       \right.
      $ 
\end{enumerate}
If $\textsl{a}$ is a variable of $G$ and the rules defining $\textsl{a}$ are given as 
\\[0.2cm]
\hspace*{1.3cm}
$\textsl{a} \rightarrow \alpha_1 \mid \cdots \mid \alpha_n$,
\\[0.2cm]
then we have
\\[0.2cm]
\hspace*{1.3cm}
$\textsl{First}(\textsl{a}) = \bigcup\limits_{i=1}^n \textsl{First}(\alpha_i)$.   \eox
\end{Definition}

\remarkEng
Note that the definitions of the function $\textsl{First}(\textsl{a})$ for variables
$\textsl{a} \in V$ and the function $\textsl{First}(\alpha)$ for strings $\alpha \in (V \cup T)^*$
are mutually recursive.  The computation of $\textsl{First}(\textsl{a})$ is best done via a 
fixpoint computation:  Start by setting $\textsl{First}(\textsl{a}) := \{\}$ for all variables $\textsl{a}\in V$ and
then continue to iterate the equations defining $\textsl{First}(\textsl{a})$ until none of the sets
$\textsl{First}(\textsl{a})$ changes any more.  The next example clarifies this idea.

\example
Wir k\"onnen f\"ur die Variablen $\textsl{a}$ der in Abbildung \ref{fig:Expr3} gezeigten Grammatik 
die Mengen $\textsl{First}(\textsl{a})$ iterativ berechnen.  Wir berechnen
die Funktion $\textsl{First}(\textsl{a})$ f\"ur die einzelnen Variablen $\textsl{a}$ am besten so, dass wir mit den
Variablen beginnen, die in der Hierarchie ganz unten stehen. 
\begin{enumerate}
\item Zun\"achst folgt aus den Regeln
      \\[0.2cm]
      \hspace*{1.3cm}
      $\textsl{factor} \rightarrow \quoted{(} \textsl{expr} \quoted{)} \mid \textsc{Number} \mid \textsc{Identifier}$,
      \\[0.2cm]
      dass jeder von \textsl{Factor} abgeleitete String entweder mit einer \"offnenden
      Klammer, einer Zahl oder einem Bezeichner beginnt:
      \\[0.2cm]
      \hspace*{1.3cm}
      $\textsl{First}(\textsl{factor}) = \{ \quoted{(}, \textsc{Number}, \textsc{Identifier}\; \}$.
\item Analog folgt aus den Regeln 
      \\[0.2cm]
      \hspace*{1.3cm}
      $\textsl{productRest} \rightarrow \quoted{*} \textsl{factor}\;\;\textsl{productRest} \;
                            \mid        \quoted{/} \textsl{factor}\;\;\textsl{productRest} \;
                            \mid        \varepsilon$,
      \\[0.2cm]
      dass ein \textsl{productRest} entweder mit dem Zeichen \qote{*} oder \qote{/} beginnt:
      \\[0.2cm]
      \hspace*{1.3cm}
      $\textsl{First}(\textsl{productRest}) = \{ \quoted{*}, \quoted{/} \}$
\item Die Regel f\"ur die Variable \textsl{product} lautet
      \\[0.2cm]
      \hspace*{1.3cm}
      $\textsl{product} \rightarrow \textsl{factor}\;\;\textsl{productRest}$.
      \\[0.2cm]
      Da die Variable \textsl{factor} nicht $\varepsilon$ erzeugend ist, sehen wir, dass
      die Menge $\textsl{First}(\textsl{product})$ mit der Menge
      $\textsl{First}(\textsl{factor})$ \"ubereinstimmt:
      \\[0.2cm]
      \hspace*{1.3cm}
      $\textsl{First}(\textsl{product}) = \{ \quoted{(}, \textsc{Number}, \textsc{Identifier}\; \}$.
\item Aus den Regeln
      \\[0.2cm]
      \hspace*{1.3cm}
      $\textsl{exprRest} \rightarrow \quoted{+} \textsl{product}\;\;\textsl{exprRest} 
                         \mid        \quoted{-} \textsl{product}\;\;\textsl{exprRest} 
                         \mid        \varepsilon$
      \\[0.2cm]
      k\"onnen wir $\textsl{First}(\textsl{exprRest})$ wie folgt berechnen:
      \\[0.2cm]
      \hspace*{1.3cm}
      $\textsl{First}(\textsl{exprRest}) = \{ \quoted{+}, \quoted{-} \}$.
\item Weiter folgt aus der Regel
      \\[0.2cm]
      \hspace*{1.3cm}
      $\textsl{expr} \rightarrow \textsl{product}\;\;\textsl{exprRest}$
      \\[0.2cm]
      und der Tatsache, dass $\textsl{product}$ nicht $\varepsilon$-erzeugend ist,
      dass die Menge $\textsl{First}(\textsl{expr})$ mit der Menge
      $\textsl{First}(\textsl{product})$ \"ubereinstimmt:
      \\[0.2cm]
      \hspace*{1.3cm}
      $\textsl{First}(\textsl{expr}) = \{ \quoted{(}, \textsc{Number}, \textsc{Identifier}\; \}$.
\item Schlie{\ss}lich folgt aus den Regeln f\"ur die syntaktische Variable \textsl{boolExpr}
      sowie der Tatsache, dass die syntaktische Variable \textsl{expr} nicht $\varepsilon$-erzeugend ist,
      dass $\textsl{First}(\textsl{boolExpr})$ mit $\textsl{First}(\textsl{expr})$ identisch ist:
      \\[0.2cm]
      \hspace*{1.3cm}
      $\textsl{First}(\textsl{boolExpr}) = \{ \quoted{(}, \textsc{Number}, \textsc{Identifier}\; \}$.
\end{enumerate}
Since we have computed the sets $\textsl{First}(\textsl{a})$ in a clever order, we did not have to perform a
proper fixpoint iteration in this example.
\eox


\begin{Definition}[$\textsl{Follow}()$]
Es sei $G = \langle V, T, R, S \rangle$ eine kontextfreie Grammatik und $\textsl{a} \in V$.
Bei der Berechnung von $\textsl{Follow}()$ wird die Grammatik zun\"achst abge\"andert,
indem wir das Symbol \qote{\symbol{36}} als neues Symbol zu der Menge $T$ der Terminale
hinzuf\"ugen.  Zu den Variablen wird das neue Symbol $\widehat{S}$ hinzugef\"ugt, das auch
gleichzeitig das neue Start-Symbol der Grammatik ist.  Zu der Menge $R$ der Regeln
f\"ugen wir die folgende Regel neu hinzu:
\\[0.2cm]
\hspace*{1.3cm}
$\widehat{S} \rightarrow S \quoted{\symbol{36}}$.
\\[0.2cm]
Das Terminal \qote{\symbol{36}} steht hierbei f\"ur das Ende der Eingabe (\textsc{Eof},
\emph{end of file}).
Weiter definieren wir
\\[0.2cm]
\hspace*{1.3cm}
 $\widehat{T} := T \cup \{ \quoted{\symbol{36}} \}$.
\\[0.2cm]
Die so ver\"anderte Grammatik bezeichnen wir als die \emph{augmentierte} Grammatik.
Dann definieren wir $\textsl{Follow}(\textsl{a})$ als die Menge aller der Token $t$, die in einer
Ableitung auf $\textsl{a}$ folgen k\"onnen:
\\[0.2cm]
\hspace*{1.3cm}
$\textsl{Follow}(\textsl{a}) := 
 \{ t \in \widehat{T} \mid \exists \beta,\gamma \in (V \cup \widehat{T})^*: 
                           \widehat{S} \Rightarrow^* \beta \textsl{a} t \gamma 
  \}
$.
\\[0.2cm]
Wenn sich aus dem Start-Symbol $\widehat{S}$ also irgendwie ein String $\beta \textsl{a} t\gamma$ ableiten l\"asst,
bei dem das Token $t$ auf die Variable $\textsl{a}$ folgt, dann ist $t$ ein Element
der Menge $\textsl{Follow}(\textsl{a})$.
\eox
\end{Definition}

\example
Wir untersuchen wieder die in Abbildung \ref{fig:Expr3} gezeigte Grammatik f\"ur arithmetische Ausdr\"ucke.
\begin{enumerate}
\item Aufgrund der neu hinzugef\"ugten Regel
      \\[0.2cm]
      \hspace*{1.3cm}
      $\widehat{S} \rightarrow \textsl{boolExpr} \quoted{\symbol{36}}$
      \\[0.2cm]
      muss die Menge $\textsl{Follow}(\textsl{boolExpr})$ das Zeichen \qote{\symbol{36}}
      enthalten.  Da die syntaktische Variable \textsl{boolExpr} sonst nirgends in der
      Grammatik vorkommt, haben wir 
      \\[0.2cm]
      \hspace*{1.3cm}
      $\textsl{Follow}(\textsl{boolExpr}) = \{ \quoted{\symbol{36}} \}$.
\item Die Grammatik-Regeln f\"ur die syntaktische Variable \textsl{boolExpr}
      zeigen uns zun\"achst, dass die Menge $\textsl{Follow}(\textsl{expr})$ die Zeichen
      \qote{==}, \qote{!=}, \qote{<=}, \qote{>=}, \qote{<}   und \qote{>} enth\"alt.  Da \textsl{expr} auch am Ende dieser Regeln steht,
      folgt weiter, dass alle Elemente aus $\textsl{Follow}(\textsl{boolExpr})$ auch auf 
      \textsl{expr} folgen k\"onnen, wir haben also auch
      \\[0.2cm]
      \hspace*{1.3cm}
      $\quoted{\symbol{36}} \in \textsl{Follow}(\textsl{expr})$.
      \\[0.2cm]
      Aufgrund der Regel 
      \\[0.2cm]
      \hspace*{1.3cm}
      $\textsl{factor} \rightarrow \quoted{(} \textsl{expr} \quoted{)}$
      \\[0.2cm]
      muss die Menge $\textsl{Follow}(\textsl{expr})$ au{\ss}erdem das Zeichen \qote{)}
      enthalten.  Also haben wir insgesamt
      \\[0.2cm]
      \hspace*{1.3cm}
      $\textsl{Follow}(\textsl{expr}) = \{ \quoted{==}, \quoted{!=}, \quoted{>=}, \quoted{<=}, \quoted{>}, \quoted{<}, \quoted{\symbol{36}}, \quoted{)} \}$.
\item Aufgrund der Regel 
      \\[0.2cm]
      \hspace*{1.3cm}
      $\textsl{expr} \rightarrow \textsl{product}\;\;\textsl{exprRest}$
      \\[0.2cm]      
      wissen wir, dass alle Terminale, die auf ein \textsl{expr} folgen k\"onnen, auch auf
      ein \textsl{exprRest} folgen k\"onnen, womit wir schon mal wissen, dass
      $\textsl{Follow}(\textsl{exprRest})$ die Token \qote{==}, \qote{!=}, \qote{<=}, \qote{>=}, \qote{<}, \qote{\symbol{36}}  und \qote{)}
      enth\"alt.   Da \textsl{exprRest} sonst nur am Ende der Regeln vorkommt, die
      \textsl{exprRest} definieren, sind das auch schon alle Token, die auf
      \textsl{exprRest} folgen k\"onnen und wir haben
      \\[0.2cm]
      \hspace*{1.3cm}
      $\textsl{Follow}(\textsl{exprRest}) = 
       \{ \quoted{==}, \quoted{!=}, \quoted{>=}, \quoted{<=}, \quoted{>}, \quoted{<}, \quoted{\symbol{36}}, \quoted{)} \}$.
\item Die Regeln 
      \\[0.2cm]
      \hspace*{1.3cm}
      $\textsl{exprRest} \rightarrow \quoted{+} \textsl{product}\;\;\textsl{exprRest} 
                         \mid        \quoted{-} \textsl{product}\;\;\textsl{exprRest}$
      \\[0.2cm]
      zeigen, dass auf ein \textsl{product} alle Elemente aus $\textsl{First}(\textsl{exprRest})$
      folgen k\"onnen, aber das ist noch nicht alles:  Da die Variable \textsl{exprRest}
      $\varepsilon$-erzeugend ist, k\"onnen zus\"atzlich auf \textsl{product} auch
      alle Token folgen, die auf \textsl{exprRest} folgen.  Damit haben wir insgesamt
      \\[0.2cm]
      \hspace*{1.3cm}
      $\textsl{Follow}(\textsl{product}) = 
      \{ \quoted{+}, \quoted{-},  \quoted{==}, \quoted{!=}, \quoted{>=}, \quoted{<=}, \quoted{>}, \quoted{<}, \quoted{\symbol{36}}, \quoted{)} \}$.
\item Die Regel
      \\[0.2cm]
      \hspace*{1.3cm}
      $\textsl{product} \rightarrow \textsl{factor}\;\;\textsl{productRest}$
      \\[0.2cm]      
      zeigt, dass alle Terminale, die auf ein \textsl{product} folgen k\"onnen, auch auf
      ein \textsl{productRest} folgen k\"onnen.
      Da \textsl{productRest} sonst nur am Ende der Regeln vorkommt, die
      \textsl{productRest} definieren, sind das auch schon alle Token, die auf
      \textsl{productRest} folgen k\"onnen und wir haben insgesamt
      \\[0.2cm]
      \hspace*{1.3cm}
      $\textsl{Follow}(\textsl{productRest}) = 
      \{ \quoted{+}, \quoted{-}, \quoted{==}, \quoted{!=}, \quoted{>=}, \quoted{<=}, \quoted{>}, \quoted{<}, \quoted{\symbol{36}}, \quoted{)} \}$.
\item Die Regeln 
      \\[0.2cm]
      \hspace*{1.3cm}
      $\textsl{productRest} \rightarrow \quoted{*} \textsl{factor}\;\;\textsl{productRest} 
                            \mid        \quoted{/} \textsl{factor}\;\;\textsl{productRest}$ 
      \\[0.2cm]
      zeigen, dass auf ein \textsl{factor} alle Elemente aus $\textsl{First}(\textsl{productRest})$
      folgen k\"onnen, aber das ist noch nicht alles:  Da die Variable \textsl{productRest}
      $\varepsilon$-erzeugend ist, k\"onnen zus\"atzlich auf \textsl{factor} auch
      alle Token folgen, die auf \textsl{productRest} folgen.  Damit haben wir insgesamt
      \\[0.2cm]
      \hspace*{1.3cm}
      $\textsl{Follow}(\textsl{factor}) = 
      \{ \quoted{*}, \quoted{/}, \quoted{+}, \quoted{-}, \quoted{==}, \quoted{!=}, \quoted{>=}, \quoted{<=}, \quoted{>}, \quoted{<}, \quoted{\symbol{36}}, \quoted{)} \}$.
      \qed
\end{enumerate}

\noindent
Das letzte Beispiel zeigt, dass die Berechnung des Pr\"adikats $\textsl{nullable}()$ 
und die Berechnung der Mengen $\textsl{First}(\textsl{a})$ und $\textsl{Follow}(\textsl{a})$ f\"ur eine syntaktische
Variable $\textsl{a}$ eng miteinander verbunden sind.  
Es sei
\\[0.2cm]
\hspace*{1.3cm}
 $\textsl{a} \rightarrow Y_1 Y_2 \cdots Y_k$ 
\\[0.2cm]
eine Grammatik-Regel.
Dann bestehen zwischen dem Pr\"adikat $\texttt{nullable}()$ und den beiden Funktionen
$\textsl{First}()$ und $\textsl{Follow}()$ die folgenden Beziehungen:
\begin{enumerate}
\item $\forall t \in T: \neg\, \textsl{nullable}(t)$.
\item $k = 0 \Rightarrow \textsl{nullable}(\textsl{a})$.
\item $\bigl(\forall i \in \{1, \cdots, k\}: \textsl{nullable}(Y_i)\bigr) \Rightarrow
       \textsl{nullable}(\textsl{a})$.

      Setzen wir hier $k=0$ so sehen wir, dass 2.~ein Spezialfall von 3.~ist.
\item $\textsl{First}(Y_1) \subseteq \textsl{First}(\textsl{a})$.
\item $\bigl(\forall j \in \{1,\cdots,i-1\}: \textsl{nullable}(Y_j)\bigr) \Rightarrow
       \textsl{First}(Y_i) \subseteq \textsl{First}(\textsl{a})$.

       Setzen wir oben $i=1$, so sehen wir, dass 4.~ein Spezialfall von 5.~ist.
\item $\textsl{Follow}(\textsl{a}) \subseteq \textsl{Follow}(Y_k)$.
\item $\bigl(\forall j \in \{i+1, \cdots, k\}: \textsl{nullable}(Y_j)\bigr) \Rightarrow 
       \textsl{Follow}(\textsl{a}) \subseteq \textsl{Follow}(Y_i)$.

      Setzen wir hier $i=k$ so sehen wir, dass 6.~ein Spezialfall von 7.~ist.
\item $\forall i \in \{1,\cdots,k-1\}:\textsl{First}(Y_{i+1}) \subseteq \textsl{Follow}(Y_i)$.
\item $\bigl(\forall j \in \{i+1, \cdots, l-1\}: \textsl{nullable}(Y_j)\bigr) \Rightarrow 
       \textsl{First}(Y_l) \subseteq \textsl{Follow}(Y_i)$.

      Setzen wir hier $l=i+1$ so sehen wir, dass 8.~ein Spezialfall von 9.~ist.
\end{enumerate}
Mit Hilfe dieser Beziehungen k\"onnen $\textsl{nullable}()$, $\textsl{First}()$ und
$\textsl{Follow}()$ iterativ \"uber eine Fixpunkt-Iteration berechnet werden:  
\begin{enumerate}
\item Zun\"achst werden die Funktionen $\textsl{First}(\textsl{a})$ und
      $\textsl{Follow}(\textsl{a})$ f\"ur jede syntaktische Variable $\textsl{a}$ mit der leeren Menge initialisiert.
      Das Pr\"adikat $\textsl{nullable}(\textsl{a})$ wird f\"ur jede syntaktische Variable auf $\texttt{false}$
      gesetzt.
\item Anschlie{\ss}end werden die oben angegebenen Regeln so lange angewendet, wie sich durch die
      Anwendung \"Anderungen ergeben. 
\end{enumerate}

\section{LL(1)-Grammatiken}
Wir k\"onnen nun die Frage beantworten, f\"ur welche Grammatiken ein Top-Down-Parser erzeugt
werden kann, der immer mit einem Token Lookahead auskommt.  

\begin{Definition}[LL(1)-Grammatik]
Eine Grammatik $G$ ist eine \emph{LL(1)-Grammatik} genau dann, wenn f\"ur jede syntaktische Variable
$\textsl{a}$, f\"ur die es in der Grammatik $G$ zwei verschiedene Regeln
\\[0.2cm]
\hspace*{1.3cm}
$\textsl{a} \rightarrow \alpha$ \quad und \quad $\textsl{a} \rightarrow \beta$ 
\\[0.2cm]
gibt, die folgenden Bedingungen erf\"ullt sind:
\begin{enumerate}
\item $\neg( \alpha \Rightarrow^* \varepsilon \wedge \beta \Rightarrow^* \varepsilon)$.

      Die R\"umpfe zweier verschiedener Regeln derselben Variablen
      d\"urfen nicht beide das leere Wort ableiten.
\item $\textsl{First}(\alpha) \cap \textsl{First}(\beta) = \{\}$.

      Die Ableitungen der R\"umpfe zweier verschiedener Regeln derselben Variablen
      d\"urfen nicht mit dem selben Token beginnen.
\item   $(\beta  \Rightarrow^* \varepsilon) \rightarrow \textsl{First}(\alpha) \cap
\textsl{Follow}(\textsl{a}) = \{\}$.

      Wenn $\beta$ den leeren String ableitet, dann m\"ussen die Mengen
      $\textsl{First}(\alpha)$ und $\textsl{Follow}(\textsl{a})$ disjunkt sein.  
      \eox
\end{enumerate}
\end{Definition}

\noindent
Wir diskutieren nun die Idee, die hinter der obigen Definition steht.
\begin{enumerate}
\item Falls das leere Wort sowohl \"uber die Regel
      \\[0.2cm]
      \hspace*{1.3cm}
      $\textsl{a} \rightarrow \alpha$ \quad als auch \"uber \quad $\textsl{a} \rightarrow \beta$
      \\[0.2cm]
      ableitbar w\"are, so wissen wir nicht, welche Regel wir anwenden sollen, wenn
      wir ein $\textsl{a}$ ableiten sollen und das n\"achste Eingabe-Token ein Element der Menge
      $\textsl{Follow}(\textsl{a})$ ist.
\item Um ein $\textsl{a}$ zu parsen und zwischen den beiden Regeln f\"ur $\textsl{a}$ unterscheiden zu k\"onnen,
      verwenden wir das folgende Rezept:

      Parsen wir ein $\textsl{a}$ und ist das Lookahead-Token ein Element der Menge
      $\textsl{First}(\alpha)$, so verwenden wir die Regel
      \\[0.2cm]
      \hspace*{1.3cm}
      $\textsl{a} \rightarrow \alpha$.
      \\[0.2cm]
      Analog verwenden wir die Regel
      \\[0.2cm]
      \hspace*{1.3cm}
      $\textsl{a} \rightarrow \beta$,
      \\[0.2cm]
      wenn das Lookahead-Token ein Element der Menge $\textsl{First}(\beta)$ ist.

      Dieses Rezept funktioniert nat\"urlich nur, wenn die Mengen $\textsl{First}(\alpha)$
      und $\textsl{First}(\beta)$ disjunkt sind.
\item Das obige Rezept um ein $\textsl{a}$ zu parsen muss in dem Fall, dass $\beta$ das leere Wort
      ableitet, wie folgt erweitert werden.
  
      Gilt $\beta \Rightarrow^* \varepsilon$ und ist das Lookahead-Token ein Element der
      Menge $\textsl{Follow}(\textsl{a})$, so verwenden wir die Regel
      \\[0.2cm]
      \hspace*{1.3cm}
      $\textsl{a} \rightarrow \beta$.
     
      Damit diese Regel nicht im Widerspruch zu den unter Punkt 2.~genannten Regeln steht,
      ben\"otigen wir die Bedingungen
      \\[0.2cm]
      \hspace*{1.3cm}
      $(\beta  \Rightarrow^* \varepsilon) \rightarrow \textsl{First}(\alpha) \cap \textsl{Follow}(\textsl{a}) = \{\}.$ 
\end{enumerate}
Insgesamt versuchen wir also dann mit einer Regel $\textsl{a} \rightarrow \alpha$ zu reduzieren,
wenn eine der beiden folgenden Bedingungen erf\"ullt ist.  In diesen Bedingungen bezeichnet
\textsl{lat} das Lookahead-Token.
\begin{enumerate}
\item $\textsl{lat} \in \textsl{First}(\alpha)$ \quad oder
\item $\alpha \Rightarrow^* \varepsilon$ und $\textsl{lat} \in \textsl{Follow}(\alpha)$.
\end{enumerate}

\remark
Falls eine Grammatik $G$ links-rekursiv ist, dann kann $G$ keine LL(1)-Grammatik sein.


\subsection{Berechnung der Parse-Tabelle}
Nach diesen Vorbereitungen k\"onnen wir nun zu einer LL(1)-Grammatik die \emph{Parse-Tabelle}
berechnen.  F\"ur eine Grammatik $G = \langle V, T, R, S \rangle$ ist die Parse-Tabelle
\\[0.2cm]
\hspace*{1.3cm}
$\textsl{parseTable}: V \times T \rightarrow 2^R$,
\\[0.2cm]
eine Funktion, so dass der Aufruf $\textsl{parseTable}(\textsl{a},t)$ einer syntaktischen Variable $\textsl{a}$ und
einem Token $t$ die Menge aller Regeln der Form  
\\[0.2cm]
\hspace*{1.3cm}
$\textsl{a} \rightarrow \alpha$
\\[0.2cm]
zuordnet, die bei einer Ableitung von $\textsl{a}$ in Frage kommen, wenn das n\"achste zu lesenden Token den
Wert $t$ hat.  Diese Funktion gen\"ugt den folgenden beiden Bedingungen:
\begin{enumerate}
\item Ist  $\textsl{a} \rightarrow \alpha$ eine Regel der Grammatik und ist $t$ ein Token aus der Menge 
      $\textsl{First}(\alpha)$, dann ist diese Regel ein Element der Menge
      $\textsl{parseTable}(\textsl{a},t)$:
      \\[0.2cm]
      \hspace*{1.3cm}
      $(\textsl{a} \rightarrow \alpha) \in R \; \wedge \; t \in \textsl{First}(\alpha) 
      \;\Rightarrow\; (\textsl{a} \rightarrow \alpha) \in \textsl{parseTable}(\textsl{a},t)$.
\item Ist $\textsl{a} \rightarrow \alpha$ eine Regel der Grammatik, wobei $\alpha$ 
      $\varepsilon$-erzeugend ist, und ist $t$ ein Token aus der Menge 
      $\textsl{Follow}(\textsl{a})$, dann ist diese Regel ein Element der Menge
      $\textsl{parseTable}(\textsl{a},t)$:
      \\[0.2cm]
      \hspace*{1.3cm}
      $(\textsl{a} \rightarrow \alpha) \in R \; \wedge \; \alpha \Rightarrow^* \varepsilon 
       \;\wedge\;t \in \textsl{Follow}(\textsl{a}) 
       \;\Rightarrow\; (\textsl{a} \rightarrow \alpha) \in \textsl{parseTable}(\textsl{a},t)$.
\end{enumerate}
Eine Grammatik ist genau dann eine $LL(1)$-Grammatik, wenn die Menge
$\textsl{parseTable}(\textsl{a},t)$ f\"ur jede syntaktische Variable $\textsl{a}$ und jedes Token $t$ maximal
eine Regel enth\"alt:
\\[0.2cm]
\hspace*{1.3cm} $G$ ist LL(1) \quad g.d.w. \quad  
$\forall \textsl{a} \in V: \forall t \in T: \textsl{card}\bigl(\textsl{parseTable}(\textsl{a},t)\bigr) \leq 1$.
\\[0.2cm]
Falls die Menge $\textsl{parseTable}(\textsl{a},t)$ leer ist, so hei{\ss}t dies einfach, dass wir beim
Parsen von $\textsl{a}$ nicht auf das Token $t$ sto{\ss}en k\"onnen.  Parsen wir also ein $\textsl{a}$ und sehen
als erstes Zeichen das Token $t$, so muss ein Syntax-Fehler vorliegen.  

\section{LL(k)-Grammatiken}

\begin{figure}[!ht]
  \begin{center}    
  \framebox{
  \framebox{
  \begin{minipage}[t]{9cm}

  \begin{eqnarray*}
  \textsl{expr}        & \rightarrow & \;\textsl{product}\;\;\textsl{exprRest}            \\[0.2cm]
  \textsl{exprRest}    & \rightarrow & \quoted{+} \textsl{product}\;\;\textsl{exprRest}   \\
                       & \mid        & \quoted{-} \textsl{product}\;\;\textsl{exprRest}   \\
                       & \mid        & \;\varepsilon                                      \\[0.2cm]
  \textsl{product}     & \rightarrow & \;\textsl{factor}\;\;\textsl{productRest}          \\[0.2cm]
  \textsl{productRest} & \rightarrow & \quoted{*} \textsl{factor}\;\;\textsl{productRest} \\
                       & \mid        & \quoted{/} \textsl{factor}\;\;\textsl{productRest} \\
                       & \mid        & \;\varepsilon                                      \\[0.2cm]
  \textsl{factor}      & \rightarrow & \quoted{(} \textsl{expr} \quoted{)}                \\
                       & \mid        & \;\textsl{Number}                                  \\
                       & \mid        & \;\textsc{Identifier}                              \\
                       & \mid        & \;\textsc{Identifier} \quoted{(} \textsl{argList} \quoted{)}
                       \\[0.2cm]
  \textsl{argList}     & \rightarrow & \textsl{expr} \;\;\textsl{argsRest}                    \\
                       & \mid        & \;\varepsilon                                      \\[0.2cm]
  \textsl{argsRest}    & \rightarrow & \quoted{,} \textsl{expr} \;\;\textsl{argsRest}         \\
                       & \mid        & \varepsilon                   
  \end{eqnarray*}
  \vspace*{-0.5cm}

  \end{minipage}}}
  \end{center}
  \caption{Arithmetische Ausdr\"ucke mit Funktions-Aufrufen.}
  \label{fig:Expr4}
\end{figure}

\noindent
Viele interessante Grammatiken sind keine $LL(1)$-Grammatiken.
Abbildung \ref{fig:Expr4} zeigt ein Beispiel.  Bei dieser Grammatik werden f\"ur arithmetische
Ausdr\"ucke auch Funktionsaufrufe der Form
\\[0.2cm]
\hspace*{1.3cm}
$f(a_1, \cdots, a_n)$
\\[0.2cm]
zugelassen.  Dabei ist $f$ ein Funktionszeichen, das syntaktisch nicht von einem Identifier zu
unterscheiden ist.  Dadurch gibt es zwischen den beiden Regeln
\\[0.2cm]
\hspace*{1.3cm}
$\textsl{factor} \rightarrow \textsc{Identifier}$ \quad und \quad
$\textsl{factor} \rightarrow \textsc{Identifier} \quoted{(} \textsl{argList} \quoted{)}$ 
\\[0.2cm]
einen Konflikt:  Soll ein \textsl{factor} geparst werden und ist das n\"achste zu lesende Zeichen ein 
\textsc{Identifier}, so ist nicht klar, welche der beiden Regeln angewendet werden soll.
Es gibt hier zwei m\"ogliche L\"osungen:  Einerseits k\"onnten wir die Grammatik durch eine
Links-Faktorisierung umschreiben.  Andererseits ist es auch m\"oglich, das Problem dadurch zu l\"osen,
dass wir bei der Entscheidung, welche der Grammatik-Regeln verwendet werden soll, zus\"atzlich das
zweite Zeichen ber\"ucksichtigen:  Handelt es sich dabei um das Zeichen $\quoted{(}$, so ist offenbar die Regel
\\[0.2cm]
\hspace*{1.3cm}
$\textsl{factor} \rightarrow \textsc{Identifier} \quoted{(} \textsl{argList} \quoted{)}$ 
\\[0.2cm]
heranzuziehen, andernfalls muss die Regel
\\[0.2cm]
\hspace*{1.3cm}
$\textsl{factor} \rightarrow \textsc{Identifier}$ 
\\[0.2cm]
verwendet werden.

Im allgemeinen Fall kann das Verfahren so erweitert werden, dass $k$ Token bei der Entscheidung,
welche Regel zu verwenden ist, als Lookahead herangezogen werden.  
Wir skizzieren die Grundz\"uge dieser Theorie.
Als erstes verallgemeinern wir die Definition der Funktion $\textsl{First}()$.  

\begin{Definition}[$\textsl{First}(k,\alpha)$]
  Es sei $G = \langle V, T, R, S \rangle$ eine kontextfreie Grammatik. Wir definieren eine Funktion
  \\[0.2cm]
  \hspace*{1.3cm}
  $\textsl{First}: \mathbb{N} \times (V \cup T)^* \rightarrow 2^{T^*}$,
  \\[0.2cm]
  so dass $\textsl{First}(k,\alpha)$ f\"ur eine nat\"urliche Zahl $k$ und einen String $\alpha$, der aus
  Terminalen und syntaktischen Variablen besteht, die Menge der Token-Strings berechnet, die
  h\"ochstens die L\"ange $k$ haben und die Pr\"afix eines von $\alpha$ abgeleiteten Strings sind.  Formal
  lautet die Definition:
  \\[0.2cm]
  \hspace*{1.3cm}
  $\textsl{First}(k, \alpha) := 
  \bigl\{ x \in T^* \mid \exists y \in T^*: \alpha \Rightarrow^* x y \wedge |x| = k \} \cup
  \{ x \in T^* \mid \alpha \Rightarrow^* x \wedge |x| < k \bigr\}.$ \eox
\end{Definition}

\example
Abbildung \ref{fig:expr-id.g} zeigt eine Grammatik f\"ur arithmetische Ausdr\"ucke.
Berechnen wir f\"ur die in dieser Grammatik auftretenden Variablen $v$ die Mengen
$\textsl{First}(2,v)$, so erhalten wir beispielsweise:
\begin{enumerate}
\item $\textsl{First}(2, \textsl{expr}) = 
      \bigl\{\, \textsc{Id}, \,\squoted{(}\, \textsc{Id},\, \squoted{((},\,
      \textsc{Id} \squoted{+},\, \textsc{Id} \squoted{-},\,
      \textsc{Id} \squoted{*},\, \textsc{Id} \squoted{/}\,
      \bigr\}$,
\item $\textsl{First}(2, \textsl{product}) = 
      \bigl\{\, \textsc{Id}, \,\squoted{(}\, \textsc{Id},\, \squoted{((},\,
      \textsc{Id} \squoted{*},\, \textsc{Id} \squoted{/}\,
      \bigr\}$,
\item $\textsl{First}(2, \textsl{factor}) = 
      \bigl\{\, \textsc{Id}, \,\squoted{(}\, \textsc{Id},\, \squoted{((}\,
      \bigr\}$.
\end{enumerate}


\begin{figure}[htbp]
  \begin{center}    
  \framebox{
  \framebox{
  \begin{minipage}[t]{9cm}

  \begin{eqnarray*}
  \textsl{expr}        & \rightarrow & \textsl{expr} \quoted{+} \textsl{product}        \\
                       & \mid        & \textsl{expr} \quoted{-} \textsl{product}        \\
                       & \mid        & \textsl{product}        \\[0.2cm]
  \textsl{product}     & \rightarrow & \textsl{product} \quoted{*} \textsl{factor}      \\ 
                       & \mid        & \textsl{product} \quoted{/} \textsl{factor}      \\ 
                       & \mid        & \textsl{factor}                                  \\[0.2cm]
  \textsl{factor}      & \rightarrow & \quoted{(} \textsl{expr} \quoted{)}              \\
                       & \mid        & \;\textsc{Id} 
  \end{eqnarray*}
  \vspace*{-0.5cm}

  \end{minipage}}}
  \end{center}
  \caption{Eine Beispiel-Grammatik f\"ur arithmetische Ausdr\"ucke.}
  \label{fig:expr-id.g}
\end{figure}


\begin{Definition}[$\textsl{Follow}(k,\textsl{a})$]
  Es sei $G = \langle V, T, R, S \rangle$ eine kontextfreie Grammatik. Wir definieren eine Funktion
  \\[0.2cm]
  \hspace*{1.3cm}
  $\textsl{Follow}: \mathbb{N} \times V \rightarrow 2^{T^*}$,
  \\[0.2cm]
  so dass $\textsl{Follow}(k,\textsl{a})$ f\"ur eine nat\"urliche Zahl $k$ und eine syntaktische
  Variable $\textsl{a}$ die Menge der Token-Strings berechnet, die
  h\"ochstens die L\"ange $k$ haben und in einer Ableitung, die vom Start-Symbol $S$ ausgeht,
  auf $\textsl{a}$ folgen k\"onnen.  Formal
  lautet die Definition:
  \\[0.2cm]
  \hspace*{1.3cm}
  $\textsl{Follow}(k, \textsl{a}) := 
  \bigl\{ x \in T^* \mid \exists \alpha, \gamma \in (T\cup V)^*: 
      S \Rightarrow^* \alpha \textsl{a} \gamma \wedge x \in \textsl{First}(k, \gamma) \bigr\}$. \eox
\end{Definition}

\example
Setzen wir das letzte Beispiel sinngem\"a{\ss} fort, so erhalten wir:
\begin{enumerate}
\item $\textsl{Follow}(2, \textsl{expr}) = 
      \bigl\{\, \varepsilon,\, 
      \squoted{)},\, 
      \squoted{))},\,
      \squoted{)*},\,
      \squoted{)+},\,
      \squoted{)-},\, 
      \squoted{)/},\, 
      \squoted{+}\textsc{Id},\,
      \squoted{-}\textsc{Id},\,
      \squoted{+(},\,
      \squoted{-(}\,
      \bigr\}$,
\item $\textsl{Follow}(2, \textsl{product}) =       
      \bigl\{\, \varepsilon,\, 
      \squoted{)},\, 
      \squoted{))},\,
      \squoted{)+},\,
      \squoted{)-},\,
      \squoted{)*},\, 
      \squoted{)/},\, 
      \squoted{+}\textsc{Id},\,
      \squoted{-}\textsc{Id},\,
      \squoted{*}\textsc{Id},\,
      \squoted{/}\textsc{Id},\,
      \squoted{+(},\,
      \squoted{-(},\,
      \squoted{*(},\,
      \squoted{/(}\,
      \bigr\}$,
\item $\textsl{Follow}(2, \textsl{factor}) = \bigl\{\, 
      \varepsilon,\, 
      \squoted{)},\, 
      \squoted{))},\,
      \squoted{)+},\,
      \squoted{)-},\,
      \squoted{)*},\, 
      \squoted{)/},\, 
      \squoted{+}\textsc{Id},\,
      \squoted{-}\textsc{Id},\,
      \squoted{*}\textsc{Id},\,
      \squoted{/}\textsc{Id},\,
      \squoted{+(},\,
      \squoted{-(},\,
      \squoted{*(},\,
      \squoted{/(}\,
      \bigr\}$.
\end{enumerate}

\begin{Definition}[Starke $LL(k)$-Grammatik]
  Eine kontextfreie Grammatik $G = \langle V, T, R, S \rangle$ ist eine 
  \emph{starke $LL(k)$-Grammatik} genau dann, wenn f\"ur je zwei verschiedene Grammatik-Regeln
  \\[0.2cm]
  \hspace*{1.3cm}
  $\textsl{a} \rightarrow \beta$ \quad und \quad $\textsl{a} \rightarrow \gamma$
  \\[0.2cm]
  aus der Menge $R$ die Bedingung 
  \\[0.2cm]
  \hspace*{1.3cm}
  $\forall \sigma, \tau \in \textsl{Follow}(k, \textsl{a}): 
   \textsl{First}(k, \beta \sigma) \cap \textsl{First}(k, \gamma \tau) = \{\}$
  \\[0.2cm]
  erf\"ullt ist. \eox
\end{Definition}

\noindent
\textbf{Erkl\"arung}:  Um die obige Definition zu verstehen, nehmen wir an, wir wollten ein $\textsl{a}$ parsen.
Wenn wir einen $LL(k)$-Parser bauen wollen, d\"urfen wir die n\"achsten $k$ Symbole der Eingabe lesen
und m\"ussen entscheiden, welche der Regeln von $\textsl{a}$ in Frage kommen.  Diese $k$ Symbole k\"onnen das
Resultat der Ableitung von $\beta$ oder $\gamma$ sein.  Wenn die von $\beta$ oder $\gamma$
abgeleiteten Strings k\"urzer als $k$ sind, so kann es sich aber auch schon um Token
handeln, die in einer Ableitung auf $\textsl{a}$ folgen, die also Elemente der Menge
$\textsl{Follow}(k,\textsl{a})$ sind.  F\"ur eine Regel
\\[0.2cm]
\hspace*{1.3cm}
$\textsl{a} \rightarrow \beta$
\\[0.2cm]
und einen String $\sigma \in \textsl{Follow}(k,\textsl{a})$
enth\"alt die Menge $\textsl{First}(k, \beta \sigma)$ alle die Strings der L\"ange
$\leq k$, die in einer Ableitung von $\textsl{a}$, welche die Regel $\textsl{a} \rightarrow \beta$ benutzt,
folgen k\"onnen.  Sind diese Mengen f\"ur verschiedene Regeln disjunkt, so l\"asst sich anhand der
$k$ folgenden Token entscheiden, welche der Regeln angewendet werden muss.
\vspace*{0.2cm}

\noindent
\textbf{Bemerkung}:
In der theoretischen Informatik gibt es neben dem Begriff der \emph{starken} $LL(k)$-Grammatik auch
noch den Begriff der (einfachen) $LL(k)$-Grammatik.  Bei einer solchen $LL(k)$-Grammatik d\"urfen bei
der Auswahl der Regel nicht nur die n\"achsten $k$ Eingabe-Token ber\"ucksichtigt werden, sondern
zus\"atzlich kann der Parser alle bisher gelesenen Token mit zu Rate ziehen.  
Dadurch kann in bestimmten F\"allen zu gegebener Variable und gegebenem Lookahead auch dann
noch eine Regel ausgew\"ahlt werden, wenn das Kriterium der starken $LL(k)$-Grammatik nicht
erf\"ullt ist.
Da dieser Begriff
wesentlich komplexer ist als der Begriff der
starken $LL(k)$-Grammatik, verzichten wir auf eine formale Darstellung des
allgemeineren Begriffs.  
Die dem allgemeineren Begriff zu Grunde liegende Theorie ist sehr ausf\"uhrlich in
\cite{aho:72} dargestellt. 

\subsection{Berechnung von $\textsl{First}()$ und $\textsl{Follow}()$}
In diesem Abschnitt zeigen wir, wir die Funktionen $\textsl{First}(k, \alpha)$ und
$\textsl{Follow}(k, \textsl{a})$ berechnet werden k\"onnen.  Dazu ben\"otigen wir verschiedene
Hilfsfunktionen, die wir vorab definieren.
\begin{enumerate}
\item Die Funktion $\textsl{prefix}(k, w)$ berechnet f\"ur eine nat\"urliche Zahl $k$
      und einen String $w$ den Pr\"afix von $w$ mit der L\"ange $k$.  Ist die L\"ange von $w$
      kleiner oder gleich $k$, so wird $w$ zur\"uck gegeben:
      \\[0.2cm]
      \hspace*{1.3cm}
      $\textsl{prefix}(k, w) = \left\{
       \begin{array}{ll}
         w[1\!:\!k]  & \mbox{falls $k < |w|$;}  \\
         w       & \mbox{sonst.}
       \end{array}
       \right.
      $
      \\[0.2cm]
      Hier bezeichnet die Notation $w[1\!:\!k]$ den Teilstring von $w$, der 
      aus den ersten $k$ Buchstaben von $w$ besteht.
\item Der Operator $+_k$ verkettet zwei Strings und bildet anschlie{\ss}end das P\"afix der
      L\"ange $k$:
      \\[0.2cm]
      \hspace*{1.3cm}
      $v +_k w = \textsl{prefix}(k, vw)$.
      \\[0.2cm]
      Hier bezeichnet $vw$ die Verkettung der Strings $v$ und $w$.
\item Die Definition des Operators $+_k$ wird auf Mengen von Strings verallgemeinert:
      \\[0.2cm]
      \hspace*{1.3cm}
      $M +_k N := \{ v +_k w \mid v \in M \wedge w \in N \}$.
\end{enumerate}

\example
Wir haben
\\[0.2cm]
\hspace*{1.7cm}
$\{ \varepsilon, \quoted{A}, \quoted{AB}, \quoted{ABC} \} +_2 
 \{ \varepsilon, \quoted{X}, \quoted{YX} \} $
\\[0.2cm]
\hspace*{1.3cm}
$= \{ \varepsilon, \quoted{A}, \quoted{AB}, \quoted{X}, \quoted{YX},
    \quoted{AX}, \quoted{AY} \}
$. \qed
\vspace*{0.2cm}

\noindent
Die Berechnung von $\textsl{First}(k,\alpha)$ f\"ur $\alpha \in (V \cup T)^*$ wird auf die
Berechnung von $\textsl{First}(k, X)$ mit $X \in V \cup T$ zur\"uck gef\"uhrt,
denn es gilt
\\[0.2cm]
\hspace*{1.3cm}
$\textsl{First}(k, X_1 X_2 \cdots X_n) = 
 \textsl{First}(k, X_1) +_k \textsl{First}(k, X_2) +_k \cdots +_k \textsl{First}(k, X_n)$.
\\[0.2cm]
F\"ur ein Terminal $t \in T$ gilt offenbar
\\[0.2cm]
\hspace*{1.3cm}
$\textsl{First}(k, t) = \{ t \}$.
\\[0.2cm]
Die Berechnung der Menge $\textsl{First}(k, \textsl{a})$ f\"ur eine syntaktische Variable $\textsl{a} \in V$
erfolgt iterativ \"uber folgenden Fixpunkt-Algorithmus:
\begin{enumerate}
\item Zun\"achst werden alle Mengen $\textsl{First}(k, \textsl{a})$ mit der leeren Menge initialisiert:
      \\[0.2cm]
      \hspace*{1.3cm}
      $\textsl{First}(k, \textsl{a}) := \{\}$.
\item Anschlie{\ss}end wird f\"ur jede Grammatik-Regel der Form 
      \\[0.2cm]
      \hspace*{1.3cm}
      $\textsl{a} \rightarrow \beta$
      \\[0.2cm]
      die Menge $\textsl{First}(k, \textsl{a})$ wie folgt erweitert:
      \\[0.2cm]
      \hspace*{1.3cm}
      $\textsl{First}(k, \textsl{a}) := \textsl{First}(k, \textsl{a}) \cup \textsl{First}(k,\beta)$.
\item Der zweite Schritt wird in einer Schleife solange durchgef\"uhrt, bis sich keine
      der Mengen $\textsl{First}(k, \textsl{a})$ mehr durch die Hinzunahme von
      $\textsl{First}(k,\beta)$ \"andert.
\end{enumerate}
Sind die Mengen $\textsl{First}(k, \textsl{a})$ berechnet, so k\"onnen wir anschlie{\ss}end die Mengen 
$\textsl{Follow}(k, \textsl{a})$ f\"ur alle syntaktischen Variablen berechnen.  Auch die Berechnung
der Mengen $\textsl{Follow}(k, \textsl{a})$ ist iterativ.  Sie erfolgt nach dem folgenden
Schema:
\begin{enumerate}
\item Zun\"achst werden alle Mengen $\textsl{Follow}(k, \textsl{a})$ mit der leeren Menge initialisiert:
      \\[0.2cm]
      \hspace*{1.3cm}
      $\textsl{Follow}(k, \textsl{a}) = \{\}$.
      \\[0.2cm]
      Anschlie{\ss}end setzen wir f\"ur das Start-Symbol $S$ der Grammatik
      \\[0.2cm]
      \hspace*{1.3cm}
      $\textsl{Follow}(k, S) = \{ \symbol{36} \}$.
      \\[0.2cm]
      Hier steht ``$\symbol{36}$'' f\"ur das Ende der Eingabe.  Die Idee ist, dass hinter dem
      Start-Symbol keine weitere Eingabe mehr kommen kann.  Beachten Sie, dass in diesem
      Fall der String ``$\symbol{36}$'' nicht aus $k$ Zeichen besteht, sondern nur aus
      einem Zeichen.
\item F\"ur jede Grammatik-Regel der Form
      \\[0.2cm]
      \hspace*{1.3cm}
      $\textsl{a} \rightarrow Y_1 Y_2 \cdots Y_l$,
      \\[0.2cm]
      f\"ur die $Y_l$ eine syntaktische Variable ist, erweitern wir die Menge
      $\textsl{Follow}(k, Y_l)$ wie folgt:
      \\[0.2cm]
      \hspace*{1.3cm}
      $\textsl{Follow}(k, Y_l) := \textsl{Follow}(k, Y_l) \cup \textsl{Follow}(k, \textsl{a})$,
      \\[0.2cm]
      denn alles, was auf ein $\textsl{a}$ folgen kann, kann auch auf ein $Y_l$ folgen.
\item F\"ur jede Grammatik-Regel der Form
      \\[0.2cm]
      \hspace*{1.3cm}
      $\textsl{a} \rightarrow Y_1 Y_2 \cdots Y_i Y_{i+1} \cdots Y_l$,
      \\[0.2cm]
      und jeden Index $i \in \{1, \cdots, l-1\}$, f\"ur den $Y_i$ eine syntaktische Variable
      ist, erweitern wir die Menge $\textsl{Follow}(k, Y_i)$ wie folgt:
      \\[0.2cm]
      \hspace*{1.3cm}
      $\textsl{Follow}(k, Y_i) := \textsl{Follow}(k, Y_i) \cup
       \bigl(\textsl{First}(k, Y_{i+1} \cdots Y_l) +_k \textsl{Follow}(k,\textsl{a})\bigr) 
      $.
      \\[0.2cm]
      Der Grund, warum wir hier noch die Menge $\textsl{Follow}(k,\textsl{a})$ anh\"angen ist der,
      dass die Strings aus  der Menge $\textsl{First}(k, Y_{i+1} \cdots Y_l)$ eventuell
      k\"urzer als $k$ sind.  In diesem Fall m\"ussen noch die Pr\"afixe von
      $\textsl{Follow}(k, \textsl{a})$ angeh\"angt werden.

      \textbf{Bemerkung}:  Beachten Sie, dass der zweite Schritt ein Spezialfall des
      dritten Schritts ist, denn wenn wir im dritten Schritt $i := l$ setzen, dann ist
      der String $Y_{i+1} \cdots Y_l$ leer und somit enth\"alt die Menge
      $\textsl{First}(k, Y_{i+1}\cdots Y_l)$ dann nur den leeren String $\varepsilon$, so
      dass der Ausdruck
      \\[0.2cm]
      \hspace*{1.3cm}
      $\bigl(\textsl{First}(k, Y_{i+1} \cdots Y_l) +_k \textsl{Follow}(k,\textsl{a})\bigr)$  \quad
      zu \quad $\textsl{Follow}(k,\textsl{a})\bigr)$
      \\[0.2cm]
      vereinfacht werden kann.  Bei der Implementierung werden wir daher nur den dritten
      Schritt umsetzen.
\item Der zweite und der dritte Schritt werden in einer Schleife solange durchgef\"uhrt, bis sich keine
      der Mengen $\textsl{Follow}(k, \textsl{a})$ mehr \"andert.
\end{enumerate}

\subsection{Implementation in \textsc{SetlX}}
Figure \ref{fig:llk.stlx:computeFirst} on page \pageref{fig:llk.stlx:computeFirst} shows an
implementation of the function \texttt{First} in \textsc{SetlX}.  We proceed to discuss the
implementation line by line.




\begin{figure}[!ht]
\centering
\begin{Verbatim}[ frame         = lines, 
                  framesep      = 0.3cm, 
                  firstnumber   = 1,
                  labelposition = bottomline,
                  numbers       = left,
                  numbersep     = -0.2cm,
                  xleftmargin   = 0.8cm,
                  xrightmargin  = 0.8cm,
                ]
    computeFirst := procedure(k, rules, variables) {
        first := initializeMap(variables);
        change := true;
        while (change) {
            change := false;    
            for ([a, body] in rules) {
                new := firstList(k, body, first);
                if (!(new <= first[a])) { 
                    change := true;
                    first[a] += new;
                }
            }
        }
        return first;
    };
\end{Verbatim}
\vspace*{-0.3cm}
\caption{The function \texttt{computeFirst}.}
\label{fig:llk.stlx:computeFirst}
\end{figure}
\begin{enumerate}
\item The first parameter $k$ is the number of lookahead tokens, while the second parameter
      \texttt{rules} is the set of all grammar rules. Here, a rule of the form
      \\[0.2cm]
      \hspace*{1.3cm}
      $\textsl{a} \rightarrow \beta$
      \\[0.2cm]
      is represented in \textsc{SetlX} as a pair of the form
      \\[0.2cm]
      \hspace*{1.3cm}
      \texttt{[ $\textsl{a}$, $\beta$ ]}.
      \\[0.2cm]
      Finally. the parameter \texttt{variables} is the set of all syntactical variables.
\item The function $\texttt{computeFirst}(k, \texttt{rules}, \mathtt{variables})$ is supposed to compute
      $\textsl{First}(k, \textsl{a})$ for all syntactical variables $\textsl{a}$.  This is done by creating a binary
      relation \texttt{first} in line 2.  At the end of the computation, the relation
      \texttt{first} will contain a pair $[\textsl{a}, \textsl{First}(k,\textsl{a})]$ for every syntactical variable
      $\textsl{a}$.  Therefore, the variable \texttt{first} codes the function \textsl{First}.

      Since the function \texttt{first} is computed via a fixpoint iteration, $\mathtt{first}[\textsl{a}]$
      is initialized to the empty set for all syntactical variables $\textsl{a}$.  This is done by the function
      \texttt{initializeMap} in line 2.  The function \texttt{initializeMap} is shown in
      Figure \ref{fig:llk.stlx:auxiliary} in line 1.
\item The computation of \texttt{First} is done in the \texttt{while}-loop that extends from line 4
      to line 13.  This \texttt{while}-loop is controlled by the Boolean flag
      \texttt{change}.  This variable is set to false at the beginning of the loop in line 5.
      If we ever find a variable \texttt{a} in line 6 such that \texttt{first[a]} gets incremented
      in line 10,
      then the variable \texttt{changed} is changed to \texttt{true} so that the iteration can keep going.
      On the other hand, if we don't find any new strings that have to be added to \texttt{first[a]}
      for any variable \texttt{a}, then we have successfully computed the function \textsl{First}
      for all variables and the fixpoint iteration can be stopped.
\item Given a rule of the form
      \\[0.2cm]
      \hspace*{1.3cm}
      $\textsl{a} \rightarrow \beta$
      \\[0.2cm]
      we know that 
      \\[0.2cm]
      \hspace*{1.3cm}
      $\textsl{First}(k, \beta) \subseteq \textsl{First}(\textsl{a})$
      \\[0.2cm]
      and therefore $\textsl{First}(k, \beta)$ has to be added to $\textsl{First}(\textsl{a})$.
      In our implementation,  $\textsl{First}(k, \beta)$ is computed by the function
      \texttt{firstList} that is shown in Figure \ref{fig:llk.stlx:firstList} on page
      \pageref{fig:llk.stlx:firstList}.
\end{enumerate}


\begin{figure}[!ht]
\centering
\begin{Verbatim}[ frame         = lines, 
                  framesep      = 0.3cm, 
                  firstnumber   = 1,
                  labelposition = bottomline,
                  numbers       = left,
                  numbersep     = -0.2cm,
                  xleftmargin   = 0.8cm,
                  xrightmargin  = 0.8cm,
                ]
    firstList := procedure(k, alpha, first) {
        match (alpha) {
            case []: 
                 return { [] };
            case [ Var(v) | r ]:
                 firstV := first[v];
                 firstR := firstList(k, r, first);
                 return unionK(firstV, firstR, k);
            case [ Token(t) | r ]:
                 firstR := firstList(k, r, first);
                 return unionK({ [t] }, firstR, k);
        }
    }; 
\end{Verbatim}
\vspace*{-0.3cm}
\caption{The implementation of \texttt{firstList}.}
\label{fig:llk.stlx:firstList}
\end{figure}

Figure \ref{fig:llk.stlx:firstList} shows the implementation of the function \texttt{firstList}.
\begin{enumerate}
\item The first parameter $k$ is the number of lookahead tokens, the second parameter \texttt{alpha}
      is a list of variables and terminals, and the last parameter \texttt{first} is a binary
      relation coding the function $\textsl{First}(k, \textsl{a})$ for all syntactical variables $\textsl{a}$.
\item If \texttt{alpha} has the form $[Y_1, \cdots, Y_n]$, then the formula to compute
      $\textsl{First}(k,\mathtt{alpha})$ is
      \\[0.2cm]
      \hspace*{1.3cm}
      $\textsl{First}(k,[Y_1, \cdots, Y_n]) =  \textsl{First}(k, Y_1) +_k \textsl{First}(k, Y_2) +_k \cdots +_k \textsl{First}(k, Y_n)$.
      \\[0.2cm]
      This sum is computed recursively.  The operator $+_k$ is implemented via the function
      \texttt{unionK} that is shown in Figure \ref{fig:llk.stlx:auxiliary}.
\end{enumerate}

\noindent
Figure \ref{fig:llk.stlx:computeFollow} on page \pageref{fig:llk.stlx:computeFollow} shows the
implementation of the function \texttt{Follow}.  
\begin{enumerate}
\item The parameter $k$ is the number of lookahead tokens, \texttt{rules} is the set of grammar
      rules, \texttt{s} is the start symbol of the grammar, \texttt{first} is a binary relation
      representing the function \textsl{First}, and \texttt{variables} is the set of all syntactical
      variables of the grammar.
\item Like the function \texttt{First}, the function \texttt{Follow} is also implemented as a binary
      relation.  This relation is stored in the variable \texttt{follow}.  At the beginning of the
      computation, for all syntactical variables $\textsl{a}$ the set $\texttt{follow}[\textsl{a}]$ is initialized as
      an empty set via the function 
      \texttt{initializeMap}.   Additionally, the end-of-file symbol ``\texttt{\symbol{36}}'' is
      added into the follow set of the start symbol \texttt{s}.
\item If we have a grammar rule of the form
      \\[0.2cm]
      \hspace*{1.3cm}
      $\textsl{a} \rightarrow Y_1 Y_2 \cdots Y_i Y_{i+1} \cdots Y_l$,
      \\[0.2cm]
      and if, furthermore, $Y_i$ is a syntactical variable, then we have to extend the set
      $\texttt{follow}(Y_i)$ as follows:
      \\[0.2cm]
      \hspace*{1.3cm}
      $\texttt{follow}(Y_i) \;\texttt{+=}\; \texttt{First}(k, [Y_{i+1}, \cdots, Y_l)]) +_k \texttt{follow}(\textsl{a})$.
      \\[0.2cm]
      This rule is implemented in line 11 and 12.  The expression $\texttt{First}(k, [Y_{i+1},\cdots, Y_l)])$ 
      is computed via the function \texttt{firstList}.
\item The variable \texttt{change} controls the fixpoint iteration: The \texttt{while}-loop keeps
      going as long as there is a syntactical variable $\textsl{a}$ such that $\mathtt{follow}[\textsl{a}]$ has changed.
\end{enumerate}


\begin{figure}[!ht]
\centering
\begin{Verbatim}[ frame         = lines, 
                  framesep      = 0.3cm, 
                  firstnumber   = 1,
                  labelposition = bottomline,
                  numbers       = left,
                  numbersep     = -0.2cm,
                  xleftmargin   = 0.8cm,
                  xrightmargin  = 0.8cm,
                ]
    computeFollow := procedure(k, rules, s, first, variables) {
        follow := initializeMap(variables);
        follow[s] := { [ "\$" ] };
        change := true;
        while (change) {
            change := false;    
            for ([a, body] in rules) {
                for (i in [1 .. #body]) {
                    match (body[i]) {
                        case Var(yi):
                            tail := firstList(k, body[i+1 ..], first);
                            new  := unionK(tail, follow[a], k);
                            if (!(new <= follow[yi])) { 
                                change := true; 
                                follow[yi] += new;
                            }
                    }
                }
            }
        }
        return follow;
    };
\end{Verbatim}
\vspace*{-0.3cm}
\caption{The function \texttt{computeFollow}.}
\label{fig:llk.stlx:computeFollow}
\end{figure}
%\$

\pagebreak 

\begin{figure}[!ht]
\centering
\begin{Verbatim}[ frame         = lines, 
                  framesep      = 0.3cm, 
                  firstnumber   = 1,
                  labelposition = bottomline,
                  numbers       = left,
                  numbersep     = -0.2cm,
                  xleftmargin   = 0.8cm,
                  xrightmargin  = 0.8cm,
                ]
    initializeMap := procedure(variables) {
        return { [a, {}] : a in variables };
    };
    prefixK := procedure(s, k) {
        if (#s <= k) {
            return s;
        }
        return s[1..k];
    };
    addK := procedure(u, v, k) {
        return prefixK(u + v, k);
    };
    unionK := procedure(s, t, k) {
        return { addK(u, v, k) : u in s, v in t };
    };    
\end{Verbatim}
\vspace*{-0.3cm}
\caption{Some auxiliary functions.}
\label{fig:llk.stlx:auxiliary}
\end{figure}

\noindent
Finally, Figure \ref{fig:llk.stlx:auxiliary} on page \pageref{fig:llk.stlx:auxiliary} shows the
implementation of some auxiliary functions. 
\begin{enumerate}
\item The function \texttt{initializeMap} takes a set of variables and creates a relation that
      assigns the empty set to all of these variables.
\item The function \texttt{prefixK} takes a list $s$ and computes the prefix of $s$
      that has a length of $k$.  If the length of $s$ is at most $k$, then $s$ is returned
      unchanged. 
\item The function $\texttt{addK}(u,v,k)$ computes $u +_k v$ for two lists $u$ and $v$.
\item The function $\texttt{unionK}(s, t, k)$ computes the set $s +_k t$ for two sets of 
      lists $s$ and $t$.
\end{enumerate}

%%% Local Variables: 
%%% mode: latex
%%% TeX-master: "formal-languages"
%%% End: 
