
\section{\textsl{JavaCC}}
Die Sprache \texttt{C} ist in vielen Bereichen von der Sprache \textsl{Java}\/ verdr\"angt
worden.  Daher ist es nur nat\"urlich, dass auch f\"ur \textsl{Java}\/ Scanner-Generatoren
entwickelt worden sind.  Hier gibt es mittlerweile eine ganze Menge verschiedener
Werkzeuge.  Von diesen hat das Werkzeuge \textsl{JavaCC}\/ \cite{kodaganallur:2004}
einen offiziellen Character, weil
es wie \textsl{Java}\/ von der Firma \textsl{Sun \/Microsystems} entwickelt worden ist und
\textsl{JavaCC}\/ ein registriertes Warenzeichen der Firma \textsl{Sun}\/ ist.  Daher habe ich
von den verschiedenen zur Auswahl stehenden Werkzeugen zun\"achst \textsl{JavaCC}\/ ausgew\"ahlt.

Eigentlich ist \textsl{JavaCC}\/ mehr als nur ein Scanner-Generator, denn mit
\textsl{JavaCC}\/ k\"onnen Sie auch einen Parser erzeugen.  Diesen Aspekt werden wir in
einem sp\"ateren Kapitel noch besprechen.  Zun\"achst begn\"ugen wir uns aber damit,
\textsl{JavaCC}\/ als Scanner-Generator einzusetzen.  

\subsection{Ein Beispiel}
Wir demonstrieren die Funktionalit\"at von \textsl{JavaCC}\/
anhand des Beispiels der Notenberechnung, wir setzen das Beispiel aus Abbildung
\ref{fig:noten.l} also nun noch einmal mit \textsl{JavaCC}\/ um.  Abbildung
\ref{fig:Klausur.jj} zeigt die Datei \texttt{Klausur.jj}, die eine Eingabe-Spezifikation
f\"ur das Werkzeug \textsl{JavaCC}\/ ist.  


\begin{figure}[!h]
\centering
\begin{Verbatim}[ frame         = lines, 
                  framesep      = 0.3cm, 
                  labelposition = bottomline,
                  numbers       = left,
                  numbersep     = -0.2cm,
                  xleftmargin   = 0.0cm,
                  xrightmargin  = 0.0cm,
                ]
    PARSER_BEGIN(Klausur)
    
    public class Klausur {
        static String  sName      = null;
        static Integer sSumPoints = 0;
        static Integer sMaxPoints = 0;
        
        public static void main(String args[]) throws ParseException {
            SimpleCharStream    stream  = new SimpleCharStream(System.in);
            KlausurTokenManager manager = new KlausurTokenManager(stream);
            Token               t;
            sMaxPoints = new Integer(args[0]);
            do {
                t = manager.getNextToken();
            } while (t.kind != 0); // end of file gives t.kind = 0 
        }
        static Double note() {
            return 7.0 - 6.0 * sSumPoints / sMaxPoints;
        }
    }
    PARSER_END(Klausur)
    
    TOKEN: {
        <KOPF:    (<LETTER>)+ ":" (~["\n"])* "\n">
      | <NAME:    (<LETTER>)+ " " (<LETTER>)+>  
                  {
                      Klausur.sName      = image.toString(); 
                      Klausur.sSumPoints = 0;
                      System.out.print(Klausur.sName);
                  }
      | <COLON:   ":" ([ " ", "\t" ])+>         
                  { 
                      System.out.print(image); 
                  }
      | <ZAHL:    "0" | ["1"-"9"] (["0"-"9"])*> 
                  { 
                      Klausur.sSumPoints += new Integer(image.toString()); 
                  }
      | <HYPHEN:  "-">
      | <EOL:     ([" ", "\t"])* "\n">          
                  { 
                      if (Klausur.sName != null) {
                          System.out.printf(" %3.1f\n", Klausur.note()); 
                      }
                  }
      | <WHITE:   [" ", "\t"]>
      | <#LETTER: ["a"-"z", "A"-"Z", "\"o", "\"a", "\"u", "\"O", "\"A", "\"U", "{\ss}"]>
    }
\end{Verbatim}
\vspace*{-0.3cm}
\caption{Ein \textsl{JavaCC}-Programm zur Notenberechnung}
\label{fig:Klausur.jj}
\end{figure}

Eine \textsl{JavaCC}-Eingabe-Spezifikation f\"ur einen Scanner besteht aus zwei Teilen:
\begin{enumerate}
\item Die \emph{Klassen-Definition} wird durch die Schl\"usselw\"orter
      ``\texttt{PARSER\_BEGIN}'' und ``\texttt{PARSER\_END}''
      eingeschlossen.  In der Abbildung erstreckt sich dieser Teil
      von Zeile 1 -- 21.
\item Die \textsl{Token-\/Definition} definiert verschiedene Tokens durch
      regul\"are Ausdr\"ucke.
      In unserem Beispiel beginnt sie mit dem Schl\"usselwort ``\texttt{TOKEN}''
      in Zeile 23, dem eine Doppelpunkt und eine \"offnende geschweifte Klammer
      ``\texttt{\{}'' folgt.   Sie endet in Zeile 45 mit der schlie{\ss}enden geschweiften
      Klammer ``\texttt{\}}''.
\end{enumerate}
Wir diskutieren die Eingabe-Spezifikation jetzt Zeile f\"ur Zeile.
\begin{enumerate}
\item In Zeile 1 folgt hinter dem Schl\"usselwort ``\texttt{PARSER\_BEGIN}'' der in Klammern
      eingeschlossene Name der Datei, die die Spezifikation enth\"alt, wobei die
      Datei-Endung weggelassen wird.  Die Datei-Endung lautet immer ``\texttt{.jj}''.
\item Anschlie{\ss}end folgt eine Klassen-Definition.  Der Name dieser Klasse muss ebenfalls
      mit dem Datei-Namen \"ubereinstimmen.
  
      In unserem Fall enth\"alt die Klasse zun\"achst die Definitionen verschiedener
      statischer Variablen.
      \begin{enumerate}
      \item Die in Zeile 4 definierte Variable \texttt{sName} enth\"alt sp\"ater den Namen 
            des Studenten, dessen Note berechnet werden soll.  Wir initialisieren diese
            Variable aber zun\"achst mit \texttt{null}.  Dadurch haben wir sp\"ater die
            M\"oglichkeit zu pr\"ufen, ob bereits Punkte eines Studenten gelesen worden sind
            oder ob bisher erst die Kopfzeilen der Eingabe-Datei gelesen wurden.
      \item Die in Zeile 5 definierte Variable \texttt{sSumPoints} gibt die Summe aller
            Punkte an, die ein Student erzielt hat.
      \item Die in Zeile 6 definierte Variable \texttt{sMaxPoints} gibt die maximal erreichbare Punktezahl an.
      \end{enumerate}
      Anschlie{\ss}end wird die Methode \texttt{main} definiert.  Diese Methode ist erforderlich, damit
      der erzeugte Scanner aufgerufen werden kann.  
      \begin{enumerate}
      \item Als erstes erzeugen wir einen \emph{Token-Manager}.  Da der Konstruktor 
            f\"ur einen Token-Manager als Eingabe ein Objekt vom Type
            \texttt{SimpleCharStream} erwartet, 
            konvertieren wir den Eingabestrom ``\texttt{System.in}'' zun\"achst in ein solches
            Objekt und erzeugen dann in Zeile 10 den Token-Manager, als Objekt der
            Klasse \texttt{KlausurTokenManager}.  
      \item Die Methode \texttt{main} bekommt die maximale Punktzahl als Argument
            \"ubergeben und speichert diese in Zeile 12 in der Variablen \texttt{sMaxPoints}
            ab.
      \item Anschlie{\ss}end lesen wir in der \texttt{while}-Schleife solange Tokens, bis wir
            das \textsl{End-Of-File}-Token lesen.  Dieses Token k\"onnen wir daran erkennen,
            dass die Member-Variable \texttt{t.kind} den Wert 0 hat.
            Diese Schleife bezeichnen wir im Folgenden als die \emph{Scanner-Schleife}.
       
            Jedes Mal, wenn der Scanner ein Token erkennt, werden die mit dem Token
            spezifizierten Aktionen ausgef\"uhrt, mehr dazu sp\"ater.
      \end{enumerate}
      Neben der Methode $\textsl{main}()$, die in jeder Parser-Klasse vorhanden
      sein muss, enth\"alt die Klasse \texttt{Klausur} noch die Definition der statischen
      Methode $\texttt{note}()$, mit der sp\"ater die Note berechnet wird.  Hier
      wird dieselbe Formel verwendet, die wir auch schon in dem entsprechenden
      \textsl{Flex}-Beispiel benutzt haben.
\item Nun folgen nach dem Schl\"usselwort \texttt{Token} und einem Doppelpunkt die
      einzelnen Token-Definitionen.  Diese werden insgesamt von geschweiften Klammern
      eingefasst und die einzelnen Token-Spezifikationen werden durch das
      Zeichen ``\texttt{|}'' voneinander getrennt.  Eine einzelne Token-Spezifikationen hat die
      Form
      \\[0.2cm]
      \hspace*{1.3cm} $\texttt{<}\textsl{Name}\texttt{:} \textsl{RegExp}\texttt{>}$ $\texttt{\{} \;\textsl{CmdList}\; \texttt{\}}$
      \\[0.2cm]
      Die Bedeutung der einzelnen Komponenten ist wie folgt:
      \begin{enumerate}
      \item \textsl{Name}\/ steht f\"ur den Namen eines Tokens.  Dieser besteht aus
            Gro{\ss}buchstaben.  
      \item \textsl{RegExp}\/ bezeichnet einen regul\"aren Ausdruck.  Die Syntax der
            regul\"aren Ausdr\"ucke weicht von der bei \textsl{Flex}\/ gebr\"auchlichen
            Syntax stark ab.  Insbesondere haben Leerzeichen und Tabulatoren keine
            Bedeutung, so dass es m\"oglich ist, komplexe regul\"are Ausdr\"ucke lesbar zu
            formatieren.  Wir werden die genaue Syntax regul\"arer Ausdr\"ucke im n\"achsten
            Unterabschnitt im Detail diskutieren,
            
            \textsl{Name}\/ und \textsl{RegExp}\/ werden zusammen in spitzen Klammern ``\texttt{<}''
            und ``\texttt{>}'' eingeschlossen und durch einen Doppelpunkt getrennt.
      \item \textsl{CmdList}\/ steht f\"ur eine Liste von Befehlen.  Diese Befehle
            werden ausgef\"uhrt, sobald Text erkannt wird, der von dem regul\"aren Ausdruck
            \textsl{RegExp}\/ erkannt wird.  Syntaktisch handelt es sich bei den Befehlen
            um einen in geschweiften Klammern eingeschlossenen Block von \textsl{Java}-Code.
      \end{enumerate}
      Wir analysieren nun die einzelnen Token-Definitionen aus Abbildung \ref{fig:Klausur.jj}.
      \begin{enumerate}
      \item Die Definition des Tokens \texttt{KOPF} in Zeile 24 spezifiziert die
            Kopfzeilen einer Ergebnis-Datei, die beispielsweise die folgende Form haben:
            \begin{verbatim}
            Klausur: Algorithmen und Datenstrukturen
            Kurs:    TIT07AIX
            \end{verbatim}
            \vspace*{-0.5cm}

            Der regul\"are Ausdruck, der in Zeile 24 auf den Doppelpunkt folgt, besteht aus vier
            Komponenten:
            \begin{enumerate}
            \item \texttt{(<LETTER>)+}
              
              Durch die spitzen Klammern wird das Token \texttt{LETTER} referenziert.
              Dieses Token wird weiter unten in Zeile 47 definiert und steht f\"ur einen beliebigen
              Buchstaben einschlie{\ss}lich eines Umlauts oder ``\texttt{{\ss}}''.  Das Token
              \texttt{LETTER} ist in runden Klammern eingeschlossen, auf die noch der Operator
              ``\texttt{+}'' folgt.  Dieser Operator steht genau wie bei \textsl{Flex}\/ f\"ur
              eine beliebige positive Anzahl von Wiederholungen.  Im Unterschied zu
              \textsl{Flex}\/ muss das Argument, auf dass sich der Operator ``\texttt{+}''
              bezieht, aber in runden Klammern eingeschlossen sein.  Insgesamt steht der
              Ausdruck ``\texttt{(<LETTER>)+}'' also f\"ur eine beliebige positive Anzahl
              von Buchstaben.
            \item \texttt{\symbol{34}:\symbol{34}}

                  In doppelten Hochkommata ``\texttt{\symbol{34}}'' eingeschlossener Text
                  wird w\"ortlich interpretiert.  Der wesentliche Unterschied zu \textsl{Flex},
                  der hier sichtbar wird, ist die Tatsache, dass innerhalb regul\"arer Ausdr\"ucke in
                  \textsl{JavaCC}\/ Leerzeichen, Tabulatoren und Zeilenumbr\"uche zul\"assig sind.
                  Diese werden ignoriert.  So wird das Leerzeichen, dass den Ausdruck
                  ``\texttt{\symbol{34}:\symbol{34}}'' von dem Ausdruck
                  ``\texttt{(<LETTER>)+}'' trennt, ignoriert.  Ich habe dieses Leerzeichen
                  zur Verbesserung der Lesbarkeit eingef\"ugt.  Das w\"are bei einer
                  \textsl{Flex}-Spezifikation so nicht m\"oglich.
            \item \texttt{(\symbol{126}[\symbol{34}\symbol{92}n\symbol{34}])*}

                  Dieser Ausdruck beschreibt eine beliebige Anzahl von Zeichen,
                  die von einem Zeilen-Umbruch ``\texttt{\symbol{92}n}'' verschieden sind.
                  Zun\"achst haben wir hier den in eckigen Klammern eingeschlossenen
                  Ausdruck ``\texttt{\symbol{92}n}''.  Eckige Klammern geben, genau wie bei
                  \textsl{Flex}, eine Menge von Zeichen an.  Im Unterschied zu \textsl{Flex}\/ 
                  m\"ussen die Zeichen aus dieser Menge allerdings alle in doppelten Hochkommata
                  ``\texttt{\symbol{34}}'' eingefasst werden.  
                  In dem obigen Fall enth\"alt die Menge nur ein einziges
                  Zeichen.  \textbf{Vor} den eckigen Klammern finden wir aber das Zeichen
                  ``\texttt{\symbol{126}}''.  Dieses Zeichen bewirkt eine Komplementbildung:
                  Der Ausdruck ``\texttt{\symbol{126}[\symbol{34}\symbol{92}n\symbol{34}]}''
                  steht also f\"ur alle Zeichen, die von einem Zeilen-Umbruch \textbf{verschieden} sind.

                  Hier weicht die Syntax ebenfalls von der bei \textsl{Flex}\/ gebr\"auchlichen
                  Syntax ab, den bei \textsl{Flex}\/ wird zur Komplementbildung der Operator
                  ``\texttt{\^}'' verwendet und dieser steht dann als erstes Zeichen \textbf{nach} der
                  \"offnenden eckigen Klammer.

                  Der ganze Ausdruck ist dann noch in runden Klammern eingefasst, auf die der
                  Operator ``\texttt{*}'' folgt.  Genau wie bei \textsl{Flex}\/ steht dieser
                  Operator f\"ur eine beliebige Anzahl von Wiederholungen.
                  Im Unterschied zu \textsl{Flex}\/ muss das Argument, auf dass sich der
                  Operator ``\texttt{*}'' bezieht, aber in runden Klammern eingeschlossen
                  sein.
            \item \texttt{\symbol{34}\symbol{92}n\symbol{34}}
          
                  Dieser Ausdruck steht f\"ur einen Zeilen-Umbruch.
            \end{enumerate}
            Hinter dem Token \texttt{KOPF} steht kein \textsl{Java}-Block.  Daher
            wird der Text, der hier erkannt wurde, einfach ignoriert.  
      \item Die Definition von \texttt{NAME} sagt, dass eine Name aus zwei Gruppen
            von Buchstaben besteht, die durch ein Leerzeichen getrennt werden.
            Wichtig ist hier, dass das Leerzeichen in dem regul\"aren Ausdruck in
            doppelte Hochkommata ``\texttt{\symbol{34}}'' eingefasst ist, denn (wie oben
            bereits erw\"ahnt) k\"onnen regul\"are Ausdr\"ucke bei \textsl{JavaCC}\/ durch das
            Einf\"ugen von Leerzeichen, Tabulatoren und Zeilenumbr\"uchen zur Verbesserung
            der Lesbarkeit formatiert werden.

            Wir diskutieren nun den \textsl{Java}-Block in den Zeilen 26 -- 30:
            Der Text, der durch den regul\"aren Ausdruck erkannt wird, ist in der Variablen
            \texttt{image} abgelegt.  Diese Variable entspricht also der \textsl{Flex}-Variablen 
            \texttt{yytext}.  Die Variable \texttt{image} hat aus Effizienzgr\"unden den Typ
            \texttt{StringBuffer} und kann mit der Methode $\textsl{toString}()$ in einen
            \texttt{String} umgewandelt werden.  Mit diesem String initialisieren wir die 
            statische Variable \texttt{sName}.  Beachten Sie, dass der Klassenname
            ``\texttt{Klausur}'' der statischen Variablen vorangestellt werden muss,
            denn der \textsl{Java}-Block  ist sp\"ater Teil der Klasse
            \texttt{KlausurTokenManager} und nicht der Klasse \texttt{Klausur}.
            
            
      \item Die Definition von \texttt{COLON} spezifiziert einen Doppelpunkt, auf 
            den Leerzeichen und Tabulatoren folgen.  Beachten Sie hier das Komma
            in der Mengen-Definition
            \begin{verbatim}
            [" ", "\t"]
            \end{verbatim}
            \vspace*{-0.5cm}

            Bei \textsl{JavaCC}\/ m\"ussen die einzelnen Zeichen einer Mengen-Definition
            durch Kommata getrennt werden.
      \item Das Token \texttt{ZAHL} steht f\"ur eine ganze Zahl.
            Dies ist entweder das Zeichen ``\texttt{0}'' oder eine beliebige Folge von
            Ziffern, die nicht mit dem Zeichen ``\texttt{0}'' beginnt.
            
            Dieses Beispiel zeigt, dass auch bei \textsl{JavaCC}\/ innerhalb einer
            Mengen-Definition Bereiche gebildet werden k\"onnen.  Dazu wird genau wie bei 
            \textsl{Flex}\/ das Minus-Zeichen ``\texttt{-}'' verwendet.
      \item Das Token \texttt{HYPHEN} steht f\"ur ein einzelnes Minus-Zeichen ``\texttt{-}''.
      \item Das Token \texttt{EOL} steht f\"ur Leerzeichen und Tabulatoren, die am
            Zeilenende auftreten.  Der abschlie{\ss}ende Zeilenumbruch geh\"ort ebenfalls noch
            zu dem Token.  Der Name \texttt{EOL} ist als Abk\"urzung f\"ur 
            \emph{\underline{e}nd \underline{o}f \underline{l}ine} gedacht.
      \item Das Token \texttt{WHITE} steht f\"ur Leerzeichen und Tabulatoren.
      \item Das Token \texttt{LETTER} ist durch das Voranstellen des
            Zeichens ``\texttt{\#}'' als \emph{privat} deklariert worden.
            Ein solches Token kann nur bei der Definition anderer Token verwendet werden,
            der Scanner gibt sp\"ater nie ein Token zur\"uck, das als privat deklariert
            wurde.  Durch die Verwendung privater Token k\"onnen regul\"are Ausdr\"ucke
            abgek\"urzt werden.  Die \emph{regul\"aren Definitionen} in \textsl{Flex}\/ haben
            dieselbe Funktion.
      \end{enumerate}
\end{enumerate}
Um das Beispiel zu \"ubersetzen, geben wir die folgenden Befehle ein:
\begin{enumerate}
\item \texttt{javacc Klausur.jj}

      Dieser Befehl erzeugt die Datei \texttt{Klausur.java}.  Zus\"atzlich werden noch die
      Dateien  
      \\[0.2cm]
      \hspace*{1.3cm}
      \texttt{KlausurConstants.java}, \texttt{KlausurTokenManager.java} und
      \texttt{Token.java} 
      \\[0.2cm]
      generiert.
\item \texttt{javac Klausur.java}      

      Damit werden die erzeugten \textsl{Java}-Dateien \"ubersetzt.
\item \texttt{java Klausur 60 < ergebnis}

      Hierdurch wird der erzeugte Scanner mit dem Argument 60 aufgerufen.
      Als Eingabe verarbeitet der Scanner dann die Datei \texttt{ergebnis}.
\end{enumerate}

\subsection{Regul\"are Ausdr\"ucke in \textsl{JavaCC}}
Die Syntax der regul\"aren Ausdr\"ucke ist f\"ur \textsl{JavaCC}\/ wie folgt:
\begin{enumerate}
\item In doppelten Hochkommata eingeschlossen Strings stehen f\"ur sich selbst.
      Beispiel:
      \\[0.2cm]
      \hspace*{1.3cm}
      \texttt{\symbol{34}else\symbol{34}}
      \\[0.2cm]
      steht f\"ur den String ``\texttt{else}''.  Genau wie bei \textsl{Flex}\/ haben
      Operator-Symbole innerhalb eines solchen Strings keine Bedeutung. 
\item Bereiche k\"onnen mit den eckigen Klammern ``\texttt{[}'' und ``\texttt{]}''
      gebildet werden, im Unterschied zu \textsl{Flex}\/ m\"ussen die einzelnen Zeichen
      allerdings in Hochkommata eingeschlossen werden.  Beispiel:
      \\[0.2cm]
      \hspace*{1.3cm}
      \texttt{[\symbol{34}0\symbol{34} - \symbol{34}9\symbol{34}]}
      \\[0.2cm]
      Dieser Ausdruck steht f\"ur eine beliebige Ziffer.
\item Die Komplementbildung erfolgt f\"ur Bereiche durch das Zeichen
      ``\texttt{\symbol{126}}'', dass nun \textbf{vor} der \"offnenden eckigen Klammer
      steht.
      Beispiel:
      \\[0.2cm]
      \hspace*{1.3cm}
      \texttt{\symbol{126}[\symbol{34}a\symbol{34} - \symbol{34}z\symbol{34}]}
      \\[0.2cm]
      Dieser Ausdruck steht f\"ur ein Zeichen, dass \textbf{kein} kleiner Buchstabe ist.
\item Der Kleene-Abschluss wird mit dem Postfix-Operator ``\texttt{*}'' gebildet.
      Im Unterschied zu \textsl{Flex}\/ muss das Argument allerdings in runden Klammern
      eingeschlossen sein.  Beispiel:
      \\[0.2cm]
      \hspace*{1.3cm}
      \texttt{([\symbol{34}0\symbol{34}-\symbol{34}9\symbol{34}])*}
      \\[0.2cm]
      Dieser Ausdruck steht f\"ur einen String, der nur aus Ziffern besteht.
\item Genau wie in \textsl{Flex}\/ gibt es die Postfix-Operatoren ``\texttt{?}'' und
      ``\texttt{+}'', deren Argumente allerdings zus\"atzlich in runden Klammern
      eingeschlossen sein m\"ussen.
\item Eine Auswahl zwischen verschiedenen Alternativen erfolgt genau wie bei \textsl{Flex}\/
      mit Hilfe des Operators ``\texttt{|}''.  Beispiel:
      \\[0.2cm]
      \hspace*{1.3cm}
      \texttt{\symbol{34}0\symbol{34} | [\symbol{34}1\symbol{34}-\symbol{34}9\symbol{34}] ([\symbol{34}0\symbol{34}-\symbol{34}9\symbol{34}])*}
      \\[0.2cm]       
      Dieser Ausdruck steht f\"ur eine einzelne 0 oder eine Zahl, die mit einer
      postiven Ziffer anf\"angt, auf die dann beliebig viele andere Ziffern folgen.
\end{enumerate}
Die Operatoren ``\texttt{\symbol{94}}'' und ``\texttt{\symbol{36}}'', mit denen in \textsl{Flex}\/
der Beginn bzw.~das Ende einer Zeile spezifiziert werden k\"onnen, gibt es in
\textsl{JavaCC}\/ nicht.  Auch der Operator ``\texttt{/}'', mit dem \emph{trailing context}
spezifiziert werden kann, fehlt in \textsl{JavaCC}.  Schlie{\ss}lich ist auch der Punkt
``\texttt{.}'', der in \textsl{Flex}\/ f\"ur ein beliebiges von ``\texttt{\symbol{92}n}''
verschiedenes Zeichen steht, in \textsl{JavaCC}\/ kein Operator.  Um ein beliebiges Zeichen
zu spezifizieren kann in \textsl{JavaCC}\/ der Ausdruck
\\[0.2cm]
\hspace*{1.3cm}
\texttt{\symbol{126}[]}
\\[0.2cm]
verwendet werden, denn dieser Ausdruck spezifiziert das Komplement eines leeren Bereichs
und das sind alle Zeichen!

\subsection{Zust\"ande in \textsl{JavaCC}}
Auch in \textsl{JavaCC}\/ gibt es Zust\"ande.  Abbildung \ref{fig:DeComment.jj} zeigt
ein \textsl{JavaCC}-Programm, mit dessen Hilfe Kommentare der Form
\\[0.2cm]
\hspace*{1.3cm}
\texttt{/*} $\cdots$ \texttt{*/}
\\[0.2cm]
aus einer \texttt{C}-Datei entfernt werden k\"onnen.  Im Unterschied zu \textsl{Flex}\/
werden Zust\"ande in \textsl{JavaCC}\/ nicht deklariert, es gibt also kein Pendant zu
``\texttt{\%x}'' bzw. ``\texttt{\%s}''.  
Au{\ss}erdem sind in \textsl{JavaCC}\/ alle Zust\"ande automatisch exklusiv.  Der Default-Zustand
hei{\ss}t jetzt ``\texttt{DEFAULT}''.

\begin{figure}[!h]
\centering
\begin{Verbatim}[ frame         = lines, 
                  framesep      = 0.3cm, 
                  labelposition = bottomline,
                  numbers       = left,
                  numbersep     = -0.2cm,
                  xleftmargin   = 0.0cm,
                  xrightmargin  = 0.0cm,
                ]
    PARSER_BEGIN(DeComment)
    
    public class DeComment {
        public static void main(String args[]) throws ParseException {
            SimpleCharStream      stream  = new SimpleCharStream(System.in);
            DeCommentTokenManager manager = new DeCommentTokenManager(stream);
            Token t;
            do {
                t = manager.getNextToken();
            } while (t.kind != 0); // end of file gives t.kind = 0 
        }
    }
    PARSER_END(DeComment)
    
    <DEFAULT> TOKEN: {
        <START: "/*"> { /* change state here */  } : ML_COMMENT
    |   <CHAR:  ~[]>  { System.out.print(image); } : DEFAULT
    }
    <ML_COMMENT> TOKEN: {
        <STOP: "*/"> {} : DEFAULT
    |   <EAT:  ~[]>  {} : ML_COMMENT
    }
\end{Verbatim}
\vspace*{-0.3cm}
\caption{Entfernung mehrzeiliger Kommentare}
\label{fig:DeComment.jj}
\end{figure}

\noindent
Das Programm in Abbildung \ref{fig:DeComment.jj} ist wie folgt aufgebaut.
\begin{enumerate}
\item Nach der Erzeugung eines Token-Managers finden wir in den Zeilen 8 -- 10
      die \emph{Scanner-Schleife}.
\item Anschlie{\ss}end finden wir in dem Programm zwei Gruppen von Token-Definitionen.
      \begin{enumerate}
      \item Die Zeilen 15 -- 18 enthalten die Regeln, die im Zustand \texttt{DEFAULT}
            verwendet werden.
            \begin{itemize}
            \item Die Regel mit dem Namen \texttt{START} erkennt den String 
                  ``\texttt{/*}'', mit dem ein mehrzeiliger Kommentar eingeleitet wird.
                  Diese Regel wechselt in den Zustand \texttt{ML\_COMMENT}.
            \item Die Regel mit dem Namen \texttt{CHAR} erkennt das Komplement
                  der leeren Zeichenmenge, also ein beliebiges Zeichen.
                  Dieses Zeichen wird unver\"andert ausgegeben.  Der Folge-Zustand
                  ist wieder der Zustand \texttt{DEFAULT}, es findet also kein
                  Zustandswechsel statt.  Daher k\"onnten wir bei dieser Regel 
                  den Rest 
                  \\[0.2cm]
                  \hspace*{1.3cm}
                  \texttt{: DEFAULT}
                  \\[0.2cm]
                  auch weglassen, denn die Angabe eines Folge-Zustands ist optional.
            \end{itemize}
      \item Die Zeilen 19 -- 22 enthalten die Regeln, die im Zustand \texttt{ML\_COMMENT}
            zum Tragen kommen.
            \begin{itemize}
            \item Die Regel mit dem Namen \texttt{STOP} erkennt den String
                  ``\texttt{*/}'', mit dem ein mehrzeiliger Kommentar beendet wird
                  und wechselt in den Zustand \texttt{DEFAULT}.
            \item Die Regel mit dem Namen \texttt{CHAR} erkennt ein beliebiges
                  Zeichen, was dann einfach verschluckt wird.
            \end{itemize}
      \end{enumerate}
\end{enumerate}
Wir erkennen an dem Beispiel, dass die allgemeine Syntax einer Regel die folgende 
Form hat:
\\[0.2cm]
\hspace*{1.3cm}
\texttt{<}\textsl{Name}\texttt{:}\textsl{Regexp}\texttt{>} 
\texttt{\{} \textsl{CmdList} \texttt{\}} \texttt{:} \textsl{State}
\\[0.2cm]
Hierbei gilt:
\begin{enumerate}
\item \textsl{Name} ist der Name, unter dem das Token sp\"ater angesprochen werden kann.
      Solange wir \textsl{JavaCC}\/ nur als Scanner verwenden, hat dieser Name keine
      Bedeutung.  Im Allgemeinen wird \textsl{JavaCC}\/ aber als \emph{Parser-Generator}
      eingesetzt und dann k\"onnen wir in den Grammatik-Regeln auf die Namen zur\"uckgreifen.
\item \textsl{Regexp} ist der regul\"are Ausdruck, der erkannt werden soll.
\item \textsl{CmdList} ist die Gruppe von Befehlen, die ausgef\"uhrt werden, sobald der regul\"are
      Ausdruck erkannt worden ist.
\item \textsl{State} ist der Zustand, in den der Automat wechseln soll, nachdem die
      Kommandos abgearbeitet sind.  Die Angabe des Folge-Zustands ist optional.
      Wird kein Folge-Zustand angegeben, so bleibt der Scanner in dem Zustand, in dem er
      gerade ist.
\end{enumerate}
\pagebreak

\exercise
Entwickeln Sie einen Assembler f\"ur die im Rechnertechnik-Skript
beschriebene Assembler-Sprache.  Das Programm soll eine Assembler-Datei in eine bin\"are
Datei \"ubersetzen.
\vspace*{0.3cm}

\noindent
\textbf{Hinweise}:  
\begin{enumerate}
\item Verwenden Sie verschiedene Zust\"ande um die einzelnen Komponenten eines Assembler-Befehls
      zu lesen.
\item Passen Sie die Klasse \texttt{Assembler}, die in dem
      Rechnertechnik-Skript vorgestellt wird, an Ihren Bedarf an und verwandeln Sie diese Klasse
      in eine konkrete Klasse.  Die abgeleiteten Klassen sind nicht mehr erforderlich.
\item Um eine 32-Bit-Zahl bin\"ar auszugeben, k\"onnen Sie die folgende Methode verwenden:
      \begin{verbatim}
public void writeBinary(OutputStream writer, int code) throws IOException
{
    int b0   = (code >>  0) & 255;
    int b1   = (code >>  8) & 255;
    int b2   = (code >> 16) & 255;
    int b3   = (code >> 24) & 255;
    writer.write(b0);
    writer.write(b1);
    writer.write(b2);
    writer.write(b3);
}
\end{verbatim}

\end{enumerate}

%%% Local Variables: 
%%% mode: latex
%%% TeX-master: "formale-sprachen"
%%% End: 
