\chapter{Introduction and Motivation}
This lecture covers both the theory of formal languages as well as the applications of this theory.
These applications are the construction of scanners, parsers, interpreters, and compilers.  Furthermore, we
will discuss a number of tools that can be used to build scanners and parsers.  In particular, we
discuss the following tools:
\begin{enumerate}
\item \textsl{JFlex} can be used to build a scanner.
\item \textsl{JavaCup} is used to generate a parser.
\item \textsc{Antlr} can build both a scanner and a parser.
\end{enumerate}
All of these tools are \emph{program generators}, i.e.~they take as input the description of a
language and produce as output a \textsl{Java} program that acts like a scanner, a parser, or both. 

Most parts of these lecture notes are currently written in the German language, while other parts 
are English.  I hope to eventually translate everything into the English language as time permits.
As I am currently rewriting parts of these lecture notes, these notes will undoubtedly contain their
fair amount of spelling errors.  If you spot an error, spelling or otherwise, I would like you to
send an email to 
\\[0.2cm]
\hspace*{1.3cm}
\href{mailto:karl.stroetmann@dhbw-mannheim.de}{\texttt{karl.stroetmann@dhbw-mannheim.de}}.
\\[0.2cm]
Alternatively, you are welcome to clone my
\href{https://github.com/karlstroetmann/Formal-Languages/}{github}
repository and send me a pull request.
There is little use telling me about a typo in person since by the time I am back at my office I
will most likely have forgotten about it.  


\section{Basic Definitions}
The central notion of this lecture is the notion of a 
\href{http://en.wikipedia.org/wiki/Formal_language}{\emph{formal language}}.
In order to be able to define this notion we need some definitions. 


\begin{Definition}[Alphabet]
An \emph{alphabet} $\Sigma$ is a finite, non-empty set of \emph{characters}:
\\[0.2cm]
\hspace*{1.3cm}
$\Sigma = \{ c_1, \cdots, c_n \}$. 
\\[0.2cm]
Sometimes, we use the term \emph{symbol} to denote a character.
\qed
\end{Definition}

\examplesEng
\begin{enumerate}
\item $\Sigma = \{ 0, 1\}$ is an alphabet that can be used to represent binary numbers.
\item $\Sigma = \{ \mathtt{a}, \cdots, \mathtt{z}, \mathtt{A}, \cdots, \mathtt{Z} \}$ 
      is the alphabet used for the English language.
\item The set $\Sigma_{\textsc{\scriptsize Ascii}} = \{ 0, 1, \cdots, 127 \}$ is known as the
      \href{http://en.wikipedia.org/wiki/ASCII}{\textsc{Ascii}-Alphabet}.  The numbers are
      interpreted as letters, digits, punctuation symbols, and control characters.
      For example, the numbers in the set $\{65, \cdots, 90 \}$ represent the letters
      $\{\mathtt{A}, \cdots, \mathtt{Z}\}$.  
      \eox
\end{enumerate}

\begin{Definition}[Strings]
Given an alphabet $\Sigma$, a \emph{string} is a list of characters from $\Sigma$.
In the theory of formal languages, these lists are written without bracket symbols and without
separating comma symbols.  If $c_1,\cdots,c_n \in \Sigma$, then we write 
\\[0.2cm]
\hspace*{1.3cm}
$w = c_1\cdots c_n$ \quad instead of \quad $w = [c_1,\cdots,c_n]$.
\\[0.2cm]
The empty string is denoted as $\varepsilon$ and  the set of all strings that can be constructed
from the alphabet $\Sigma$ is written as $\Sigma^*$.
\qed
\end{Definition}

\examplesEng
\begin{enumerate}
\item Assume that $\Sigma = \{0, 1\}$.  If we define
      \\[0.2cm]
      \hspace*{1.3cm}
      $w_1 := 01110$ \quad and \quad $w_2 := 11001$,
      \\[0.2cm]
      then both $w_1$and $w_2$ are strings.  Therefore we have
      \\[0.2cm]
      \hspace*{1.3cm}
      $w_1 \in \Sigma^*$ \quad and \quad $w_2 \in \Sigma^*$.
\item Assume that $\Sigma = \{\mathtt{a}, \;\cdots\!,\; \mathtt{z}\}$.   If we define
      \\[0.2cm]
      \hspace*{1.3cm}
      $w := \mathtt{example}$,
      \\[0.2cm]
      then we have $w \in \Sigma^*$. \eox
\end{enumerate}
The \emph{length} of a string $w$ is defined as the number of characters composing $w$.
The length of $w$ is written as $|w|$.  We use square brackets to extract the characters from a string:
Given a string $w$ and a natural number $i \leq |w|$, we agree that $w[i]$ denotes the $i$-th
character of the string $w$.   As $1$ is the first natural number, we start to count the characters at 1.  
Therefore,  $w[1]$ is the first character of the string $w$.  Although it seems quite natural to
start counting at $1$, many modern programming languages, e.g.~\texttt{C}, \textsl{Java}, and
\textsl{Python} instead start counting at $0$.  A notable exception is 
\href{http://randoom.org/Software/SetlX}{\textsc{SetlX}}.

Next, we define the \emph{concatenation} of two strings $w_1$ and $w_2$ as the string $w$ that
results from appending the string $w_2$ at the end of  $w_1$.  The concatenation of $w_1$ and $w_2$
is written as $w_1 + w_2$ or sometimes even shorter as $w_1w_2$.  
\vspace*{0.3cm}

\exampleEng
If $\Sigma = \{\mathtt{0},\mathtt{1}\}$ and, furthermore,  $w_1 = \mathtt{01}$ and $w_2 =
\mathtt{10}$, then we have
\\[0.2cm]
\hspace*{1.3cm}
$w_1+w_2 = \mathtt{0110}$ \quad and \quad $w_2+w_1 = \mathtt{1001}$.  \eox


\begin{Definition}[Formal Language] \hspace*{\fill} \linebreak
If $\Sigma$ is an alphabet, then a subset $L \subseteq \Sigma^*$
is called a \emph{formal language}.  \qed
\end{Definition}

The previous definition is very general.  As the lecture proceeds, we will define several
specializations of this concept.  For us, the two most important specializations are 
\href{http://en.wikipedia.org/wiki/Regular_language}{regular languages} and
\href{http://en.wikipedia.org/wiki/Context-free_language}{context-free languages}, 
because these two categories  play the most prominent role in computer science. 


\examplesEng
\begin{enumerate}
\item Assume that $\Sigma = \{\mathtt{0},\mathtt{1}\}$.  Define
      \\[0.2cm]
      \hspace*{1.3cm}
      $L_\mathbb{N} = \{ \mathtt{1}+w \mid w \in \Sigma^* \} \cup \{ \mathtt{0} \}$
      \\[0.2cm]
      Then $L_\mathbb{N}$ is the language consisting of all strings that can be interpreted as
      natural numbers.  The language contains all strings from $\Sigma^*$  that start with 
      the character \texttt{1} as well as the string \texttt{0}, which only contains the character
      \texttt{0}.  For example, we have
      \\[0.2cm]
      \hspace*{1.3cm}
      $\mathtt{100} \in L_\mathbb{N}$, \quad but \quad $\mathtt{010} \not\in L_\mathbb{N}$.
      \\[0.2cm]
      Let us define a function 
      \\[0.2cm]
      \hspace*{1.3cm}
      $\textsl{value}: L_\mathbb{N} \rightarrow \mathbb{N}$
      \\[0.2cm]
      on the set $L_\mathbb{N}$.  We define $\textsl{value}(w)$ by induction on the length of $w$.
      We call $\textsl{value}(w)$ the \emph{interpretation} of $w$.  The idea is that
      $\textsl{value}(w)$ computes the number represented by the string $w$:
      \begin{enumerate}
      \item $\textsl{value}(\mathtt{0}) = 0$, $\textsl{value}(1) = 1$,
      \item $|w| > 0 \rightarrow \textsl{value}(w\mathtt{0}) = 2 \cdot \textsl{value}(w)   $,
      \item $|w| > 0 \rightarrow \textsl{value}(w\mathtt{1}) = 2 \cdot \textsl{value}(w) + 1$.
      \end{enumerate}
\item Again we have $\Sigma = \{0,1\}$. Define the language $L_\mathbb{P}$
      to be the set of all strings from $L_\mathbb{N}$ that are prime numbers:
      \\[0.2cm]
      \hspace*{1.3cm}
      $L_\mathbb{P} := \{ w \in L_\mathbb{N} \mid \textsl{value}(w) \in \mathbb{P} \}$
      \\[0.2cm]
      Here, $\mathbb{P}$ denotes the set of \emph{prime numbers}, which is the set of all natural
      numbers $p$ bigger than $1$ that have no positive divisor other than $1$ or $p$:
      \\[0.2cm]
      \hspace*{1.3cm}
      $\mathbb{P} = \bigl\{ p \in \mathbb{N} \;\big|\; \{ t \in \mathbb{N} \mid \exists k \in
      \mathbb{N} : k \cdot t = p \} = \{1, p\} \bigr\}$.
\item Define $\Sigma_{\textsc{\scriptsize Ascii}} = \{ 0, \cdots, 127\}$.  Furthermore, define $L_C$
      as the set of all strings of the form
      \\[0.2cm]
      \hspace*{1.3cm}
      \texttt{char* $f$(char* $x$) \{ $\cdots$ \}}
      \\[0.2cm]
      that are, furthermore, valid \texttt{C} function definitions.
      Therefore,  $L_C$ contains all those strings that can be interpreted as a \texttt{C} function $f$
      such that $f$ takes a single argument which is a string and returns a value which is also a
      string.
\item Define $\Sigma := \Sigma_{\textsc{\scriptsize Ascii}} \cup \{\dag\}$, where
      $\mathtt{\dag}$ is some new symbol that is different from all symbols in
      $\Sigma_{\textsc{\scriptsize Ascii}}$.
      The universal language $L_u$ is the set of all strings of the form
      \\[0.2cm]
      \hspace*{1.3cm}
      $p$\dag$x$\dag$y$
      \\[0.2cm]
      such that
      \begin{enumerate}
      \item $p \in L_C$,
      \item $x,y \in \Sigma_{\textsc{Ascii}}^*$,
      \item if $f$ is the function that is represented by $p$, then $f(x)$ yields the result $y$.
            \eox
      \end{enumerate}
\end{enumerate}
The examples given above demonstrate that the notion of a formal language is very broad.
While it is easy to recognize the strings of the language $L_\mathbb{N}$, it is quite a bit more
difficult to decide whether a string is a member of 
$L_\mathbb{P}$ or $L_C$.  Finally, there can be no algorithm that is able to decide whether a string
$w$ is an element of the language $L_u$. 

\subsection{Programming Languages Used in this Lecture}
The tools presented in this lecture are based on 
\textsl{Java}.  For some of the more complex algorithms discussed in this lecture
it would be a pain to implement them in \textsl{Java}.\footnote{
For somebody used to a high-level language like \textsl{Python} or \textsc{SetlX},
it is actually a pain to implement anything in \textsl{Java}.
}
Hence, I will implement these algorithms in \textsc{SetlX}.  \textsc{SetlX} is a set-based
programming language which is very suitable for rapid prototyping.  Often, even very complex
algorithms can be implemented in a few lines of \textsc{SetlX}.  You are not expected to be able to
program in \textsc{SetlX} yourself.  However, you are expected to understand programs written in
\textsc{SetlX}.  

\textsc{SetlX} programs are about as concise as the pseudo code found in many books.  However, in
contrast to pseudo code a \textsc{SetlX} program can be executed.  In order to execute
\textsc{SetlX} programs you have to install the \textsc{SetlX} interpreter.  This interpreter is
found at the following location:
\\[0.2cm]
\hspace*{1.3cm}
\href{http://randoom.org/Software/SetlX}{{http://randoom.org/Software/SetlX}}
\\[0.2cm]
There is a tutorial explaining the language at the following address:
\\[0.2cm]
\hspace*{1.3cm}
\href{http://wwwlehre.dhbw-stuttgart.de/~stroetma/SetlX/tutorial.pdf}{\texttt{http://wwwlehre.dhbw-stuttgart.de/\symbol{126}stroetma/SetlX/tutorial.pdf}}
\\[0.2cm]
I would like to encourage you to install \textsc{SetlX} since I think it is quite beneficial if you
are able to run the programs presented in this lecture. 
\pagebreak

\section{Literature}
In addition to these lecture notes there are two good books that I would like to recommend:
\begin{enumerate}
\item \emph{Introduction to Automata Theory, Languages, and Computation}
      \cite{hopcroft:06}

      This book is the bible with respect to the theory of formal languages and it contains all the theoretical
      results discussed in this lecture. 
      Obviously, we will only be able to cover a small part of the results discussed in this book.
\item \emph{Compilers --- Principles, Techniques and Tools}
      \cite{aho:2006}

      This book is the bible with respect to the theory of compilers.  It also covers a fair amount of
      the theory of formal languages.
\end{enumerate}


%%% Local Variables: 
%%% mode: latex
%%% TeX-master: "formal-languages.tex"
%%% End: 
