\chapter{Introduction and Motivation}
This lecture covers both the theory of formal languages as well as the applications of this theory.
These applications are the construction of scanners, parsers, interpreters, and compilers.  Furthermore, we
will discuss a number of tools that can be used to build scanners and parsers.  In particular, we
discuss the following tools:
\begin{enumerate}
\item \textsl{JFlex} can be used to build a scanner.
\item \textsl{JavaCup} is used to generate a parser.
\item \textsc{Antlr} can build both a scanner and a parser.
\end{enumerate}
All of these tools are \emph{program generators}, i.e.~they take as input the description of a
language and produce as output a \textsl{Java} program that acts like a scanner, a parser, or both. 

Most parts of these lecture notes will be in German, while some parts will be written in English.
I hope to eventually translate everything into English as time permits.  As I am currently rewriting
parts of these lecture notes, these notes will undoubtedly contain their fair amount of spelling
errors.  If you spot an error, spelling or otherwise, I would like you to send an email to
\\[0.2cm]
\hspace*{1.3cm}
\href{mailto:karl.stroetmann@dhbw-mannheim.de}{\texttt{karl.stroetmann@dhbw-mannheim.de}}.
\\[0.2cm]
There is little use telling me these error in person since by the time I am back at my office I will
most likely have forgotten them.


\section{Basic Definitions}
The central notion of this lecture is the notion of a \emph{formal language}.
In order to be able to define this notion we need some definitions. 


\begin{Definition}[Alphabet]
An \emph{alphabet} $\Sigma$ is a finite, non-empty set of \emph{characters}:
\\[0.2cm]
\hspace*{1.3cm}
$\Sigma = \{ c_1, \cdots, c_n \}$. 
\\[0.2cm]
Sometimes, we use the term \emph{symbol} to denote a character.
\qed
\end{Definition}

\examplesEng
\begin{enumerate}
\item $\Sigma = \{ 0, 1\}$ is an alphabet that can be used represent binary numbers.
\item $\Sigma = \{ \mathtt{a}, \cdots, \mathtt{z}, \mathtt{A}, \cdots, \mathtt{Z} \}$ 
      is the alphabet used in the English language.
\item The set $\Sigma_{\textsc{\scriptsize Ascii}} = \{ 0, 1, \cdots, 127 \}$ is known as the
      \href{http://en.wikipedia.org/wiki/ASCII}{\textsc{Ascii}-Alphabet}.  The numbers are
      interpreted as letters, digits, punctuation symbols, and control characters.
      For example, the numbers in the set $\{65, \cdots, 91 \}$ represent the letters
      $\{\mathtt{A}, \cdots, \mathtt{Z}\}$.  
      \eox
\end{enumerate}

\begin{Definition}[Strings]
Given an alphabet $\Sigma$, a \emph{string} is a list of characters from $\Sigma$.
In the theory of formal languages, these lists are written without bracket symbols and with
separating comma symbols.  If $c_1,\cdots,c_n \in \Sigma$, then we write 
\\[0.2cm]
\hspace*{1.3cm}
$w = c_1\cdots c_n$ \quad instead of \quad $w = [c_1,\cdots,c_n]$.
\\[0.2cm]
The empty string is denoted as $\varepsilon$ and  the set of all strings is written as $\Sigma^*$.
\qed
\end{Definition}

\examplesEng
\begin{enumerate}
\item Assume that $\Sigma = \{0, 1\}$.  If we define
      \\[0.2cm]
      \hspace*{1.3cm}
      $w_1 := 01110$ \quad and \quad $w_2 := 11001$,
      \\[0.2cm]
      then both $w_1$and $w_2$ are strings.  Therefore we have
      \\[0.2cm]
      \hspace*{1.3cm}
      $w_1 \in \Sigma^*$ \quad and \quad $w_2 \in \Sigma^*$.
\item Assume that $\Sigma = \{\mathtt{a}, \;\cdots\!,\; \mathtt{z}\}$.   If we define
      \\[0.2cm]
      \hspace*{1.3cm}
      $w := \mathtt{example}$,
      \\[0.2cm]
      then we have $w \in \Sigma^*$. \eox
\end{enumerate}
The \emph{length} of a string $w$ is defined as the number of characters composing $w$.
The length of $w$ is written as $|w|$.  We use square brackets to extract the characters from a string:
Given a string $w$ and positive natural number $i \leq |w|$, we agree that $w[i]$ denotes the $i$-th
character of the string $w$.   Note that we start to count at 1.  Therefore,  $w[1]$ is the first
character of the string $w$.  Although it seems quite natural to start counting at $1$, many
programming languages, e.g.~\texttt{C} and \textsl{Java}, instead start counting at $0$.

Next, we define the \emph{concatenation} of two strings $w_1$ and $w_2$ as the string $w$ that
results from appending the string $w_2$ at the end of  $w_1$.  The concatenation of $w_1$ and $w_2$
is written as $w_1 + w_2$ or sometimes even shorter as $w_1w_2$.
\vspace*{0.3cm}

\exampleEng
If $\Sigma = \{\mathtt{0},\mathtt{1}\}$ and, furthermore,  $w_1 = \mathtt{01}$ and $w_2 =
\mathtt{10}$, then we have
\\[0.2cm]
\hspace*{1.3cm}
$w_1+w_2 = \mathtt{0110}$ \quad and \quad $w_2+w_1 = \mathtt{1001}$.  \eox


\begin{Definition}[Formal Language] \hspace*{\fill} \linebreak
If $\Sigma$ is an alphabet, then a subset $L \subseteq \Sigma^*$
is called a \emph{formal language}.  \qed
\end{Definition}

An important topic of this lecture is the question how we can specify formal languages in a way that
these specifications can be manipulated in a computer.


\examplesEng
\begin{enumerate}
\item Assume that $\Sigma = \{\mathtt{0},\mathtt{1}\}$.  Define
      \\[0.2cm]
      \hspace*{1.3cm}
      $L_\mathbb{N} = \{ \mathtt{1}+w \mid w \in \Sigma^* \} \cup \{ \mathtt{0} \}$
      \\[0.2cm]
      Then $L_\mathbb{N}$ is the language consisting of all strings that can be interpreted as
      natural numbers.  The language contains all strings from $\Sigma^*$  that start with 
      the character \texttt{1} as well as the string \texttt{0}, which only contains the character
      \texttt{0} besteht.  For example, we have
      \\[0.2cm]
      \hspace*{1.3cm}
      $\mathtt{100} \in L_\mathbb{N}$, \quad but \quad $\mathtt{010} \not\in L_\mathbb{N}$.
      \\[0.2cm]
      Let us define a function 
      \\[0.2cm]
      \hspace*{1.3cm}
      $\textsl{value}: L_\mathbb{N} \rightarrow \mathbb{N}$
      \\[0.2cm]
      on the set $L_\mathbb{N}$.  We define $\textsl{value}(w)$ by induction on the length of $w$.
      We call $\textsl{value}(w)$ the \emph{interpretation} of $w$.  The idea is that
      $\textsl{value}(w)$ computes the number represented by the string $w$:
      \begin{enumerate}
      \item $\textsl{value}(\mathtt{0}) = 0$, $\textsl{value}(1) = 1$,
      \item $|w| > 0 \rightarrow \textsl{value}(w\mathtt{0}) = 2 \cdot \textsl{value}(w)   $,
      \item $|w| > 0 \rightarrow \textsl{value}(w\mathtt{1}) = 2 \cdot \textsl{value}(w) + 1$.
      \end{enumerate}
\item Again we have $\Sigma = \{0,1\}$. Define the language $L_\mathbb{P}$
      to be the set of all strings from $L_\mathbb{N}$ that are prime numbers:
      \\[0.2cm]
      \hspace*{1.3cm}
      $L_\mathbb{P} := \{ w \in L_\mathbb{N} \mid \textsl{value}(w) \in \mathbb{P} \}$
      \\[0.2cm]
      here, $\mathbb{P}$ denotes the set of \emph{prime numbers}, which is the set of all natural
      numbers $p$ bigger than $1$ that are not evenly divided by any number other than $1$ or $p$:
      \\[0.2cm]
      \hspace*{1.3cm}
      $\mathbb{P} = \bigl\{ p \in \mathbb{N} \;\big|\; \{ t \in \mathbb{N} \mid \exists k \in
      \mathbb{N} : k \cdot t = p \} = \{1, p\} \bigr\}$.
\item Define $\Sigma_{\textsc{\scriptsize Ascii}} = \{ 0, \cdots, 127\}$.  Furthermore, define $L_C$
      as the set of all \texttt{C} function definitions that have the following form:
      \\[0.2cm]
      \hspace*{1.3cm}
      \texttt{char* $f$(char* $x$) \{ $\cdots$ \}}
      \\[0.2cm]
      Therefore,  $L_C$ contains all those strings that can be interpreted as a \texttt{C} function $f$
      such that $f$ takes a single argument which is a string and returns a value which is also a
      string.
\item Define $\Sigma := \Sigma_{\textsc{\scriptsize Ascii}} \cup \{\texttt{\dag}\}$, where
      $\mathtt{\dag}$ is some new symbol that is different form all symbols in
      $\Sigma_{\textsc{\scriptsize Ascii}}$.
      The universal language $L_u$ is the set of all strings of the form
      \\[0.2cm]
      \hspace*{1.3cm}
      \texttt{$p$\dag$x$\dag$y$}
      \\[0.2cm]
      such that
      \begin{enumerate}
      \item $p \in L_C$,
      \item $x,y \in \Sigma_{\textsc{Ascii}}^*$,
      \item if $f$ is the function that is represented by $p$, then $f(x)$ yields the result $y$.
            \eox
      \end{enumerate}
\end{enumerate}
The examples given above demonstrate that the notion of a formal language is quite general.
While it is easy to recognize the strings of the language $L_\mathbb{N}$, it is quite a bit more
difficult to decide whether a string is a member of 
$L_\mathbb{P}$ or $L_C$.  Finally, the can be no algorithm that is able to decide whether a string
$w$ is an element of the language $L_u$. 
\pagebreak

\section{Literature}
In addition to these lecture notes there are two good books that I would like to recommend:
\begin{enumerate}
\item \emph{Introduction to Automata Theory, Languages, and Computation}
      \cite{hopcroft:06}

      This book is the bible on formal languages and it contains all the theoretical results discussed in this lecture.
      Obviously, we will only be able to cover a small part of the results discussed in this book.
\item \emph{Compilers --- Principles, Techniques and Tools}
      \cite{aho:2006}

      This is the bible with respect to the theory of compilers.
\end{enumerate}

\subsection{Programming Languages Used in this Lecture}
The tools presented in this lecture are based on 
\textsl{Java}.  For some of the more complex algorithms discussed in this lecture
it would be a pain to implement them in \textsl{Java}.
Hence, I will implement these algorithms in \textsc{SetlX}.  \textsc{SetlX} is a set-based
programming language which is very suitable for rapid prototyping.  Often, even very complex
algorithms can be implemented in a few lines of \textsc{SetlX}.  You are not expected to be able to
program in \textsc{SetlX} yourself.  However, you are expected to understand programs written in
\textsc{SetlX}.  

\textsc{SetlX} programs are about as concise as the pseudo code found in many books.  However, in
contrast to pseudo code a \textsc{SetlX} program can be executed.  In order to execute
\textsc{SetlX} programs you have to install the \textsc{SetlX} interpreter.  This interpreter is
found at the following location:
\\[0.2cm]
\hspace*{1.3cm}
\href{http://wwwlehre.dhbw-stuttgart.de/~stroetma/SetlX/setlX.php}{\texttt{http://wwwlehre.dhbw-stuttgart.de/\symbol{126}stroetma/SetlX/setlX.php}}
\\[0.2cm]
There is a tutorial explaining the language at the following address:
\\[0.2cm]
\hspace*{1.3cm}
\href{http://wwwlehre.dhbw-stuttgart.de/\symbol{126}stroetma/SetlX/tutorial.pdf}{\texttt{http://wwwlehre.dhbw-stuttgart.de/\symbol{126}stroetma/SetlX/tutorial.pdf}}


%%% Local Variables: 
%%% mode: latex
%%% TeX-master: "formal-languages.tex"
%%% End: 
