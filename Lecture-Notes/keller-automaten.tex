\chapter{Keller-Automaten}
In diesem Kapitel stellen wir die \emph{Keller-Automaten} (engl.~\emph{push-down automata})
vor.  Wir werden sehen, dass dieses Automaten-Modell zu dem Konzept der kontextfreien Sprachen
\"aquivalent ist:  Eine Sprache $L$ ist genau dann kontextfrei, wenn es einen
nicht-deterministischen Keller-Automaten gibt, der die Sprache $L$ erkennt.  Insofern verhalten
sich Keller-Automaten zu den kontextfreien Sprachen genauso, wie sich die endlichen Automaten
zu den regul\"aren Sprachen verhalten.  Im Vergleich zur Theorie der endlichen Automaten hat die
Theorie der Keller-Automaten allerdings einen Sch\"onheitsfehler:  W\"ahrend die deterministischen
endlichen Automaten zu den nicht-deterministischen endlichen Automaten \"aquivalent sind, sind
die nicht-deterministischen Keller-Automaten m\"achtiger als die deterministischen
Keller-Automaten, denn es gibt kontextfreie Sprachen, die sich zwar mit einem
nicht-deterministischen Keller-Automaten erkennen lassen, f\"ur die es aber keinen
deterministischen Keller-Automaten gibt, der diese Sprache erkennt.  Allerdings werden wir
sehen, dass die meisten f\"ur die Praxis interessanten kontextfreien Sprachen bereits von
deterministischen Keller-Automaten erkannt werden.
    
\section{Definition eines Keller-Automaten}
Im Allgemeinen werden Keller-Automaten als endliche Automaten definiert, die zus\"atzlich mit
einem Kellerspeicher (engl.~\emph{stack}) ausgestattet sind, auf dem Werte abgelegt werden k\"onnen.  Der
Zustand eines Automaten setzt sich dann aus zwei Komponenten zusammen:
\begin{enumerate}
\item Dem Inhalt des Kellerspeichers und
\item dem Zustand des zugrunde liegenden endlichen Automaten.
\end{enumerate}
Es zeigt sich aber, dass der Kellerspeicher als Ged\"achtnis bereits ausreichend ist, eine weitere
Komponente wird nicht ben\"otigt.  Wir werden uns in unseren Betrachtungen zun\"achst auf diese Art von
Keller-Automaten beschr\"anken.

\begin{Definition}[Keller-Automat]
  Ein \emph{Keller-Automat} $A$ ist ein 4-Tupel
  \\[0.2cm]
  \hspace*{1.3cm}
  $A = \langle \Sigma, \Gamma, \delta, S_0 \rangle$,
  \\[0.2cm]
  wobei die Komponenten die folgende Bedeutung haben:
  \begin{enumerate}
  \item $\Sigma$ ist das \emph{Eingabe-Alphabet}, der Keller-Automat verarbeitet also W\"orter aus
        $\Sigma^*$.
  \item $\Gamma$ ist das \emph{Keller-Alphabet}.

        In der Praxis ist $\Gamma$ h\"aufig eine Obermenge von $\Sigma$.
  \item $\delta$ ist die \emph{\"Ubergangs-Funktion}.  
        Als Argument bekommt diese Funktion entweder einen Buchstaben
        aus dem Eingabe-Alphabet $\Sigma$ und einen Buchstaben aus dem Keller-Alphabet
        $\Gamma$, oder das Zeichen $\varepsilon$ und einen Buchstaben aus $\Gamma$.
        Als Ausgabe liefert diese Funktion eine Menge von Strings aus dem Keller-Alphabet,
        es gilt also
        \\[0.2cm]
        \hspace*{1.3cm}
        $\delta: (\Sigma \cup \{ \varepsilon \}) \times \Gamma \rightarrow \textsl{Pow}\bigl(\Gamma^*\bigr)$.
        \\[0.2cm]
        Hier bezeichnet $\textsl{Pow}(\Gamma^*)$ die Potenz-Menge von $\Gamma^*$, also die
        Menge aller Teilmengen von $\Gamma^*$ und $\Gamma^*$ bezeichnet die Menge der
        W\"orter \"uber $\Gamma$.
  \item $S_0$ ist das Start-Symbol, das ein Element des Keller-Alphabets $\Gamma$ ist:
        \\[0.2cm]
        \hspace*{1.3cm}
        $S_0 \in \Gamma$.
        \\[0.2cm]
        Zu Beginn der Rechnung eines Keller-Automaten liegt im Kellerspeicher des
        Keller-Automaten genau dieses Start-Symbol und sonst nichts.  \qed
  \end{enumerate}
\end{Definition}

Wie funktioniert nun ein Keller-Automat?  Ein Wort $w \in \Sigma^*$ wird von dem
Keller-Automaten wie folgt verarbeitet. 
\begin{enumerate}
\item Zun\"achst wird in dem leeren Kellerspeicher das Start-Symbol $S_0$ abgelegt.
\item Anschlie{\ss}end wird das Wort Buchstabe f\"ur Buchstabe gelesen, wobei sich der 
      Kellerspeicher des Automaten wie folgt ver\"andern kann:
      \begin{enumerate}
      \item Ist $C$ der Buchstabe, der im Kellerspeicher (also auf dem Stack) ganz oben liegt, 
            und gilt $\alpha \in \delta(\varepsilon, C)$, dann wird der Buchstabe $C$ 
            aus dem Kellerspeicher entfernt und die Buchstaben aus $\alpha$ werden so eingekellert
            dass der erste Buchstabe aus $\alpha$  oben auf dem Stack liegt.
            Da $\alpha$ auch das leere Wort sein kann, kann der Stack dabei auch schrumpfen.

            In diesem Fall wird kein Buchstabe des Wortes $w$ gelesen.
      \item Ist $C$ der Buchstabe, der im Kellerspeicher (also auf dem Stack) ganz oben liegt, ist
            $b$ der n\"achste zu lesende Buchstabe aus dem Wort $w$ und gilt
            $\alpha \in \delta(b, C)$, dann wird der Buchstabe $b$ gelesen, der Buchstabe
            $C$ wird aus dem Kellerspeicher entfernt und anschlie{\ss}end werden die Buchstaben aus
            $\alpha$ eingekellert. 
      \end{enumerate}
      Keller-Automaten sind nicht-deterministisch, denn beispielsweise kann f\"ur       
      $\alpha_1 \not= \alpha_2$ sowohl 
      \\[0.2cm]
      \hspace*{1.3cm}
      $\alpha_1 \in \delta(b,C)$, \quad als auch \quad $\alpha_2 \in \delta(b,C)$ gelten,
      \\[0.2cm]
      und dann hat der Automat die Wahl, ob er $\alpha_1$ oder $\alpha_2$ einkellert.
      Genauso kann es sein, dass gleichzeitig
      \\[0.2cm]
      \hspace*{1.3cm}
      $\alpha_1 \in \delta(\varepsilon,C)$ \quad und \quad $\alpha_2 \in \delta(b,C)$
      \\[0.2cm]
      gilt.  In diesem Fall hat der Automat die M\"oglichkeit, entweder das Zeichen
      $C$ im Kellerspeicher durch das Wort $\alpha_1$ zu ersetzen, ohne dabei einen Buchstaben zu lesen, oder
      der Automat kann den Buchstaben $b$ lesen und das Zeichen $C$ durch das Wort $\alpha_2$
      ersetzen.
\item Sind alle Buchstaben des Wortes $w$ gelesen, und ist gleichzeitig der Kellerspeicher geleert worden,
      dann akzeptiert der Keller-Automat das Wort $w$.
\end{enumerate}
Formal definieren wir die \emph{Konfiguration} eines Keller-Automaten $A = \langle \Sigma, \Gamma, \delta, S_0 \rangle$,
als ein Paar
\\[0.2cm]
\hspace*{1.3cm}
$\langle v, \alpha \rangle \in \Sigma^* \times \Gamma^*$,
\\[0.2cm]
dass aus einem Wort $v$ des Eingabe-Alphabets und einem weiteren Wort
$\alpha$  des Keller-Alphabets besteht.  Die Idee ist, dass $v$ den Teil des Eingabewortes
darstellt, der noch nicht gelesen wurde, w\"ahrend $\alpha$ den Zustand des Kellerspeicher  beschreibt. 
Wir definieren nun f\"ur zwei Konfigurationen $\pair(u,\alpha)$ und $\pair(v,\beta)$ des
Keller-Automaten $A$ eine
Relation 
\\[0.2cm]
\hspace*{1.3cm}
$\pair(u,\alpha) \vdash_A \pair(v,\beta)$
\\[0.2cm]
durch die folgenden beiden Klauseln:
\begin{enumerate}
\item $\alpha \in \delta(\varepsilon, C) \;\rightarrow\; \pair(w, C\beta) \vdash_A \pair(w,\alpha \beta)$.

      Hier hat der Keller-Automat $A$ einen $\varepsilon$-\"Ubergang,
      bei dem das Stack-Symbol $C$ durch den String $\alpha$ ersetzt wird.
      Das Wort $w$ bleibt in diesem Fall unver\"andert.
      
\item $\alpha \in \delta(b, C) \;\rightarrow\; \pair(bu, C\beta) \vdash_A \pair(u,\alpha \beta)$.

      In diesem Fall hat der Keller-Automat $A$ f\"ur den Buchstaben $b$ und das
      Stack-Symbol einen \"Ubergang, bei dem $b$ gelesen und das Stack-Symbol $C$ durch
      den String $\alpha$ ersetzt wird. 
\end{enumerate}
Wir bezeichnen den transitiven und reflexiven Abschluss der Relation $\vdash_A$ mit
$\vdash_A^*$.  Bei Diskussionen, in denen klar ist, um welchen Keller-Automaten $A$ es
sich handelt, lassen wir den Index $A$ auch weg und schreiben einfach $\vdash^*$ an Stelle
von $\vdash_A^*$.  Damit sind wir nun in der Lage, die von einem Keller-Automaten 
\\[0.2cm]
\hspace*{1.3cm}
$A = \langle \Sigma, \Gamma, \delta, S_0 \rangle$
\\[0.2cm]
akzeptierte Sprache $L(A)$ zu definieren:
\\[0.2cm]
\hspace*{1.3cm}
$L(A) := \bigl\{ w \in \Sigma^* \mid \pair(w,S_0) \vdash_A^* \pair(\varepsilon, []) \bigr\}$,
\\[0.2cm]
wobei wir hier mit $[]$ den leeren Kellerspeicher bezeichnen.  Ein Wort $w$ wird also genau dann
von dem Keller-Automaten $A$ akzeptiert,  wenn der Keller-Automat beim Lesen dieses Wortes
seinen Kellerspeicher leeren kann.

\example
Wir definieren den Keller-Automat $A = \langle \Sigma, \Gamma, \delta, S \rangle$ wie
folgt:
\begin{enumerate}
\item $\Sigma := \{ \texttt{a}, \texttt{b} \}$.
\item $\Gamma := \{ \texttt{a}, \texttt{b}, S \}$.
\item $\delta(\varepsilon, \texttt{a}) := \{ \}$, \quad 
      $\delta(\varepsilon, \texttt{b}) := \{ \}$, \quad
      $\delta(\varepsilon, S) := \{ \texttt{a}S\texttt{a}, \texttt{b}S\texttt{b}, \texttt{a}, \texttt{b}, \varepsilon \}$,

      $\delta(\texttt{a}, \texttt{a}) := \{ \varepsilon \}$, \quad
      $\delta(\texttt{a}, \texttt{b}) := \{ \}$, \quad
      $\delta(\texttt{a}, S) := \{ \}$, 

      $\delta(\texttt{b}, \texttt{a}) := \{ \}$, \quad
      $\delta(\texttt{b}, \texttt{b}) := \{ \varepsilon \}$, \quad
      $\delta(\texttt{b}, S) := \{ \}$.

      Beachten Sie, dass es einen subtilen Unterschied zwischen den F\"allen
      \\[0.2cm]
      \hspace*{1.3cm}
      $\delta(\texttt{a}, \texttt{a}) := \{ \varepsilon \}$ \quad und \quad
      $\delta(\texttt{a}, \texttt{b}) := \{ \}$
      \\[0.2cm]
      gibt:  Falls der n\"achste Buchstaben des zu verarbeitenden Wortes ein ``\texttt{a}''
      ist und falls au{\ss}erdem ein ``\texttt{a}'' oben auf 
      dem Stack liegt, dann wird das ``\texttt{a}'' aus dem Wort gelesen und 
      das ``\texttt{a}'' auf dem Stack wird durch das leere Wort $\varepsilon$ ersetzt 
      und damit entfernt. 
      
      Ist hingegen  der n\"achste Buchstaben des zu verarbeitenden Wortes ein ``\texttt{a}''
      und liegt  ein ``\texttt{b}'' oben auf dem Stack, so gibt es f\"ur den Automaten
      keinen \"Ubergang und damit kann der Buchstabe ``\texttt{a}'' nicht gelesen werden.
      In diesem Fall sagen wir auch, dass der Automat \emph{stirbt}.
\item Das Start-Symbol ist $S$.
\end{enumerate}
Ist $w$ ein Wort, so bezeichnen wir mit $w^r$ das
Wort, das aus $w$ entsteht, wenn wir $w$ von hinten lesen.  Der erste Buchstabe von $w$
ist also der letzte Buchstabe von $w^r$, der zweite Buchstabe von $w$ ist der vorletzte
Buchstabe von $w^r$ und allgemein gilt f\"ur ein Wort der L\"ange $n$
\\[0.2cm]
\hspace*{1.3cm}
$|w| = n \;\rightarrow\; \forall i \in \{1,\cdots,n\}:w^r[i] = w[n + 1 - i]$. 
\\[0.2cm]
Hier bezeichnet $|w|$ die L\"ange des Wortes und $w[i]$ bezeichnet den $i$-ten Buchstaben.
W\"orter, f\"ur die $w^r = w$ gilt, werden als \emph{Palindrome} bezeichnet.
Ein einfaches Beispiel ist das Wort 
\\[0.2cm]
\hspace*{1.3cm}
\texttt{aabaa}.
\\[0.2cm]
Wir behaupten, dass der oben definierte Automat $A$ genau die Palindrome erkennt, es gilt also
\\[0.2cm]
\hspace*{1.3cm}
$L(A) = \{ w \in \Sigma^* \mid w = w^r \}$
\\[0.2cm]
Wir zeigen, wie der Automat das Wort 
\\[0.2cm]
\hspace*{1.3cm}
\texttt{aabaa}
\\[0.2cm]
verarbeitet.
\begin{enumerate}
\item Der Automat startet mit der Konfiguration
      \\[0.2cm]
      \hspace*{1.3cm}
      $\pair(\texttt{aabaa},S)$ 
\item Da das oberste Symbol auf dem Stack ein $S$ ist, kann der Automat die Tatsache, dass
      \\[0.2cm]
      \hspace*{1.3cm}
      $\texttt{a}S\texttt{a} \in \delta(\varepsilon,S)$
      \\[0.2cm]
      ausnutzen und einen $\varepsilon$-\"Ubergang durchf\"uhren.  Dabei wird dann die
      folgende Konfiguration erreicht:
      \\[0.2cm]
      \hspace*{1.3cm}
      $\pair(\texttt{aabaa},\texttt{a}S\texttt{a})$
\item Jetzt ist der erste Buchstabe des zu lesenden Wortes ein \texttt{a} und gleichzeitig
      liegt ein \texttt{a} auf dem Stack.  Also verwenden wir
      \\[0.2cm]
      \hspace*{1.3cm}
      $\varepsilon \in \delta(\texttt{a}, \texttt{a})$
      \\[0.2cm]
      und entfernen damit sowohl das erste ``\texttt{a}'' aus dem zu lesenden Wort als
      auch das ``\texttt{a}'' auf dem Stack und erhalten die Konfiguration
      \\[0.2cm]
      \hspace*{1.3cm}
      $\pair(\texttt{abaa}, S\texttt{a})$.
\item Wir benutzen wieder $\texttt{a}S\texttt{a} \in \delta(\varepsilon,S)$ und erhalten
      die Konfiguration
      \\[0.2cm]
      \hspace*{1.3cm}
      $\pair(\texttt{abaa}, \texttt{a}S\texttt{aa})$.
\item Nun verwenden wir $\varepsilon \in \delta(\texttt{a}, \texttt{a})$ und erhalten
      \\[0.2cm]
      \hspace*{1.3cm}
      $\pair(\texttt{baa}, S\texttt{aa})$.
\item Diesmal benutzen wir $\texttt{b} \in \delta(\varepsilon,S)$ und erhalten
      \\[0.2cm]
      \hspace*{1.3cm}
      $\pair(\texttt{baa}, \texttt{baa})$.
\item Jetzt verwenden wir einmal $\varepsilon \in \delta(\texttt{b}, \texttt{b})$ und zweimal
      $\varepsilon \in \delta(\texttt{a}, \texttt{a})$ und haben dann die Konfiguration
      \\[0.2cm]
      \hspace*{1.3cm}
      $\pair(\varepsilon, [])$.
      \\[0.2cm]
      Damit ist  gezeigt, dass $\texttt{aabaa} \in L(A)$ gilt. \qed
\end{enumerate}

\section{Von  kontextfreien Sprachen zu Keller-Automaten}
Die Menge $L_P(\Sigma)$ der Palindrome \"uber einem Alphabet $\Sigma$
\\[0.2cm]
\hspace*{1.3cm}
$L_P(\Sigma) := \bigl\{ w \in \Sigma^* \mid w^r = w \bigr\}$
\\[0.2cm]
kann auch durch eine kontextfreie Grammatik beschrieben werden.  Definieren wir die
kontextfreie Grammatik $G$ durch
\\[0.2cm]
\hspace*{1.3cm}
$G = \langle \bigl\{S\}, \{ \texttt{a},\texttt{b}\}, R, S\bigr\rangle$,
\\[0.2cm]
wobei die Regeln durch
\\[0.2cm]
\hspace*{1.3cm}
$S \rightarrow \texttt{a}S\texttt{a} \mid \texttt{b}S\texttt{b} \mid \texttt{a} \mid \texttt{b} \mid \varepsilon$,
\\[0.2cm]
so kann gezeigt werden, dass gilt:
\\[0.2cm]
\hspace*{1.3cm}
$L(G) = L_P(\{\texttt{a},\texttt{b}\})$.
\\[0.2cm]
Vergleichen wir die Definition der Grammatik mit der Definition des Keller-Automaten aus
dem letzten Abschnitt, so erkennen wir ein Muster, dass wir jetzt allgemein formulieren.

\begin{Definition}[$A(G)$]
  Es sei $G = \langle V, T, R, S \rangle$ eine kontextfreie Grammatik.  Dann definieren
  wir den \emph{von $G$ erzeugten Keller-Automaten $A(G)$} wie folgt:
  \\[0.2cm]
  \hspace*{1.3cm}
  $A(G) = \langle T, V \cup T, \delta, S \rangle$,
  \\[0.2cm]
  wobei die \"Ubergangs-Funktion $\delta$ durch die folgenden Klauseln definiert wird.
  \begin{enumerate}
  \item F\"ur jede Regel $A \rightarrow \beta$ aus der Grammatik $G$ gilt $\beta \in \delta(\varepsilon, A)$,
        denn wir definieren
        \\[0.2cm]
        \hspace*{1.3cm}
        $\delta(\varepsilon, A) := \bigl\{ \beta \mid (A \rightarrow \beta) \in R \bigr\}$
        \quad f\"ur alle $A \in V$.
  \item F\"ur jeden Buchstaben $b \in T$ gilt $\varepsilon \in \delta(b,b)$, denn wir definieren
        \\[0.2cm]
        \hspace*{1.3cm}
        $\delta(b,c) := \left\{
        \begin{array}{ll}
          \{ \varepsilon \} & \mbox{falls $b = c$;} \\
          \{\}              & \mbox{sonst.}
        \end{array}\right.
        $
        \qed
  \item In allen anderen F\"allen liefert die Funktion $\delta$ als Ergebnis die leere Menge.
        \begin{enumerate}
        \item $\forall b \in T: \delta(\varepsilon, b) = \{\}$,
        \item $\forall b \in T, A \in V: \delta(b, A) = \{\}$.
        \end{enumerate}
  \end{enumerate}
\end{Definition}

\remark
Der im letzten Abschnitt angegebene Automat $A$ ist genau der von der oben angegebenen
Grammatik $G$ erzeugte Automat $A(G)$.
\vspace*{0.3cm}

Um zeigen zu k\"onnen, dass der Automat $A(G)$ genau die Sprache akzeptiert, die von der
Grammatik $G$ beschrieben wird, ben\"otigen wir den Begriff der
\emph{Links-Ableitung} :  Informal ist das ein Ableitungsschritt, bei dem die
linkeste Variable durch eine Regel ersetzt wird.  Formal definieren wir diesen Begriff wie
folgt:  
\begin{enumerate}
\item Es sei $G = \langle T, V, R, S \rangle$ eine kontextfreie Grammatik. Weiter sei 
\item $A \in V$ eine syntaktische Variable,
\item $u A \beta \in (V \cup T)^*$ sei ein String aus Variablen und Terminalen mit $u \in
  T^*$ \ und 
\item $(A \rightarrow \gamma) \in R$ sei eine Regel der Grammatik $G$.
\end{enumerate}
Dann schreiben wir \\[0.2cm]
\hspace*{1.3cm}
$u A \beta \lmderiv u \gamma\beta$. 
\\[0.2cm]
und  sagen, dass das Wort $u\gamma\beta$ durch einen Schritt
einer \emph{Links-Ableitung} aus dem Wort $uA\beta$ hervorgegangen ist.
Der entscheidende Unterschied gegen\"uber einem normalen Ableitungs-Schritt ist der, dass
alle Zeichen, die links von der ersetzten Variablen $A$ stehen, Terminale sein m\"ussen: 
$u \in T^*$.  Ist allgemein $w$ ein Wort einer Sprache $L(G)$ und ist $S$ das
Start-Symbol, dann erhalten wir eine \emph{Links-Ableitung} von $w$, wenn wir eine
Ableitung w\"ahlen, bei der in jedem Schritt die linkeste Variable durch die rechte Seite
einer Grammatik-Regel ersetzt wird.

\example
Eine Grammatik 
$G = \langle \{ E, P, F\}, \{ \squoted{+}, \squoted{*}, \squoted{1}, \squoted{2}, \squoted{3} \}, R, E \rangle$ 
f\"ur arithmetische Ausdr\"ucke habe die folgenden Regeln $R$:
\\[0.2cm]
\hspace*{1.3cm}
$E \rightarrow E \quoted{+} P \;|\; P$, \quad
$P \rightarrow P \quoted{*} F \;|\; F$, \quad
$F \rightarrow \quoted{1} \;|\; \quoted{2} \;|\; \quoted{3}$.
\\[0.2cm]
Der String ``\texttt{1+2*3}'' hat dann die folgende Links-Ableitung:
\\[0.2cm]
\hspace*{1.3cm}
$
\begin{array}[t]{lcl}
  E & \lmderiv & E \quoted{+} P \\[0.3cm]
    & \lmderiv & P \quoted{+} P \\[0.3cm]
    & \lmderiv & F \quoted{+} P \\[0.3cm]
    & \lmderiv & \quoted{1} \quoted{+} P \\[0.3cm]
    & \lmderiv & \quoted{1} \quoted{+} P \quoted{*} F \\[0.3cm]
    & \lmderiv & \quoted{1} \quoted{+} F \quoted{*} F \\[0.3cm]
    & \lmderiv & \quoted{1} \quoted{+} \quoted{2} \quoted{*} F \\[0.3cm]
    & \lmderiv & \quoted{1} \quoted{+} \quoted{2} \quoted{*} \quoted{3} 
\end{array}
$
\\[0.2cm]
Die folgende Ableitung ist hingegen keine Links-Ableitung, denn im zweiten Schritt wird
zum Beispiel nicht die linkeste Variable $E$ sondern stattdessen die Variable $P$ ersetzt:
\\[0.2cm]
\hspace*{1.3cm}
$
\begin{array}[t]{lcl}
  E & \Rightarrow & E \quoted{+} P \\[0.3cm]
    & \Rightarrow & E \quoted{+} P \quoted{*} F \\[0.3cm]
    & \Rightarrow & P \quoted{+} P \quoted{*} F \\[0.3cm]
    & \Rightarrow & P \quoted{+} P \quoted{*} \quoted{3} \\[0.3cm]
    & \Rightarrow & F \quoted{+} P \quoted{*} \quoted{3} \\[0.3cm]
    & \Rightarrow & \quoted{1} \quoted{+} P \quoted{*} \quoted{3} \\[0.3cm]
    & \Rightarrow & \quoted{1} \quoted{+} F \quoted{*} \quoted{3} \\[0.3cm]
    & \Rightarrow & \quoted{1} \quoted{+} \quoted{2} \quoted{*} \quoted{3}  \\[0.3cm]
\end{array}
$

\remark
Falls wir ein Wort $w$ aus dem Start-Symbol $S$ einer Grammatik $G = \langle V, T, R, S \rangle$ ableiten
k\"onnen, so k\"onnen wir die Ableitungs-Schritte immer so umsortieren, dass wir hinterher eine Links-Ableitung 
haben. Daher gilt
\\[0.2cm]
\hspace*{1.3cm}
$L(G) = \{ w \in T^* \mid S \;
           \raisebox{-2.3mm}{$\stackrel{\mbox{\large$\Rightarrow$}}{\mbox{\scriptsize l}}^*$}\; w \}$.

\begin{Satz}
  Ist $G = \langle V, T, R, S \rangle$ eine kontextfreie Grammatik, so gilt
  \\[0.2cm]
  \hspace*{1.3cm}
  $L\bigl(A(G)\bigr) = L(G)$,
  \\[0.2cm]
  der von $G$ erzeugte Automat erkennt also genau die Sprache, die durch $G$ beschrieben wird.
\end{Satz}

\noindent
\textbf{Beweis}:  \ Wir zeigen zun\"achst, dass
\\[0.2cm]
\hspace*{1.3cm}
$L(G) \subseteq L\bigl(A(G)\bigr)$
\\[0.2cm]
gilt.  Wir nehmen also an, dass $w \in L(G)$ ist und m\"ussen zeigen, dass der Automat $A(G)$
das Wort $w$ akzeptiert.  Wir nehmen nun eine Ableitung des Wortes $w$, bei der immer die 
linkeste Variable ersetzt wird.  Eine solche Ableitung hat die Form
\\[0.2cm]
\hspace*{1.3cm}
$S = u_1A_1\beta_1 \lmderiv u_2A_2\beta_2 \lmderiv \cdots \lmderiv u_nA_n\beta_n \lmderiv w$,
\\[0.2cm]
wobei gilt
\begin{enumerate}
\item $u_1 = \varepsilon$, $A_1 = S$, $\beta_1 = \varepsilon$.
\item $u_i \in T^*$ \quad f\"ur alle $i=1,\cdots,n$

      links von den syntaktischen Variablen $A_i$ stehen also nur Terminale.
\item $A_i \in V$ \quad f\"ur alle $i=1,\cdots,n$,

      $A_i$ ist also die syntaktische Variable in dem String $u_iA_i\beta_i$, die am
      weitesten links steht. 
\end{enumerate}
Wir zeigen nun durch Induktion \"uber $i$, dass es f\"ur alle $i=1,\cdots,n$  Strings $v_i \in T^*$ 
gibt, so dass 
\\[0.2cm]
\hspace*{1.3cm}
$\pair(w,S) \vdash^* \pair(v_i, A_i\beta_i)$ und $u_iv_i = w$
\\[0.2cm]
gilt, der String $v_i$ ist also der Rest, der von dem String $w$ \"ubrig bleibt, wenn das Pr\"afix
$u_i$ entfernt wird.
\begin{enumerate}
\item[I.A.] $i=1$: Die Start-Konfiguration ist 
            \\[0.2cm]
            \hspace*{1.3cm}
            $\pair(w,S) = \pair(w,u_1A_1\beta_1)$
            \\[0.2cm]
            und damit muss gelten
            \\[0.2cm]
            \hspace*{1.3cm}
            $u_1 = \varepsilon$, \quad $A_1 = S$, \quad und \quad $\beta_1 = \varepsilon$.  
            \\[0.2cm]
            Also setzen wir $v_1 := w$ und haben trivialerweise
            \\[0.2cm]
            \hspace*{1.3cm}
            $\pair(w,S) \vdash^* \pair(w,S) = \pair(v_1,u_1A_1\beta_1)$ \quad und $u_1v_1 = w$.
\item[I.S.] $i \mapsto i + 1$: 
            Nach Induktions-Voraussetzung wissen wir, dass
            \begin{equation}
              \label{eq:e1}
            \pair(w, S) \vdash^* \pair(v_i,A_i\beta_i) \quad \mbox{und} \quad u_iv_i = w
            \end{equation}
            gilt.  Wegen
            \begin{equation}
              \label{eq:e2}
            u_iA_i\beta_i \lmderiv u_{i+1}A_{i+1}\beta_{i+1} \quad \mbox{und}\quad u_i \in T^*
            \end{equation}
            muss es eine Regel 
            \\[0.2cm]
            \hspace*{1.3cm}
            $(A_i \rightarrow \gamma_i) \in R$
            \\[0.2cm]
            geben, so dass bei dem Ableitungs-Schritt $A_i$ durch $\gamma_i$ ersetzt
            worden ist.  Damit gilt also
            \begin{equation}
              \label{eq:e3}
            u_{i+1}A_{i+1}\beta_{i+1} = u_i\gamma_i\beta_i.              
            \end{equation}
            Wegen (\ref{eq:e2}) muss $u_i$ ein Pr\"afix von $u_{i+1}$ sein.  Also gibt es einen
            String $t_{i+1}\in T^*$, so dass
            \\[0.2cm]
            \hspace*{1.3cm}
            $u_i\, t_{i+1} = u_{i+1}$
            \\[0.2cm]
            gilt.  Schneiden wir in Gleichung (\ref{eq:e3}) auf beiden Seiten den Pr\"afix
            $u_i$ ab, so erhalten wir
            \begin{equation}
              \label{eq:e4}
              \gamma_i \beta_i = t_{i+1} A_{i+1}\beta_{i+1}.
            \end{equation}
            Aufgrund der Linksableitung
            \\[0.2cm]
            \hspace*{1.3cm}
            $S \lmderiv\!\!^* u_{i+1}A_{i+1}\beta_{i+1} \lmderiv\!\!^* w$,
            \\[0.2cm] 
            muss $u_{i+1}$ ein Pr\"afix von $w$ sein und daher gibt es genau ein $v_{i+1} \in T^*$, so dass
            \\[0.2cm]
            \hspace*{1.3cm}
            $u_{i+1} v_{i+1} = w$ 
            \\[0.2cm]
            gilt.  Setzen wir hier einerseits $u_{i+1} = u_i t_{i+1}$ und andererseits $w = u_iv_i$ ein, so
            finden wir 
            \\[0.2cm]
            \hspace*{1.3cm}
            $u_i t_{i+1} v_{i+1} = u_i v_i$.
            \\[0.2cm]
            Schneiden wir auf beiden Seiten dieser Gleichung $u_i$ ab, so folgt
            \\[0.2cm]
            \hspace*{1.3cm}
            $t_{i+1} v_{i+1} = v_i$.
            \\[0.2cm]
            Nach Konstruktion des Keller-Automaten $A(G)$ folgt aus $(A_i \rightarrow \gamma_i) \in R$
            \\[0.2cm]
            \hspace*{1.3cm}
            $\gamma_i \in \delta(\varepsilon, A_i)$
            \\[0.2cm]
            und nach Definition eines Keller-Automaten gilt
            \begin{equation}
              \label{eq:e5}
              \pair(v_i,A_i \beta_i) \vdash \pair(v_i,\gamma_i \beta_i).
            \end{equation}
            Wir haben oben gesehen, dass
            $v_i = t_{i+1}v_{i+1}$ gilt und Gleichung (\ref{eq:e4}) zeigt $\gamma_i \beta_i = t_{i+1}A_{i+1}\beta_{i+1}$.
            Also k\"onnen wir  (\ref{eq:e5}) auch als
            \begin{equation}
              \label{eq:e6}
              \pair(v_i,A_i \beta_i) \vdash \pair(t_{i+1}v_{i+1},t_{i+1}A_{i+1} \beta_{i+1})
            \end{equation}
            schreiben.  Nun gilt nach Kunstruktion von $A(G)$ f\"ur alle Buchstaben
            $b \in T$
            \\[0.2cm]
            \hspace*{1.3cm}
            $\varepsilon \in \delta(b,b)$.
            \\[0.2cm]
            Damit haben wir
            \\[0.2cm]
            \hspace*{1.3cm}
            $\pair(t_{i+1}v_{i+1},t_{i+1}A_{i+1} \beta_{i+1}) \vdash^* \pair(v_{i+1},A_{i+1} \beta_{i+1})$
            \\[0.2cm]
            und zusammen mit (\ref{eq:e1}) und \ref{eq:e6}) folgt daraus
            \\[0.2cm]
            \hspace*{1.3cm}
            $\pair(w,S) \vdash^* \pair(v_{i+1},A_{i+1} \beta_{i+1})$,
            \\[0.2cm]
            was f\"ur den Induktions-Schritt zu zeigen war und die Induktion ist abgeschlossen.
\end{enumerate}
Setzen wir oben $i=n$, so haben wir gezeigt, dass
\begin{equation}
  \label{eq:e7}
  \pair(w,S) \vdash^* \pair(v_n, A_n\beta_n) \quad \mbox{und} \quad u_nv_n = w
\end{equation}
gilt und wissen au{\ss}erdem, dass 
\\[0.2cm]
\hspace*{1.3cm}
$u_n A_n \beta_n \lmderiv w$
\\[0.2cm]
gilt. Also gibt es eine Regel $A_n \rightarrow \gamma_n$ in $R$ und es muss
\\[0.2cm]
\hspace*{1.3cm}
$u_n \gamma_n \beta_n = w$
\\[0.2cm]
gelten.  Daraus folgt aber wegen $u_n v_n = w$
\\[0.2cm]
\hspace*{1.3cm}
$\gamma_n \beta_n = v_n$.
\\[0.2cm]
Nach Konstruktion von $A(G)$ gilt $\gamma_n \in \delta(\varepsilon, A_n)$ und wir
haben
\\[0.2cm]
\hspace*{1.3cm}
$\pair(v_n, A_n\beta_n) \vdash \pair(v_n, \gamma_n \beta_n) = \pair(v_n, v_n)$.
\\[0.2cm]
Da nach Kunstruktion von $A(G)$ f\"ur alle Buchstaben
$b \in T$ gilt $\varepsilon \in \delta(b,b)$,  haben wir
\\[0.2cm]
\hspace*{1.3cm}
$\pair(v_n, v_n) \vdash^* \pair(\varepsilon, [])$
\\[0.2cm]
und damit ist insgesamt
\\[0.2cm]
\hspace*{1.3cm}
$\pair(w,S) \vdash^* \pair(\varepsilon, [])$
\\[0.2cm]
gezeigt und wir sehen, dass $w \in L\bigl(A(G)\bigr)$ gilt.
\vspace*{0.3cm}

Um den Beweis abzuschlie{\ss}en, zeigen wir nun die umgekehrte Richtung und weisen 
\\[0.2cm]
\hspace*{1.3cm}
$L\bigl(A(G)\bigr) \subseteq L(G)$
\\[0.2cm]
nach.  Wir nehmen also an, dass $w \in L\bigl(A(G)\bigr)$ gilt und weisen nach, dass
daraus $w \in L(G)$ folgt.
Erinnern wir uns an die Definition der von einem Keller-Automaten akzeptierten
Sprache, sowie an die Definition von $L(G)$, so sagt die Voraussetzung $w \in
L\bigl(A(G)\bigr)$ dass aus
\\[0.2cm]
\hspace*{1.3cm}
$\pair(w,S) \vdash^* \pair(\varepsilon,[])$ \quad die Konklusion \quad $S \Rightarrow^* w$
\\[0.2cm]
folgt.  Wir m\"ussen also die Beziehung
\\[0.2cm]
\hspace*{1.3cm}
$\pair(w,S) \vdash^* \pair(\varepsilon,[]) \;\rightarrow\; S \Rightarrow^* w$
\\[0.2cm]
nachweisen.  Um diese Behauptung zeigen zu k\"onnen, verallgemeinern wir sie:  Wir werden
f\"ur jede syntaktische Variable $X$ und jedes Wort $w \in T^*$ zeigen, dass
\\[0.2cm]
\hspace*{1.3cm}
$\pair(w,X) \vdash^* \pair(\varepsilon,[]) \;\rightarrow\; X \Rightarrow^* w$
\\[0.2cm]
gilt.  Der Beweis dieser Behauptung erfolgt durch Induktion \"uber die Anzahl $n$ der
Berechnungs-Schritte des Keller-Automaten $A(G)$.
\begin{enumerate}
\item[I.A.] $n=1$: $X$ kann sowohl eine syntaktische Variable als auch ein Terminal sein.
            Wir behandeln diese beiden F\"alle getrennt:
            \begin{enumerate}
            \item $X \in V$, also ist $X$ eine syntaktische Variable.

                  Die Zustands-\"Uberg\"ange des Keller-Automaten $A(G)$, bei denen ein Buchstabe
                  des Wortes gelesen wird, setzen alle voraus, dass auf dem Stack 
                  derselbe Buchstabe liegt.
                  Falls der Keller-Automaten nur einen Schritt braucht, um von der
                  Konfiguration $\pair(w,X)$ zur Konfiguration $\pair(\varepsilon,[])$,
                  so kann also kein Buchstabe gelesen worden sein.
                  Daher muss $w = \varepsilon$ gelten und andererseits muss
                  die Grammatik $G$ eine Regel der Form
                  \\[0.2cm]
                  \hspace*{1.3cm}
                  $X \rightarrow \varepsilon$
                  \\[0.2cm]
                  enthalten.  Daraus folgt dann aber sofort
                  \\[0.2cm]
                  \hspace*{1.3cm}
                  $X \Rightarrow \varepsilon$
                  \\[0.2cm]
                  und wegen $w= \varepsilon$ ist das die Behauptung.
            \item $X \in T$, also ist $X$ ein Terminal.

                  Nach Konstruktion des Keller-Automaten $A(G)$ folgt aus
                  der Voraussetzung
                  \\[0.2cm]
                  \hspace*{1.3cm}
                  $\pair(w,X) \rightarrow \pair(\varepsilon,[])$,
                  \\[0.2cm]
                  dass das Wort $w$ nur aus dem Buchstaben $X$ besteht: $w = X$.  Dann
                  gilt trivialerweise
                  \\[0.2cm]
                  \hspace*{1.3cm}
                  $X \Rightarrow^* w$.
            \end{enumerate}
\item[I.S.] $1,\cdots, n \mapsto n+1$:  Da nun bei der Rechnung 
            \\[0.2cm]
            \hspace*{1.3cm}
            $\pair(w,X) \vdash^* \pair(\varepsilon,[])$
            \\[0.2cm]
            mehr als ein Schritt durchgef\"uhrt wird, kann $X$ nur eine syntaktische Variable sein.            
            Wir nehmen also an, dass der Automat die Rechnung
            \\[0.2cm]
            \hspace*{1.3cm}
            $\pair(w,X) \vdash^* \pair(\varepsilon,[])$
            \\[0.2cm]
            in $n+1$ Schritten durchf\"uhrt.  Da $X$ eine Variable ist,
            muss der erste Schritt der Rechnung des Keller-Automaten $A(G)$ die Form
            \\[0.2cm]
            \hspace*{1.3cm}
            $\pair(w,X) \vdash \pair(w,\gamma)$
            \\[0.2cm]
            haben, wobei $X \rightarrow \gamma$ eine Regel der Grammatik sein muss.  Wir
            zerlegen die rechte Seite $\gamma$ dieser Regel in ihre Buchstaben und schreiben
            \\[0.2cm]
            \hspace*{1.3cm}
            $\gamma = y_1 y_2 \cdots y_k$, \quad mit $y_i \in T \cup V$.
            \\[0.2cm]
            Da insgesamt 
            \\[0.2cm]
            \hspace*{1.3cm}
            $\pair(w,X) \vdash^* \pair(\varepsilon, [])$ 
            \\[0.2cm]
            gilt, wissen wir 
            \\[0.2cm]
            \hspace*{1.3cm}
            $\pair(w,y_1 y_2 \cdots y_k) \vdash^* \pair(\varepsilon, [])$,
            \\[0.2cm]
            wobei der Keller-Automat insgesamt $n$ Schritte durchf\"uhrt.  Offenbar entfernt
            der Keller-Automat $A(G)$ nacheinander die Zeichen $y_1 y_2 \cdots y_k$ vom
            Stack und liest dabei das Wort $w$.  Damit k\"onnen wir die Rechnung des
            Keller-Automaten wie folgt aufspalten:
            \\[0.2cm]
            \hspace*{1.3cm}
            $\pair(w,y_1 y_2 \cdots y_k) \vdash^* \pair(w_2,y_2 \cdots y_k) \vdash^*
            \cdots \vdash^* \pair(w_i, y_i \cdots y_k) \vdash^* \cdots \vdash^* \pair(\varepsilon, [])$,
            \\[0.2cm]
            Dabei bezeichnet $w_i$ den Teil des Wortes $w$, der noch nicht gelesen ist,
            wenn die Zeichen $y_1 \cdots y_{i-1}$ vom Stack entfernt worden sind.
            Offenbar k\"onnen wir 
            das Wort $w$ so in Teilstrings $x_1 \cdots x_k$ zerlegen, dass $x_i$ der
            Teilstring von $w$ ist, der von dem Automaten gelesen wird, um das Zeichen
            $y_i$ vom Stack zu entfernen, es gilt also
            \\[0.2cm]
            \hspace*{1.3cm}
            $w = x_1 x_2 \cdots x_k$ \quad und \quad $w_i = x_i x_{i+1} \cdots x_k$.
            \\[0.2cm]
            Daraus folgt aber, dass das Lesen von $x_i$ das Symbol $y_i$ vom Stack
            entfernt:
            \begin{equation}
              \label{eq:e8}
              \pair(\underbrace{x_i x_{i+1} \cdots x_k}_{w_i}, y_i y_{i+1} \cdots y_{k})
              \vdash^* \pair(\underbrace{x_{i+1} \cdots x_k}_{w_{i+1}}, y_{i+1} \cdots y_{k})
            \end{equation}
            Da der Keller-Automat immer nur das oberste Zeichen auf dem Stack anschauen
            kann, k\"onnen die Zeichen $y_{i+1} \cdots y_{k}$ bei dieser Rechnung keine
            Rolle spielen, und genauso spielt auch der Teil des Strings, der bei dieser
            Rechnung nicht gelesen wird, keine Rolle.  Damit folgt aus (\ref{eq:e8}), dass 
            \begin{equation}
              \label{eq:e9}
              \pair(x_i, y_i) \vdash^* \pair(\varepsilon, [])                          
            \end{equation}
            gilt und f\"ur diese Rechnung ben\"otigt der Keller-Automat genausoviele
            Berechnungs-Schritte wie in (\ref{eq:e8}) und das sind h\"ochstens $n$ Schritte.
            Damit k\"onnen wir auf (\ref{eq:e9}) die Induktions-Voraussetzung anwenden und es gilt
            \begin{equation}
              \label{eq:e10}
              y_i \Rightarrow^* x_i
            \end{equation}
            Damit haben wir insgesamt folgende Ableitung von $w$:
            \begin{eqnarray*}            
            X & \Rightarrow\;\; & \gamma = y_1 y_2 \cdots y_k               \\
              & \Rightarrow^* & x_1 y_2 \cdots y_k                        \\
              &               & \vdots                                    \\
              & \Rightarrow^* & x_1 \cdots x_i y_{i+1} y_{i+1} \cdots y_k \\
              & \Rightarrow^* & x_1 \cdots x_i x_{i+1} y_{i+2} \cdots y_k \\
              &               & \vdots                                    \\
              & \Rightarrow^* & x_1 \cdots x_k = w  
            \end{eqnarray*}
            und haben damit die Induktion und den Beweis abgeschlossen. \qed
\end{enumerate}

\exercise
Die Grammatik $G$ sei wie folgt gegeben:
\\[0.2cm]
\hspace*{1.3cm}
$G = \langle \{S,A\}, \{ \texttt{x}, \texttt{y}\}, R, S \rangle$,
\\[0.2cm]
wobei die Regeln $R$ wie folgt gegeben sind:
\begin{eqnarray*}
  S & \rightarrow & \texttt{x}S\texttt{y} \mid A \\
  A & \rightarrow & \texttt{y}A\texttt{x} \mid S \mid \varepsilon
\end{eqnarray*}
Konstruieren Sie einen Keller-Automaten, der die Sprache $L(G)$ akzeptiert.
\pagebreak

\section{Von Keller-Automaten zu kontextfreien Sprachen}
Nach dem letzten Abschnitt wissen wir, dass das Konzept der Keller-Automaten mindestens so
m\"achtig ist, wie das Konzept der kontextfreien Sprachen.
Wir zeigen in diesem Abschnitt die umgekehrte Richtung und weisen nach, dass alle
Sprachen, die von Keller-Automaten erkannt werden k\"onnen, auch durch eine Grammatik
beschrieben werden k\"onnen.

\begin{Satz} \label{satz:keller2grammar}
  Es sei $A = \langle \Sigma, \Gamma, \delta, S \rangle$ ein Keller-Automat.  Wir nehmen
  ohne Beschr\"ankung der Allgemeinheit an, dass die Mengen $\Sigma$ und $\Gamma$ disjunkt
  sind.  Dies k\"onnen wir immer erreichen, indem wir die Symbole aus $\Gamma$ umbenennen.
  Dann konstruieren wir eine Grammatik 
  \\[0.2cm]
  \hspace*{1.3cm}
  $G(A) := \langle \Gamma, \Sigma, R, S \rangle$,
  \\[0.2cm]
  indem wir die Menge $R$ der Regeln wie folgt definieren:
  \begin{enumerate}
  \item Ist $b \in \Sigma$, $X \in \Gamma$ und gilt $\gamma \in \delta(b,X)$,
        dann enth\"alt $R$ die Regel
        \\[0.2cm]
        \hspace*{1.3cm}
        $X \rightarrow b\gamma$.
  \item Ist $X \in \Gamma$ und gilt $\gamma \in \delta(\varepsilon, X)$,
        dann enth\"alt $R$ die Regel 
        \\[0.2cm]
        \hspace*{1.3cm}
        $X \rightarrow \gamma$.
  \end{enumerate}
  Insgesamt setzen wir also
  \\[0.2cm]
  \hspace*{1.3cm}
  $R := \bigl\{ X \rightarrow b\gamma \mid b\in\Sigma \wedge X\in\Gamma \wedge \gamma \in \delta(b,X) \bigr\} \cup 
        \bigl\{ X \rightarrow \gamma \mid X\in\Gamma \wedge \gamma \in \delta(\varepsilon,X) \bigr\}$.
  \\[0.2cm]
  Dann gilt 
  \\[0.2cm]
  \hspace*{1.3cm}
  $L\bigl(G(A)\bigr) = L(A)$.
\end{Satz}

\noindent
\textbf{Beweis}:
Wir zerlegen den Beweis in zwei Teile und zeigen zun\"achst, dass
\\[0.2cm]
\hspace*{1.3cm}
$L\bigl(G(A)\bigr) \subseteq L(A)$
\\[0.2cm]
gilt.  Diese Behauptung ist \"aquivalent zu
\\[0.2cm]
\hspace*{1.3cm}
$S \Rightarrow^* w \;\rightarrow\; \pair(w,S) \vdash^* \pair(\varepsilon, [])$
\\[0.2cm]
Wir zeigen eine etwas allgemeinere Behauptung.  Wir zeigen durch Induktion nach $n$, dass
f\"ur alle $X \in \Gamma$ und $w \in \Sigma^*$ folgendes gilt:
\\[0.2cm]
\hspace*{1.3cm}
$X \Rightarrow^n w \;\rightarrow\; \pair(w,X) \vdash^* \pair(\varepsilon, [])$
\\[0.2cm]
Hier dr\"uckt die Notation $X \Rightarrow^n w$ aus, dass das Wort $w$ aus der Variablen $X$ 
in $n$ Schritten abgeleitet werden kann.  
\begin{enumerate}
\item[I.A.] $n = 1$.  Dann gibt es in der Grammatik $G(A)$ eine Regel der Form 
            \\[0.2cm]
            \hspace*{1.3cm}
            $X \rightarrow w$
            \\[0.2cm]
            Es gibt zwei M\"oglichkeiten, wie diese Regel entstanden sein kann.
            \begin{enumerate}
            \item $w = b \gamma$ mit $b \in \Sigma$ und $\gamma \in \delta(b,X)$.

                  Einerseits gilt $w \in \Sigma^*$, andererseits folgt aus $\gamma \in \delta(b,X)$,
                  dass $\gamma \in \Gamma^*$ gelten muss.  Da $\gamma$ ein Suffix von $w$ ist, gilt
                  auch $\gamma \in \Sigma^*$, also insgesamt
                  \\[0.2cm]
                  \hspace*{1.3cm}
                  $\gamma \in \Sigma^* \cap \Gamma^*$.
                  \\[0.2cm]
                  Nun haben wir aber vorausgesetzt, dass das Eingabe-Alphabet $\Sigma$ und das
                  Stack-Alphabet $\Gamma$ disjunkt sind.  Damit ist die obige Gleichung nur m\"oglich,
                  falls
                  \\[0.2cm]
                  \hspace*{1.3cm}
                  $\gamma = \varepsilon$
                  \\[0.2cm]
                  gilt.  Also haben wir insgesamt
                  \\[0.2cm]
                  \hspace*{1.3cm}
                  $w = b \in \Sigma$ \quad und \quad $\varepsilon \in \delta(b,X)$.
                  \\[0.2cm]
                  Daraus folgt aber sofort
                  \\[0.2cm]
                  \hspace*{1.3cm}
                  $\pair(b, X) \vdash \pair(\varepsilon, [])$.
            \item $w = \gamma$ und $\gamma \in \delta(\varepsilon, X)$.

                  Einerseits gilt dann $w \in \Sigma^*$, andererseits folgt aus 
                  $\gamma \in \delta(\varepsilon,X)$, dass $\gamma \in \Gamma^*$ gelten muss und genau wie im 
                  ersten Fall k\"onnen wir daraus auf \\[0.2cm]
                  \hspace*{1.3cm}
                  $\gamma = \varepsilon$
                  \\[0.2cm]
                  schlie{\ss}en. Daraus folgt dann
                  \\[0.2cm]
                  \hspace*{1.3cm}
                  $\pair(\varepsilon, X) \vdash \pair(\varepsilon, [])$.
            \end{enumerate}
\item[I.S.] $n \mapsto n+1$.  Wieder gibt es zwei M\"oglichkeiten f\"ur die erste verwendete Regel.

            \begin{enumerate}
            \item Die erste bei der Ableitung $X \Rightarrow^{n+1} w$ verwendete Regel hat die Form
                  \\[0.2cm]
                  \hspace*{1.3cm} $X \rightarrow b\gamma$ \quad mit $b \in \Sigma$ und $\gamma \in \delta(b,X)$.
                  \\[0.2cm]
                  Insgesamt haben wir dann
                  \\[0.2cm]
                  \hspace*{1.3cm} $X \Rightarrow b \gamma \Rightarrow^n w$, \quad also $w = bv$ mit $\gamma
                  \Rightarrow^n v$.
                  \\[0.2cm]
                  Dann haben $\gamma$ und $v$ die Form
                  \\[0.2cm]
                  \hspace*{1.3cm} $\gamma = Y_1 \cdots Y_k$, \quad $v = u_1 \cdots u_k$ \quad und \quad $Y_i
                  \Rightarrow^* u_i$ \quad f\"ur alle $i=1,\cdots,k$.
                  \\[0.2cm]
                  Auf $Y_i \Rightarrow^* u_i$ k\"onnen wir die Induktions-Voraussetzung anwenden und sehen, dass
                  \\[0.2cm]
                  \hspace*{1.3cm} 
                  $\pair(u_i, Y_i) \vdash^* \pair(\varepsilon, [])$ \quad f\"ur alle $i=1,\cdots,k$
                  \\[0.2cm]
                  gilt. Setzen wir diese Rechnungen zusammen, so haben wir
                  \\[0.2cm]
                  \hspace*{1.3cm} 
                  $\pair(u_1 \cdots u_k, Y_1 \cdots Y_k) \vdash^* \pair(\varepsilon, [])$, \quad also 
                  \quad $\pair(v,\gamma) \vdash^* \pair(\varepsilon, [])$.
                  \\[0.2cm]
                  Damit finden wir insgesamt
                  \\[0.2cm]
                  \hspace*{1.3cm}
                  $\pair(w,X) = \pair(bv,X) \vdash \pair(v,\gamma) \vdash^* \pair(\varepsilon, [])$
                  \\[0.2cm]
                  und das war die Behauptung.
            \item Die erste bei der Ableitung $X \Rightarrow^{n+1} w$ verwendete Regel hat die Form
                  \\[0.2cm]
                  \hspace*{1.3cm} $X \rightarrow \gamma$ \quad mit $\gamma \in \delta(\varepsilon,X)$.
                  \\[0.2cm]
                  Insgesamt haben wir dann
                  \\[0.2cm]
                  \hspace*{1.3cm} $X \Rightarrow \gamma \Rightarrow^n w$.
                  \\[0.2cm]
                  Dann haben $\gamma$ und $w$ die Form
                  \\[0.2cm]
                  \hspace*{1.3cm} $\gamma = Y_1 \cdots Y_k$, \quad $w = u_1 \cdots u_k$ \quad und
                  \quad $Y_i \Rightarrow^* u_i$ \quad f\"ur alle $i=1,\cdots,k$.
                  \\[0.2cm]
                  Die Ableitung $Y_i \Rightarrow^* u_i$ kann h\"ochstens aus $n$ Schritten bestehen
                  und daher k\"onnen wir die Induktions-Voraussetzung anwenden und sehen, dass
                  \\[0.2cm]
                  \hspace*{1.3cm} 
                  $\pair(u_i, Y_i) \vdash^* \pair(\varepsilon, [])$ \quad f\"ur alle $i=1,\cdots,k$
                  \\[0.2cm]
                  gilt. Setzen wir diese Rechnungen zusammen, so haben wir
                  \\[0.2cm]
                  \hspace*{1.3cm} 
                  $\pair(u_1 \cdots u_k, Y_1 \cdots Y_k) \vdash^* \pair(\varepsilon, [])$, \quad also 
                  \quad $\pair(w,\gamma) \vdash^* \pair(\varepsilon, [])$.
                  \\[0.2cm]
                  Damit finden wir insgesamt
                  \\[0.2cm]
                  \hspace*{1.3cm}
                  $\pair(w,X) \vdash \pair(w,\gamma) \vdash^* \pair(\varepsilon, [])$
                  \\[0.2cm]
                  und wieder ist die Behauptung gezeigt.
            \end{enumerate}
            Damit ist der Induktions-Beweis abgeschlossen.
\end{enumerate}
Um den Beweis abzuschlie{\ss}en, zeigen wir nun, dass
\\[0.2cm]
\hspace*{1.3cm}
$L(A) \subseteq L\bigl(G(A)\bigr)$
\\[0.2cm]
gilt.  Dazu zeigen wir, dass f\"ur alle $X \in \Gamma$ folgendes gilt:
\\[0.2cm]
\hspace*{1.3cm}
$\pair(w,X) \vdash^n \pair(\varepsilon, []) \;\rightarrow\; X \Rightarrow^n w$.
\\[0.2cm]
Diesen Beweis f\"uhren wir durch Induktion nach $n$.
\begin{enumerate}
\item[I.A.] $n = 1$.  Dann ist $w$ entweder das leere Wort oder $w$ besteht aus einem einzigen Buchstaben 
            $b \in \Sigma$.  Wir untersuchen die beiden F\"alle getrennt.
            \begin{enumerate}
            \item $w = b \in \Sigma$.  Dann gilt $\varepsilon \in \delta(b,X)$.
                  Nach Konstruktion der Grammatik $G(A)$ gibt es eine Regel
                  \\[0.2cm]
                  \hspace*{1.3cm}
                  $X \rightarrow b \varepsilon$ \quad also \quad $X \rightarrow b$.
                  \\[0.2cm]
                  Daraus folgt sofort $X \Rightarrow b = w$.
            \item $w = \varepsilon$.  Dann gilt $\varepsilon \in \delta(\varepsilon,X)$.
                  Nach Konstruktion der Grammatik $G(A)$ gibt es dann eine Regel
                  \\[0.2cm]
                  \hspace*{1.3cm}
                  $X \rightarrow \varepsilon$.
                  \\[0.2cm]
                  Daraus folgt sofort $X \Rightarrow \varepsilon = w$.
            \end{enumerate}
\item[I.S.] $n \mapsto n+1$.  Wieder gibt es zwei F\"alle, die wir getrennt betrachten.
            \begin{enumerate}
            \item Die Rechnung $\pair(w,X) \vdash^n \pair(\varepsilon, [])$
                  f\"angt mit dem \"Ubergang $\gamma \in \delta(b,X)$ an. Dann
                  muss $w = bv$ gelten und wir haben
                  \\[0.2cm]
                  \hspace*{1.3cm}
                  $\pair(w, X) = \pair(bv, X) \vdash \pair(v, \gamma) \vdash^n \pair(\varepsilon, [])$.
                  \\[0.2cm]
                  Dann hat $\gamma$ die Form $\gamma = Y_1 \cdots Y_k$, $v$ l\"asst sich zerlegen
                  als $v = u_1 \cdots u_k$ und es ist $u_i$ gerade der Teil des Wortes $v$, der
                  gelesen wird, um die Variable $Y_i$ vom Stack zu entfernen, es gilt also
                  \\[0.2cm]
                  \hspace*{1.3cm}
                  $\pair(u_i, Y_i) \vdash^* \pair(\varepsilon, [])$.
                  \\[0.2cm]
                  Diese Rechnung kann h\"ochstens $n$ Schritte ben\"otigen.  Also haben wir
                  nach Induktions-Voraussetzung
                  \\[0.2cm]
                  \hspace*{1.3cm}
                  $Y_i \Rightarrow^* u_i$ \quad f\"ur alle $i = 1, \cdots, k$.
                  \\[0.2cm]
                  Wegen $\gamma \in \delta(b,X)$ enth\"alt die Grammatik $G(A)$ die Regel $X \rightarrow b \gamma$ und
                  damit haben wir insgesamt
                  \\[0.2cm]
                  \hspace*{1.3cm}
                  $X \Rightarrow b \gamma = bY_1 \cdots Y_k \Rightarrow^* b u_1 \cdots u_k = bv = w$.
            \item Die Rechnung $\pair(w,X) \vdash^n \pair(\varepsilon, [])$
                  f\"angt mit dem \"Ubergang $\gamma \in \delta(\varepsilon,X)$ an. Dann
                  haben wir
                  \\[0.2cm]
                  \hspace*{1.3cm}
                  $\pair(w, X) \vdash \pair(w, \gamma) \vdash^n \pair(\varepsilon, [])$.
                  \\[0.2cm]
                  Dann hat $\gamma$ die Form $\gamma = Y_1 \cdots Y_k$, $w$ l\"asst sich zerlegen
                  als $w = u_1 \cdots u_k$ und es ist $u_i$ gerade der Teil des Wortes $w$, der
                  gelesen wird, um die Variable $Y_i$ vom Stack zu entfernen, es gilt also
                  \\[0.2cm]
                  \hspace*{1.3cm}
                  $\pair(u_i, Y_i) \vdash^* \pair(\varepsilon, [])$.
                  \\[0.2cm]
                  Diese Rechnung kann h\"ochstens $n$ Schritte ben\"otigen.  Also haben wir
                  nach Induktions-Voraussetzung wieder
                  $Y_i \Rightarrow^* u_i$  f\"ur alle $i = 1, \cdots, k$.
                  Wegen $\gamma \in \delta(\varepsilon,X)$ enth\"alt die Grammatik $G(A)$ die Regel 
                  $X \rightarrow \gamma$ und 
                  damit haben wir insgesamt
                  \\[0.2cm]
                  \hspace*{1.3cm}
                  $X \Rightarrow \gamma = Y_1 \cdots Y_k \Rightarrow^* u_1 \cdots u_k = w$. \qed
            \end{enumerate}
\end{enumerate}

\exercise
Der Keller-Automat $A$ sei wie folgt definiert:
\\[0.2cm]
\hspace*{1.3cm}
$A = \langle \Sigma, \Gamma, \delta, S \rangle$ 
\\[0.2cm] 
Dabei gelte:
\begin{enumerate}
\item $\Sigma := \{ \texttt{a}, \texttt{b} \}$.
\item $\Gamma := \{ A, B, S \}$.
\item $\delta(\varepsilon, A) := \{ \}$, \quad 
      $\delta(\varepsilon, B) := \{ \}$, \quad
      $\delta(\varepsilon, S) := \{ ASA, BSB, A, B, \varepsilon \}$,

      $\delta(\texttt{a}, A) := \{ \varepsilon \}$, \quad
      $\delta(\texttt{a}, B) := \{ \}$, \quad
      $\delta(\texttt{a}, S) := \{ \}$, 

      $\delta(\texttt{b}, A) := \{ \}$, \quad
      $\delta(\texttt{b}, B) := \{ \varepsilon \}$, \quad
      $\delta(\texttt{b}, S) := \{ \}$.
\end{enumerate}
Verwenden Sie das im Beweis von Satz \ref{satz:keller2grammar} verwendete Verfahren um eine
Grammatik zu konstruieren, die dieselbe Sprache beschreibt, die von dem Automaten $A(G)$ akzeptiert
wird.  

%%% Local Variables: 
%%% mode: latex
%%% TeX-master: "formale-sprachen"
%%% End: 
